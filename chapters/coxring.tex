\chapter{Anneaux de Cox}

\section{Un exemple introductif}

Soit $X=\proj(B/I)$ une variété projective et irréductible, où $B=k[x_0,...,x_n]$ et $I\subset B$ un idéal homogène. A la différence du cas affine, l'algèbre graduée $B/I$ des coordonnées homogènes de $X$ n'est pas un invariant. Par exemple, $\PP^1_k=\proj(k[x_0,x_1])$ est isomorphe à tous ses plongements de Veronese, ces isomorphismes provenant de morphismes injectifs mais non surjectifs entre les algèbres de coordonnées homogènes correspondantes.

On considère maintenant le cas où $X$ est donnée par une immersion fermée $X\xhookrightarrow{i}\PP^n_k $ avec $i(X)=\proj(B/I)$. On note d'après \ref{morphismeproj} que cela revient à se donner un diviseur très ample $D$ tel que $\Oo_X(D)\simeq i^*\Oo_{\PP^n_k}(1)$. Considérons l'application canonique $\alpha: \oplus_{l\in\ZZ}\Oo_{\PP^n_k}(l)\rightarrow i_*i^*\oplus_{l\in \ZZ}\Oo_{\PP^n_k}(l)\simeq i_*\oplus_{l\in \ZZ}\Oo_X(lD)$. Sur les sections globales cela donne $\alpha(\PP^n_k): k[x_0,...,x_n]\rightarrow \Gamma_*(\Oo_X(D))=\Ss(X)$, ou $\Ss$ est le faisceau d'algèbres divisorielles sur $X$ associé à $\ZZ D$. Au degré zéro, $\alpha_0=i^\sharp$, on en déduit $\ker \alpha=\Gamma_*(\Ii _{i(X)})=I$ d'après \ref{}, puis $\Im \alpha \simeq B/I$. On a ainsi retrouvé l'algèbre des coordonnées du plongement à partir d'un $\Oo_X$-module inversible, c'est à dire un élément de son groupe de Picard, qui est un objet intrinsèquement défini sur $X$.

D'après \cite{Hartshorne} ex II.5.14, $\Ss(X)$ est la clôture intégrale de $B/I$. Ainsi, $B/I$ est normal si et seulement si $\alpha(\PP^n_k)$ est surjective, on dit alors que le plongement est projectivement normal. Cette remarque montre en particulier que $\Ss(X)$ est une $k$-algèbre de type fini. On redémontre maintenant ce fait en donnant un éclairage géométrique sur ces constructions. Le fibré en droites $L=\spec_{\PP^n_k}(\sym(\Oo_{\PP^n_k}(1))) = \spec_{\PP^n_k}(\oplus_{l\geq 0}\Oo_{\PP^n_k}(l))$ est l'éclatement en l'origine de $\AAA^{n+1}$. En effet, on montre facilement que le $\Oo_{\PP^n_k}$-module des sections de l'éclatement en l'origine de $\AAA^{n+1}$ vu comme fibré en droites sur $\PP_k^n$ est $ \Oo_{\PP^n_k}(-1)$, ce qui permet de conclure d'après la discussion \ref{linebundle}. On a le diagramme commutatif suivant, où $L_X$ est le fibré en droites Spec$_X($Sym$(\Oo_X(D))$, $j$ est une immersion fermée (comme recollement d'immersions fermées) et $\pi$ est propre.

	\begin{center}
	\begin{tikzcd}
  		L_X \arrow[r, hook, "j"] \arrow[d, "p_X"]& L \arrow[d, "p"] \arrow[r, "\pi"]& \AAA^{n+1}_k\\ 
  		X \arrow[r, hook, "i"] & \PP^n_k &
	\end{tikzcd}\\
	\end{center}


On considère comme en \ref{linebundle} l'action de $\GG_m$ sur les fibres de $L$, $L_0$ l'ensemble des points fixes, et $L^\times$ son complémentaire. $L^\times$ est constitué de recollements de schémas isomorphes à Spec$_{U_i}( \Oo_{U_i}[t,t^{-1}])\simeq U_i \times_k \GG_m$, ce qui donne $L^\times=$ Spec$_{\PP^n_k}(\oplus_{l\in \ZZ}\Oo_{\PP^n_k}(lD))$. De plus, $L^\times$ est isomorphe à $\AAA^{n+1}\setminus\lbrace 0 \rbrace$ par restriction de $\pi$. D'autre part, $p$ se restreint en une application $p^\times:L^\times \rightarrow \PP^n_k$ qui est le quotient géométrique de l'action de $\GG_m$. On résume cela dans le diagramme ci-dessous:

	\begin{center}
	\begin{tikzcd}
  		\widetilde{X}:=L_X^\times \arrow[r, hook, "j"] \arrow[d, "p_X^\times"]& L^\times \arrow[d, "p^\times"] \arrow[r, "\simeq"]& \AAA^{n+1}_k\setminus \lbrace 0 \rbrace\\ 
  		X \arrow[r, hook, "i"] & \PP^n_k &
	\end{tikzcd}\\
	\end{center}

Notons encore $\pi$ la restriction $\pi j$, c'est une application propre. Ainsi, $\pi_*\Oo_{L_X}$ est un $\Oo_{\AAA_k^{n+1}}$-module cohérent. On en déduit que $\Oo(L_X)=\pi_*\Oo_{L_X}(\AAA_k^{n+1})$ est un $k[x_0,...,x_n]$-module de type fini, c'est en particulier une $k$-algèbre de type fini. De plus comme les $\Oo_X(lD)$ n'ont pas de sections globales non nulles pour $l<0$, on en déduit que $\Oo(\widetilde{X})=\Gamma(X,\oplus_{l\in \ZZ}\Oo_X(lD))=\Oo(L_X)$ est de type fini sur $k$. C'est de plus une algèbre $\NN$-graduée, son spectre $\bar{X}$ est donc muni d'une action de $\GG_m$ avec un unique point fixe $p_0$ appartenant à l'adhérence de tout orbite. Enfin, pour tout $f\in \Gamma(X,\Oo_X(lD))$ avec $f$ non nulle et $l>0$, on a  $\Oo_{\widetilde{X}}(p_X^{\times-1}(X_f))=\Gamma(X_f,\oplus_{k\in \ZZ}\Oo_X(lD))=\Gamma(X,\oplus_{k\in \ZZ}\Oo_X(lD))_f=\Oo(L_X)_f$. L'unique maximal contenant chaque $f$ correspond à $p_0$, on voit ainsi que ces spectres se recollent en $\widetilde{X}=\bar{X}\setminus \lbrace p_0 \rbrace$, où $\bar{X}=\spec\Oo(\widetilde{X})$.\\

Prenons maintenant un exemple concret, soit $X\subset \PP^3_k$ la quadrique projective d'équation en coordonnées homogènes $x_1x_2=x_3x_4$, et $D$ le diviseur des zéros de la section $x_4\in\Oo_X(1)$. $D$ est un diviseur de Cartier effectif sur $X$ que l'on peut décrire sur les ouverts standards par $(U_i, x_4/x_i)_{1\leq i\leq 4}$. De plus, $D$ est très ample car $\Oo_X(D)\simeq \Oo_X(1)$. En effet, en regardant ces $\Oo_X$-modules comme sous-modules de $k(x_1,...,x_4)$, on voit l'isomorphisme en multipliant les générateurs locaux $x_i/x_4$ de $\Oo_X(D)$ par $x_4$. L'algèbre des coordonnées homogènes de $X$ est normale, donc en reprenant les notations précédentes on obtient que $\bar{X}$ est le cone affine d'équation $t_1t_2-t_3t_4$ dans $\AAA^4$ et $\widetilde{X}=\bar{X}\setminus\lbrace 0 \rbrace$ le cône épointé. 

On va maintenant utiliser cette construction pour calculer le groupe des classes $\clg(\bar{X})\simeq \clg(\widetilde{X})$. En tant que fibré en droite sur $X$ on voit en utilisant \cite{Hartshorne} II.6.6 que Cl$(X)\simeq$ Cl$(L_X)$, et le pullback de $D$ par cet isomorphisme est exactement $\pi^{-1}(0)$. Puis on a $\widetilde{X}=L_X \setminus \pi^{-1}(0)$ où le diviseur exceptionnel $\pi^{-1}(0)$ est isomorphe à $X$. D'après ce qui précède et comme $\pi^{-1}(0)$ est irréductible et de codimension $1$ dans $\widetilde{X}$, on a d'après \ref{divexactseq} une suite exacte:
\begin{center}
$ 0\rightarrow\ZZ\xrightarrow{\phi}$ Cl$(X) \rightarrow$ Cl$(\widetilde{X}) \rightarrow 0$, $\phi(1)=D$
\end{center}
Par ailleurs dans \cite{Hartshorne} II.6.6.1 on calcule Cl$(X)\simeq \ZZ^2$ avec $D=(1,1)$. On en déduit Cl$(\widetilde{X})\simeq \ZZ$ avec deux générateurs de somme nulle, $D_1=\mathcal{V}(p_1)$ avec $p_1=(t_1,t_4)$ et $D_2=\mathcal{V}(p_2)$ avec $p_2=(t_2,t_4)$.

Calculons maintenant les sections globales du faisceau d'algèbres divisorielles $\Ss$ sur $Y:=\widetilde{X}$ associé à $K=\ZZ D_2$. On a $\Gamma(Y, \Oo_Y(-D_2))=\Gamma(\bar{X}, \Oo_{\bar{X}}(-D_2))=p_2$ d'après \ref{divaff}. Puis comme comme le cône épointé $Y$ est lisse, $D_2$ est Cartier dessus. Ainsi $\Oo_Y(-D_2)$ est un faisceau inversible sur $Y$ d'inverse $\Oo_Y(D_2)$, ce dernier étant la restriction à $Y$ du $\Oo_{\bar{X}}$-module $(t_4^{-1}p_1)^{\widetilde{}}$. En effet, on a $p_1p_2=(t_4)$ car ce sont deux idéaux radicaux définissant le même fermé.  On a donc $\Gamma(Y, \Oo_Y(D_2))=t_4^{-1}p_1=(1, t_4^{-1}t_1)$. Ainsi $\Gamma(Y,S)$ est engendré en tant que $\Oo(Y)$-algèbre par les éléments algébriquement libres $(z_1,z_2,z_3,z_4):=(1, t_4^{-1}t_1,t_2,t_4)$. De plus d'après les équations suivantes la composante homogène de degré zéro est inclus dans $k[z_1,z_2,z_3,z_4]$:
$$ z_2z_4=t_1,\, z_1z_3=t_2,\, z_2z_3=t_3,\, z_1z_4=t_4$$

On a donc $\Gamma(X, S)=k[z_1,z_2,z_3,z_4]$, c'est une algèbre de polynômes naturellement graduée par deg$(z_1)$ = deg$(z_2)=1$, deg($z_3$) $=$ deg($z_4$)$=-1$. Cette $\ZZ$-graduation se traduit en une action de $\GG_m$ sur $Y$ donnée  concrètement par $\lambda.z=(\lambda z_1,\lambda z_2,\lambda^{-1} z_3,\lambda^{-1} z_4)$. L'algèbre des invariants sous cette action est $k[z_1z_2,z_1z_4,z_2z_3,z_2z_4]$ ce qui donne un isomorphisme $\spec(\Gamma(Y,S)^{\GG_m})\simeq Y$.

On vient ainsi de voir dans cet exemple qu'un faisceau d'algèbres divisorielles $\Ss$ bien choisi permet de retrouver $Y$ comme bon quotient d'une $H$-variété construite à partir de ce faisceau (où $H$ définit la graduation de $\Ss$). C'est ce qu'on va étudier de manière générale dans cette partie.

\section{Cas d'un groupe des classes sans torsion}

Dans cette partie on définit l'anneau de Cox d'une variété $X$ normale irréductible avec un groupe des classes libre de type fini. On est en particulier dans le cadre de l'exemple précédent.

\subsection{Faisceau et anneau de Cox}

\begin{cons}[Faisceau de Cox, Anneau de Cox]\ref{consFreeCoxRing}
Soit $X$ une variété normale irréductible avec un groupe des classes libre de type fini et un sous-groupe $K\subset \clg(X)$ se projetant isomorphiquement sur Cl$(X)$. Il existe de tels $K$ car $\clg(X)$ est libre de type fini donc la projection admet des sections. On définit le faisceau de Cox sur $X$, noté $\Rr$ comme le faisceau d'algèbres divisorielles associé à $K$. Cette description ne dépend qu'à isomorphisme près du choix de $K$. L'anneau de Cox de $X$ est l'algèbre des sections globales du faisceau de Cox. 
\end{cons}
\begin{proof}
Soient $K,K'$ deux sous-groupes de $\wdiv(X)$ se projetant isomorphiquement sur $\clg(X)$, et $\Rr,\Rr'$ les faisceaux de Cox correspondants. On choisit une base $(D_1,...,D_s)$ de $K$. Cette base définit une section de la projection $c:\wdiv(X)\rightarrow \clg(X)$ et on peut modifier cette section par des éléments du noyau, cela fournissant autant de bases de sous-groupes de $\wdiv(X)$ se projetant isomorphiquement sur $\clg(X)$. Par ailleurs, chaque $D_i+\Ker c$ rencontre nécessairement $K'$ sinon on aurait $\rg K' < \rg \clg(X)$ comme rang de modules libres de type fini. On choisit ainsi $f_1,...,f_s\in k(X)$ tels que $(D_i-\divi(f_i))_i$ forme une base de $K'$. On définit un morphisme $\alpha:K\rightarrow k(X)^*, a_1D_1+...+a_sD_s\mapsto f_1^{a_1}...f_s^{a_s}$. Avec cela, l'isomorphisme linéaire faisant correspondre les bases de $K$ et $K'$ est $\widetilde{\psi}:K\rightarrow K',\, D\mapsto -\divi(\alpha(D))+D$. Enfin on définit un isomorphisme d'algèbres divisorielles $(\psi, \widetilde{\psi}): \Rr\rightarrow\Rr'$ en posant $f\in \Gamma(U, \Rr_D)\mapsto \alpha(D)f$.
\end{proof}

\begin{ex}
Soit $X=\proj B/I$ une variété projective normale et irréductible, où $B=k[x_0,...,x_n]$ et $I\subset B$ un idéal homogène. Si $X$ est projectivement normale, c'est à dire que $B/I$ est intégralement clos, et que $\clg(X)$ est libre de rang $1$ engendré par l'intersection de $X$ avec un hyperplan de $\PP^n_k$. Alors $\Rr(X)=B/I=\Oo(\bar{X})$ où $\bar{X}$ est le cône affine dans $\AAA^{n+1}$ correspondant à $X$.
\end{ex}

\subsection{Propriétés algébriques de l'anneau de Cox}

\begin{prop}
Soit $X$ une variété normale irréductible avec un groupe des classes libre de type fini. Alors:
\begin{enumerate}
\item L'anneau de Cox $\Rr(X)$ est factoriel.
\item Le groupe des unités de l'anneau de Cox est $\Rr(X)^\times=\Gamma(X,\Oo^\times)$
\end{enumerate}
\end{prop}
\begin{proof}
\begin{enumerate}
\item On peut supposer $X$ lisse car $X_{sing}$ est de codimension $\geq 2$ donc $\Gamma(X, \Rr)=\Gamma(X_{reg}, X)$. Ainsi $\widetilde{X}$ est lisse (par vérification locale immédiate) et le théorème \ref{clgtrivial} s'applique. Soit $f\in \Rr(X)=p_*\Oo_{\widetilde{X}}(X)=\Oo_{\widetilde{X}}(\widetilde{X})$. On a donc le diviseur effectif $\divi(f)= \sum n_iD_i=\sum n_i\divi(f_i)=\divi(\prod f_i^{n_i})$ où les $D_i$ sont des diviseurs premiers, chaque $f_i$ est irréductible et appartient à $\Rr(X)$ d'après \ref{caracfaisceaustructdiv}. L'unicité de l'écriture sur $\wdiv(X)$ donne l'unicité de l'écriture $f=u\prod f_i^{n_i}$ où $u\in \Rr(X)^\times$.
\item Une inclusion est évidente. Pour l'autre prenons $f\in\Rr(X)^\times$, qui est homogène d'après \ref{Coxgraded}, disons de degré $D$. Alors, $fg=1\in \Rr_0(X)$ pour un $g\in\Rr(X)^\times$ homogène de degré $-D$. Ainsi on a $0=\divi_0(1)=\divi_{D-D}(fg)=\divi_D(f)+\divi_{-D}(g)$. Les deux derniers diviseurs étant effectifs, on a $\divi_D(f)=0$ et donc $D=-\divi(f)$, ce qui donne $f\in\Rr_0(X)=\Oo(X)^\times$ d'où le résultat.
\end{enumerate}
\end{proof}

\begin{ex}
Reprenons le cas du cône affine de l'exemple introductif.  $\Rr(Y)$ est l'algèbre de polynômes sur $k$ à $4$ indéterminés, donc est factoriel en particulier.  Par ailleurs, $\Oo(Y)^{\times}=k^*$ car en inversant par exemple $t_1$  on obtient $Y_{t_1}\simeq \AAA^{1}\setminus \lbrace0\rbrace\times\AAA^{2}$, d'où le résultat.
\end{ex}

\begin{prop}
Soit $X$ vérifiant $(\dagger)$. Alors:
\begin{enumerate}
\item Soient $f\in \Rr_D(X)$ et $g\in \Rr_E(X)$ non nulles. Alors $f\mid g\iff \divi_D(f)\leq\divi_E(g)$.
\item Soit $f\in \Rr_D(X)$ non nulle. Alors $f$ est premier si et seulement si $\divi_D(f)$ est premier.
\end{enumerate}
\end{prop}
\begin{proof}
Soit $f,g$ des fonctions régulières non-nulles sur $\widetilde{X}$. Comme $\widetilde{X}$ est intègre, on peut les voir comme des éléments de $k(X)^*$. Ainsi $f\mid g$ dans $\Rr(X)$ si et seulement si $\divi(f^{-1}g)\geq 0$. Prenant $f,g$ comme dans l'énoncé et en utilisant \ref{pstarprincipal}, que c'est équivalent à $\divi_D(f)\leq\divi_E(g)$. La preuve du deuxième énoncé est similaire.
\end{proof}

\section{Cas d'un groupe des classes avec torsion}

\subsection{Faisceau et anneau de Cox}

\begin{cons}\label{conscoxtorsion}
Soit $X$ une variété irréductible, normale, telle que $\Gamma(X,\Oo^\times)=k^*$ et $\clg(X)$ est de type fini. On se fixe un sous-groupe $K\subset \wdiv(X)$ tel que la projection $c:K\rightarrow \clg(X)$ est surjective. Soit $K^0:=\ker c$ et $\chi:K^0\rightarrow k(X)^*$ un caractère, c'est à dire un morphisme de groupes tel que $\divi(\chi(E))=E,\,\forall E\in K^0$. Soit $\Ss$ le faisceau d'algèbres divisorielles associé à $K$. Soit $\Ii$ le faisceau d'idéaux défini par l'image du morphisme 
$$\bigoplus_{E\in K^0}\Ss\rightarrow \Ss,\, E\mapsto 1-\chi(E),\,\,\, 1\in \Ss_0$$
Le faisceau de Cox associé à $K$ et $\chi$ est le faisceau quotient $\Rr:=\Ss/\Ii$. C'est une $\Oo_X$-algèbre quasi-cohérente et $\clg(X)$-graduée de la manière suivante: 
$$\Rr=\bigoplus_{[D]\in \clg(X)}\Rr_{[D]},\,\,\,\,\,\, \Rr_{[D]}:=\pi(\bigoplus_{D'\in c^{-1}([D])}\Ss_{D'}),\,\,\,\, \text{où }\pi:\Ss\rightarrow \Rr\text{ est la projection}$$
L'anneau de Cox associé à $K$ et $\chi$ est l'anneau des sections globales de $\Rr$.
\end{cons}

La $\clg(X)$-graduation de $\Rr$ annoncée ci-dessus n'est pas évidente à priori. On clarifie cela dans la proposition ci-dessous.

\begin{prop}
Avec les notation de la construction \ref{conscoxtorsion}, $\Ss$ est naturellement muni d'une $\clg(X)$-graduation:
$$\Ss=\bigoplus_{[D]\in\clg(X)}\Ss_{[D]},\,\,\,\,\,\,\, \Ss_{[D]}:=\bigoplus_{D'\in c^{-1}([D])}\Ss_{D'}$$
Soit $f\in\Gamma(U,\Ii)$ et $D\in K$, alors la composante $\clg(X)$-homogène $f_{[D]}\in \Gamma(U,\Ss_{[D]})$ s'écrit de manière unique sous la forme:
$$f_{[D]}=\sum_{E\in K^0}(1-\chi(E))f_E, \text{  où } f_E\in\Gamma(U,\Ss_D), \text{ et }\chi(E)\in \Gamma(U, \Ss_{-E})$$
En particulier, $\Ii$ est un faisceau d'idéaux $\clg(X)$-homogènes, et $\pi$ est un morphisme de $\Oo_X$-algèbres $\clg(X)$-graduées. De plus, si $f\in \Gamma(U,\Ii)$ est $K$-homogène, alors c'est la section nulle.
\end{prop}
\begin{proof}
On montre d'abord l'unicité. Si on obtient une telle écriture alors les composantes $K$-homogènes  $f_{[D]}$ sont facilement identifiables, il s'agit pour chaque degré $D-E\in c^{-1}([D])$ de $-\chi(E)f_E$.

Montrons l'existence. Par définition de $\Ii$, chaque germe $f_{[D],x}$ a une représentation sur un voisinage ouvert $U_x$ de $x$ par une section 
$$g=\sum_{E\in K^0}(1-\chi(E))g_E \in \Gamma(U_x, \Ii)\text{, où }g_E\in \Gamma(U_x, \Ss_{[D]})$$
On écrit la décomposition en composantes $K$-homogènes de chaque $g_E$:
$$g_E=\sum_{D'\in D+K^0}g_{E,D'}, \text{ où } g_{E,D'}\in\Gamma(U_x, \Ss_{D'})$$
La section $g'_{E,D'}:=\chi(D'-D)g_{E,D'}$ est $K$-homogène de degré $D$ et on a l'identité 
$$(1-\chi(E))g_{E,D'}=(1-\chi(E+D-D'))g'_{E,D'}-(1-\chi(D-D'))g'_{E,D'}$$
On obtient ainsi l'écriture désirée localement sur des ouverts qui recouvrent $X$. Par irréductibilité de $X$ on obtient l'unicité de l'écriture globalement en recollant ces sections.
\end{proof}

\begin{cor}
Supposons $K$ de type fini, et soit $E_1,...,E_s$ une base de $K^0$. Alors pour tout ouvert $U\subset X$, l'idéal $\Gamma(U,\Ii)$ est engendré par $1-\chi(E_i)$, $1\leq i\leq s$.
\end{cor}
\begin{proof}
C'est une conséquence de la proposition précédente et des identités suivantes:
$$1-\chi(E+E')=(1-\chi(E))+(1-\chi(E'))\chi(E)$$
$$1-\chi(-E)=(1-\chi(E))(-\chi(-E))$$
\end{proof}

\begin{prop}
Avec les notations de \ref{conscoxtorsion}, on a pour tout $D\in K$ un isomorphisme de faisceaux $\pi_{|\Ss_D}:\Ss_D\rightarrow \Rr_{[D]}$.
\end{prop}
\begin{proof}

\end{proof}