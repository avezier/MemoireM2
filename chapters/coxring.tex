\chapter{Anneaux de Cox}

\section{Un exemple introductif}




Considérons une variété irréductible $X$ est un diviseur très ample $D$ tel que $\mathcal{O}_X(D)\simeq i^*\mathcal{O}_{\PP^n_k}(1)$ où $X\xhookrightarrow{i}\PP^n_k $ est une immersion fermée. Le fibré en droites $L=$ Spec$_{\PP^n_k}($Sym$(\mathcal{O}_{\PP^n_k}(1))) = \oplus_{k\geq 0}\mathcal{O}_{\PP^n_k}(k)$ est l'éclatement en l'origine de $\AAA^{n+1}$. En effet, on montre facilement que le $\mathcal{O}_{\PP^n_k}$-module des sections de l'éclatement en l'origine de $\AAA^{n+1}$ vu comme fibré en droites sur $\PP_k^n$ est $ \mathcal{O}_{\PP^n_k}(-1)$, ce qui permet de conclure d'après la discussion \ref{linebundle}. On a le diagramme commutatif suivant, où $L_X$ est le fibré en droites Spec$_X($Sym$(\mathcal{O}_X(D))$, $j$ est une immersion fermée (comme recollement d'immersions fermées) et $\pi$ est propre.

	\begin{center}
	\begin{tikzcd}
  		L_X \arrow[r, hook, "j"] \arrow[d, "p_X"]& L \arrow[d, "p"] \arrow[r, "\pi"]& \AAA^{n+1}_k\\ 
  		X \arrow[r, hook, "i"] & \PP^n_k &
	\end{tikzcd}\\
	\end{center}


On considère comme en \ref{linebundle} l'action de $\GG_m$ sur les fibres de $L$, $L_0$ l'ensemble des points fixes, et $L^\times$ son complémentaire. $L^\times$ est constitué de recollements de schémas isomorphes à Spec$_{U_i}( \mathcal{O}_{U_i}[t,t^{-1}])\simeq U_i \times_k \GG_m$, ce qui donne $L^\times=$ Spec$_{\PP^n_k}(\oplus_{k\in \ZZ}\mathcal{O}_{\PP^n_k}(kD))$. De plus, $L^\times$ est isomorphe à $\AAA^{n+1}\setminus\lbrace 0 \rbrace$ par restriction de $\pi$. D'autre part, $p$ se restreint en une application $p^\times:L^\times \rightarrow \PP^n_k$ qui est le quotient géométrique de l'action de $\GG_m$. On résume cela dans le diagramme ci-dessous:

	\begin{center}
	\begin{tikzcd}
  		\widetilde{X}:=L_X^\times \arrow[r, hook, "j"] \arrow[d, "p_X^\times"]& L^\times \arrow[d, "p^\times"] \arrow[r, "\simeq"]& \AAA^{n+1}_k\setminus \lbrace 0 \rbrace\\ 
  		X \arrow[r, hook, "i"] & \PP^n_k &
	\end{tikzcd}\\
	\end{center}

Notons encore $\pi$ la restriction $\pi j$, c'est une application propre. Ainsi, $\pi_*\mathcal{O}_{L_X}$ est un $\mathcal{O}_{\AAA_k^{n+1}}$-module cohérent. On en déduit que $\mathcal{O}(L_X)=\pi_*\mathcal{O}_{L_X}(\AAA_k^{n+1})$ est un $k[x_0,...,x_n]$-module de type fini, c'est en particulier une $k$-algèbre de type fini. De plus comme les $\mathcal{O}_X(kD)$ n'ont pas de sections globales non nulles pour $k<0$, on en déduit que $\mathcal{O}(\widetilde{X})=\Gamma(X,\oplus_{k\in \ZZ}\mathcal{O}_X(kD))=\mathcal{O}(L_X)$ est de type fini sur $k$. C'est de plus une algèbre $\NN$-graduée, son spectre $\bar{X}$ est donc muni d'une action de $\GG_m$ avec un unique point fixe $p_0$ appartenant à l'adhérence de tout orbite. 

Enfin, considérons l'application canonique $\alpha: \oplus_{l\in\ZZ}\mathcal{O}_{\PP^n_k}(l)\rightarrow i_*i^*\oplus_{l\in \ZZ}\mathcal{O}_{\PP^n_k}(l)=i_*\oplus_{l\in \ZZ}\mathcal{O}_X(l)$. Sur les sections globales cela donne $\alpha(\PP^n_k): k[x_0,...,x_n]\rightarrow \mathcal{O}_{\widetilde{X}}(\widetilde{X})$.\\
Sur chaque $U_i=D_+(x_i)$, on a $\mathcal{O}_{\widetilde{X}}(p_X^{\times-1}(U_i))=\Gamma(U_i,\oplus_{k\in \ZZ}\mathcal{O}_X(kD))=\Gamma(X,\oplus_{k\in \ZZ}\mathcal{O}_X(kD))_{x_i}=\mathcal{O}_{L_X}(L_X)_{x_i}$. L'unique maximal contenant chaque $x_i$ correspond à $p_0$, on voit ainsi que ces spectres se recollent en $\widetilde{X}=\bar{X}\setminus \lbrace p_0 \rbrace$.\\

Prenons un exemple concret, soit $X\subset \PP^3_k$ la quadrique projective d'équation en coordonnées homogènes $xy=zw$. L'algèbre des coordonnées homogènes est normale, donc en reprenant les notations précédentes on obtient que $\bar{X}$ est le cone affine d'équation $t_1t_2-t_3t_4$ dans $\AAA^4$ et $\widetilde{X}=\bar{X}\setminus\lbrace 0 \rbrace$ le cône épointé. On va utiliser cette construction pour calculer le groupe des classes Cl$(\bar{X})=$ Cl$(\widetilde{X})$. En tant que fibré en droite sur $X$ on voit en utilisant\cite{Hartshorne} II.6.6 que Cl$(X)$ = Cl$(L_X)$ et on a $\widetilde{X}=L_X \setminus \pi^{-1}(0)$ où le diviseur exceptionnel $\pi^{-1}(0)$ est isomorphe à $X$. On a donc une suite exacte:
\begin{center}
$0\rightarrow \ZZ\rightarrow$ Cl$(X) \rightarrow$ Cl$(\widetilde{X}) \rightarrow 0$
\end{center}
On voit donc que Cl$(Y)$ admet deux générateurs de somme nulle, $D_1=\mathcal{V}(p_1)$, avec $p_1=(t_1,t_4)$ et $D_2=\mathcal{V}(p_2)$ avec $p_2=(t_2,t_4)$. Calculons maintenant les section globales du faisceau d'algèbres divisorielles $S$ associé à $K=\ZZ D_2$. On vérifie facilement en regardant les valuations dans les deux anneaux locaux que $D_1+D_2=$ div$(t_4)$. $p_2$ est CARTIER car loacalement principalAinsi $p_1p_2=(t_4)$, ce qui montre que $p_2$ est inversible avec $p_2^{-1}=t_4^{-1}p_1=(1, t_4^{-1}t_1)$. Par ailleurs, on voit en inversant par exemple $t_1$  que $X_{t_1}\simeq \AAA^{1}\setminus \lbrace0\rbrace\times\AAA^{2}$. On en déduit que $\mathcal{O}_X(X)^{\times}=k^*$. En notant $(z_1,z_2,z_3,z_4)=(1, t_4^{-1}t_1,t_2,t4)$, on a donc $\Gamma(X, S)=k[z_1,z_2,z_3,z_4]$. C'est une algèbre de polynômes, car les $z_i$ sont algébriquement libres, et on qu'elle est factorielle comme attendu (\ref{}) . De plus, elle est naturellement graduée par deg$(z_1)$ = deg$(z_2)=1$, deg($z_3$) $=$ deg($z_4$)$=-1$. Cette $\ZZ$-graduation se traduit en une action de $\GG_m$ sur $Y$ donnée  concrètement par $\lambda.z=(\lambda z_1,\lambda z_2,\lambda^{-1} z_3,\lambda^{-1} z_4)$. L'algèbre des invariants sous cette action est $k[z_1z_2,z_1,z_4,z_2z_3,z_2z_4]$ ce qui donne un isomorphisme $\spec(\Gamma(Y,S))^{\GG_m}\simeq Y$. Or comme $Y$ est affine, $\spec(\Gamma(Y,S))$ n'est autre que 

\section{Groupe des classes sans torsion}


\section{Groupe des classes avec torsion}
