\chapter{Anneaux de Cox}

\section{Un exemple introductif}
\label{ExIntroCox}
Soit $X=\proj(B/I)$ une variété projective normale et irréductible, où $B=k[x_0,...,x_d]$ et $I\subset B$ est un idéal premier homogène et radical. A la différence du cas affine, l'algèbre graduée $B/I$ des coordonnées homogènes de $X$ n'est pas un invariant. Par exemple, $\PP^1_k=\proj(k[x_0,x_1])$ est isomorphe à tous ses plongements de Veronese, ces isomorphismes provenant de morphismes injectifs mais non surjectifs entre les algèbres de coordonnées homogènes correspondantes.

On considère maintenant le cas où $X$ est donnée par une immersion fermée $X\xhookrightarrow{i}\PP^d_k $ avec $i(X)\simeq \proj B/I$. Considérons l'application canonique $\alpha: \oplus_{n\in\ZZ}\Oo_{\PP^d_k}(n)\rightarrow i_*i^*\oplus_{n\in \ZZ}\Oo_{\PP^d_k}(n)\simeq i_*\oplus_{n\in \ZZ}\Oo_X(n)$. Sur les sections globales cela donne $\alpha(\PP^d_k): k[x_0,...,x_d]\rightarrow \Gamma_*(\Oo_X)$. Au degré zéro, $\alpha_0=i^\sharp$, on en déduit $\Ker \alpha(\PP^d_k)=\Gamma_*(\Ii _{i(X)})=I$ car $I$ est radical (\ref{FQCProj} et \cite[ex II.5.10]{Hartshorne}). Ainsi, $\Im \alpha(\PP^d_k)$ est l'algèbre des coordonnées homogènes du plongement de $X$ dans $\PP_k^d$. On l'a déterminée à partir d'un $\Oo_X$-module inversible, c'est à dire un élément de son groupe de Picard, qui est un objet intrinsèquement défini sur $X$. Notons qu'il existe un diviseur $D\in \wdiv(X)$ tel que $\Oo_X(D)\simeq\Oo_X(1)$, et qu'alors $\Gamma_*(\Oo_X)$ est isomorphe à l'anneau des sections globales du faisceau d'algèbres divisorielles sur $X$ associé à $\ZZ D$.

D'après \cite[ex II.5.14]{Hartshorne} , $\Gamma_*(\Oo_X)$ est la clôture intégrale de $B/I$. Ainsi, $B/I$ est normal si et seulement si $\alpha(\PP^d_k)$ est surjective, on dit alors que le plongement est projectivement normal. Cette remarque montre en particulier que $\Gamma_*(\Oo_X)$ est une $k$-algèbre de type fini (\ref{IntClosureFinite}). On redémontre maintenant ce fait en donnant un éclairage géométrique sur ces constructions. Le fibré en droites $L=\spec_{\PP^d_k}(\sym(\Oo_{\PP^d_k}(1))) = \spec_{\PP^d_k}(\oplus_{n\geq 0}\Oo_{\PP^d_k}(n))$ est l'éclatement en l'origine de $\AAA^{d+1}$. En effet, on montre facilement que le $\Oo_{\PP^d_k}$-module des sections de l'éclatement en l'origine de $\AAA^{d+1}$ vu comme fibré en droites sur $\PP_k^d$ est $ \Oo_{\PP^d_k}(-1)$, ce qui permet de conclure d'après la discussion suivant \ref{linebundle}. On a le diagramme commutatif suivant, où $L_X$ est le fibré en droites $\spec_X(\sym(\Oo_X(1))$, $j$ est une immersion fermée (comme recollement d'immersions fermées) et $\pi$ est propre.

	\begin{center}
	\begin{tikzcd}
  		L_X \arrow[r, hook, "j"] \arrow[d, "p_X"] \arrow[rd, phantom, "\square"]& L \arrow[d, "p"] \arrow[r, "\pi"]& \AAA^{d+1}_k\\ 
  		X \arrow[r, hook, "i"] & \PP^d_k &
	\end{tikzcd}\\
	\end{center}


On considère comme en \ref{linebundle} l'action de $\GG_m$ sur les fibres de $L$, l'ensemble des points fixes $L_0$, et $L^\times$ son complémentaire. En considérant un recouvrement $(U_i)_i\in I$ qui trivialise $\Oo_{\PP^d_k}(1)$, on voit que $L^\times$ est constitué de recollements de schémas isomorphes à $\spec_{U_i}( \Oo_{U_i}[t,t^{-1}])\simeq U_i \times_k \GG_m$, ce qui donne $L^\times=\spec_{\PP^d_k}(\oplus_{n\in \ZZ}\Oo_{\PP^d_k}(n))$. De plus, $L^\times$ est isomorphe à $\AAA^{d+1}\setminus\lbrace 0 \rbrace$ par restriction de $\pi$. D'autre part, $p$ se restreint en une application $p^\times:L^\times \rightarrow \PP^d_k$ qui est le quotient géométrique de l'action de $\GG_m$. On résume cela dans le diagramme ci-dessous:

	\begin{center}
	\begin{tikzcd}
  		\widetilde{X}:=L_X^\times \arrow[r, hook, "j"] \arrow[d, "p_X^\times"]& L^\times \arrow[d, "p^\times"] \arrow[r, "\simeq"]& \AAA^{d+1}_k\setminus \lbrace 0 \rbrace\\ 
  		X \arrow[r, hook, "i"] & \PP^d_k &
	\end{tikzcd}\\
	\end{center}

Notons encore $\pi$ la restriction $\pi j$, c'est une application propre. Ainsi, $\pi_*\Oo_{L_X}$ est un $\Oo_{\AAA_k^{d+1}}$-module cohérent. On en déduit que $\Oo(L_X)=\pi_*\Oo_{L_X}(\AAA_k^{d+1})$ est un $k[x_0,...,x_d]$-module de type fini, c'est en particulier une $k$-algèbre de type fini. De plus comme les faisceaux $\Oo_X(n)$ n'ont pas de sections globales non nulles pour $n<0$, on en déduit que $\Oo(\widetilde{X})=\Gamma(X,\oplus_{n\in \ZZ}\Oo_X(n))=\Oo(L_X)$ est de type fini sur $k$. C'est de plus une algèbre $\NN$-graduée, son spectre $\bar{X}$ est donc muni d'une action de $\GG_m$ avec un unique point fixe $p_0$ appartenant à l'adhérence de tout orbite. Pour tout $f\in \Gamma(X,\Oo_X(n))$ avec $f$ non nulle et $n>0$, on a  $\Oo_{\widetilde{X}}(p_X^{\times-1}(X_f))=\Gamma(X_f,\oplus_{n\in \ZZ}\Oo_X(n))=\Gamma(X,\oplus_{n\in \ZZ}\Oo_X(n))_f=\Oo(L_X)_f$. L'unique maximal contenant chaque $f$ correspond à $p_0$, on voit ainsi que ces spectres se recollent en $\widetilde{X}=\bar{X}\setminus \lbrace p_0 \rbrace$, où $\bar{X}=\spec\Oo(\widetilde{X})$. Enfin, $(X,p^\times_X)$ est le quotient géométrique de $\widetilde{X}$ pour l'action de $\GG_m$.\\

Prenons maintenant un exemple concret, soit $X\subset \PP^3_k$ la quadrique projective d'équation en coordonnées homogènes $x_1x_2=x_3x_4$, et $D$ le diviseur des zéros de la section $x_4\in\Oo_X(1)$. Le diviseur $D$ est de Cartier sur $X$, on peut le décrire sur les ouverts standards par $(U_i, x_4/x_i)_{1\leq i\leq 4}$. De plus, $D$ est très ample car $\Oo_X(D)\simeq \Oo_X(1)$. En effet, en regardant ces $\Oo_X$-modules comme sous-modules de $k(x_1,...,x_4)$, on voit l'isomorphisme en multipliant les générateurs locaux $x_i/x_4$ de $\Oo_X(D)$ par $x_4$. L'algèbre des coordonnées homogènes de $X$ est normale d'après \ref{SerreCritere2} et \ref{CritereReg}, donc en reprenant les notations précédentes on obtient que $\bar{X}$ est le cone affine d'équation $t_1t_2-t_3t_4$ dans $\AAA^4$ et $\widetilde{X}=\bar{X}\setminus\lbrace 0 \rbrace$ le cône épointé. 

On va maintenant utiliser cette construction pour calculer le groupe des classes $\clg(\bar{X})\simeq \clg(\widetilde{X})$. En tant que fibré en droites sur $X$ on voit en utilisant \cite[II.6.6]{Hartshorne} que Cl$(X)\simeq$ Cl$(L_X)$, et l'image inverse de $D$ par cet isomorphisme est exactement $\pi^{-1}(0)$. Puis on a $\widetilde{X}=L_X \setminus \pi^{-1}(0)$ où le diviseur exceptionnel $\pi^{-1}(0)$ est isomorphe à $X$. D'après ce qui précède et comme $\pi^{-1}(0)$ est irréductible et de codimension $1$ dans $\widetilde{X}$, on a d'après \ref{divexactseq} une suite exacte:
\begin{center}
$ 0\rightarrow\ZZ\xrightarrow{\phi}$ Cl$(X) \rightarrow$ Cl$(\widetilde{X}) \rightarrow 0$, où $\phi(1)=D$
\end{center}
Par ailleurs dans \cite[II.6.6.1]{Hartshorne} on calcule $\clg(X)\simeq \ZZ^2$ avec $D=(1,1)$. On en déduit $\clg(\widetilde{X})\simeq \ZZ$ avec deux générateurs de somme nulle, $D_1=\mathcal{V}(\mathfrak{p}_1)$ avec $\mathfrak{p}_1=(t_1,t_4)$ et $D_2=\mathcal{V}(\mathfrak{p}_2)$ avec $\mathfrak{p}_2=(t_2,t_4)$.

Calculons maintenant les sections globales du faisceau d'algèbres divisorielles $\Ss$ sur $\bar{X}$ associé à $K=\ZZ D_2$, on note également $\Ss$ sa restriction à $\widetilde{X}$ en se rappelant \ref{isomorphismcodim2CorXreg}. On a $\Ss_{-D_2}(\bar{X})=\mathfrak{p}_2$ d'après \ref{divaff}. Le diviseur $D_2$ est de Cartier sur le cône épointé lisse $\widetilde{X}$. Ainsi $\Ss_{-D_2}$ est un faisceau inversible sur $\widetilde{X}$ d'inverse $\Oo_{\widetilde{X}}(D_2)$, ce dernier étant la restriction à $\widetilde{X}$ du $\Oo_{\bar{X}}$-module $(t_4^{-1}\mathfrak{p}_1)^{\widetilde{}}$. En effet, on a $\mathfrak{p}_1\mathfrak{p}_2=(t_4)$ car ceux sont deux idéaux radicaux définissant le même fermé.  On a donc $\Ss_{D_2}(\widetilde{X})=t_4^{-1}\mathfrak{p}_1=(1, t_4^{-1}t_1)$. Les applications naturelles $\Ss_{\pm D2}(\widetilde{X})^{\otimes n}\xhookrightarrow{} \Ss_{\pm nD2}(\widetilde{X})$ avec $n>0$, ne sont pas surjectives à priori. On est donc contraint d'examiner localement la situation sur un recouvrement bien choisi. On a quatre ouverts affines $\widetilde{X}_{t_i}$ qui recouvrent $\widetilde{X}$, et $\Ss_{\pm nD2}(\widetilde{X})=\cap_i \Ss_{\pm nD2}(\widetilde{X}_{t_i})$, où l'intersection est prise dans $k(\widetilde{X})$. On a d'après \ref{FQCProp2}, $\Ss_{\pm nD_2}(\widetilde{X}_{t_i})=\Ss_{\pm nD_2}(\widetilde{X})_{t_i}$, et on calcule 
$$\Ss_{-nD_2}(\widetilde{X}_{t_1})=(t_4^n)_{t_1},\,\Ss_{-nD_2}(\widetilde{X}_{t_2})=(1)_{t_2},\,\Ss_{-nD_2}(\widetilde{X}_{t_3})=(t_1^n)_{t_3},\,\Ss_{-nD_2}(\widetilde{X}_{t_4})=(1)_{t_4}$$
$$\Ss_{nD_2}(\widetilde{X}_{t_i})=(1,t_4^{-1}t_1,(t_4^{-1}t_1)^2,...,(t_4^{-1}t_1)^n)_{t_i}$$

On voit que l'on a $\cap_i \Ss_{\pm nD2}(\widetilde{X}_{t_i})\subset \Ss_{\pm D2}(\widetilde{X})^{\pm n}$ et on en déduit que l'application $\Ss_{\pm D2}(\widetilde{X})^{\otimes n}\xhookrightarrow{} \Ss_{\pm nD2}(\widetilde{X})$ est un isomorphisme. Ainsi $\Gamma(\bar{X},\Ss)$ est engendré en tant que $\Oo(\bar{X})$-algèbre par les éléments algébriquement libres $(z_1,z_2,z_3,z_4):=(1, t_4^{-1}t_1,t_2,t_4)$. De plus d'après les équations suivantes, la composante homogène de degré zéro est incluse dans $k[z_1,z_2,z_3,z_4]$:
$$ z_2z_4=t_1,\, z_1z_3=t_2,\, z_2z_3=t_3,\, z_1z_4=t_4$$

On a donc $\Gamma(\bar{X}, \Ss)=k[z_1,z_2,z_3,z_4]$, c'est une algèbre de polynômes naturellement graduée par $\deg z_1=\deg z_2=1$, $\deg z_3=\deg z_4=-1$. Cette $\ZZ$-graduation se traduit en une action de $\GG_m$ sur $\AAA^4=\spec \Gamma(\bar{X}, \Ss)$ donnée  concrètement par $\lambda.z=(\lambda z_1,\lambda z_2,\lambda^{-1} z_3,\lambda^{-1} z_4)$. L'algèbre des invariants sous cette action est $k[z_1z_2,z_1z_4,z_2z_3,z_2z_4]$ ce qui donne un isomorphisme $\spec(\Gamma(\bar{X},\Ss)^{\GG_m})\simeq \widetilde{X}$.

On vient ainsi de voir dans cet exemple qu'un faisceau d'algèbres divisorielles $\Ss$ bien choisi permet de retrouver $\bar{X}$ comme bon quotient d'une $H$-variété construite à partir de ce faisceau, où $H$ est le groupe diagonalisable définissant la graduation de $\Ss$. C'est ce qu'on va étudier de manière générale dans cette partie.

\section{Faisceau et anneau de Cox dans le cas sans torsion}

\subsection{Le spectre relatif d'une algèbre divisorielle}


On souhaiterait réaliser géométriquement le faisceau d'algèbre divisorielles $\Ss$ d'une variété normale irréductible $X$ associé à un sous-groupe $K\subset \wdiv(X)$ de type fini, donc libre de rang fini. Une idée naturelle est de prendre le spectre relatif $(\widetilde{X}, p)$ de ce faisceau d'algèbres quasi-cohérent. Toutefois, ce spectre relatif ne définira pas une variété si $\Ss$ n'est pas localement de type fini. Sous certaines conditions, on pourra s'en assurer, par exemple en supposant $X$ lisse.

\begin{rem}\label{lisseGoodQuotient}
Supposons $X$ lisse et irréductible. Tous ses diviseurs sont de Cartier, et en choisissant une $\ZZ$-base $D_1,...,D_s$ de $K$ on a un isomorphisme
\begin{center}
$\widetilde{X}=\spec_X(\bigoplus_{a_1\in\ZZ}\Oo_X(a_1D_1)\otimes_{\Oo_X}...\otimes_{\Oo_X}\bigoplus_{a_s\in\ZZ}\Oo_X(a_sD_s))\simeq L_1^\times\times_X...\times_X L_s^\times$
\end{center}
où, avec les notations de l'exemple d'introduction, $L_i$ est le fibré en droites correspondant à $\Oo_X(D_i)$. 

Tout point de $X$ admet un voisinage ouvert $U$ sur lequel chacun des diviseurs $D_i$ est principal, égal disons à $\divi(f_i)$. Cet ouvert trivialise chaque fibré $L_i$ et on obtient un isomorphisme $p^{-1}(U)\simeq \GG_m^s\times U$ qui provient de l'isomorphisme d'algèbres graduées
\begin{center}
$\Oo_X(U)\otimes_k k[t_1^{\pm},...,t_s^{\pm}] \rightarrow \mathcal{R}(U),\, g\otimes t_1^{\nu_1}...t_s^{\nu_s}\mapsto gf_1^{-\nu_1}...f_s^{-\nu_s}$
\end{center}

En notant $H:=\spec k[K]$, si bien que $H\simeq \GG_m^s$, on peut résumer la situation dans le digramme commutatif ci-dessous auquel on se référera ultérieurement. Les flèches sont $H$-équivariantes et $H$ agit sur le produit par multiplication sur le premier facteur. En particulier, $(\widetilde{X},p)$ est un $H$-fibré principal:
\begin{center}
	\begin{tikzcd}
		p^{-1}(U) \arrow[r,"\simeq"] \arrow[d,"p"] & H\times U \arrow[dl, "pr_U"] \\
		U
	\end{tikzcd}
\end{center}
\end{rem}



\begin{cons}\label{specrelatifdivi}
Soit $X$ une variété normale et irréductible, $K\subset \wdiv(X)$ un sous-groupe de type fini, $\Ss$ le faisceau d'algèbres divisorielles associé. On suppose que $\Ss$ est localement de type fini. Alors le spectre relatif $\widetilde{X}=\spec_X(\Ss)$ muni de son morphisme structural $p$ est naturellement équipé d'une action du tore $H:=\spec k[K]$ pour laquelle $(X,p)$ est un bon quotient.
\end{cons}
\begin{proof}
En effet, $p$ est affine par construction. Puis, pour tout ouvert affine $U\subset X$, on a $p^{-1}(U)=\spec \Ss(U)$, où $\Ss(U)$ est une $\Oo_X(U)$-algèbre $K$-graduée. On a $\Ss(U)_0=\Oo_X(U)$, ce qui donne des quotients locaux par $H$ qui se recollent globalement en $p$.
\end{proof}

\begin{prop}
Avec les données de \ref{specrelatifdivi}, supposons de plus que tout diviseur soit de Cartier (par exemple si X est lisse). Alors $H$ agit librement sur $\widetilde{X}$.
\end{prop}
\begin{proof}
C'est immédiat avec le diagramme de \ref{lisseGoodQuotient}.
\end{proof}


\begin{prop}\label{preimagecodim2}
Avec les données de \ref{specrelatifdivi}, $\widetilde{X}$ est une variété irréductible et normale. De plus, pour tout fermé $A\subset X$ de codimension $\geq 2$, $p^{-1}(A)$ est aussi de codimension $\geq 2$.
\end{prop}
\begin{proof}
Tout d'abord, $\widetilde{X}$ est séparé comme spectre relatif sur une variété. Ensuite, on recouvre l'ouvert des points réguliers $X_{reg}$ par un nombre fini d'ouverts $U_i$ comme dans le diagramme ci-dessus, ce qui est possible car tout ouvert de $X$ est quasi-compact. Ainsi les $p^{-1}(U_i)$ sont irréductibles, et leur réunion $p^{-1}(X_{reg})$ également car leur intersection est non-vide. De plus, $p^{-1}(X_{reg})$ est lisse car c'est vrai localement par le diagramme. On recouvre maintenant $X$ par des ouverts affines $V_1,...,V_s$ et on a d'après \ref{isomorphismcodim2}, $\Ss(V_i\cap X_{reg})=\Ss(V_i)=\Oo_{\widetilde{X}}(p^{-1}(V_i))$. Ces anneaux sont intègralement clos car $p^{-1}(X_{reg})$ est irréductible et lisse. Comme les $p^{-1}(V_i)$ recouvrent $\widetilde{X}$ on en déduit la normalité et l'irréductibilité.
Pour la dernière assertion, c'est une conséquence directe du fait que $p_*\Oo_{\widetilde{X}}=\Ss$ et de \ref{cohomcodimgeq2}.
\end{proof}

Remarquons qu'une section de $s\in \Ss(U)$ homogène de degré $D\in K$ sur un ouvert $U\subset X$ peut être vue à la fois comme une fonction rationnelle sur $X$ vérifiant $\divi(s)+D\geq 0$ et comme une fonction régulière sur $\widetilde{X}$ qui est homogène de degré $D$ pour l'action de $H$. On rappelle que $\divi_D(s):=\divi(s)+D$ est le diviseur des zéros de $s$ sur $X$. C'est un diviseur effectif, on note $\Vv_X(D,s)$ sont support, et $X_{D,s}$ le complémentaire de ce support. On explore maintenant les relations entre ces points de vue. Notons tout d'abord que comme $p$ est un morphisme dominant de variétés irréductibles, on peut définir l'image inverse $p^*(D)$ d'un diviseur de Cartier simplement par l'image inverse de ses équations locales. Pour un diviseur de Weil $D$, on considère sa restriction $D'$ à $X_{reg}$ et on définit $p^*(D)$ comme l'unique diviseur de Weil correspondant à $p^*(D')$ via l'isomorphisme $\wdiv(\widetilde{X})\simeq\wdiv(p^{-1}(X_{reg}))$ (\ref{isomorphismcodim2CorXreg}). Cela définit un morphisme de groupes injectif $\wdiv(X)\xhookrightarrow{}\wdiv(\widetilde{X})$, qui préservent les diviseurs principaux et l'effectivité. On obtient ainsi un morphisme de groupes $\clg(X)\rightarrow \clg(\widetilde{X})$.

\begin{prop}\label{pstarPrincipalCartier}
Les données sont celles de \ref{specrelatifdivi}. Pour tout $D\in K$ de Cartier et $s\in\Ss_D(X)\subset \Oo(\widetilde{X})$, on a $p^*(D)=\divi(s-p^\sharp(s))$, où $s$ est d'abord vue comme une fonction régulière sur $\widetilde{X}$, puis comme un élément de $k(X)$.
\end{prop}
\begin{proof}
On écrit $D=(U_i, f_i)$, on a ainsi $s_i:=s_{|U_i}=\alpha_i f_i^{-1}$ avec $\alpha_i\in \Oo_X(U_i)$, et localement on a $p^*(\divi_D(s)_{|U_i})=p^*(\divi(\alpha_i))=\divi(p^\sharp(\alpha_i))=\divi(\alpha_i)=\divi(\alpha_i f_i^{-1})=\divi(s_{|U_i})$, l'avant dernière égalité étant due au fait que $f_i$ est inversible sur $p^{-1}(U_i)$. En effet, sur $U_i$ on a $D=\divi(f_i)$ donc $f_i\in \Ss_{-D}(U_i)\subset\Oo_{\widetilde{X}}(p^{-1}(U_i))$ et $f_i^{-1}\in \Ss_D(U_i)\subset\Oo_{\widetilde{X}}(p^{-1}(U_i))$.
\end{proof}

\begin{prop}\label{pstarprincipal}
Les données sont celles de \ref{specrelatifdivi}. Pour tout $D\in K$ et $s\in\Ss_D(X)\subset \Oo(\widetilde{X})$, on a $\divi(s)=p^*(\divi_D(s))$. Si de plus $X_{D,s}$ est affine, on a $\Vv_{\widetilde{X}}(s)=p^{-1}(\Vv_X(D,s))$.
\end{prop}
\begin{proof}
On peut supposer pour la première assertion que $X$ est lisse en utilisant \ref{preimagecodim2} et \ref{isomorphismcodim2CorXreg}. Ainsi $\widetilde{X}$ est également lisse d'après \ref{lisseGoodQuotient}. Le résultat découle alors de la proposition précédente.

Pour la deuxième assertion, il faut montrer $p^{-1}(X_{D,s})=\widetilde{X}_s$. On remarque que $s^{-1}\in \Ss_{-D}(X_{D,s})$ donc $s$ est inversible sur $p^{-1}(X_{D,s})$, ce qui montre $p^{-1}(X_{D,s})\subset \widetilde{X}_s$. Par ailleurs, $\divi_D(s)$ est de Cartier sur $X_{reg}$ et son image inverse est l'image inverse de ses équations locales, on obtient donc $p^{-1}(X_{D,s})\cap p^{-1}(X_{reg})=\widetilde{X}_s \cap p^{-1}(X_{reg})$, d'après la première assertion. Ainsi $\widetilde{X}_s \setminus p^{-1}(X_{D,s})$ est le complémentaire d'un ouvert affine, et il est de codimension $\geq 2$, donc est vide d'après \ref{codimaffinenormal}.
\end{proof}

\begin{prop}\label{CoxIsomorphismLocalisation}
Les données sont celles de \ref{specrelatifdivi}. Soit $D\in K$ et $f\in\Gamma(X,\Ss_D)$ non-nul. Alors on a un isomorphisme canonique d'algèbres $K$-graduées
\begin{center}
$\Gamma(X_{D,f},\Ss)\simeq \Gamma(X,\Ss)_f$
\end{center}
\end{prop}
\begin{proof}
On peut supposer que $X$ est lisse, d'après \ref{isomorphismcodim2CorXreg}. Dans ce cas on a $\Vv_{\widetilde{X}}(f)=p^{-1}(\Vv_X(D,f))$, et on conclut en utilisant \ref{FQCProp2} et le fait que $p_*\Oo_{\widetilde{X}}=\Ss$.
\end{proof}

\begin{cor}\label{zerosspecrelatif}
Les données sont celles de \ref{specrelatifdivi}. Soit $\tilde{x}\in\widetilde{X}$ tel que $H.\tilde{x}$ est fermé dans $\widetilde{X}$. Pour tout $D\in K, f\in \Ss_D(X)$ non-nulle, on a:
$$f(\tilde{x})=0 \iff p(\tilde{x})\in \supp(\divi_D(f))$$
\end{cor}
\begin{proof}
Montrons dans un premier temps que l'on a $\supp(p^*(D))\subset p^{-1}(\supp(D))$. Soit $x\in \supp (p^*(D))$, comme $\supp(p^*(D))=\overline{\supp_{p^{-1}(X_{reg})}(p^*(D_{|X_{reg}}))}$, on a $x\in\Vv_{\widetilde{X}}(p^\sharp(f_i))$ pour un $i\in I$, où l'on a écrit $D_{|X_{reg}}=(U_i, f_i)_{i\in I}$. On en déduit que $p(x)\in \Vv_X(f_i)$, c'est à dire $p(x)\in \overline{\supp_{X_{reg}}(D_{|X_{reg}})}=\supp(D)$. Maintenant, comme $p$ est surjectif, on trouve $p(\supp(p^*(D)))\subset \supp(D)$. De plus, $p(\supp(p^*(D)))$ et $\supp(D)$ coïncident sur l'ouvert dense $X_{reg}$ et $p(\supp(p^*(D)))$ est fermé d'après \ref{goodquotientthm}. On a donc l'égalité:$$p(\supp(p^*(D)))=\supp(D)$$
Ainsi en appliquant \ref{pstarprincipal}, on a $p(\supp(\divi(f)))= \supp(\divi_D(f))$ et donc $f(\tilde{x})=0 \implies p(\tilde{x})\in\supp \divi_D(f)$. Réciproquement, si $p(\tilde{x})\in\supp \divi_D(f)$, on a $p(\tilde{x})=p(\tilde{x}')$ pour un $\tilde{x}'\in \supp(\divi(f))$. Toujours d'après \ref{goodquotientthm} et en utilisant que $H.\tilde{x}$ est fermé on obtient $H.\tilde{x}\subset\overline{H.\tilde{x}'}$ ce qui prouve $f(\tilde{x})=0$ car $f$ est nulle sur $H.\tilde{x}'$, étant homogène et s'annulant en $\tilde{x}'$.
\end{proof}

Toujours avec les données de \ref{specrelatifdivi}, on établit maintenant un résultat important sur le groupe des classes de $\widetilde{X}$. On a d'abord besoin d'un lemme préliminaire assurant l'existence de $H$-linéarisations sur les fibrés en droites sur $\widetilde{X}$.

\begin{lem}\label{existenceGLinTore}
Soit $X$ une variété irréductible munie d'une action d'un tore $H$, et $(L,\pi)$ un fibré en droites sur $X$. Alors il existe une $H$-linéarisation de cette action sur $L$.
\end{lem}
\begin{proof}
On note $p_2:H\times X\rightarrow X$ la projection. En utilisant \cite[II.6.6]{Hartshorne}, on obtient un isomorphisme $p_2^*:\pic(X)\rightarrow \pic(H\times X)$, d'où avec les notations de \ref{existenceGLin}, $\alpha^*(L)\simeq p_2^*(M)$ pour un certain fibré en droites $M$ sur $X$. En prenant l'image inverse par $e\times \id$ on obtient un isomorphisme de fibré en droites $L\simeq M$. On a ainsi un isomorphisme de fibrés en droites $\alpha^*(L)\simeq p_2^*(L)$, ce qui conclut la preuve d'après \ref{existenceGLin}.
\end{proof}

\begin{thm}\label{clgtrivial}
Les données sont celles de \ref{specrelatifdivi}, en supposant de plus que $X$ est lisse. Les assertions suivantes sont équivalentes:
\begin{enumerate}
\item La projection $c:K\rightarrow \clg(X)$ est surjective.
\item Le groupe des classes $\clg(\widetilde{X})$ est trivial.
\end{enumerate}
\end{thm}
\begin{proof}
Supposons la projection $c$ surjective. Soit $\widetilde{D}\in \wdiv(\widetilde{X})$ un diviseur. Il est de Cartier par hypothèse et on veut montrer qu'il est principal. On a une action du tore $H=\spec k[K]$ sur $\widetilde{X}$ par la $K$-graduation de $\Oo_{\widetilde{X}}$. Soit $L$ le fibré en droites sur $\widetilde{X}$ associé au faisceau inversible $\Ss_{\widetilde{D}}$. En utilisant \ref{existenceGLinTore}, on munit $L$ d'une H-linéarisation et on obtient un $H$-module $\Ss_{\widetilde{D}}(\widetilde{X})$ d'après \ref{GmoduleSectionsGlobalesGlin}. On en déduit que pour tout $h\in H$ et $f\in \Ss_{\widetilde{D}}(\widetilde{X})$, on a $\divi_{\widetilde{D}}(h.f)=h.\divi_{\widetilde{D}}(f)$. D'après \ref{GroupeDiagCarac}, on peut supposer que $f$ est un vecteur propre pour l'action de $H$, et on a ainsi construit un diviseur $\divi_{\widetilde{D}}(f)$ fixe pour l'action de $H$ et linéairement équivalent à $\widetilde{D}$. On peut donc supposer que $\widetilde{D}$ est $H$-invariant. En utilisant le diagramme de \ref{lisseGoodQuotient} qui s'applique localement et en remarquant que $H$ agit transitivement sur lui même on conclut que via cet isomorphisme, $\widetilde{D}$ est de la forme $H\times D$, pour un diviseur $D\in \wdiv(X)$ et que $\widetilde{D}=p^*(D)$. Or, par hypothèse $D$ est linéairement équivalent à un $D'\in K$, on a ainsi le résultat car $p^*(D')$ est principal d'après la proposition précédente.

Supposons maintenant $\clg(\widetilde{X})$ trivial. Soit $D\in\wdiv(X)$, on veut montrer que $D$ est linéairement équivalent à un diviseur $D'\in K$, et on remarque que l'on peut supposer $D$ effectif. L'image inverse $p^*(D)$ est principal par hypothèse et effectif. Donc $p^*(D)=\divi(f)$ pour un élément $f\in \Oo(\widetilde{X})$, et on affirme que $f$ est homogène pour la $K$-graduation. On introduit pour montrer cela une fonction régulière inversible sur $H\times \widetilde{X}$:
$$F:H\times \widetilde{X}\rightarrow k,\,(h,x)\mapsto f(h.x)/f(x)$$
En effet, $p^*(D)$ est $H$-invariant en tant qu'image inverse par l'application quotient $p$. On en déduit que pour tout $h\in H$, on a $\divi(x\mapsto f(h.x)/f(x))=0$, ce qui prouve que $F\in \Oo(H\times \widetilde{X})^*$ d'après \ref{caracfaisceaustructdiv}. Comme $H$ est connexe en tant que tore, on peut appliquer le résultat \ref{rosenlicht1}. Ainsi il existe $\chi \in X^*(H)$ et $g\in\Oo(\widetilde{X})^*$ tel que $F(h,x)=\chi(h)g(x)$ pour tout $h\in H$ et $x\in \widetilde{X}$. Fixons $h=e$, on obtient $g=1$, d'où $f(h.x)=\chi(h)f(x)$. Ainsi, $f\in \Gamma(X, \Ss_{D'})$ pour un diviseur $D'\in K$. Avec \ref{pstarprincipal}, on tire l'égalité $p^*(D)=\divi(f)=p^*(\divi_{D'}(f))$, et on conclut que $D=D'+\divi(f)$, ce qu'on voulait montrer.

\end{proof}

\begin{rem}
On remarque que si on ne suppose plus que $X$ est lisse, l'implication $1\implies 2$ reste vraie dans le théorème précédent. En effet, comme $p^{-1}(X_{sing})$ est de codimension $\geq 2$ on peut alors supposer $X$ lisse, car on a un isomorphisme $\clg(\widetilde{X})\simeq\clg(\widetilde{X}\setminus p^{-1}(X_{sing}))$. 
\end{rem}


\begin{cor}\label{specrelatifQuasiAff}
Les données sont celles de \ref{specrelatifdivi}. Alors $\widetilde{X}$ est quasi-affine.
\end{cor}
\begin{proof}
On recouvre $X$ par des ouverts affine $X_1,...,X_r$. D'après \ref{codimaffinenormal} chaque $X\setminus X_i$ est purement de codimension $1$, et donc le support d'un diviseur effectif. D'après \ref{SectionDivCartierPicardGroup}, on a $X\setminus X_i=\supp(\divi_{D_i}(f_i))$ pour un couple $(\Oo_X(D_i),f_i)$, où $f_i\in\Gamma(X,\Oo_X(D_i))$. Ainsi, d'après \ref{pstarprincipal}, $\widetilde{X}$ est recouvert par des $\widetilde{X}_{f_i}$ qui sont affines car les $X_i=X_{D_i,f_i}$ le sont. Maintenant, notons que la propriété de finitude locale du faisceau structural de $\widetilde{X}$ reste vrai sur tout recouvrement affine de $\widetilde{X}$, on a donc en particulier pour tout $i$, $\Oo(\widetilde{X}_{f_i})=k[(g_{ij})_{1\leq j\leq n}]$. Comme $f_i$ est inversible sur chaque $\widetilde{X}_{f_i}$, on ne change rien en multipliant les $g_{ij}$ par une puissance $f_i^m$, ce qui permet de supposer que les $g_{ij}$ proviennent de sections globales en prenant un $m$ suffisamment grand, car pour tout $i$ on a $\Oo({\widetilde{X}_{f_i}})=\Oo({\widetilde{X}})_{f_i}$ d'après \ref{FQCProp2}. On a ainsi construit une sous-algèbre de type fini $R:=k[(g_{ij})_{1\leq i\leq r, 1\leq j\leq n}]$ de $\Oo(\widetilde{X})$ telle pour tout $i$, on a $R_{f_i}=\Oo({\widetilde{X}_{f_i}})=\Oo({\widetilde{X}})_{f_i}$. On en déduit des immersions ouvertes $\widetilde{X}_{f_i}\xhookrightarrow{} \spec(R)$ qui se recollent en une immersion ouverte $\widetilde{X}\xhookrightarrow{} \spec(R)$ d'où le résultat.
\end{proof}

\begin{cor}
Les données sont celles de \ref{specrelatifdivi}. Soient $x\in X$, $K_x^{prin}\subset K$ le sous-groupe des diviseurs localement principaux en $x$, et $\tilde{x}\in p^{-1}(x)$ tel que $H.\tilde{x}$ est fermé. Alors le stabilisateur $H_{\tilde{x}}$ est égal à $\spec k[K/K_x^{prin}]$.
\end{cor}
\begin{proof}
Comme $H$ agit localement sur $\widetilde{X}$ on peut se placer sur un voisinage affine de $x$, on suppose ainsi $X$ et $\widetilde{X}$ affines. Ainsi, en appliquant \ref{staborbitegroup}, on a $H_{\tilde{x}}=\spec k[K/K_{\tilde{x}}]$, où $K_{\tilde{x}}$ est le groupe d'orbite de $\tilde{x}$. Supposons que $D\in K$ soit principal sur ce voisinage de $x$, c'est à dire $D=\divi(g)$ avec $g\in k(X)$. Choisissons $\alpha \in \Oo(X)$ tel que $\alpha(x)\neq 0$. Alors $f:=\alpha g^{-1}\in \Ss_D(X)$ et $\divi_D(f)=\divi(\alpha)$. En utilisant \ref{zerosspecrelatif}, on obtient $f(\tilde{x})\neq 0$ et donc $D\in K_{\tilde{x}}$. Réciproquement, prenons $D\in K_{\tilde{x}}$, c'est à dire $f(\tilde{x})\neq 0$ pour un $f\in \Ss_D(X)$. Alors $x\notin \divi_D(f)$, et on a $D=-\divi(f)$ au voisinage de $x$. Cela montre que $D$ est localement principal.
\end{proof}




\subsection{Faisceau et anneau de Cox}

Dans cette partie on définit l'anneau de Cox d'une variété $X$ normale irréductible avec un groupe des classes libre de type fini. C'est en particulier le cas de l'exemple introductif

\begin{cons}[Faisceau de Cox, Anneau de Cox]\label{consFreeCoxRing}
Soit $X$ une variété normale irréductible avec un groupe des classes libre de type fini. Soit $K\subset \wdiv(X)$ un sous-groupe se projetant isomorphiquement sur $\clg(X)$ (il existe de tels $K$ car $\clg(X)$ est libre de type fini donc la projection admet des sections). On définit le faisceau de Cox sur $X$, noté $\Rr$ comme le faisceau d'algèbres divisorielles associé à $K$. Cette description ne dépend qu'à isomorphisme près du choix de $K$. L'anneau de Cox de $X$ est l'algèbre des sections globales du faisceau de Cox. 
\end{cons}
\begin{proof}
Soient $K,K'$ deux sous-groupes de $\wdiv(X)$ se projetant isomorphiquement sur $\clg(X)$, et $\Rr,\Rr'$ les faisceaux de Cox correspondants. On choisit une base $(D_1,...,D_s)$ de $K$. Cette base définit une section de la projection $c:\wdiv(X)\rightarrow \clg(X)$ et on peut modifier cette section par des éléments du noyau, ce qui fournit autant de bases de sous-groupes de $\wdiv(X)$ se projetant isomorphiquement sur $\clg(X)$. Par ailleurs, chaque $D_i+\Ker c$ rencontre nécessairement $K'$ sinon on aurait $\rg K' < \rg \clg(X)$ comme rang de modules libres de type fini. On choisit ainsi $f_1,...,f_s\in k(X)$ tels que $(D_i-\divi(f_i))_i$ forme une base de $K'$, et on définit un morphisme $\alpha:K\rightarrow k(X)^*, a_1D_1+...+a_sD_s\mapsto f_1^{a_1}...f_s^{a_s}$. Avec cela, l'isomorphisme linéaire faisant correspondre les bases de $K$ et $K'$ est $\widetilde{\psi}:K\rightarrow K',\, D\mapsto D-\divi(\alpha(D))$. Enfin, on définit un isomorphisme d'algèbres divisorielles $(\psi, \widetilde{\psi}): \Rr\rightarrow\Rr'$ en posant $f\in \Gamma(U, \Rr_D)\mapsto \alpha(D)f$.
\end{proof}

\begin{ex}
Soit $X=\proj B/I$ une variété projective normale et irréductible, où $B=k[x_0,...,x_n]$ et $I\subset B$ un idéal premier homogène et radical. On suppose que $X$ est projectivement normale, c'est à dire que $B/I$ est intégralement clos, et que $\clg(X)$ est libre de rang $1$ engendré par l'intersection de $X$ avec un hyperplan de $\PP^n_k$. Alors $\Rr(X)=B/I=\Oo(\bar{X})$ où $\bar{X}$ est le cône affine dans $\AAA^{n+1}$ correspondant à $X$. Voir l'exemple \ref{ExIntroCox}.
\end{ex}

\subsection{Propriétés algébriques de l'anneau de Cox}

\begin{thm}\label{coxFreeFactoriel}
Soit $X$ une variété normale irréductible dont le groupe des classes est libre de type fini. Alors:
\begin{enumerate}
\item L'anneau de Cox $\Rr(X)$ est factoriel.
\item Le groupe des unités de l'anneau de Cox est $\Rr(X)^*=\Oo(X)^*$
\end{enumerate}
\end{thm}
\begin{proof}
\begin{enumerate}
\item On peut supposer $X$ lisse d'après \ref{isomorphismcodim2CorXreg}. Soit $f\in \Rr(X)=p_*\Oo_{\widetilde{X}}(X)=\Oo_{\widetilde{X}}(\widetilde{X})$. En vertu du théorème \ref{clgtrivial} le diviseur de $f$ s'écrit $\divi(f)= \sum n_iD_i=\sum n_i\divi(f_i)=\divi(\prod f_i^{n_i})$ où les $D_i$ sont des diviseurs premiers, chaque $f_i$ est irréductible et appartient à $\Rr(X)$ d'après \ref{caracfaisceaustructdiv}. L'unicité de l'écriture sur $\wdiv(X)$ donne l'unicité de l'écriture $f=u\prod f_i^{n_i}$ où $u\in \Rr(X)^*$.
\item L'inclusion $\Oo(X)^*\subset\Rr(X)^*$ est évidente par définition. Pour l'autre prenons $f\in\Rr(X)^*$, qui est homogène d'après \ref{GradedProp1}, disons de degré $D$. Alors, $fg=1\in \Rr_0(X)$ pour un $g\in\Rr(X)^*$ homogène de degré $-D$. Ainsi on a $0=\divi_0(1)=\divi_{D-D}(fg)=\divi_D(f)+\divi_{-D}(g)$. Les deux derniers diviseurs étant effectifs, on a $\divi_D(f)=0$ et donc $D=-\divi(f)$, ce qui donne $f\in\Rr_0(X)=\Oo(X)$ d'où le résultat.
\end{enumerate}
\end{proof}

\begin{ex}
Reprenons le cas du cône affine de l'exemple introductif.  L'anneau de Cox $\Rr(\bar{X})$ est l'algèbre de polynômes sur $k$ à $4$ indéterminés, donc est factoriel en particulier.  Par ailleurs, $\Oo(\bar{X})^{\times}=k^*$ car en inversant par exemple $t_1$  on obtient $\bar{X}_{t_1}\simeq \AAA^{1}\setminus \lbrace0\rbrace\times\AAA^{2}$, et les fonctions régulières inversibles sur cet ouvert sont nécessairement constantes.
\end{ex}

\begin{prop}\label{divisibilitePropsFreeCox}
Soit $X$ une variété normale irréductible dont le groupe des classes est libre de type fini.  Alors:
\begin{enumerate}
\item Soient $f\in \Rr_D(X)$ et $g\in \Rr_E(X)$ non nulles. Alors $f\mid g\iff \divi_D(f)\leq\divi_E(g)$.
\item Soit $f\in \Rr_D(X)$ non nulle. Alors $f$ est premier si et seulement si $\divi_D(f)$ est premier.
\end{enumerate}
\end{prop}
\begin{proof}
Soit $f,g$ des fonctions régulières non-nulles sur $\widetilde{X}$. Comme $\widetilde{X}$ est intègre, on peut les voir comme des éléments de $k(X)^*$. Ainsi $f\mid g$ dans $\Rr(X)$ si et seulement si $\divi(f^{-1}g)\geq 0$. Prenant $f,g$ comme dans l'énoncé et en utilisant \ref{pstarprincipal}, on voit que c'est équivalent à $\divi_D(f)\leq\divi_E(g)$. La preuve du deuxième énoncé est similaire.
\end{proof}

\section{Faisceau et anneau de Cox dans le cas général}

\subsection{Faisceau et anneau de Cox}

On souhaite maintenant définir l'anneau de Cox d'une variété $X$ dont le groupe des classes admet éventuellement de la torsion. On suppose comme toujours $X$ normale irréductible, et $\clg(X)$ de type fini. L'idée est de construire le faisceau d'algèbres divisorielles $\Ss$ associé à un sous-groupe $K\subset \wdiv(X)$ se projetant surjectivement sur $\clg(X)$, puis d'identifier des composantes homogènes $\Ss_{D_1}$ et  $\Ss_{D_2}$ si $D_1=D_2+E$, où $E\in \Ker (c:K\rightarrow \clg(X))$. Pour cela on remarque que l'on a, d'après \ref{isomophismeOTimes}, un isomorphisme $\Ss_{D_1}\simeq \Ss_{D_2}\otimes_{\Oo_X}\Ss_E$. Posons $E=\divi(g)$ avec $g\in k(X)$, et considérons $f_2\in \Ss_{D_2}(U)$ une section sur un ouvert $U\subset X$. Par l'isomorphisme, on voit que $f_2$ est de la forme $g_{|U}f_1$, où $f_1\in \Ss_{D_1}(U)$. Pour identifier $f_1$ et $f_2$, il faut imposer $g_{|U}=1_{|U}$, où $1\in\Ss_0(X)$, c'est ce que l'on fait dans la construction ci-dessous.

\begin{cons}[Faisceau de Cox, Anneau de Cox]\label{conscoxtorsion}
Soit $X$ une variété normale irréductible, telle que $\Oo(X)^*=k^*$ et $\clg(X)$ est de type fini. On se fixe un sous-groupe $K\subset \wdiv(X)$ tel que la projection $c:K\rightarrow \clg(X)$ est surjective. Soit $K^0:=\Ker c$ et $\chi:K^0\rightarrow k(X)^*$ un caractère, c'est à dire un morphisme de groupes tel que $\divi(\chi(E))=E$, pour tout $E\in K^0$. Pour construire un tel caractère, on peut considérer une base $(D_1,...,D_s)$ de $K^0$ est des éléments $f_i\in k(X)$ tels que $D_i=\divi(f_i)$. On définit le caractère en posant $\chi(D_i)=f_i$, il vérifie bien la condition voulue.

Soit $\Ss$ le faisceau d'algèbres divisorielles sur $X$ associé à $K$. Soit $\Ii$ le faisceau d'idéaux de $\Ss$ défini par l'image du morphisme 
$$\bigoplus_{E\in K^0}\Ss\rightarrow \Ss,\, E\mapsto 1-\chi(E),\,\,\, \text{ où }1\in \Ss_0(X)$$
Le faisceau de Cox associé à $K$ et $\chi$ est le faisceau quotient $\Rr:=\Ss/\Ii$. C'est une $\Oo_X$-algèbre quasi-cohérente et $\clg(X)$-graduée de la manière suivante: 
\begin{center}
$\Rr=\bigoplus_{[D]\in \clg(X)}\Rr_{[D]},\,\,\,\,\,\, \Rr_{[D]}:=\pi(\bigoplus_{D'\in c^{-1}([D])}\Ss_{D'}),\,\,\,\,\,$ où $\pi:\Ss\rightarrow \Rr$ est la projection
\end{center}

\noindent L'anneau de Cox associé à $K$ et $\chi$ est l'anneau des sections globales de $\Rr$.
\end{cons}

La $\clg(X)$-graduation de $\Rr$ annoncée ci-dessus n'est pas évidente à priori. On clarifie cela dans la proposition ci-dessous.

\begin{prop}\label{coxidealform}
Avec les notation de la construction \ref{conscoxtorsion}, $\Ss$ est naturellement muni d'une $\clg(X)$-graduation:
$$\Ss=\bigoplus_{[D]\in\clg(X)}\Ss_{[D]},\,\,\,\,\,\,\, \Ss_{[D]}:=\bigoplus_{D'\in c^{-1}([D])}\Ss_{D'}$$
Soit $U\subset X$ un ouvert, $f\in\Gamma(U,\Ii)$ et $D\in K$. La composante $\clg(X)$-homogène $f_{[D]}\in \Gamma(U,\Ss_{[D]})$ s'écrit de manière unique sous la forme d'une somme finie:
$$f_{[D]}=\sum_{E\in K^0}(1-\chi(E))f_E, \text{  où } f_E\in\Gamma(U,\Ss_D), \text{ et }\chi(E)\in \Gamma(U, \Ss_{-E})$$
En particulier, $\Ii$ est un faisceau d'idéaux $\clg(X)$-homogènes, et $\pi$ est un morphisme de $\Oo_X$-algèbres $\clg(X)$-graduées. De plus, si $f\in \Gamma(U,\Ii)$ est $K$-homogène, alors c'est la section nulle.
\end{prop}
\begin{proof}
Supposons que l'on obtienne une telle écriture. Alors les composantes $K$-homogènes de $f_{[D]}$ sont facilement identifiables, et définissent les éléments $f_E\in\Gamma(U,\Ss_D)$ pour $E\neq 0$. Une telle écriture est donc unique.

Montrons l'existence. Par définition de $\Ii$, chaque germe $f_{[D],x}$ admet une représentation sur un voisinage ouvert $U_x$ de $x$ par une section 
$$g=\sum_{E\in K^0}(1-\chi(E))g_E \in \Gamma(U_x, \Ii)\text{, où }g_E\in \Gamma(U_x, \Ss_{[D]})$$
On écrit la décomposition en composantes $K$-homogènes de chaque section $g_E$:
$$g_E=\sum_{D'\in D+K^0}g_{E,D'}, \text{ où } g_{E,D'}\in\Gamma(U_x, \Ss_{D'})$$
La section $g'_{E,D'}:=\chi(D'-D)g_{E,D'}$ est $K$-homogène de degré $D$ et on a l'identité 
$$(1-\chi(E))g_{E,D'}=(1-\chi(E+D-D'))g'_{E,D'}-(1-\chi(D-D'))g'_{E,D'}$$
On obtient ainsi l'écriture désirée localement sur des ouverts qui recouvrent $X$. Par irréductibilité de $X$, on obtient l'unicité de l'écriture globalement en recollant ces sections.
\end{proof}

\begin{cor}\label{genreatorsCoxIdeal}
Supposons $K$ de type fini. Soit $E_1,...,E_s$ une base de $K^0$. Alors pour tout ouvert $U\subset X$, l'idéal $\Gamma(U,\Ii)$ est engendré par $1-\chi(E_i)$, $1\leq i\leq s$.
\end{cor}
\begin{proof}
C'est une conséquence de la proposition précédente et des identités suivantes:
$$1-\chi(E+E')=(1-\chi(E))+(1-\chi(E'))\chi(E)$$
$$1-\chi(-E)=(1-\chi(E))(-\chi(-E))$$
\end{proof}

\subsection{Invariance de l'anneau de Cox}

Notre objectif est maintenant de vérifier que l'anneau de Cox construit en \ref{conscoxtorsion} est indépendant des choix faits à isomorphisme près. On aura donc une construction intrinsèque car ne dépendant à isomorphisme près que de données intrinsèques de la variété (faisceau structural, groupe des classes).

\begin{lem}\label{LemmeCoxRelevelement}
Avec les données de \ref{conscoxtorsion}. Soit $U\subset X$ un ouvert, $f\in \Ss(U)$ une section $\clg(X)$-homogène de degré $[D]$, où $D\in K$. Il existe $f'\in \Ss_D(U)$ tel que $f-f'\in\Ii(U)$.
\end{lem}
\begin{proof}
Cela se voit grâce à l'astuce d'écriture ci-dessous. En effet, le premier terme dans le membre de droite est homogène de degré $D$, le second appartient à $\Ii(U)$.
$$f=\sum_{D'\in D+K^0}f_{D'}=\sum_{D'\in D+K^0}\chi(D'-D)f_{D'}+ \sum_{D'\in D+K^0}(1-\chi(D'-D))f_{D'}$$
\end{proof}


\begin{prop}\label{coxsheafiso2}
Avec les données de \ref{conscoxtorsion}, on a pour tout $D\in K$ un isomorphisme de faisceaux $\pi_{|\Ss_D}:\Ss_D\rightarrow \Rr_{[D]}$.
\end{prop}
\begin{proof}
Considérons pour $x\in X$ le morphisme induit entre les tiges $\pi_x:\Ss_{D,x}\rightarrow \Rr_{[D],x}$, et montrons que c'est un isomorphisme. D'après \ref{coxidealform}, seul le germe nul de $\Ss_{D,x}$ se projette sur $0$, on a donc l'injectivité. Pour la surjectivité, il suffit de montrer que toute section $f$ de $\Ss_{[D]}$ sur un voisinage $U$ de $x$ s'écrit, éventuellement sur un voisinage $V$ de $x$ plus petit, sous la forme $f'+s$ avec $f'\in \Ss_D(V)$, et $s\in \Ii(V)$. Il suffit d'appliquer le lemme précédent.
\end{proof}

\begin{prop}\label{coxsheafiso}
Avec les données de \ref{conscoxtorsion}, on a pour tout ouvert $U\subset X$ un isomorphisme canonique $\Rr(U)\simeq \Ss(U)/\Ii(U)$.
\end{prop}
\begin{proof}
L'application canonique $\psi_U:\Gamma(U,\Ss)/\Gamma(U,\Ii)\rightarrow\Gamma(U,\Ss/\Ii)$ est $\clg(X)$-homogène et injective, montrons qu'elle est surjective. Soit $h\in\Gamma(U,\Ss/\Ii)$, c'est donc un recollement de sections locales $h_i:=\psi_{U_i}(g_i)$, avec $U=\cup_{i\in I}U_i$, et telles que la restriction de $g_i-g_j$ à $U_i\cap U_j$ appartient à $\Ii(U_i\cap U_j)$ pour tout couple $(i,j)\in I$. Soit $g_{i,[D]}\in \Ss_{[D]}(U_i)$ la composante $\clg(X)$-homogène de degré $[D]$ de $g_i$. Comme $\Ii$ est $\clg(X)$-gradué d'après \ref{coxidealform}, on a $g_{j,[D]}-g_{i,[D]}\in\Ii(U_i\cap U_j)$ pour tout couple $(i,j)\in I$. De plus, d'après le lemme \ref{LemmeCoxRelevelement}, il existe des éléments $f_{i,D}\in \Ss_D(U_i)$ tels que $g_{i,[D]}-f_{i,D}\in \Ii(U_i)$, et on a $f_{i,D}-f_{j,D}\in\Ii(U_i\cap U_j)$. Ainsi $f_{i,D}=f_{j,D}$ sur $U_i\cap U_j$ d'après \ref{coxidealform}, et on obtient que les $f_{i,D}$ se recollent en une section $f_D\in \Ss_D(U)$ telle que $\psi(f_D)=h_{[D]}$. Finalement, $f=\sum f_D$ est un antécédent de $h$ par $\psi$.
\end{proof}

\begin{thm}\label{InvarianceCoxTorsion}
Avec les données de \ref{conscoxtorsion}. Soit $K', \chi'$ un autre choix de sous-groupe et de caractères. Alors les faisceaux de Cox associés sont isomorphes en tant que faisceaux d'algèbres $\clg(X)$-graduées.
\end{thm}
\begin{proof}
Dans un premier temps, montrons que l'on peut se ramener à des faisceaux de Cox définis par un sous-groupe de $\wdiv(X)$ de type fini. Soit donc $K_1\subset K$ de type fini se projetant surjectivement sur $\clg(X)$. Un tel $K_1$ existe car il suffit par exemple de prendre le sous-groupe de $K$ engendré par des antécédents de générateurs de $\clg(X)$ par $c$. La restriction de $\chi$ à $K_1$ définit un caractère $\chi_1:K_1^0\rightarrow k(X)^*$. L'inclusion $K_1\subset K$ définit une injection $\Ss_1\rightarrow \Ss$, qui envoie l'idéal $\Ii_1$ défini par $\chi_1$ dans $\Ii$. Cela donne une injection $\clg(X)$-graduée $\Rr_1\rightarrow \Rr$, où $\Rr_1$ est le faisceau de Cox associé à $K_1$ et $\chi_1$. Mais d'après \ref{LemmeCoxRelevelement}, toute section $\clg(X)$-homogène de $\Rr$ peut se représenter par un section $K_1$-homogène, cela donne la surjectivité de ce morphisme.

On suppose donc $K$ de type fini. Montrons que tout choix de caractères $\chi,\chi':K^0\rightarrow k(X)^*$ définit un isomorphisme entre les faisceaux de cox $\Rr$ et $\Rr'$ associés. Notons que $\chi^{-1}\chi'$ envoie $K^0$ dans $\Oo(X)^*=k^*$. On peut ainsi étendre $\chi^{-1}\chi'$ en un caractère $\theta:K\rightarrow k^*$. En effet considérons une $\ZZ$-base $(D_1,....,D_r)$ de $K$ adaptée à $K^0$. Il existe alors des entiers positifs $n_1,...n_s$ avec $s\leq k$ tels que $§n_1D_1,...,n_sD_s$ soit une $\ZZ$-base de $K^0$. Soit $\alpha_i := \chi^{-1}\chi'(n_iD_i)$ pour $1\leq i\leq s$. Choisissons une racine $n_i$-ième de $\alpha_i$ que l'on note $\beta_i$, pour $1\leq i\leq s$. En posant $\theta(D_i)=\beta_i$ pour $1\leq i\leq s$, et $\theta(D_i)=1$ pour $s+1\leq i\leq r$ on définit un prolongement de $\chi^{-1}\chi'$. Ce prolongement permet de définir un automorphisme $K$-gradué $(\alpha,\id)$ de $\Ss$ en posant:
$$\alpha_D:\Ss_D\rightarrow \Ss_D,\,\,\, f\mapsto \theta(D)f$$
Cet automorphisme envoie $\Ii'$ sur $\Ii$ par construction et induit donc un isomorphisme entre $\Ss/\Ii'$ et $\Ss/\Ii$.

Pour finir considérons deux sous-groupes de type fini $K,K'\subset \wdiv(X)$ se projetant chacun surjectivement sur $\clg(X)$. Soit $(D_1,...,D_r)$ une $\ZZ$-base de $K$. On choisit $E_i\in c^{-1}([D_i])\cap K'=(D_i+\Ker c)\cap K'$ pour $1\leq i\leq r$ et on définit un morphisme $\tilde{\alpha}:K\rightarrow K'$ en posant $\alpha(D_i)=E_i$. Comme $E_i$ est de la forme $D_i-\divi(f_i)$ avec $f_i\in k(X)^*$, on voit que $\tilde{\alpha}$ est de la forme $\tilde{\alpha}(D)=D-\divi(\eta(D))$, où $\eta:K\rightarrow k(X)^*$ est un morphisme de groupes. Choisissons un caractère $\chi':K'^0\rightarrow k(X)^*$ associé à $K'$. Alors, pour $D\in K^0$, on a
$$D-\divi(\eta(D))=\tilde{\alpha}(D)=\divi(\chi'(\tilde{\alpha}(D)))$$
On obtient que $D$ est le diviseur de la fonction $\chi(D):=\chi'(\tilde{\alpha}(D))\eta(D)$, et cette formule définit un caractère $\chi:K^0\rightarrow k(X)^*$. On obtient ainsi un morphisme $K$-gradué $(\alpha,\tilde{\alpha}):\Ss\rightarrow \Ss'$ entre les faisceaux associés à $K$ et $K'$ défini par
$$\alpha_D:\Ss_D\rightarrow \Ss'_{\tilde{\alpha}(D)},\,\,\, f\mapsto\eta(D)f$$
Ce morphisme envoie l'idéal $\Ii$ défini par $\chi$ dans l'idéal $\Ii'$ défini par $\chi'$, et induit un un morphisme injectif $\Rr\rightarrow \Rr'$. Comme les $\alpha_D$ sont des isomorphismes de $\Oo_X$-modules on voit que ce morphisme est également surjectif en utilisant \ref{coxsheafiso2}.
\end{proof}

\begin{rem}
Avec les données de \ref{conscoxtorsion}. Si on se donne un point régulier $x\in X$, on peut définir de manière canonique le faisceau de Cox de la variété pointée $(X, x)$. Soit $K^x:=\lbrace D\in\wdiv(X)\mid x\notin \supp D\rbrace$.  Ce sous-groupe de $\wdiv(X)$ se projette surjectivement sur $\clg(X)$. En effet pour tout $[D]\in \clg(X)$, choisissons $D\in c^{-1}([D])$. Si $D\notin K^x$, alors comme $\Oo_{X,x}$ est factoriel, on a localement $D=\divi(f)$ avec $f\in k(X)$. Ainsi, $D-\divi(f)\in K^x$ et $c(D-\divi(f)) =[D]$. Pour tout $E\in K^{x,0}$, on a $E=\divi(f_E)$ pour une fonction rationnelle $f_E\in k(X)$ définie sur un voisinage de $x$, car $x\notin \supp E$. Si l'on impose $f_E(x)=1$, alors $f_E$ est unique car $\Oo(X)^*=k^*$ et on définit ainsi un caractère $\chi^x:K^x\rightarrow k(X)^*,\,E\mapsto f_E$. On note $\Rr^x$ le faisceau de Cox obtenu canoniquement à partir de ces données.
\end{rem}

\subsection{Exemples}

\begin{ex}
Considérons la surface affine $X\subset \AAA_k^3$ d'équation $t_1t_2-t_3^2$. Cette surface est irréductible et normale d'après \ref{SerreCritere2} et \ref{CritereReg}. On note $f_i$ la fonction régulière sur $X$ associée à la restriction de la coordonnée $t_i$ pour $1\leq i\leq 3$. Soit $D_2:=\Vv_X(f_2,f_3)$, on a facilement $\Oo_{D_2}(D_2)\simeq k[f_1]$ donc $D_2$ est un fermé irréductible de codimension 1 de $X$, c'est donc un diviseur premier et $(f_2,f_3)$ est son point générique. On a $(f_2,f_3)\Oo(X)_{(f_2,f_3)}=(f_1f_2,f_3)\Oo(X)_{(f_2,f_3)}=(f_3^2,f_3)\Oo(X)_{(f_2,f_3)}=(f_3)\Oo(X)_{(f_2,f_3)}$ donc $f_3$ est une uniformisante dans l'anneau de valuation discrète $\Oo(X)_{(f_2,f_3)}$. Dans cet anneau, on a $f_2=f_1^{-1}f_3^2$, d'où $v_{D_2}(f_2)=2$, puis $\divi(f_2)=2D_2$ car $\supp \divi(f_2)=D_2$. On a de même $\divi(f_1)=2D_1$, et $\divi(f_3)=D_1+D_2$.

Par ailleurs, $\Oo(X\setminus D_2)=\Oo(X)_{f_2}\simeq k[f_2^{\pm 1},f_3]$ est factoriel donc $\clg(X\setminus D_2)=0$ d'après \ref{AffUFDClgTrifial}. Notons que cela montre aussi que $\Oo(X)^*=k^*$. D'après \ref{divexactseq}, $\clg(X)$ est engendré par $D_2$. Montrons que $D_2$ n'est pas principal. Soit l'idéal maximal $\mathfrak{m}:=(f_1,f_2,f_3)$. Le $k$-espace vectoriel $\mathfrak{m}/\mathfrak{m}^2$ est de dimension $3$ engendré par les classes $\bar{f_1},\bar{f_2},\bar{f_3}$ des fonctions $f_i$ modulo $\mathfrak{m}^2$. Mais $(f_2,f_3)\subset \mathfrak{m}$, et son image dans $\mathfrak{m}/\mathfrak{m}^2$ contient $\bar{f_2}$ et $\bar{f_3}$, il ne peut donc pas être principal. Ainsi, $\clg(X)\simeq \ZZ/2\ZZ$ et on a deux générateurs de somme nulle $D_1$ et $D_2$.

Calculons le faisceau d'algèbres divisorielles $\Ss$ associé à $K=\ZZ D_2$. D'après ce qui précède on a $\Ss_{2D_2}(X)=(f_2^{-1})$ et $\Ss_{-2D_2}(X)=(f_2)$ comme $\Oo(X)$-modules. On a $\Ss_{-D_2}(X)=(f_2,f_3)$, puis d'après \ref{isomophismeOTimes} et \ref{FQCProps3} $2$, on a $\Ss_{D_2}(X)=(f_2,f_3)(f_2^{-1})=(1,f_2^{-1}f_3)$. Posons 
$$g_1:=1\in\Ss_{D_2}(X), \,\,g_2:=f_2^{-1}f_3\in\Ss_{D_2}(X),\,\,g_3:=f_2^{-1}\in \Ss_{2D_2}(X), \,\,g_4:=f_2\in\Ss_{-2D_2}(X)$$
Compte tenu des relations ci-dessous, $\Ss(X)$ est engendrée comme $k$-algèbre par $g_1,g_2,g_3$ et $g_4$.
\begin{center}
$f_1=g_2^2g_4\in\Ss_0(X)=\Oo(X)$, de même,  $f_2=g_1^2g_4$ et $f_3=g_1g_2g_4$
\end{center}
Soit $\widetilde{X}=\spec_X(\Ss)$ le spectre relatif de $\Ss$ muni de son morphisme structural $p$. Comme $(X,p)$ est un bon quotient pour l'action de $\GG_m$ sur $\widetilde{X}$ défini par la $\ZZ$-graduation $\deg(g_1)=1,\deg(g_2)=1,\deg(g_3)=2$ et $\deg(g_4)=-2$, on a $\dim \widetilde{X}=\dim X+ 1$. De plus, $\dim \widetilde{X}=\dim \Ss(X)$ car on a une immersion ouverte $\widetilde{X}\xhookrightarrow{} \spec \Ss(X)$. On en déduit que $g_3g_4=1$ engendre toutes les relations dans $\Ss(X)$ car ce dernier est intègre. On a ainsi un isomorphisme d'algèbres graduées $\Ss(X)\simeq k[t_1,t_2,t_3, t_3^{-1}]$, où les $t_i$ sont des indéterminées et $\deg(t_i):=\deg(g_i)$.

Le noyau de la projection $\ZZ D_2\rightarrow \clg(X)$ est $K^0:=2\ZZ D_2$, et un caractère est donnée par $\chi: K^0\rightarrow k(X),\,\,2nD_2\mapsto f_2^n$. L'idéal $\Ii$ est engendré en tant que $\Oo_X$-module par la section globale $1-g_4$. Finalement l'anneau de Cox de $X$ est donné à ismorphisme près par
$$\Rr(X)\simeq k[t_1,t_2,t_3^{\pm}]/(1-t_3^{-1})\simeq k[t_1,t_2]$$
la $\clg(X)$-graduation étant donnée par $\deg(t_1)=\deg(t_2)=[D_2]$.
\end{ex}

\begin{ex}\label{CoxRingEx3}
Considérons la surface affine $X\subset \AAA_k^3$ d'équation $t_1^2-t_2t_3-1$. Cette surface est lisse et irréductible. On note $f_i$ la fonction régulière sur $X$ associée à la restriction de la coordonnée $t_i$ pour $1\leq i\leq 3$. Considérons les diviseurs premiers
$$D_+=\Vv_X(f_1-1,f_2)=\lbrace 1\rbrace\times\lbrace 0\rbrace\times k$$
$$D_-=\Vv_X(f_1+1,f_2)=\lbrace -1\rbrace\times\lbrace 0\rbrace\times k$$
Dans chaque anneau local $\Oo_{X,D_\pm}$, on voit facilement que $f_2$ est une uniformisante, et on obtient $\divi(f_2)=D_++D_-$. De plus, $X\setminus\supp \divi(f_2)=X_{f_2}\simeq \AAA^{1*}_k\times \AAA^1$ car $\Oo(X_{f_2})\simeq k[f_1,f_2^{\pm 1}]$. Cela montre que $\Oo(X)^*=k^*$ et que $\clg(X)$ est engendré par $[D+]$, d'après \ref{AffUFDClgTrifial}, \ref{divexactseq}, et le fait que $D_+$ et $-D_-$ soient linéairement équivalents. Supposons qu'il existe $n>0$ tel que $n[D_+]=0$, alors $nD_+=\divi(f)$ avec $f\in\Oo(X)$ car $D_+$ est effectif. On a de fait $f_2^n=fh$ avec $h\in\Oo(X)$ tel que $\divi(h)=nD_-$. Si on munit $k[t_1,t_2,t_2]$ de la graduation $\deg(t_1)=0,\,\deg(t_2)=1,\,\deg(t_3)=-1$ alors l'équation $t_1^2-t_2t_3-1$ est homogène, est $\Oo(X)$ hérite d'une graduation induite avec $\deg(f_i)=\deg(t_i)$. Tout élément de degré positif est multiple de $f_2$, en particulier on voit dans l'écriture $f_2^n=fh$ que $f$ ou $h$ est nécessairement multiple de $f_2$, ce qui est impossible vu leur diviseur respectif. On conclut que $\clg(X)=\ZZ[D_+]$.

La variété $X$ est naturellement munie d'une action de $G:=\lbrace \pm 1\rbrace\simeq \ZZ/2\ZZ$ par $x\mapsto -x$. D'après, \ref{goodquotientthm} on a l'existence du bon quotient $\pi:X\rightarrow Y:=X//G$, qui est de plus irréductible et normal. L'algèbre des invariants est facilement identifiable, on obtient  $Y=\spec k[f_1^2,f_2^2,f_3^2,f_1f_2,f_1f_3,f_2f_3]$, avec $\pi$ donné par l'inclusion de cette algèbre. On montre maintenant que $\clg(Y)\simeq \ZZ/2\ZZ$, engendré par la classe de $D:=\pi(D+)$. On a une action naturelle de $G$ sur $\wdiv(X)$ et l'orbite de $D_+$ est $\lbrace D_-,D_+\rbrace$. Ainsi, $G$ agit sur $X_{f_2}$ et le quotient $\pi(X_{f_2})$ a pour algèbre d'invariants $k[f_1^2,f_2^{\pm 2}]$. Ce dernier anneau est factoriel, on en déduit que $[D]$ engendre $\clg(Y)$. Par ailleurs, $D$ n'est pas principal sinon $\pi^*(D)=D_++D_-$ serait le diviseur d'une fonction $G$-invariante, ce qui n'est pas le cas. Enfin, comme on a $2D=\divi(f_2^2)$, le résultat est démontré.

Déterminons maintenant l'anneau de Cox de $Y$. Calculons le faisceau d'algèbres divisorielles $\Ss$ sur $Y$ associé à $K=\ZZ D$. On a $\Ss_{-D}(Y)=\Ii_Y(D)=\pi^{\sharp}(Y)^{-1}((f_2))=k[f_1^2,f_2^2,f_3^2,f_1f_2,f_1f_3,f_2f_3]\cap (f_2)=(f_1f_2,f_2^2,f_2f_3)$. On voit facilement que $D$ est de Cartier car il est principal localement en $0$ et aussi sur les ouverts affines principaux associés à $f_1f_2, f_2f_3$ et $f_1f_3$ qui recouvrent $Y\setminus \lbrace 0\rbrace$. Par exemple on a 
$$(f_1f_2,f_2^2,f_2f_3)_{f_1f_3}=(f_1^2f_2f_3,f_1^2f_2^2f_3^2,f_2f_3)_{f_1f_3}=((f_2f_3+1)f_2f_3,f_2f_3)_{f_1f_3}=(f_2f_3)_{f_1f_3}$$
Ainsi, $\Ss_{-D}(Y)$ est inversible et on a $\Ss_{D}(Y)=\Ss_{-D}(Y)^{-1}=(\Oo(Y):\Ss_{-D}(Y))$. On trouve facilement des sections 
$$a_1:=1,\,\,\,a_2:=f_1f_2^{-1},\,\,\,a_3:=f_2^{-1}f_3\in \Ss_D(Y)$$
Or rappelons que $\Ss_{-D}(Y)$ est engendré par les sections
$$b_1:=f_1f_2,\,\,\,b_2:=f_2^2,\,\,\,b_3:=f_2f_3$$
Donc compte tenu de la relation $a_2b_1-a_1b_3=1$ on conclut que $ \Ss_D(Y)=(a_1,a_2,a_3)$. Et comme
$$f_1^2=a_2b_1,\,\,\,f_2^2=a_1b_2,\,\,\,f_3^2=a_3b_3,\,\,\,f_1f_2=a_1b_1,\,\,\,f_1f_3=a_3b_1,\,\,\,f_2f_3=a_1b_3\in\Ss_0(Y)$$
on trouve que $\Ss(Y)$ est engendrée comme $k$-algèbre par $a_1,a_2,a_3,b_1,b_2,b_3$. Prenons le caractère $\chi:K^0\rightarrow k(Y)^*,2nD\mapsto f_2^{2n}$. L'idéal $\Ii(Y)$ de $\Oo(Y)$ est engendré par $1-f_2^2$ avec $1\in \Ss_0(Y)$ et $f_2^2\in \Ss_{-2D}(Y)$, d'après \ref{genreatorsCoxIdeal}, et $\Rr(Y)$ est engendré par les classes $\overline{a_i}$ des $a_i$ module $\Ii(Y)$. Or on remarque que
$$a_1-b_2=(1-f_2^2)a_1,\,\,\,a_2-b_1=(1-f_2^2)a_2,\,\,\,a_3-b_3=(1-f_2^2)a_3$$
et on en déduit que $\Rr(Y)$ est engendré par $z_1:=\overline{a_2}=\overline{b_1},z_2:=\overline{a_1}=\overline{b_2},z_3=\overline{a_3}=\overline{b_3}$. La relation $z_1^2-z_2z_3-1=0$, engendre toute les relations pour des raisons de dimension. Ainsi $\Rr(Y)\simeq \Oo(X)$, en particulier l'anneau de Cox n'est pas factoriel car le groupes des classes de $X$ n'est pas trivial.

\end{ex}

\section{Propriétés algébriques de l'anneau de Cox} 

\subsection{Intégrité et normalité}

Notre objectif est de montrer que les anneaux de Cox sont intégralement clos. Remarquons tout de suite que l'on peut travailler avec une variété lisse pour cela.

\begin{prop}\label{isomcodime2CoxTorsion}
On considère les données de la construction \ref{conscoxtorsion}. Alors pour tous ouverts $V\subset U\subset X$ tels que $U\setminus V$ est de codimension $\geq 2$ dans $U$, la restriction $\Rr(U)\rightarrow \Rr(V)$ est un isomorphisme.
\end{prop}
\begin{proof}
On a déjà vu l'isomorphisme $\Ss(U)\rightarrow \Ss(V)$ en \ref{isomorphismcodim2}. De plus, la proposition \ref{coxidealform} montre que la restriction $\Ii(U)\rightarrow \Ii(V)$ est un isomorphisme. On conclut en appliquant la proposition \ref{coxsheafiso}.
\end{proof}

\begin{cons}[Car. zéro]\label{conscoxtorsionlisse}
On considère les données de la construction \ref{conscoxtorsion}. On suppose de plus que $X$ est lisse. On considère les spectres relatifs $(\widetilde{X}, p)=\spec_X \Ss$ et $(\widehat{X},p_X)=\spec_X \Rr$. De plus, on peut toujours supposer $K$ de type fini et ces spectres sont munis respectivement d'une action du tore $H:=\spec k[K]$ et du groupe diagonalisable $H_X:=\spec k[\clg(X)]$. On a une immersion fermée $\widehat{X}\xhookrightarrow{i}\widetilde{X}$ d'image $\Vv(\Ii)$ qui donne un diagramme commutatif

	\begin{center}
	\begin{tikzcd}
  		\widehat{X}\arrow[rd, swap, "p_X"] \arrow[rr, "i"] & & \widetilde{X} \arrow[ld, "p"] \\ 
  		& X &
	\end{tikzcd}
	\end{center}
De même, $H_X$ se plonge dans $H$, et l'immersion fermée $i$ est $H_X$ équivariante pour ce plongement. Les morphismes $p_X$ et $p$ sont des quotients géométriques. En particulier, $(p_X, \widehat{X})$ et $(p, \widetilde{X})$ sont des fibrés principaux. Enfin, $\widehat{X}$ est une variété lisse et quasi-affine.
\end{cons}
\begin{proof}
Les actions de $H$ et $H_X$ sont dues aux graduations respectives de $\Ss$ et $\Rr$ par $K$ et $\clg(X)$. Pour tout ouvert affine $U\subset X$, notons $U_1:=p_X^{-1}(U)$ et $U_2:=p^{-1}(U)$. La projection $\Ss(U)\rightarrow \Rr(U)$ donne localement une immersion fermée $U_1\xhookrightarrow{} U_2$ d'image $\Vv_{U_2}(\Ii(U_2))$. De plus, ces immersions sont graduées par la projection $c:K\rightarrow \clg(X)$, et se recollent en une immersion $i$ d'image $\Vv(\Ii)$ qui est $H_X$-équivariante. Les mêmes arguments qu'en \ref{specrelatifdivi} nous donne que $p_X$ et $p$ sont des bon quotients car on a

$$(p_X)_*(\Oo_{\widehat{X}})_0\simeq \Rr_0\simeq\Oo_X\simeq\Ss_0\simeq  p_*(\Oo_{\widetilde{X}})_0$$
Puis la remarque \ref{lisseGoodQuotient} nous dit que ce sont en fait des quotients géométriques. Ainsi, le morphisme $\Gamma:H_X\times \widehat{X}\rightarrow \widehat{X}\times_X\widehat{X}$ de \ref{GPrincipalBundleDef} est surjectif. Comme $H_X$ agit librement, $\Gamma$ est bijectif. On va voir plus loin que $\widehat{X}$ est normale, donc $\widehat{X}\times_X \widehat{X}$ également. Ainsi, $\Gamma$ est un isomorphisme. De plus, on va également voir que $p_X$ est plat. Comme il est surjectif on obtient que $(\widehat{X}, p_X)$ est un $H_X$-fibré principal. On a déjà vu que $(\widetilde{X},p)$ était un $H$-fibré principal.

Montrons maintenant que $\widehat{X}\simeq \Vv_{\widetilde{X}}(\Ii)$ est une variété. C'est un schéma de type fini sur $k$. De plus, il est séparé sur $k$ car \ref{sepCritere} est vérifié par construction, et comme $X$ est séparé sur $k$, on peut appliquer \ref{sepCritere2}. Il reste à montrer que ce schéma est réduit. On peut travailler localement sur un ouvert affine de la forme $\widetilde{U}=p^{-1}(U)=\spec(\Ss(U))$ pour un ouvert affine $U$ de $X$. Soit $f\in\sqrt{\Ii(U)}$, comme c'est un idéal $\clg(X)$-homogène, il contient les composantes $\clg(X)$-homogènes de $f$. Soit $f_{[D]}$ l'une d'entre elles, elle s'écrit $f_{[D]}=f_D+g$ avec $f_D\in \Ss_D(U)$ et $g\in\Ii(U)$ d'après \ref{coxsheafiso2}. On obtient facilement $f_D^n\in \Ii(U)$ pour un $n>0$. Cela impose $f_D^n=0$ compte tenu de \ref{coxidealform} et donc $f_D=0$. Finalement, $f\in\Ii(U)$ ce qui montre que $\Ii(U)$ est radical.

La variété $\widehat{X}$ est quasi-affine car $\widetilde{X}$ l'est d'après \ref{specrelatifQuasiAff}. Montrons pour finir qu'elle est lisse. L'immersion $i$ est $H_X$ équivariante si bien que l'on peut supposer que  $\widehat{X}$ est une sous-variété $H_X$-stable de $\widetilde{X}$ pour l'action induite de $H_X\subset H$. De plus, on peut travailler localement et donc supposer que $\widetilde{X}\simeq H\times X$ avec l'action de $H$ sur le facteur de gauche, d'après \ref{lisseGoodQuotient} car $X$ est lisse. Supposons que $H_X$ soit un tore, alors c'est un facteur direct de $H$ d'après \ref{GDiagSubtorusFacteurDirect}, c'est à dire qu'il existe un sous-tore $H'\subset H$ tel que $ H_X\times H'\simeq H$ via le morphisme naturel $(h,h')\mapsto hh'$. Alors $\widehat{X}$ est égal à $H_X\times S$ pour une certaine sous-variété $S$ de $H'\times X$ qui est isomorphe à $X$ via la projection $p$. On obtient le diagramme ci-dessous, et donc le résultat.

	\begin{center}
	\begin{tikzcd}
  		\widehat{X}=H_X\times S \arrow[r, hook,"\subset"] \arrow[d, "p_S"] & \widetilde{X}=H_X\times H'\times X \arrow[d, "p"] \\ 
  		S \arrow[r, "\simeq"] & X
	\end{tikzcd}
	\end{center}

Dans le cas général on peut d'après \ref{lisseGoodQuotient} trouver un sous-tore $H'$ de $H$ tel que $H_XH'=H$ et $H_X\cap H'$ est fini. Grâce à cela on va pouvoir construire un revêtement étale $S$ de $X$ tel que $\widehat{X}$ soit le quotient de $H_X\times S$ par le groupe fini $H_X\cap H'$. Comme $S$ sera lisse en tant que revêtement étale d'une variété lisse d'après \ref{EtaleLisse}, le résulta suivra d'après \ref{GITEtaleFiniteGroup}. On considère le carré cartésien ci-dessous, où $(\widetilde{X},\pi)$ est le quotient par $H_X\cap H'$

	\begin{center}
	\begin{tikzcd}
  		Y=\pi^{-1}(\widehat{X}) \arrow[r, hook,"\subset"] \arrow[d, "\pi_Y"] \arrow[rd, phantom, "\square"]& H_X\times H'\times X \arrow[d, "\pi"] \\ 
  		\widehat{X} \arrow[r, "\subset"] & \widetilde{X}=H\times X
	\end{tikzcd}
	\end{center}

Comme dans le cas facile où $H_X$ était facteur direct, on a $Y=H_X\times S$, où $S$ est une sous-variété de $H'\times X$ telle que $X$ est le quotient de $S$ par $H_X\cap H'$. On a également que $(\widehat{X}, \pi_Y)$ est le quotient de $Y$ par $H_X\cap H'$ d'après \ref{goodquotientthm}. Ainsi $\pi_Y$ est étale d'après \ref{GITEtaleFiniteGroup}. Finalement on obtient le diagramme commutatif ci-dessous, où $\pi_S$ est également étale pour la même raison que $\pi_Y$, ce qui finit la preuve.

	\begin{center}
	\begin{tikzcd}
  		& Y=H_X\times S \arrow[r, hook,"\subset"] \arrow[d, "\pi_Y=/H_X\cap H'"] \arrow[ddl, "p_S"]& H_X\times H'\times X \arrow[d, "\pi=/H_X\cap H"] \\ 
  		& \widehat{X} \arrow[d, "p_X=/H_X"] \arrow[r, "\subset"]& \widetilde{X}=H\times X \arrow[dl, "p=/H"] \\
  		S \arrow[r,"\pi_S=/H_X\cap H'"] & X &
	\end{tikzcd}
	\end{center}


\end{proof}

\begin{thm}[Car. zéro]
On considère les données de la construction \ref{conscoxtorsion}. On suppose de plus que $\Rr$ est localement de type fini. Alors pour tout ouvert $U\in X$, l'anneau $\Rr(U)$ est intégralement clos.
\end{thm}
\begin{proof}
On peut supposer $X$ lisse donc être dans le contexte de \ref{conscoxtorsionlisse}. Alors $\widehat{X}$ est lisse donc normale. Il reste donc à prouver que les anneaux $\Rr(U)$ sont intègres pour tout ouvert $U$ de $X$. Si on prouve l'irréductibilité de $\widehat{X}$, ce sera donc un schéma intègre est on aura le résultat d'après \ref{SchemaIntegreCritere}. Comme $p_X$ est surjectif, une de ses composantes irréductibles $\widehat{X}_1$ domine $X$ car une variété irréductible ne peut être réunion de fermés irréductibles de codimensions non-nulles. Soit $U$ un ouvert de $X$, si $p_X^{-1}(U)$ est irréductible alors il est inclus dans $\widehat{X}_1$. Par ailleurs, comme $p_X^{-1}(U)$ est isomorphe à la sous variété de $p^{-1}(U)$ définie par $\Ii(U)$. Il suffit de montrer que l'on peut recouvrir $X$ par des ouverts $U$ tels que $\Vv_{p^{-1}(U)}(\Ii(U))$ est irréductible, on aura alors $\widehat{X}_1=\widehat{X}$.

On peut supposer $K$ de type fini d'après \ref{InvarianceCoxTorsion}. Soit $(D_1,...,D_s)$ une base de $K$ telle que $(n_1D_1,...,n_lD_l)$ soit une base de $K^0$ avec $1\leq l\leq s$. Une telle base existe d'après la théorie des $\ZZ$-modules. On peut également supposer que les diviseurs $D_i$ ne sont pas des multiples dans $K$, quitte à remplacer $K$ si nécessaire. On choisit un ouvert affine $U\subset X$ tel que chaque diviseur $D_i$ est principal, disons égal à $\divi(f_i)$ avec $f_i\in k(X)$. D'après le corollaire \ref{genreatorsCoxIdeal}, l'idéal $\Ii(U)$ est engendré par les générateurs $1-\chi(n_iD_i)$ pour $1\leq i\leq l$. L'isomorphisme ci-dessous identifie $p^{-1}(U)$ à $U\times \GG_m^s$. 
\begin{center}
$\Oo_X(U)\otimes_k k[t_1^{\pm},...,t_s^{\pm}] \rightarrow \mathcal{R}(U),\, g\otimes t_1^{\nu_1}...t_s^{\nu_s}\mapsto gf_1^{-\nu_1}...f_s^{-\nu_s}$
\end{center}
Par cet identification, $p_X^{-1}(U)$ est donné par les équations $1-\chi(n_iD_i)f_i^{n_i}t_i^{n_i}$ pour $1\leq i\leq l$, et on va montrer que cet ouvert affine est irréductible. On doit pour cela vérifier que l'algèbre ci-dessous est intègre
\begin{center}
$\Oo_{\widehat{X}}(p_X^{-1}(U))\simeq \frac{\Oo_X(U)[t_1^{\pm 1},...,t_s^{\pm 1}]}{(1-\chi(n_iD_i)f_i^{n_i}t_i^{n_i})_{1\leq i\leq l}}$
\end{center}
On peut le faire par quotients successifs, où à chaque étape $i<l$ on effectue un quotient
\begin{center}
$\frac{A_{i-1}[t_i^{\pm 1}]}{(1-\chi(n_iD_i)f_i^{n_i}t_i^{n_i})}[t_{i+1}^{\pm 1},...,t_s^{\pm 1}]$,
\end{center}
où $A_{i-1}$ est l'algèbre des coordonnées d'une sous-variété irréductible de $U\times\GG_m^{i-1}$, avec $A_0=\Oo_X(U)$, et l'équation $1-\chi(n_iD_i)f_i^{n_i}t_i^{n_i}$ est vue dans l'algèbre $A_{i-1}[t_i^{\pm 1}]$. On peut travailler dans le corps des fractions $K_{i-1}$ de $A_{i-1}$, et supposer $f_i$ inversible. En faisant le changement de variable $u_i=f_it_i$, on est donc ramené à montrer que $1-\chi(n_iD_i)u_i^{n_i}$ est irréductible dans $K_{i-1}[u_i^{\pm 1}]$. En utilisant le lemme suivant, on doit donc montrer que $\chi(n_iD_i)$ n'est pas une puissance dans $K_{i-1}$. Il est suffisant de le faire dans $k(X)$ car $\chi(n_iD_i)$ est indépendant des coordonnées $t_j$ pour $j\neq i$. Donc si $\chi(n_iD_i)$ était une puissance dans $K_{i-1}$, il le serait nécessairement dans $k(X)$, en attribuant des valeurs arbitraires aux coordonnées $t_j$. Supposons par l'absurde que $\chi(n_iD_i)$ soit une puissance dans $k(X)$. On obtient alors $n_iD_i=k_i\divi(h_i)$ pour un élément $h_i\in k(X)$. Or les diviseurs $D_i$ ne sont pas multiples dans $\wdiv(X)$ donc $k_i$ divise $n_i$, ce qui entraine que $n_i/k_iD_i$ est principal, c'est une contradiction avec le fait que $(n_1D_1,...,n_lD_l)$ soit une base de $K^0$. Finalement, comme $X$ est lisse, on peut la recouvrir par de tels ouverts, ce qui conclut la preuve.
\end{proof}

\begin{lem}[Car. zéro]
Soit $K$ un corps de caractéristique zéro contenant toutes les racines de l'unité. Soit $a\in K$ qui ne soit pas une puissance non-triviale. Alors pour tout entier $n> 0$, le polynôme $1-at^n$ est irréductible dans $K[t^{\pm 1}]$.
\end{lem}
\begin{proof}
Dans une clôture algébrique de $K$, notons $b=\sqrt[n]{a}$ une racine $n$-ième de $a$. On a donc $1-at^n=\prod_{i=0}^{n-1}(1-\xi^i bt)$. Si on pouvait écrire une factorisation non-triviale $1-at^n=h_1(t)h_2(t)$ dans $K[t^{\pm 1}]$, alors $b^k\in K$ pour un entier $1<k<n$ car $K$ contient toutes les racines de l'unité. En notant $d=k\wedge n$ on obtient $b^d\in K$, ce qui montre que $a$ est une puissance non-triviale. C'est une contradiction.
\end{proof}

\subsection{Localisation et groupe des unités}

\begin{cons}\label{conscoxtorsionDDiviseur}
Dans la situation de la construction \ref{conscoxtorsion}, considérons un diviseur $D\in K$ et une section globale homogène $f\in\Rr_{[D]}(X)$. D'après la proposition \ref{coxsheafiso2}, il existe un unique élément $\tilde{f}\in\Ss_D(X)$ tel que $\pi(X)(\tilde{f})=f$. On définit le $[D]$-diviseur de $f$ comme le diviseur de Weil effectif
$$\divi_{[D]}(f):=\divi_D(\tilde{f})=\divi(\tilde{f})+D$$
Ce diviseur ne dépend ni du choix de $D\in c^{-1}([D])$, ni des choix faits en \ref{conscoxtorsion}.
\end{cons}
\begin{proof}
Soit $f\in\Rr_{[D]}(X)$, et deux isomorphismes $\phi_i:\Oo_X(D_i)\rightarrow \Rr_{[D]}$. On note $\tilde{f}_i$ les deux sections telles que $\phi(\tilde{f}_i)=f$. En raisonnant sur l'ouvert régulier de $X$, on voit que l'isomorphisme $\phi_2^{-1}\phi_1$ est donné par la multiplication par un élément $h\in k(X)^*$ satisfaisant $\divi(h)=D_1-D_2$. On a ainsi $\tilde{f}_2=\tilde{f}_1h$ et on obtient facilement les égalités
$$D_2+\divi(\tilde{f}_2)=D_1+\divi(\tilde{f}_1)$$
\end{proof}


\begin{prop}\label{diviCoxTorsFormulas}
Avec les notations de la proposition précédentes, on a:
\begin{enumerate}
\item Pour tout diviseur effectif $E\in\wdiv(X)$, il existe $[D]\in\clg(X)$ et $f\in\Rr_{[D]}(X)$ tels que $E=\divi_{[D]}(f)$.
\item Soit $[D]\in\clg(X)$ et $f\in\Rr_{[D]}(X)$ non-nulle. Alors $\divi_{[D]}(f)=0\implies [D]=0$ dans $\clg(X)$.
\item Pour tous $f\in\Rr_{[D_1]}(X)$ et $g\in\Rr_{[D_2]}(X)$, on a $\divi_{[D_1]+[D_2]}(fg)=\divi_{[D_1]}(f)+\divi_{[D_2]}(g)$.
\end{enumerate}
\end{prop}
\begin{proof}
La première assertion est claire d'après \ref{SectionDivCartierPicardGroup}. Pour la deuxième, si $\divi_{[D]}(f)=0$, alors $\divi_{[D]}(\tilde{f})=0$ pour un certain représentant $\tilde{f}\in\Ss_D(X)$ de $f$. Cela montre que $D$ est principal. La troisième assertion est claire par définition.
\end{proof}

On remarque que le système linéaire complet associé à un diviseur $D$ d'une variété projective irréductible et normale s'écrit
$$|D|=\lbrace\divi_{[D]}(f)\mid f\in \Rr_{[D]}(X)\rbrace = \PP(\Rr_{[D]}(X))$$

\begin{defn}
Dans la situation de la construction \ref{conscoxtorsion}. Pour tout élément non-nul $f\in\Rr_{[D]}(X)$, on définit la $[D]$-localisation de $X$ par $f$ comme l'ouvert de $X$
$$X_{[D],f}:=X\setminus\supp(\divi_{[D]}(f))$$
\end{defn}

\begin{prop}
Dans la situation de la construction \ref{conscoxtorsion}. Pour tout élément non-nul $f\in\Rr_{[D]}(X)$ on a isomorphisme canonique
$$\Gamma(X_{[D],f},\Rr)\simeq \Gamma(X,\Rr)_f$$
\end{prop}
\begin{proof}
C'est une conséquence de \ref{CoxIsomorphismLocalisation}, \ref{coxsheafiso} et du fait que la localisation est compatible avec le passage au quotient.
\end{proof}

On s'intéresse maintenant aux unités de l'anneau de Cox. Le résultat suivant montre en particulier que les unités de l'anneau de Cox d'une variété projective irréductible et normale sont les fonctions constantes.

\begin{prop}
Dans la situation de la construction \ref{conscoxtorsion},
\begin{enumerate}
\item Tout élément homogène inversible de $\Rr(X)$ est constant.
\item Si $\Gamma(X, \Oo)=k$, tout élément inversible de $\Rr(X)$ est constant.
\end{enumerate}
\end{prop}
\begin{proof}
Considérons $f\in\Rr_{[D]}(X)$ inversible. Pour la première assertion, l'inverse $g$ de $f$ appartient à $\Rr_{-[D]}(X)$ et on a $fg=1$ dans $\Rr_0(X)^*=\Oo(X)^*=k^*$. D'après \ref{diviCoxTorsFormulas} on a $0=\divi_0(fg)=\divi_{[D]}(f)+\divi_{[-D]}(g)$. Comme les deux derniers diviseurs sont effectifs, ils sont nécessairement nuls. Toujours d'après \ref{diviCoxTorsFormulas} on obtient $[D]=0$, et donc $f\in k^*$.
Pour la seconde assertion, on doit montrer que tout $f\in\Rr(X)^*$ est de degré $0$. On considère la $K_0\oplus K_t$ où $K_t$ est le sous $\ZZ$-module de torsion de $K$.
On considère la graduation "épaissie"
$$\Rr(X)=\bigoplus_{w\in K_0}R_w,\,\,\,\,\,\, R_w:=\bigoplus_{u\in K_t}\Rr(X)_{w+u}$$
Avec cette graduation, on peut appliquer \ref{GradedProp1} et en déduire que $f$ et $f^{-1}$ sont homogènes de degré respectifs $w\in K_0$ et $-w$. On écrit les décompositions de $f$ et $f^{-1}$ en composantes $\clg(X)$-homogènes
$$f=\sum_{u\in K_t}f_{w+u},\,\,\,\,\,\, f^{-1}=\sum_{u\in K_t}f^{-1}_{-w-u}$$
Comme $fg=1$ on a $f_{w+u_0}f^{-1}_{-w-u_0}\neq 0$ pour au moins un degré $u_0\in K_t$. En tant que fonction régulière sur $X$, le produit $f_{w+u_0}f^{-1}_{-w-u_0}$ est une constante non-nulle par hypothèse. On en déduit $w+u_0=0$ comme plus haut et donc $w=0$. Ainsi, tous les degrés $w+u$ sont de torsion dans $\clg(X)$. Pour $u\neq 0$, on a donc $n(w+u)=0$ pour un certain entier $n>0$, puis en utilisant que $\Oo(X)=k$, on a $0=\divi_{0}(f_{w+u}^n)=\divi_{n(w+u)}(f_{w+u}^n)=n\divi_{w+u}(f_{w+u})$. On en déduit $f_{w+u}=0$ d'après \ref{diviCoxTorsFormulas}, puis que $f$ est homogène de degré $0$ donc constante.
\end{proof}

\subsection{Propriétés de divisibilité}

Dans le cas d'un groupe des classes de type fini sans torsion on a vu que l'anneau de Cox etait factoriel. Ce n'est pas le cas en général en présence de torsion comme le montre l'exemple \ref{CoxRingEx3}. Toutefois, si l'on se restreint aux éléments homogènes on peut obtenir des critères intéressants de divisibilité. 

\begin{defn}
Soit $K$ un groupe abélien et $R$ une $k$-algèbre $K$-graduée intègre.
\begin{enumerate}
\item Soit $f\in R$ un élément non-nul et non-inversible. On dit que $f$ est $K$-premier si il est homogène et si $f\mid gh$ avec $g$ et $h$ homogènes implique que $f\mid g$ ou $f\mid h$.
\item On dit que $R$ est $K$-factoriel si tout élément $f$ homogène non-nul et non-inversible est produit d'éléments $K$-premiers.
\item On dit d'un idéal $\mathfrak{a}$ qu'il est $K$-principal si il est engendré par un élément homogène.
\item On dit d'un idéal $\mathfrak{a}$ qu'il est $K$-premier si il est homogène et pour tous éléments $f,g\in R$ homogènes tels que $fg\in\mathfrak{a}$, on ait $f\in\mathfrak{a}$ ou $g\in\mathfrak{a}$.
\item On dit qu'un idéal $K$-premier $\mathfrak{a}$ est de $K$-hauteur $d$ si cet entier est la longueur maximale des chaines $\mathfrak{a}_0\subset \mathfrak{a}_1\subset ...\subset \mathfrak{a}_k=\mathfrak{a}$ d'idéaux $K$-premiers.
\end{enumerate}
\end{defn}

On peut considérer ces notions d'un point de vue géométrique. Soit $H=\spec k[K]$ un groupe diagonalisable et $X$ une $H$-variété affine irréductible et normale. Alors $\wdiv(X)$ est naturellement muni d'une action de $H$ par 
$$h.\sum_i a_iD_i=\sum_i a_i(h.D_i)$$
Considérons le support $Y$ d'un diviseur, c'est à dire la réunion des supports d'un nombre fini de diviseurs premiers. Ce support est $H$-stable si et seulement si l'idéal $\mathfrak{a}:=\Ii_X(Y)$ qu'il définit est $K$-homogène. De plus, si $\mathfrak{a}$ est $K$-premier, alors il est irréductible en tant qu'idéal $K$-homogène, c'est à dire que l'on ne peut pas l'écrire comme intersection d'idéaux $K$-homogènes qui le contiennent strictement. On voit que cela entraine que l'action de $H$ est transitive sur les diviseurs premiers envisagés. Réciproquement, supposons que l'action est transitive et montrons que $\mathfrak{a}$ et $K$-premier. Cela revient à montrer que $\Oo(Y)$ est "$K$-intègre". Soient $f,g\in \Oo(Y)$ des éléments $K$-homogènes non-nuls, et supposons par l'absurde que $fg=0$. Alors $Y=\Vv_Y(f)\cup\Vv_Y(g)$, et chacun de ces deux fermés est $H$-stable. Par hypothèse on a nécessairement $Y=\Vv_Y(f)$, quitte à renommer $f$ en $g$, et donc $f=0$, c'est une contradiction. On introduit donc la version géométrique de ces concepts:

\begin{defn}
Soit $H=\spec k[K]$ un groupe diagonalisable et $X$ une $H$-variété irréductible et normale.
\begin{enumerate}
\item Un diviseur $H$-premier est un diviseur de Weil non-nul $\sum_i D_i$  tel que $H$ permute transitivement les diviseurs premiers $D_i$.
\item Un fonction rationnelle $f\in k(X)$ est $H$-homogène si elle est régulière sur un ouvert $H$-invariant de $X$, et $H$-homogène sur cet ouvert.
\item On dit que $X$ est $H$-factoriel si tout diviseur de Weil $H$-invariant est le diviseur d'une fonction rationnelle $H$-homogène.
\end{enumerate}
\end{defn}

\begin{prop}
Soit $H=\spec k[K]$ un groupe diagonalisable et $X$ une $H$-variété irréductible quasi-affine et normale. Considérons l'algèbre $K$-graduée $R:=\Oo(X)$. Les conditions suivantes sont équivalentes: 
\begin{enumerate}
\item Tout idéal $K$-premier de $K$-hauteur $1$ de $R$ est $K$-principal.
\item La variété $X$ est $H$-factoriel.
\item L'algèbre $R$ est $K$-factoriel.
\end{enumerate}
De plus si une de ces assertions est vérifiée, alors un élément homogène non-nul et non-inversible $f\in R$ est $K$-premier si et seulement si le diviseur $\divi(f)$ est $H$-premier, et tout diviseur $H$-premier est de la forme $\divi(f)$ pour un élément $K$-premier $f\in R$.
\end{prop}
\begin{proof}
Supposons $1$ et montrons $2$. Soit $D$ un diviseur de Weil $H$-invariant. On vérifie immédiatement que l'on peut l'écrire de manière unique $D=a_1D_1+...+a_rD_r$ où les $D_i$ sont des diviseurs $H$-premiers. Si $\mathfrak{a}_i:=\Ii_X(\supp(D_i))$ est de $K$-hauteur $1$, alors par hypothèse $D_i=\divi(f_i)$ avec $f_i\in R$ homogène, puis on obtient $D=\divi(f_1^{a_1}...f_r^{a_r})$. On voit donc qu'il suffit de montrer que $\mathfrak{a}_i$ est de $K$-hauteur $1$. Supposons que $X$ soit affine, on montrera ensuite que l'on peut faire cette simplification. Par l'absurde, supposons qu'il existe un idéal $K$-premier non-nul $\mathfrak{a}$ strictement contenu dans $\mathfrak{a}_i$. On peut donc trouver $f\in \mathfrak{a}_i\setminus\mathfrak{a}$	 que l'on peut de plus choisir homogène. Remarquons que $\Vv(\mathfrak{a})$ est de codimension $1$ car il contient $\Vv(\mathfrak{a}_i)$ et que $\mathfrak{a}$ est non-nul. Sa décomposition en composantes irréductibles s'écrit donc $\Vv(\mathfrak{a})=D'_1\cup...\cup D'_s\cup Z_1\cup ...\cup Z_t$, où les $D'_i$ sont les diviseurs premiers intervenant dans l'écriture de $D_i$. Notons $Z=Z_1\cup ...\cup Z_t$, alors $f$ s'annule sur $\supp(D_i)$ mais pas sur $Z$ car sinon une puissance de $f$ appartiendrai à $\mathfrak{a}$, et donc $f$ aussi car cet idéal est $K$-premier. Par ailleurs, $W:=\overline{Z\setminus \supp(D_i)}$ est un fermé propre non-vide de $X$ (et de $\Vv(\mathfrak{a})$), qui est de plus $H$-invariant. On peut donc trouver une fonction $g$ homogène non-nulle s'annulant sur $Z$ mais pas sur $\supp(D_i)$. On obtient ainsi $(fg)^n\in\mathfrak{a}$ pour un $n>0$ alors que $f\notin \mathfrak{a}$ et $g\notin\mathfrak{a}$, contradiction. Enfin, revenons au cas où $X$ est quasi-affine. En utilisant \ref{QuasiAffCritere}, on plonge $X$ dans une $H$-variété affine $Y$ normale et irréductible, par une immersion ouverte $H$-équivariante telle qu'il existe des fonctions régulières homogènes $f\in\mathfrak{a}_i\setminus \mathfrak{a}$, et $h\in\mathfrak{a}$ qui s'étendent en des fonctions régulières sur $Y$. Alors l'idéal $\mathfrak{b}=\mathfrak{a}\cap \Oo(Y)$ est $K$-premier, non-nul, et strictement contenu dans l'idéal $K$-premier $\mathfrak{b}_i=\mathfrak{a}_i\cap \Oo(Y)$. On a de plus $\Vv_Y(\mathfrak{b}_i)=\overline{D_i}$ qui est un diviseur de Weil $H$-premier de $Y$. On s'est donc ramené au cas affine.

Supposons $2$ et montrons $3$. Soit $f\in R\setminus R^*$ non-nul. Écrivons $\divi(f)=D_1+...+D_r$ où les $D_i$ sont $H$-premiers. Par hypothèse, $D_i=\divi(f_i)$ pour un élément $f_i\in R$ homogène, qui est de plus $K$-premier. On en déduit $f=uf_1...f_r$ où $u$ est une unité homogène de $R$. C'est ce qu'on voulait montrer.

Enfin, supposons $3$ et considérons un idéal $\mathfrak{a}$ $K$-premier de hauteur $1$. Soit $f\in\mathfrak{a}$ non-nul et non-inversible. Par hypothèse on peut trouver un facteur $K$-premier $f_1$ de $f$ qui de plus appartient à $\mathfrak{a}$. On obtient ainsi une chaine d'idéaux $K$-premiers $\lbrace 0\rbrace\varsubsetneq (f_1)\subset \mathfrak{a}$, ce qui force $(f_1)=\mathfrak{a}$.
\end{proof}

\begin{cor}
Avec les notations et hypothèses de la proposition précédente. Supposons de plus que $R^*=k^*$, et $R$ factoriel. Alors $R$ est $K$-factoriel.
\end{cor}
\begin{proof}
Considérons un diviseur $H$-invariant $D$ de $X$. Comme $X$ est quasi-affine, on peut la voire comme une sous-variété ouverte de $\spec R$ d'après \ref{QuasiAffCritere}. Par hypothèse, on a alors $D=\divi(f)$ avec $f\in \fract(R)$. Soit $h\in H$, comme $D$ est $H$-invariant, on a $\divi(h.f)=\divi(f)$ ce qui entraine $h.f=\chi(h)f$ pour un certain $\chi\in X^*(H)$, car $R^*=k^*$. Ainsi $f$ est $H$-homogène ce qui conclut la preuve d'après la proposition précédente car $X$ est $H$-factoriel.
\end{proof}

Muni de ces définitions on va pouvoir généraliser les résultats de divisibilité dans l'anneau de Cox obtenus en \ref{divisibilitePropsFreeCox} dans le cas sans torsion.

\begin{lem}
Avec les données de \ref{conscoxtorsionlisse}, on a pour tout élément non-nul $f\in \Rr_{[D]}(X)$
$$\divi(f)=p^*_X(\divi_{[D]}(f))$$
\end{lem}
\begin{proof}
Avec les notations de \ref{conscoxtorsionlisse}, soit $D\in c^{-1}([D])$ et $\tilde{f}\in\Ss_D(X)$ se projetant sur $f$. Le diagramme commutatif de \ref{conscoxtorsionlisse} donne
$$\divi(f)=i^*(\divi(\tilde{f}))=i^*(p^*(\divi_D(\tilde{f})))=p_X^*(\divi_{[D]}(f))$$
\end{proof}

\begin{prop}
Soit $X$ une variété normale irréductible telle que $\Oo(X)^*=k^*$ et $\clg(X)$ est de type fini.
\begin{enumerate}
\item Soient $f\in \Rr_{[D]}(X)$ et $g\in \Rr_{[E]}(X)$ non-nuls. Alors $f\mid g\iff \divi_{[D]}(f)\leq\divi_{[E]}(g)$.
\item Soient $f\in \Rr_{[D]}(X)$ et $g\in \Rr_{[E]}(X)$ non-nuls. Alors ces éléments sont associés si et seulement si $\divi_{[D]}(f)=\divi_{[E]}(g)$. Dans ce cas, on a $[D]=[E]$.
\item Soit $f\in \Rr_D(X)$ non-nuls. Alors $f$ est $\clg(X)$-premier si et seulement si $\divi_D(f)$ est $\clg(X)$-premier.
\end{enumerate}
\end{prop}
\begin{proof}
En utilisant \ref{isomcodime2CoxTorsion} on peut supposer $X$ lisse et donc se placer dans la situation de la construction \ref{conscoxtorsionlisse}. Alors la variété quasi-affine $\widehat{X}$ est lisse et son algèbre de fonction régulières est $\Rr(X)$. Ainsi on a $f\mid g$ dans $\Rr(X)$ si et seulement si $\divi(f)\leq \divi(g)$ dans $\wdiv(\widehat{X})$. D'après le lemme précédent, c'est équivalent à $\divi_{[D]}(f)\leq\divi_{[E]}(g)$. Avec cela on obtient directement $2$.

Pour le point $3$, supposons $\divi_{[D]}(f)$ premier. Comme ce diviseur est non-trivial, $\divi(f)$ l'est aussi d'après le lemme précédent. Ainsi $f$ est non-nul et non-inversible. Si $f$ divise un produit $f_1f_2$ d'éléments homogènes non-nuls de $\Rr(X)$ alors d'après $1$, il divise l'un des deux car $\divi_{[D]}(f)$ étant premier, il est nécessairement inférieur ou égal à l'un des deux diviseurs effectifs  $\divi_{[D]}(f_i)$. Ainsi $f$ est $\clg(X)$-premier. Réciproquement, supposons par l'absurde que $f$ est $\clg(X)$-premier et $\divi_{[D]}(f)$ non-premier. Alors $\divi_{[D]}(f)=D_1+D_2$ avec chaque $D_i$ non-nul et effectif. D'après \ref{diviCoxTorsFormulas} $1$), on obtient des $f_i\Rr_{[D_i]}(X)$ tels que $\divi_{[D_i]}(f_i)=D_i$. D'après $1$), $f$ divise $f_1f_2$ mais ne divise aucun des facteurs, contradiction.
\end{proof}

En particulier, la seconde assertion nous indique que l'élément $f\in \Rr_{[D]}(X)$ avec $E=\divi_{[D]}(f)$ obtenu en \ref{diviCoxTorsFormulas} 1) pour un diviseur effectif $E$ donné est unique à association près. On l'appel la section canonique de $E$. On termine par une généralisation du théorème \ref{coxFreeFactoriel}.

\begin{thm}
Avec les données de \ref{conscoxtorsion}, L'anneau de Cox $\Rr(X)$ est $\clg(X)$-factoriel.
\end{thm}
\begin{proof}
Soit $f\in\Rr_{[D]}(X)$ non-nul et non-inversible. On écrit $\divi_{[D]}(f)=D_1+...+D_r$ où les diviseurs $D_i$ sont premiers et effectifs. D'après \ref{diviCoxTorsFormulas} on obtient des éléments $f_i\in\Rr_{[D_i]}(X)$ tels que $\divi_{[D_i]}(f_i)=D_i$. Or d'après la proposition précédente, chaque $f_i$ est $\clg(X)$-premier et on a $f=uf_1...f_r$ où $u\in\Rr(X)^*$.
\end{proof}

\begin{rem}
Avec les données de \ref{conscoxtorsion}, l'application $f\mapsto \divi_{[D]}(f)$ induit un isomorphisme entre le monoïde des éléments homogènes de $\Rr(X)$ modulo les inversibles et le monoïde $\wdiv^+(X)$ des diviseurs de Weil effectifs. Le fait que $\Rr(X)$ soit factoriel reflète le fait tout diviseur de Weil effectif s'écrit de manière unique comme une combinaison linéaire à coefficients dans $\NN$ de diviseur premiers.
\end{rem}

\begin{ex}
Dans l'exemple \ref{CoxRingEx3}, l'anneau de Cox est $\ZZ/2\ZZ$-factoriel mais pas factoriel.
\end{ex}

