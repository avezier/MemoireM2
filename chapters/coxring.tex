\chapter{Anneaux de Cox}

\section{Un exemple introductif}

\subsection{Le spectre relatif d'un faisceau d'algèbres quasi-cohérentes}

\begin{cons}\label{relspec}[Spectre relatif]

\end{cons}

\begin{ex}\label{relspec}
Si on se donne un faisceau inversible $\mathcal{L}$ sur un $k$-schéma $X$, le fibré en droites qu'on lui associe dans la discussion \ref{linebundle} n'est autre que Spec$_X($Sym$(\mathcal{L}))$, où Sym$(\mathcal{L}))$ est l'algèbre symétrique associée à $\mathcal{L}$ sur $\mathcal{O}_X$. En effet, on recolle les Spec$_{U_i}($Sym$(\mathcal{O}_{U_i}))\simeq $ Spec$_{U_i}( \mathcal{O}_{U_i}[t])\simeq U_i \times_k \AAA^1_k$, où $U_i$ est un recouvrement qui trivialise $\mathcal{L}$.
\end{ex}

\subsection{Exemple d'une variété projective}

Considérons l'espace projectif $\PP^n_k=$Proj$[x_0,...,x_n]$ et le diviseur premier $D$ correspondant à l'hyperplan $x_0=0$. Le fibré en droites $L=$ Spec$_{\PP^n_k}($Sym$(\mathcal{O}_{\PP^n_k}(D)))\simeq \oplus_{k\geq 0}\mathcal{O}_{\PP^n_k}(kD)$ est isomorphe à l'éclatement en l'origine de $\AAA^{n+1}$.

\section{Groupe des classes sans torsion}


\section{Groupe des classes avec torsion}
