\chapter{Anneaux de Cox}

\section{Un exemple introductif}

Soit $X=\proj(B/I)$ une variété projective et irréductible, où $B=k[x_0,...,x_n]$ et $I\subset B$ un idéal homogène. A la différence du cas affine, l'algèbre graduée $B/I$ des coordonnées homogènes de $X$ n'est pas un invariant. Par exemple, $\PP^1_k=\proj(k[x_0,x_1])$ est isomorphe à tous ses plongements de Veronese alors que les algèbres de coordonnées homogènes correspondantes sont strictement contenues dans $k[x_0,x_1]$.\\ 
On considère maintenant le cas où $X$ est donnée par une immersion fermée $X\xhookrightarrow{i}\PP^n_k $ avec $i(X)=\proj(B/I)$. On note d'après \ref{morphismeproj} que cela revient à se donner un diviseur très ample $D$ tel que $\Oo_X(D)\simeq i^*\Oo_{\PP^n_k}(1)$. Considérons l'application canonique $\alpha: \oplus_{l\in\ZZ}\Oo_{\PP^n_k}(l)\rightarrow i_*i^*\oplus_{l\in \ZZ}\Oo_{\PP^n_k}(l)=i_*\oplus_{l\in \ZZ}\Oo_X(l)$. Sur les sections globales cela donne $\alpha(\PP^n_k): k[x_0,...,x_n]\rightarrow \Gamma_*(\Oo_X(D))=S(X)$, ou $S$ est le faisceau d'algèbres divisorielles associée $\ZZ D$. Au degré zéro, $\alpha_0=i^\sharp$, on en déduit $\ker \alpha=\Gamma_*(\Ii _{i(X)})=I$ d'après \ref{}, puis $\Im \alpha \simeq B/I$. On a ainsi retrouvé l'algèbre des coordonnées du plongement à partir d'un $\Oo_X$-module inversible, c'est à dire un élément de son groupe de Picard, qui est un objet intrinsèquement défini sur $X$.\\
D'après \cite{Hartshorne} ex II.5.14, $S(X)$ est la clôture intègrale de $B/I$. Ainsi, $B/I$ est normal si et seulement si $\alpha(\PP^n_k)$ est surjective, on dit alors que le plongement est projectivement normal. Cette remarque montre en particulier que $S(X)$ est une $k$-algèbre de type fini. On redémontre maintenant ce fait en donnant un éclairage géométrique sur ces constructions. Le fibré en droites $L=$ Spec$_{\PP^n_k}($Sym$(\Oo_{\PP^n_k}(1))) = \oplus_{l\geq 0}\Oo_{\PP^n_k}(l)$ est l'éclatement en l'origine de $\AAA^{n+1}$. En effet, on montre facilement que le $\Oo_{\PP^n_k}$-module des sections de l'éclatement en l'origine de $\AAA^{n+1}$ vu comme fibré en droites sur $\PP_k^n$ est $ \Oo_{\PP^n_k}(-1)$, ce qui permet de conclure d'après la discussion \ref{linebundle}. On a le diagramme commutatif suivant, où $L_X$ est le fibré en droites Spec$_X($Sym$(\Oo_X(D))$, $j$ est une immersion fermée (comme recollement d'immersions fermées) et $\pi$ est propre.

	\begin{center}
	\begin{tikzcd}
  		L_X \arrow[r, hook, "j"] \arrow[d, "p_X"]& L \arrow[d, "p"] \arrow[r, "\pi"]& \AAA^{n+1}_k\\ 
  		X \arrow[r, hook, "i"] & \PP^n_k &
	\end{tikzcd}\\
	\end{center}


On considère comme en \ref{linebundle} l'action de $\GG_m$ sur les fibres de $L$, $L_0$ l'ensemble des points fixes, et $L^\times$ son complémentaire. $L^\times$ est constitué de recollements de schémas isomorphes à Spec$_{U_i}( \Oo_{U_i}[t,t^{-1}])\simeq U_i \times_k \GG_m$, ce qui donne $L^\times=$ Spec$_{\PP^n_k}(\oplus_{l\in \ZZ}\Oo_{\PP^n_k}(lD))$. De plus, $L^\times$ est isomorphe à $\AAA^{n+1}\setminus\lbrace 0 \rbrace$ par restriction de $\pi$. D'autre part, $p$ se restreint en une application $p^\times:L^\times \rightarrow \PP^n_k$ qui est le quotient géométrique de l'action de $\GG_m$. On résume cela dans le diagramme ci-dessous:

	\begin{center}
	\begin{tikzcd}
  		\widetilde{X}:=L_X^\times \arrow[r, hook, "j"] \arrow[d, "p_X^\times"]& L^\times \arrow[d, "p^\times"] \arrow[r, "\simeq"]& \AAA^{n+1}_k\setminus \lbrace 0 \rbrace\\ 
  		X \arrow[r, hook, "i"] & \PP^n_k &
	\end{tikzcd}\\
	\end{center}

Notons encore $\pi$ la restriction $\pi j$, c'est une application propre. Ainsi, $\pi_*\Oo_{L_X}$ est un $\Oo_{\AAA_k^{n+1}}$-module cohérent. On en déduit que $\Oo(L_X)=\pi_*\Oo_{L_X}(\AAA_k^{n+1})$ est un $k[x_0,...,x_n]$-module de type fini, c'est en particulier une $k$-algèbre de type fini. De plus comme les $\Oo_X(lD)$ n'ont pas de sections globales non nulles pour $l<0$, on en déduit que $\Oo(\widetilde{X})=\Gamma(X,\oplus_{l\in \ZZ}\Oo_X(lD))=\Oo(L_X)$ est de type fini sur $k$. C'est de plus une algèbre $\NN$-graduée, son spectre $\bar{X}$ est donc muni d'une action de $\GG_m$ avec un unique point fixe $p_0$ appartenant à l'adhérence de tout orbite. Enfin, pour tout $f\in \Gamma(X,\Oo_X(lD))$ avec $f$ non nulle et $l>0$, on a  $\Oo_{\widetilde{X}}(p_X^{\times-1}(X_f))=\Gamma(X_f,\oplus_{k\in \ZZ}\Oo_X(lD))=\Gamma(X,\oplus_{k\in \ZZ}\Oo_X(lD))_f=\Oo(L_X)_f$. L'unique maximal contenant chaque $f$ correspond à $p_0$, on voit ainsi que ces spectres se recollent en $\widetilde{X}=\bar{X}\setminus \lbrace p_0 \rbrace$.\\

Prenons maintenant un exemple concret, soit $X\subset \PP^3_k$ la quadrique projective d'équation en coordonnées homogènes $x_1x_2=x_3x_4$, et $D$ le diviseur des zéros de la section $x_4\in\Oo_X(1)$. $D$ est un diviseur de Cartier effectif sur $X$ que l'on peut décrire sur les ouverts standards par $(U_i, x_4/x_i)_{1\leq i\leq 4}$. L'algèbre des coordonnées homogènes de $X$ est normale, donc en reprenant les notations précédentes on obtient que $\bar{X}$ est le cone affine d'équation $t_1t_2-t_3t_4$ dans $\AAA^4$ et $\widetilde{X}=\bar{X}\setminus\lbrace 0 \rbrace$ le cône épointé. On va utiliser cette construction pour calculer le groupe des classes Cl$(\bar{X})=$ Cl$(\widetilde{X})$. En tant que fibré en droite sur $X$ on voit en utilisant \cite{Hartshorne} II.6.6 que Cl$(X)\simeq$ Cl$(L_X)$, et le pullback de $D$ par cet isomorphisme est exactement $\pi^{-1}(0)$. Puis on a $\widetilde{X}=L_X \setminus \pi^{-1}(0)$ où le diviseur exceptionnel $\pi^{-1}(0)$ est isomorphe à $X$. D'après ce qui précède et comme $\pi^{-1}(0)$ est irréductible et de codimension $1$ dans $\widetilde{X}$, on a d'après \ref{divexactseq} une suite exacte:
\begin{center}
$ 0\rightarrow\ZZ\xrightarrow{\phi}$ Cl$(X) \rightarrow$ Cl$(\widetilde{X}) \rightarrow 0$, $\phi(1)=D$
\end{center}
Par ailleurs dans \cite{Hartshorne} II.6.6.1 on calcule Cl$(X)\simeq \ZZ^2$ avec $D=(1,1)$. On en déduit Cl$(\widetilde{X})\simeq \ZZ$ avec deux générateurs de somme nulle, $D_1=\mathcal{V}(p_1)$, avec $p_1=(t_1,t_4)$ et $D_2=\mathcal{V}(p_2)$ avec $p_2=(t_2,t_4)$.\\ 
Calculons maintenant les sections globales du faisceau d'algèbres divisorielles $S$ sur $Y:=\widetilde{X}$ associé à $K=\ZZ D_2$. On a $\Gamma(Y, \Oo_Y(-D_2))=p_2$ d'après \ref{divaff}. Ensuite, le cône épointé $Y^*:=Y\setminus \lbrace 0 \rbrace$ est lisse donc $D_2$ est Cartier dessus. Par ailleurs l'origine est de codimension 3, on peut donc voir $\Oo_Y(-D_2)$ comme un faisceau inversible sur $Y^*$, avec son inverse $\Oo_Y(D_2)$ étant la restriction à $Y^*$ du $\Oo_Y$-module $(t_4^{-1}p_1)^{\widetilde{}}$. En effet, on vérifie facilement en regardant les valuations dans les deux anneaux locaux que $D_1+D_2=$ div$(t_4)$, d'où $p_1p_2=(t_4)$.  On a donc $\Gamma(Y, \Oo_Y(D_2))=t_4^{-1}p_1=(1, t_4^{-1}t_1)$. Puis d'après \ref{hartshorne II.6.13}, $\Gamma(Y,S)$ est engendré en tant que $\Oo(Y)$-algèbre par les éléments algébriquement libres $(z_1,z_2,z_3,z_4):=(1, t_4^{-1}t_1,t_2,t_4)$. De plus d'après les équations suivantes la composante homogène de degré zéro est inclus dans $k[z_1,z_2,z_3,z_4]$:
$$ z_2z_4=t_1,\, z_1z_3=t_2,\, z_2z_3=t_3,\, z_1z_4=t_4$$

On a donc $\Gamma(X, S)=k[z_1,z_2,z_3,z_4]$, c'est une algèbre de polynômes naturellement graduée par deg$(z_1)$ = deg$(z_2)=1$, deg($z_3$) $=$ deg($z_4$)$=-1$. Cette $\ZZ$-graduation se traduit en une action de $\GG_m$ sur $Y$ donnée  concrètement par $\lambda.z=(\lambda z_1,\lambda z_2,\lambda^{-1} z_3,\lambda^{-1} z_4)$. L'algèbre des invariants sous cette action est $k[z_1z_2,z_1,z_4,z_2z_3,z_2z_4]$ ce qui donne un isomorphisme $\spec(\Gamma(Y,S)^{\GG_m})\simeq Y$. 
Enfin on remarque comme attendu que $\Gamma(X, S)$ est factoriel et $\Oo(Y)^{\times}=k^*$ comme attendu (\ref{}). En effet, en inversant par exemple $t_1$  que $X_{t_1}\simeq \AAA^{1}\setminus \lbrace0\rbrace\times\AAA^{2}$. On en déduit que $\Oo(Y)^{\times}=k^*$.\\

On voudrait associer à $X$ un objet construit de manière intrinsèque (i.e. indépendamment de son plongement dans un espace projectif) et qui caractérise $X$ à isomorphisme près.
 

\section{Groupe des classes sans torsion}

Dans cette partie on définit l'anneau de Cox d'une variété normale irréductible avec un groupe des classes libre de type fini. On est donc en particulier dans le cadre de l'exemple précédent.

\subsection{Définition de l'anneau de Cox}

\begin{cons}[Faisceau de Cox, Anneau de Cox]
Soit $X$ une variété normale irréductible avec un groupe des classes libre de type fini. On fixe un sous-groupe $K\subset $ Cl$(X)$ se projetant isomorphiquement sur Cl$(X)$, ce qui est possible par la théorie des $\ZZ$-modules de type fini. On définit le faisceau de Cox sur $X$, noté $\Rr$ comme le faisceau d'algèbres divisorielles associé à $K$. Cette description ne dépend qu'à isomorphisme près du choix de $K$. L'anneau de Cox de $X$ est l'algèbre des sections globales du faisceau de Cox. 
\end{cons}
\begin{proof}
todo
\end{proof}



\subsection{Le spectre relatif de $\mathcal{R}$}

On souhaiterait réaliser géométriquement le faisceau de Cox. Une idée naturelle est de prendre le spectre relatif $(\widetilde{X}, p)$ de ce faisceau d'algèbres quasi-cohérent. Toutefois, ce spectre relatif ne définira pas une variété en général. Il faudrait pouvoir recouvrir $X$ par un nombre fini d'ouverts affines $U_i$ tels que $\mathcal{R}(U_i)$ soit de type fini réduit, on dit alors que $\Rr$ est localement de type fini. Sous certaines conditions, on pourra s'en assurer. Par exemple si $X$ est lisse, tous les diviseurs sont Cartier. Dans ce cas notons $D_1,...,D_s$ une base de $K$ et $U$ un ouvert sur lequel chaque $D_i$ est principal. On a localement un isomorphisme d'algèbres graduées:

$$\Oo_X(U)\otimes_k k[t_1^{\pm},...,t_s^{\pm}] \rightarrow \mathcal{R}(U),\, g\otimes t^{\nu_1}...t_s^{\nu_s}\mapsto gf_1^{-\nu_1}...f_s^{-\nu_s}$$

Par recollement on obtient que $\widetilde{X}$ est le produit $L_1^\times\times...\times L_s^\times$ où, avec les notation de l'exemple d'introduction, $L_i$ est le fibré en droite correspondant à $\Oo_X(D_i)$.\\
Par ailleurs, notons que dans le cas où $\Rr$ est localement de type fini, son spectre relatif est naturellement muni d'une action de $H:=\spec k[K]$ pour laquelle $(X,p)$ est un bon quotient.  En effet, $p$ est affine par construction et sur chaque $U_i$, $\Rr(U_i)$ est $K$-graduée avec pour éléments homogènes de degré zéro $\Rr(U_i)_0=\Oo_X(U_i)$. Ces quotients locaux coïncident aux intersections et se recollent globalement en $p$. On remarque que dans le cas où $X$ est lisse l'isomorphisme ci-dessus nous dit que localement on a un diagramme:
\begin{center}
	\begin{tikzcd}
		p^{-1}(U) \arrow[r,"\simeq"] \arrow[d,"p"] & H\times U \arrow[dl, "pr_U"] \\
		U
	\end{tikzcd}
\end{center}

Les flèches sont $H$-équivariantes et $H$ agit sur le produit par multiplication sur le premier facteur.

\begin{prop}
Soit $X$ une variété normale irréductible avec un groupe des classes libre de type fini, et $\Rr$ localement de type fini. Alors $\widetilde{X}$ est une variété irréductible et normale. De plus, pour tout fermé $A\subset X$ de codimension $\geq 2$, $p^{-1}(A)$ est aussi de codimension $\geq 2$.
\end{prop}
\begin{proof}
Tout d'abord, $\widetilde{X}$ est séparé comme spectre relatif sur une variété.\\
On recouvre l'ouvert des points réguliers $X_{reg}$ par des un nombre fini de comme dans le diagramme ci-dessus $U_i$, ce qui est possible car $U$ est quasi-compact. Ainsi les $p^{-1}(U_i)$ sont irréductibles, et leur réunion $p^{-1}(X_{reg})$ également car leur intersection est non-vide. De plus, $p^{-1}(X_{reg})$ est lisse car c'est vrai localement par le diagramme.\\
On recouvre maintenant $X_{reg}$ par des ouverts affines $V_1,...,V_s$ et on a d'après \ref{codimesingnormal} et \ref{Cox 4.0.15}, $S(V_i\setminus X_{reg})=S(V_i)=\Oo_{\widetilde{X}}(p^{-1}(V_i))$.\\
Pour la dernière assertion, c'est une conséquence directe du fait que $p_*\Oo_{\widetilde{X}}=S$ et de \ref{}.
\end{proof}

Remarquons qu'une section de $s\in S(U)$ homogène de degré $D\in K$ sur un ouvert $U\subset X$ peut être vue à la fois comme une fonction rationnelle sur $X$ vérifiant $\divi(s)+D\geq 0$ et comme une fonction régulière sur $\widetilde{X}$ qui est homogène de degré $D$ pour l'action de $H$. De plus, si $D$ est Cartier, $\divi(s)+D$ est le diviseur des zéros de $s$ sur $X$. Dans tous les cas on adopte la notation $\divi_D(s)$ pour ce diviseur. On explore maintenant les relation entre ces points de vue. Notons tout d'abord que comme $p$ est un morphisme dominant de variétés irréductibles, on peut définir le pullback $p^*(D)$ d'un diviseur de Cartier simplement par le pullback de ses équations locales. Pour un diviseur de Weil $D$, on considère sa restriction $D'$ à $X_{reg}$ et on définit $p^*(D)=\overline{p^*(D')}$, $p^*(D')$ étant alors un diviseur sur $p^{-1}(X_{reg})$. Le pullback envoie les diviseurs principaux sur des diviseurs principaux et on obtient un morphisme de groupes $\clg(X)\rightarrow \clg(\widetilde{X})$.

\begin{prop}
Soit $X$ une variété normale irréductible avec un groupe des classes libre de type fini, et $\Rr$ localement de type fini. Pour tout $D\in K$ et $s\in\Rr_D(X)$, on a $\divi(s)=p^*(\divi(s)+D)$. Si de plus $X_s$ est affine, on a $\Supp(\divi(f))=p^{-1}(\Supp(\divi_D(f))$.
\end{prop}
\begin{proof}
Comme $\widetilde{X}\setminus p^{-1}(X_{reg})$ est de codimension $\geq 2$ on peut supposer pour ce problème $X$ et donc $\widetilde{X}$ lisse. On écrit $D=(U_i, f_i)$, on a ainsi $s_i:=s_{|U_i}=\alpha_i f_i^{-1}$ avec $\alpha_i\in \Oo_X(U_i)$, et localement on a $p^*(\divi_D(s)_{|U_i})=p^*(\divi(\alpha_i))=\divi(p^\sharp(\alpha_i))=\divi(\alpha_i)=\divi(\alpha_i f_i^{-1})=\divi(f^{-1}_{|U_i})$, l'avant dernière égalité étant due au fait que $f_i$ est inversible sur $p^{-1}(U_i)$, en effet, sur $U_i$, on a $D=\divi(f_i)$ donc $f_i\in \Rr_{-D}(U_i)\subset\Oo_{\widetilde{X}(p^{-1}(U_i))}$ et $f_i^{-1}\in \Rr_D(U_i)\subset\Oo_{\widetilde{X}(p^{-1}(U_i))}$.\\
La deuxième assertion, il faut montrer $p^{-1}(X{D,f})=\widetilde{X}_f$. On remarque $f^{-1}\in \Rr_{-D}{X_{D,f}}$ donc $f$ est inversible sur $p^{-1}(X_{D,f})$, ce qui montre $p^{-1}(X_{D,f})\subset \widetilde{X}_f$. Par ailleurs, $\divi_D(f)$ est Cartier sur $X_{reg}$ et son pullback est le pullback de ses équations locales, on obtient donc $p^{-1}(X_{D,f})\cap p^{-1}(X_{reg})=\widetilde{X}_f \cap p^{-1}(X_{reg})$. Ainsi $\widetilde{X}_f \setminus p^{-1}(X_{D,f})$ est le complémentaire d'un ouvert affine de codimension $\geq 2$, donc est vide d'après \ref{}.
\end{proof}


\section{Groupe des classes avec torsion}
