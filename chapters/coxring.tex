\chapter{Anneaux de Cox}

\section{Un exemple introductif}

Soit $X=\proj(B/I)$ une variété projective et irréductible, où $B=k[x_0,...,x_n]$ et $I\subset B$ un idéal homogène. A la différence du cas affine, l'algèbre graduée $B/I$ des coordonnées homogènes de $X$ n'est pas un invariant. Par exemple, $\PP^1_k=\proj(k[x_0,x_1])$ est isomorphe à tous ses plongements de Veronese alors que les algèbres de coordonnées homogènes correspondantes sont strictement contenues dans $k[x_0,x_1]$.\\ 
On considère maintenant le cas où $X$ est donnée par une immersion fermée $X\xhookrightarrow{i}\PP^n_k $ avec $i(X)=\proj(B/I)$. On note d'après \ref{morphismeproj} que cela revient à se donner un diviseur très ample $D$ tel que $\Oo_X(D)\simeq i^*\Oo_{\PP^n_k}(1)$. Considérons l'application canonique $\alpha: \oplus_{l\in\ZZ}\Oo_{\PP^n_k}(l)\rightarrow i_*i^*\oplus_{l\in \ZZ}\Oo_{\PP^n_k}(l)=i_*\oplus_{l\in \ZZ}\Oo_X(l)$. Sur les sections globales cela donne $\alpha(\PP^n_k): k[x_0,...,x_n]\rightarrow \Gamma_*(\Oo_X(D))=S(X)$, ou $S$ est le faisceau d'algèbres divisorielles associée $\ZZ D$. Au degré zéro, $\alpha_0=i^\sharp$, on en déduit $\ker \alpha=\Gamma_*(\Ii _{i(X)})=I$ d'après \ref{}, puis $\Im \alpha \simeq B/I$. On a ainsi retrouvé l'algèbre des coordonnées du plongement à partir d'un $\Oo_X$-module inversible, c'est à dire un élément de son groupe de Picard, qui est un objet intrinsèquement défini sur $X$.\\
D'après \cite{Hartshorne} ex II.5.14, $S(X)$ est la clôture intègrale de $B/I$. Ainsi, $B/I$ est normal si et seulement si $\alpha(\PP^n_k)$ est surjective, on dit alors que le plongement est projectivement normal. Cette remarque montre en particulier que $S(X)$ est une $k$-algèbre de type fini. On redémontre maintenant ce fait en donnant un éclairage géométrique sur ces constructions. Le fibré en droites $L=$ Spec$_{\PP^n_k}($Sym$(\Oo_{\PP^n_k}(1))) = \oplus_{l\geq 0}\Oo_{\PP^n_k}(l)$ est l'éclatement en l'origine de $\AAA^{n+1}$. En effet, on montre facilement que le $\Oo_{\PP^n_k}$-module des sections de l'éclatement en l'origine de $\AAA^{n+1}$ vu comme fibré en droites sur $\PP_k^n$ est $ \Oo_{\PP^n_k}(-1)$, ce qui permet de conclure d'après la discussion \ref{linebundle}. On a le diagramme commutatif suivant, où $L_X$ est le fibré en droites Spec$_X($Sym$(\Oo_X(D))$, $j$ est une immersion fermée (comme recollement d'immersions fermées) et $\pi$ est propre.

	\begin{center}
	\begin{tikzcd}
  		L_X \arrow[r, hook, "j"] \arrow[d, "p_X"]& L \arrow[d, "p"] \arrow[r, "\pi"]& \AAA^{n+1}_k\\ 
  		X \arrow[r, hook, "i"] & \PP^n_k &
	\end{tikzcd}\\
	\end{center}


On considère comme en \ref{linebundle} l'action de $\GG_m$ sur les fibres de $L$, $L_0$ l'ensemble des points fixes, et $L^\times$ son complémentaire. $L^\times$ est constitué de recollements de schémas isomorphes à Spec$_{U_i}( \Oo_{U_i}[t,t^{-1}])\simeq U_i \times_k \GG_m$, ce qui donne $L^\times=$ Spec$_{\PP^n_k}(\oplus_{l\in \ZZ}\Oo_{\PP^n_k}(lD))$. De plus, $L^\times$ est isomorphe à $\AAA^{n+1}\setminus\lbrace 0 \rbrace$ par restriction de $\pi$. D'autre part, $p$ se restreint en une application $p^\times:L^\times \rightarrow \PP^n_k$ qui est le quotient géométrique de l'action de $\GG_m$. On résume cela dans le diagramme ci-dessous:

	\begin{center}
	\begin{tikzcd}
  		\widetilde{X}:=L_X^\times \arrow[r, hook, "j"] \arrow[d, "p_X^\times"]& L^\times \arrow[d, "p^\times"] \arrow[r, "\simeq"]& \AAA^{n+1}_k\setminus \lbrace 0 \rbrace\\ 
  		X \arrow[r, hook, "i"] & \PP^n_k &
	\end{tikzcd}\\
	\end{center}

Notons encore $\pi$ la restriction $\pi j$, c'est une application propre. Ainsi, $\pi_*\Oo_{L_X}$ est un $\Oo_{\AAA_k^{n+1}}$-module cohérent. On en déduit que $\Oo(L_X)=\pi_*\Oo_{L_X}(\AAA_k^{n+1})$ est un $k[x_0,...,x_n]$-module de type fini, c'est en particulier une $k$-algèbre de type fini. De plus comme les $\Oo_X(lD)$ n'ont pas de sections globales non nulles pour $l<0$, on en déduit que $\Oo(\widetilde{X})=\Gamma(X,\oplus_{l\in \ZZ}\Oo_X(lD))=\Oo(L_X)$ est de type fini sur $k$. C'est de plus une algèbre $\NN$-graduée, son spectre $\bar{X}$ est donc muni d'une action de $\GG_m$ avec un unique point fixe $p_0$ appartenant à l'adhérence de tout orbite. Enfin, pour tout $f\in \Gamma(X,\Oo_X(lD))$ avec $f$ non nulle et $l>0$, on a  $\Oo_{\widetilde{X}}(p_X^{\times-1}(X_f))=\Gamma(X_f,\oplus_{k\in \ZZ}\Oo_X(lD))=\Gamma(X,\oplus_{k\in \ZZ}\Oo_X(lD))_f=\Oo(L_X)_f$. L'unique maximal contenant chaque $f$ correspond à $p_0$, on voit ainsi que ces spectres se recollent en $\widetilde{X}=\bar{X}\setminus \lbrace p_0 \rbrace$.\\

Prenons maintenant un exemple concret, soit $X\subset \PP^3_k$ la quadrique projective d'équation en coordonnées homogènes $x_1x_2=x_3x_4$, et $D$ le diviseur des zéros de la section $x_4\in\Oo_X(1)$. $D$ est un diviseur de Cartier effectif sur $X$ que l'on peut décrire sur les ouverts standards par $(U_i, x_4/x_i)_{1\leq i\leq 4}$. L'algèbre des coordonnées homogènes de $X$ est normale, donc en reprenant les notations précédentes on obtient que $\bar{X}$ est le cone affine d'équation $t_1t_2-t_3t_4$ dans $\AAA^4$ et $\widetilde{X}=\bar{X}\setminus\lbrace 0 \rbrace$ le cône épointé. On va utiliser cette construction pour calculer le groupe des classes Cl$(\bar{X})=$ Cl$(\widetilde{X})$. En tant que fibré en droite sur $X$ on voit en utilisant \cite{Hartshorne} II.6.6 que Cl$(X)\simeq$ Cl$(L_X)$, et le pullback de $D$ par cet isomorphisme est exactement $\pi^{-1}(0)$. Puis on a $\widetilde{X}=L_X \setminus \pi^{-1}(0)$ où le diviseur exceptionnel $\pi^{-1}(0)$ est isomorphe à $X$. D'après ce qui précède et comme $\pi^{-1}(0)$ est irréductible et de codimension $1$ dans $\widetilde{X}$, on a d'après \ref{divexactseq} une suite exacte:
\begin{center}
$ 0\rightarrow\ZZ\xrightarrow{\phi}$ Cl$(X) \rightarrow$ Cl$(\widetilde{X}) \rightarrow 0$, $\phi(1)=D$
\end{center}
Par ailleurs dans \cite{Hartshorne} II.6.6.1 on calcule Cl$(X)\simeq \ZZ^2$ avec $D=(1,1)$. On en déduit Cl$(\widetilde{X})\simeq \ZZ$ avec deux générateurs de somme nulle, $D_1=\mathcal{V}(p_1)$, avec $p_1=(t_1,t_4)$ et $D_2=\mathcal{V}(p_2)$ avec $p_2=(t_2,t_4)$.\\ 
Calculons maintenant les sections globales du faisceau d'algèbres divisorielles $S$ sur $Y:=\widetilde{X}$ associé à $K=\ZZ D_2$. On a $\Gamma(Y, \Oo_Y(-D_2))=\Gamma(\bar{X}, \Oo_{\bar{X}}(-D_2))=p_2$ d'après \ref{divaff}. Puis comme comme le cône épointé $Y$ est lisse, $D_2$ est Cartier dessus. Ainsi $\Oo_Y(-D_2)$ est un faisceau inversible sur $Y$ d'inverse $\Oo_Y(D_2)$, ce dernier étant la restriction à $Y$ du $\Oo_{\bar{X}}$-module $(t_4^{-1}p_1)^{\widetilde{}}$. En effet, on a $p_1p_2=(t_4)$ car ce sont deux idéaux radicaux définissant le même fermé.  On a donc $\Gamma(Y, \Oo_Y(D_2))=t_4^{-1}p_1=(1, t_4^{-1}t_1)$. Ainsi $\Gamma(Y,S)$ est engendré en tant que $\Oo(Y)$-algèbre par les éléments algébriquement libres $(z_1,z_2,z_3,z_4):=(1, t_4^{-1}t_1,t_2,t_4)$. De plus d'après les équations suivantes la composante homogène de degré zéro est inclus dans $k[z_1,z_2,z_3,z_4]$:
$$ z_2z_4=t_1,\, z_1z_3=t_2,\, z_2z_3=t_3,\, z_1z_4=t_4$$

On a donc $\Gamma(X, S)=k[z_1,z_2,z_3,z_4]$, c'est une algèbre de polynômes naturellement graduée par deg$(z_1)$ = deg$(z_2)=1$, deg($z_3$) $=$ deg($z_4$)$=-1$. Cette $\ZZ$-graduation se traduit en une action de $\GG_m$ sur $Y$ donnée  concrètement par $\lambda.z=(\lambda z_1,\lambda z_2,\lambda^{-1} z_3,\lambda^{-1} z_4)$. L'algèbre des invariants sous cette action est $k[z_1z_2,z_1z_4,z_2z_3,z_2z_4]$ ce qui donne un isomorphisme $\spec(\Gamma(Y,S)^{\GG_m})\simeq Y$.\\ 
On vient ainsi de voir dans cet exemple qu'un faisceau d'algèbres divisorielles $S$ bien choisi permet de retrouver $X$ comme bon quotient d'une $H$-variété construite à partir de ce faisceau (où $H$ définit la graduation de $S$). C'est ce qu'on va étudier de manière générale dans cette partie.

\section{Cas d'un groupe des classes sans torsion}

Dans cette partie on définit l'anneau de Cox d'une variété $X$ normale irréductible avec un groupe des classes libre de type fini. On est en particulier dans le cadre de l'exemple précédent.

\subsection{Faisceau et anneau de Cox}

\begin{cons}[Faisceau de Cox, Anneau de Cox]
Soit $X$ normale irréductible avec un groupe des classes libre de type fini et un sous-groupe $K\subset \clg(X)$ se projetant isomorphiquement sur Cl$(X)$. Il existe de tels $K$ car $\clg(X)$ est libre de type fini. On définit le faisceau de Cox sur $X$, noté $\Rr$ comme le faisceau d'algèbres divisorielles associé à $K$. Cette description ne dépend qu'à isomorphisme près du choix de $K$. L'anneau de Cox de $X$ est l'algèbre des sections globales du faisceau de Cox. 
\end{cons}
\begin{proof}
Soient $K,K'$ deux sous-groupes de $\wdiv(X)$ se projetant isomorphiquement sur $\clg(X)$, et $\Rr,\Rr'$ les faisceaux de Cox correspondants. On choisit une base $(D_1,...,D_s)$ de $K$. Cette base définit une section de la projection $c:\wdiv(X)\rightarrow \clg(X)$ et on peut modifier cette section par des éléments du noyau, cela fournissant autant de bases de sous-groupes de $\wdiv(X)$ se projetant isomorphiquement sur $\clg(X)$. Par ailleurs, chaque $D_i+\Ker c$ rencontre nécessairement $K'$ sinon on aurait $\rg K' < \rg \clg(X)$ comme rang de modules libres de type fini. On choisit ainsi $f_1,...,f_s\in k(X)$ tels que $(D_i-\divi(f_i))_i$ forme une base de $K'$. On définit un morphisme $\alpha:K\rightarrow k(X)^*, a_1D_1+...+a_sD_s\mapsto f_1^{a_1}...f_s^{a_s}$. Avec cela, l'isomorphisme linéaire faisant correspondre les bases de $K$ et $K'$ est $\widetilde{\psi}:K\rightarrow K',\, D\mapsto -\divi(\alpha(D))+D$. Enfin on définit un isomorphisme d'algèbres divisorielles $(\psi, \widetilde{\psi}): \Rr\rightarrow\Rr'$ en posant $f\in \Gamma(U, \Rr_D)\mapsto \alpha(D)f$.
\end{proof}



\subsection{Le spectre relatif de $\mathcal{R}$}

On souhaiterait réaliser géométriquement le faisceau de Cox d'une variété $X$ comme dans la partie précédente. Une idée naturelle est de prendre le spectre relatif $(\widetilde{X}, p)$ de ce faisceau d'algèbres quasi-cohérent. Toutefois, ce spectre relatif ne définira pas une variété en général. Il faudrait pouvoir recouvrir $X$ par un nombre fini d'ouverts affines $U_i$ tels que $\mathcal{R}(U_i)$ soit de type fini réduit, on dit alors que $\Rr$ est localement de type fini. Sous certaines conditions, on pourra s'en assurer. Par exemple si $X$ est lisse, tous les diviseurs sont Cartier. Dans ce cas notons $D_1,...,D_s$ une base de $K$ et $U$ un ouvert sur lequel chaque $D_i$ est principal. On a localement un isomorphisme d'algèbres graduées:

$$\Oo_X(U)\otimes_k k[t_1^{\pm},...,t_s^{\pm}] \rightarrow \mathcal{R}(U),\, g\otimes t_1^{\nu_1}...t_s^{\nu_s}\mapsto gf_1^{-\nu_1}...f_s^{-\nu_s}$$

Par recollement on obtient que $\widetilde{X}$ est le produit $L_1^\times\times...\times L_s^\times$ où, avec les notation de l'exemple d'introduction, $L_i$ est le fibré en droite correspondant à $\Oo_X(D_i)$.\\
Par ailleurs, notons que dans le cas où $\Rr$ est localement de type fini, son spectre relatif est naturellement muni d'une action de $H:=\spec k[K]$ pour laquelle $(X,p)$ est un bon quotient.  En effet, $p$ est affine par construction et sur chaque $U_i$, $\Rr(U_i)$ est $K$-graduée avec pour éléments homogènes de degré zéro $\Rr(U_i)_0=\Oo_X(U_i)$. Ces quotients locaux coïncident aux intersections et se recollent globalement en $p$. On remarque que dans le cas où $X$ est lisse, l'isomorphisme ci-dessus nous dit que localement on a un diagramme commutatif dans lequel les flèches sont $H$-équivariantes et $H$ agit sur le produit par multiplication sur le premier facteur:
\begin{center}
	\begin{tikzcd}
		p^{-1}(U) \arrow[r,"\simeq"] \arrow[d,"p"] & H\times U \arrow[dl, "pr_U"] \\
		U
	\end{tikzcd}
\end{center}

Dans la suite de cette partie on considère les conditions et données suivantes sur la variété $X$, qui nous permet de parler de son faisceau de Cox et la variété définie par son spectre relatif:

\begin{description}
\item [($\dagger$)] $X$ est normale irréductible avec un groupe des classes libre de type fini. On se fixe un sous-groupe $K\subset \clg(X)$ se projetant isomorphiquement sur Cl$(X)$. On suppose que le faisceau de Cox est localement de type fini.
\end{description}

\begin{prop}\label{preimagecodim2}
Soit $X$ vérifiant $(\dagger)$. Alors $\widetilde{X}$ est une variété irréductible et normale. De plus, pour tout fermé $A\subset X$ de codimension $\geq 2$, $p^{-1}(A)$ est aussi de codimension $\geq 2$.
\end{prop}
\begin{proof}
Tout d'abord, $\widetilde{X}$ est séparé comme spectre relatif sur une variété. Ensuite, on recouvre l'ouvert des points réguliers $X_{reg}$ par un nombre fini d'ouverts $U_i$ comme dans le diagramme ci-dessus , ce qui est possible car tout ouvert de $X$ est quasi-compact. Ainsi les $p^{-1}(U_i)$ sont irréductibles, et leur réunion $p^{-1}(X_{reg})$ également car leur intersection est non-vide (\ref{}). De plus, $p^{-1}(X_{reg})$ est lisse car c'est vrai localement par le diagramme.\\
On recouvre maintenant $X_{reg}$ par des ouverts affines $V_1,...,V_s$ et on a d'après \ref{codimesingnormal} et \ref{extregularnormal}, $S(V_i\setminus X_{reg})=S(V_i)=\Oo_{\widetilde{X}}(p^{-1}(V_i))$.\\
Pour la dernière assertion, c'est une conséquence directe du fait que $p_*\Oo_{\widetilde{X}}=\Rr$ et de \ref{cohomologie locale}.
\end{proof}

\begin{cor}
Soit $X$ vérifiant $(\dagger)$. Alors pour tout ouverts $V\subset U\subset X$ tels que $U\setminus V$ soit de codimension $\geq 2$ dans $U$, on a $\Rr_{|V}\simeq \Rr_{|U}$.
\end{cor}
\begin{proof}
On applique la propriété précédente et \ref{extregularnormal}
\end{proof}

Remarquons qu'une section de $s\in S(U)$ homogène de degré $D\in K$ sur un ouvert $U\subset X$ peut être vue à la fois comme une fonction rationnelle sur $X$ vérifiant $\divi(s)+D\geq 0$ et comme une fonction régulière sur $\widetilde{X}$ qui est homogène de degré $D$ pour l'action de $H$. De plus, si $D$ est Cartier, $\divi(s)+D$ est le diviseur des zéros de $s$ sur $X$. Dans tous les cas on adopte la notation $\divi_D(s)$ pour ce diviseur effectif et $X_{D,s}$ pour le complémentaire de son support. On explore maintenant les relation entre ces points de vue. Notons tout d'abord que comme $p$ est un morphisme dominant de variétés irréductibles, on peut définir le pullback $p^*(D)$ d'un diviseur de Cartier simplement par le pullback de ses équations locales. Pour un diviseur de Weil $D$, on considère sa restriction $D'$ à $X_{reg}$ et on définit $p^*(D)$ comme l'unique diviseur de Weil à correspondant à $p^*(D')$ via l'isomorphisme $\wdiv(\widetilde{X})\simeq\wdiv(p^{-1}(X_{reg}))$. Le pullback envoie les diviseurs principaux sur des diviseurs principaux et on obtient un morphisme de groupes $\clg(X)\rightarrow \clg(\widetilde{X})$.

\begin{prop}\label{pstarprincipal}
Soit $X$ vérifiant $(\dagger)$. Pour tout $D\in K$ et $s\in\Rr_D(X)$, on a $\divi(s)=p^*(\divi_D(s))$.\\ Si de plus $X_s$ est affine, on a $\supp(\divi(s))=p^{-1}(\supp(\divi_D(s))$.
\end{prop}
\begin{proof}
Comme $\widetilde{X}\setminus p^{-1}(X_{reg})$ est de codimension $\geq 2$ on peut supposer pour ce problème $X$ et donc $\widetilde{X}$ lisse. On écrit $D=(U_i, f_i)$, on a ainsi $s_i:=s_{|U_i}=\alpha_i f_i^{-1}$ avec $\alpha_i\in \Oo_X(U_i)$, et localement on a $p^*(\divi_D(s)_{|U_i})=p^*(\divi(\alpha_i))=\divi(p^\sharp(\alpha_i))=\divi(\alpha_i)=\divi(\alpha_i f_i^{-1})=\divi(f^{-1}_{|U_i})$, l'avant dernière égalité étant due au fait que $f_i$ est inversible sur $p^{-1}(U_i)$, en effet, sur $U_i$, on a $D=\divi(f_i)$ donc $f_i\in \Rr_{-D}(U_i)\subset\Oo_{\widetilde{X}(p^{-1}(U_i))}$ et $f_i^{-1}\in \Rr_D(U_i)\subset\Oo_{\widetilde{X}(p^{-1}(U_i))}$.\\
La deuxième assertion, il faut montrer $p^{-1}(X_{D,s})=\widetilde{X}_s$. On remarque $s^{-1}\in \Rr_{-D}{X_{D,s}}$ donc $s$ est inversible sur $p^{-1}(X_{D,s})$, ce qui montre $p^{-1}(X_{D,s})\subset \widetilde{X}_s$. Par ailleurs, $\divi_D(s)$ est Cartier sur $X_{reg}$ et son pullback est le pullback de ses équations locales, on obtient donc $p^{-1}(X_{D,s})\cap p^{-1}(X_{reg})=\widetilde{X}_s \cap p^{-1}(X_{reg})$. Ainsi $\widetilde{X}_s \setminus p^{-1}(X_{D,s})$ est le complémentaire d'un ouvert affine de codimension $\geq 2$, donc est vide d'après \ref{}.
\end{proof}

\begin{cor}\label{zerosspecrelatif}
Soit $X$ vérifiant $(\dagger)$. Soit $\tilde{x}\in\widetilde{X}$ tel que $H.\tilde{x}$ est fermé dans $\widetilde{X}$. Pour tout $D\in K, f\in \Rr_D(X)$ non-nulle, on a:
$$f(\tilde{x})=0 \iff p(\tilde{x})\in \supp(\divi_D(f))$$
\end{cor}
\begin{proof}
Remarquons que l'on a $\supp(p^*(D))\subset p^{-1}(\supp(D))$. Puis comme $p$ est surjective, on trouve $p(\supp(p^*(D)))\subset \supp(D)$. En effet, on peut supposer $D$ effectif et on a $p(\supp(p^*(D)))=p(\overline{\supp(p^*(D'))})\subset\overline{p(\supp(p^*(D')))}$, où on a noté $D'=D\cap X_{reg}$. Or $x\in \supp(p^*(D'))\iff x\in \Vv_{p^{-1}(U_i)}(p^\sharp(f_i))$ en écrivant $D'=(U_i,f_i)_i$. On en déduit $p(x)\in \Vv_{U_i}(f_i)$, d'où $p(x)\in\supp(D')$. Comme $\overline{\supp(D')}=\supp(D)$, on a bien le résultat annoncé. De plus, $p(\supp(p^*(D)))$ et $\supp(D)$ coïncident sur l'ouvert dense $X_{reg}$ et $p(\supp(p^*(D)))$ est fermé d'après \ref{goodquotientthm}. On a donc l'égalité:$$p(\supp(p^*(D)))=\supp(D)$$
Ainsi en appliquant la proposition précédente, on a $p(\supp(\divi(f)))= \supp(\divi_D(f))$ et donc $f(\tilde{x})=0 \implies p(\tilde{x})\in\supp \divi_D(f)$. Réciproquement, si $p(\tilde{x})\in\supp \divi_D(f)$, on a $p(\tilde{x})=p(\tilde{x}')$ pour un $\tilde{x}'\in \supp(\divi(f))$. Toujours d'après \ref{goodquotientthm} et en utilisant que $H.\tilde{x}$ est fermé on obtient $H.\tilde{x}\subset\overline{H.\tilde{x}'}$ ce qui prouve $f(\tilde{x})=0$ car $f$ est nulle sur $H.\tilde{x}'$, étant homogène et s'annulant en $\tilde{x}'$.
\end{proof}

On établit maintenant un résultat important sur le groupe des classes de $\widetilde{X}$ sous l'hypothèse ($\dagger$). On mentionne tout de suite une réciproque dans le cas où $X$ est lisse. Soit $\widetilde{X}$ le spectre relatif du faisceau d'algèbres divisorielles associé à un sous-groupe de type fini $K\subset\wdiv(X)$. Dans ce cas, si $\clg(\widetilde{X})=0$, alors la projection $K\rightarrow \clg(\widetilde{X})$ est surjective (voir \cite{coxrings} 1.3.3).



\begin{thm}\label{clgtrivial}
Soit $X$ vérifiant $(\dagger)$. On a $\clg(\widetilde{X})=0$.
\end{thm}
\begin{proof}
Comme $p^{-1}(X_{sing})$ est de codimension $\geq 2$ on peut supposer $X$ et donc $\widetilde{X}$ lisse, on a en effet un isomorphisme $\clg(\widetilde{X})\simeq\clg(\widetilde{X}\setminus p^{-1}(X_{sing}))$. Soit $\widetilde{D}\in \wdiv(\widetilde{X})$ un diviseur. Il est de Cartier par hypothèse et on veut montrer qu'il est principal. $\widetilde{X}$ est muni d'une action de $\spec k[K]$ par la $K$-graduation de $\Oo_{\widetilde{X}}$. On étend cette graduation au faisceau structural Sym($\Oo_{\widetilde{X}}(\widetilde{D})$) du fibré en droites $(L,\pi)$ associé à $\widetilde{D}$ en posant localement $\deg(t)=0$ sur les ouverts affines $U_i$ où $\pi^{-1}(U_i)\simeq \Oo_{U_i}[t]$. Cela munit $L$ d'une action de $H$ dont on voit qu'elle est $\pi$-équivariante et linéaire sur les fibres (localement c'est l'identité entre les fibres). On a ainsi muni $L$ d'une H-linéarisation et on obtient un $H$-module $\Gamma(\widetilde{X}, \Oo_{\widetilde{X}}(\widetilde{D}))$ (Cf \ref{}). On en déduit que pour tout $h\in H$, $f\in \Gamma(\widetilde{X}, \Oo_{\widetilde{X}}(\widetilde{D}))$, on a $\divi_{\widetilde{D}}(h.f)=h.\divi_{\widetilde{D}}(f)$. Choisissant $f$ homogène pour cette représentation (cf \ref{}) on a ainsi construit un diviseur $\divi_{\widetilde{D}}(f)$ fixe pour l'action de $H$ et linéairement équivalent à $D$. On peut donc supposer que $\widetilde{D}$ est $H$-invariant. En utilisant le diagramme du début de cette partie qui s'applique localement ici et en remarquant que $H$ agit transitivement sur lui même on conclut que via cette isomorphisme, $\widetilde{D}$ est de la forme $H\times Z$. On en déduit en posant $D=p(\widetilde{D})$ que $\widetilde{D}=p^*(D)$. Or, par hypothèse $D$ est linéairement équivalent à un $D'\in K$, on a ainsi le résultat car $p^*(D')$ est principal d'après la proposition précédente.
\end{proof}

\begin{cor}
Soit $X$ vérifiant $(\dagger)$. Alors $\widetilde{X}$ est quasi-affine.
\end{cor}
\begin{proof}
On recouvre $X$ par des ouverts affine $X_1,...,X_r$. D'après \ref{codimaffinenormal} chaque $X\setminus X_i$ est purement de codimension $1$, c'est donc le support d'un diviseur effectif $D_i\in \wdiv(X)$. Sur $X_{reg}$, on a donc $\supp(D_i)=\supp(\divi_{D_i}(1))$ où $1$ est vu comme une section globale de $\Oo_X(D_i)$. Cette égalité reste vrai sur $X$ grâce aux isomorphismes $\Gamma(X ,\Oo_X(D_i)\simeq\Gamma(X_{reg} ,\Oo_X(D_i))$ et $\wdiv(X)\simeq\wdiv(X_{reg})$. Ainsi, d'après \ref{pstarprincipal}, $\widetilde{X}$ est recouvert par des $\widetilde{X}_{f_i}$ qui sont affines car les $X_i=X_{D_i,1}$ le sont. Maintenant, notons que la propriété de finitude locale du faisceau structural de $\widetilde{X}$ reste vrai sur tout recouvrement affine de $\widetilde{X}$, on a donc en particulier pour tout $i$, $\Oo(\widetilde{X}_{f_i})=k[(g_{ij})_{1\leq j\leq n}]$. Comme $f_i$ est inversible sur chaque $\widetilde{X}_{f_i}$, on ne change rien en multipliant les $g_{ij}$ par une puissance $f_i^m$, ce qui permet de supposer que les $g_{ij}$ proviennent de sections globales en prenant un $m$ suffisamment grand (Cf \ref{}). On a ainsi construit une sous-algèbre de type fini $R=k[(g_{ij})_{1\leq i\leq r, 1\leq j\leq n}]$ de $\Oo(\widetilde{X})$ telle que pour tout $i$, on a $R_{f_i}=\Oo({\widetilde{X}_{f_i}})=\Oo({\widetilde{X}})_{f_i}$, la dernière égalité venant de \ref{cox 1.3.1.7}. On en déduit des immersions ouvertes $\widetilde{X}_{f_i}\xhookrightarrow{} \spec(R)$ qui se recollent en une immersion ouverte $\widetilde{X}\xhookrightarrow{} \spec(R)$ d'où le résultat.
\end{proof}

\begin{cor}
Soit $X$ vérifiant $(\dagger)$, $x\in X$, $K_x^0\subset K$ le sous-groupe des diviseurs localement principaux en $x$, et $\tilde{x}\in p^{-1}(x)$ tel que $H.\tilde{x}$ est fermé. Alors le stabilisateur $H_{\tilde{x}}$ est égal à $\spec k[K/K_x^0]$.
\end{cor}
\begin{proof}
Comme $H$ agit localement sur $\widetilde{X}$ on peut se placer sur un voisinage affine de $x$, on suppose ainsi $X$ et $\widetilde{X}$ affines. Ainsi, en appliquant \ref{staborbitegroup}, on a $H_{\tilde{x}}=\spec k[K/K_{\tilde{x}}]$, où $K_{\tilde{x}}$ est le groupe d'orbite de $\tilde{x}$. Supposons que $D\in K$ soit principal sur ce voisinage de $x$, c'est à dire $D=\divi(g)$ avec $g\in k(X)$. Choisissons $\alpha \in k[X]$ tel que $\alpha(x)\neq 0$. Alors $f:=\alpha g^{-1}\in \Rr_D(X)$ et $\divi_D(f)=\divi(\alpha)$. En utilisant \ref{zerosspecrelatif}, on obtient $f(\tilde{x})\neq 0$ et donc $D\in K_{\tilde{x}}$. Réciproquement, prenons $D\in K_{\tilde{x}}$, c'est à dire $f(\tilde{x})\neq 0$ pour un $f\in \Rr_D(X)$. Alors $x\notin \divi_D(f)$, et on a $D=-\divi(f)$ au voisinage de $x$. Cela qui montre que $D$ est localement principal.
\end{proof}

\begin{cor}
Soit $X$ vérifiant $(\dagger)$, et telle que tout diviseur soit de Cartier, par exemple si X est lisse. Alors $H$ agit librement sur $\widetilde{X}$.
\end{cor}


\subsection{Propriétés algébriques de l'anneau de Cox}

\begin{prop}
Soit $X$ vérifiant $(\dagger)$. Alors:
\begin{enumerate}
\item L'anneau de Cox $\Rr(X)$ est factoriel.
\item Le groupe des unités de l'anneau de Cox est $\Rr(X)^\times=\Gamma(X,\Oo^\times)$
\end{enumerate}
\end{prop}
\begin{proof}
\begin{enumerate}
\item On peut supposer $X$ lisse car $X_{sing}$ est de codimension $\geq 2$ donc $\Gamma(X, \Rr)=\Gamma(X_{reg}, X)$. Ainsi $\widetilde{X}$ est lisse (par vérification locale immédiate) et le théorème \ref{clgtrivial} s'applique. Soit $f\in \Rr(X)=p_*\Oo_{\widetilde{X}}(X)=\Oo_{\widetilde{X}}(\widetilde{X})$. On a donc le diviseur effectif $\divi(f)= \sum n_iD_i=\sum n_i\divi(f_i)=\divi(\prod f_i^{n_i})$ où les $D_i$ sont des diviseurs premiers, chaque $f_i\in \Rr(X)$ d'après \ref{caracfaisceaustructdiv} et est irréductible. L'unicité de l'écriture sur $\wdiv(X))$ donne l'unicité de l'écriture $f=u\prod f_i^{n_i}$ où $u\in \Rr(X)^\times$.
\item Une inclusion est évidente. Pour l'autre prenons $f\in\Rr(X)^\times$, qui est homogène d'après \ref{Coxgraded}, disons de degré $D$. Alors, $fg=1\in \Rr_0(X)$ pour un $g\in\Rr(X)^\times$ homogène de degré $-D$. Ainsi on a $0=\divi_0(1)=\divi_{D-D}(fg)=\divi_D(f)+\divi_{-D}(g)$. Les deux derniers diviseurs étant effectifs, on a $\divi_D(f)=0$ et donc $D=-\divi(f)$, ce qui donne $f\in\Rr_0(X)=\Oo(X)^\times$ d'où le résultat.
\end{enumerate}
\end{proof}

\begin{ex}
Reprenons le cas du cône affine de l'exemple introductif.  $\Rr(Y)$ est l'algèbre de polynômes sur $k$ à $4$ indéterminés, donc est factoriel en particulier.  Par ailleurs, $\Oo(Y)^{\times}=k^*$ car en inversant par exemple $t_1$  on obtient $Y_{t_1}\simeq \AAA^{1}\setminus \lbrace0\rbrace\times\AAA^{2}$, d'où le résultat.
\end{ex}

\begin{prop}
Soit $X$ vérifiant $(\dagger)$. Alors:
\begin{enumerate}
\item Soient $f\in \Rr_D(X)$ et $g\in \Rr_E(X)$ non nulles. Alors $f\mid g\iff \divi_D(f)\leq\divi_E(g)$.
\item Soit $f\in \Rr_D(X)$ non nulle. Alors $f$ est premier si et seulement si $\divi_D(f)$ est premier.
\end{enumerate}
\end{prop}
\begin{proof}
Soit $f,g$ des fonctions régulières non-nulles sur $\widetilde{X}$. Comme $\widetilde{X}$ est intègre, on peut les voir comme des éléments de $k(X)^*$. Ainsi $f\mid g$ dans $\Rr(X)$ si et seulement si $\divi(f^{-1}g)\geq 0$. Prenant $f,g$ comme dans l'énoncé et en utilisant \ref{pstarprincipal}, que c'est équivalent à $\divi_D(f)\leq\divi_E(g)$. La preuve du deuxième énoncé est similaire.
\end{proof}

\section{Groupe des classes avec torsion}
