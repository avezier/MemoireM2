% !TEX root = ../FundationsDataScience.tex

\section*{Introduction}

L'objectif du stage était d'entrer dans le sujet des anneaux de Cox en suivant la présentation de Arzhantzev, Derenthal, Hausen et Laface dans le chapitre $1$ du livre \cite{coxrings}. Il m'a fallu dans un premier temps acquérir des notions de bases en théorie des invariants, que j'ai pu trouver dans la note de synthèse \cite{LAGBrion}. Puis, je me suis naturellement intéressé à la théorie des diviseurs sur une variété algébrique, qui constitue un autre ingrédient important de la construction des anneaux de Cox. La meilleure exposition que j'ai pu trouver sur ce sujet est celle du livre \cite{Hartshorne}, ce qui m'a amené à me familiariser avec le langage des schémas. J'ai pu ensuite regarder les variétés
Enfin, il m'a fallu éclaircir plusieurs points du livre. Le premier était de comprendre pourquoi la bijection naturelle $H/H_x\simeq H.x$, où $H.x$ est l'orbite d'un point $x$ d'une $H$-variété sous l'action d'un groupe algébrique diagonalisable, était un isomorphisme de variétés.

\section*{Remerciements}

Je souhaite remercier Michel BRION pour le sujet passionnant qu'il m'a proposé, sa disponibilité, ainsi que pour l'aide précieuse qu'il m'a apporté tout au long de ce travail.

\section*{Conventions}
\label{sec:conventions}

\begin{itemize}
\item Sauf mention explicite du contraire, $k$ désigne un corps algébriquement clos de caractéristique zéro. Les résultats où l'hypothèse sur la caractéristique est nécessaire seront clairement balisés.
\item Un anneau désigne un anneau commutatif unitaire.
\item Un groupe algébrique désigne un groupe algébrique affine.
\end{itemize}


