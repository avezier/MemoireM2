\chapter{Préliminaires}

\section{Résultats d'algèbre commutative}

\subsection{Extensions entières d'anneaux}

\begin{defn}\label{normalring}
Soit $A$ un anneau intègre. $A$ est dit intégralement clos si il est égal à sa clôture intégrale. Soit $B$ un anneau, on dit que $B$ est normal si tout localisé de $B$ en idéal premier est un anneau intègre et intégralement clos. On note qu'un anneau intègre est intégralement clos si et seulement si il est normal.
\end{defn}

\begin{thm}\label{factonormal}
Soit $A$ un anneau intègre noetherien intégralement clos. Alors
\begin{enumerate}
\item Tous les diviseurs premiers d'un idéal principal non-nul sont de hauteur 1.
\item $A=\cap_{p \textrm{ premier, } \textrm{ht}(p)=1} A_p$
\end{enumerate}
\end{thm}
\begin{proof}
Prendre la preuve dans \cite{Matsumura} thm 11.5 p81
\end{proof}


\begin{thm}\label{UFDPID}
Soit $A$ un anneau intègre noetherien intégralement clos. \\
Alors, $A$ est factoriel $\iff$ Tout idéal premier de hauteur 1 est principal.
\end{thm}
\begin{proof}
\end{proof}

\subsection{Anneaux locaux, Anneaux de valuation discrète}

\begin{thm}\label{reglocufd}
Un anneau local régulier est factoriel.
\end{thm}
\begin{proof}
Voir \cite{Matsumura} 20.3
\end{proof}

\begin{thm}\label{ideauxinversibles}
Soit $A$ un anneau intègre et $I$ un idéal fractionnaire de $A$. Les assertions suivantes sont équivalentes:
\begin{enumerate}
\item $I$ est inversible
\item $I$ est un $A$-module projectif
\item $I$ est de type fini, et pour tout idéal maximal $m$ de $A$, l'idéal fractionnaire $I_m$ de $A_m$ est principal.
\end{enumerate}
\end{thm}
\begin{proof}
Voir \cite{Matsumura} 11.3
\end{proof}

\section{Algèbres graduées}



\section{Variétés algébriques}

\subsection{Généralités}

Dans ce mémoire, on travaille dans la catégorie des $k$-schémas réduits séparés de type fini sur $k$, où $k$ est un corps algébriquement clos de caractéristique zéro fixé. Ces objets sont appelés des $k$-variétés algèbriques, ou tout simplement variétés.  Cette catégorie est équivalente à la catégorie des $k$-variétés algébriques au sens de \cite{LAGSpringer} en ne considérant que les points fermés. D'ailleurs, par un point d'une variété $X$, on entendra point fermé, sauf mention du contraire. Soit $X_0$ le sous espace des points fermés de $X$ muni de la topologie induite. Alors les treillis des ouverts des topologies de $X$ et $X_0$ sont isomorphes. On bénéficie ainsi des résultats sur les morphismes et la dimension démontrés par exemple dans \cite{LAGSpringer} ou \cite{MumfordRedBook} chap I.

On note que la sous-catégorie pleine des variétés affines est anti-équivalente à celle des $k$-algèbre de type fini réduites via le foncteur sections globales, noté $k[.]$ dans ce cas pour coller aux notations traditionnelles. Un schéma de type fini sur un corps est noetherien, on en déduit que toute partie localement fermée d'une variété admet une unique structure de variété, et tout fermé se décompose de manière unique en une union finie de sous-variétés fermées irréductibles maximales. Enfin, le produit sur $k$ préserve l'irréductibilité.

\begin{cons}\label{gluevar}[Recollement de variétés]
Soit $(X_i)_{u\in I}$ une famille finie de variétés. Supposons $\forall i,j$ on ait des ouverts $X_{ij}\subset X_i$ et des isomorphismes $f_{ij}:X_{ij}\rightarrow X_{ji}$ tels que $\forall i,j,k\in I$ on ait:
	\begin{enumerate}
	\item $X_{ii}=X_i$ et $f_{ii}=id$
	\item $f_{ij}^{-1}(X_{ji} \cap X_{jk}) =  X_{ij} \cap X_{ik}$
	\item Le diagramme suivant commute:
	$$
		\begin{tikzcd}
		  X_{ij}\cap X_{ik} \arrow[rd, "f_{ij}"] \arrow[rr, "f_{ik}"] & & X_{ki}\cap X_{kj}\\
  		 & X_{ji}\cap X_{jk} \arrow[ru, "f_{jk}"] &
	\end{tikzcd}
	$$
	\end{enumerate}
Alors on peut définir une variété X comme la réunion disjointe des $X_i$ modulo la relation d'équivalence $x\sim x' \iff §x\in X_{ij}, x'\in X_{ji}$ et $f_{ij}(x)=x')$. On note $f_i:X_i\rightarrow X$ les applications canoniques, et on munit 
$X$ de la topologie finale associée aux $f_i$. On vérifie que chaque $f_i$ est une immersion ouverte dont on note $U_i$ l'image. Le faisceau structural est définit en recollant les $\mathcal{O}_{U_i}:=f_{i\star}\mathcal{O}_{X_i}$, ses sections sur un ouvert $W\subset X$ étant $$
\mathcal{O}_X(W) =
\{
(s_i)_{i \in I} \mid
s_i \in \mathcal{O}_{U_i}(W \cap U_i),
f_{ij}(s_i|_{W \cap U_i \cap U_j}) = s_j|_{W \cap U_i \cap U_j}
\}
$$
\end{cons}



\begin{defn}
Soit $X, Y$ des variétés. Supposons que $X$ soit une variété sur $Y$ par un morphisme $f:X\rightarrow Y$. On dit que $X$ est séparé sur $Y$ si $f$ est séparé, c'est à dire que l'image du morphisme diagonal dans $X\times_Y X$ est fermé.
\end{defn}

\begin{prop}
	\begin{enumerate}
	\item la composition de deux morphismes séparés est séparé.
	\item Lemme 25.21.8 stacks
	\end{enumerate}
\end{prop}
\begin{proof}
\end{proof}

On déduit de la proposition précédente que le spectre relatif d'une variété $X$ est séparé.

\subsection{Dimension}

\begin{thm}\label{dimsousvariete}
Soit $X$ une variété irréductible, $U\subset X$ un ouvert non-vide et $f\in \Oo_X(U)^*$ non inversible. Soit $Z$ une composante irréductible de $\lbrace x \in U \mid f(x)=0 \rbrace$. Alors $\dime Z= \dime(X)-1$.
\end{thm}
\begin{proof}
Voir \cite{MumfordRedBook} I.7 Th.2, après réduction au cas $X$ affine, la preuve consiste en une réduction au cas facile où $k[X]$ est factoriel. 
\end{proof}

\subsection{Normalité}

\begin{defn}
Une variété est dite normale si tous ses anneaux locaux sont intègres et intégralement clos.
\end{defn}

\begin{prop}\label{normaluniondisjointe}
Une variété normale est union disjointe de ses composantes irréductibles.
\end{prop}
\begin{proof}
Si un point $p\in X$ d'une variété se situe à l'intersection de deux composantes irréductibles, l'anneau local en $p$ contient au moins deux premiers minimaux et n'est donc pas intègre.
\end{proof}

On note que une variété irréductible $X$ est normale si et seulement si pour tout ouvert affine $U\subset X$, $\mathcal{O}_X(U)$ est normal au sens de la définition \ref{normalring}.

\begin{prop}\label{codimesingnormal}
Le lieu singulier d'une variété normale est un fermé de codimension $\geq 2$
\end{prop}
\begin{proof}

\end{proof}


\begin{prop}\label{extregularnormal}
Soit $X$ une variété normale irréductible. Pour toute sous-variété fermée $Y$ de codimension $\geq 2$, la restriction $\Oo(X)\rightarrow \Oo(X\setminus Y)$ est un isomorphisme.
\end{prop}
\begin{proof}
On peut traiter le problème localement et supposer $X=\spec A$ affine. On considère $f$ régulière sur $U:=X\setminus Y$. On remarque que tout $p\in \spec A$ de hauteur $1$ appartient à $U$. En effet, dans le cas contraire il contiendrait les idéaux premiers correspondants aux composantes irréductibles de $Y$, ce qui est impossible car ils sont de hauteur $\geq 2$. On en déduit, en tenant compte de l'irréductibilité de $X$, des injections $\Oo_X(U)\xhookrightarrow{} \Oo_p=A_p$ dans les tiges qui peuvent être vues comme des inclusions dans le corps des fonctions rationnelles de $X$. On a ainsi en tenant compte de \ref{factonormal} un morphisme $\Oo_X(U)\xhookrightarrow{} \cap_p A_p=A=\Oo(X)$ qui est inverse de la restriction, d'où le résultat.
\end{proof}

\begin{prop}\label{codimaffinenormal}
Soit $X$ une variété affine irréductible. Pour toute variété irréductible $Y$ contenant $X$, le complémentaire $Y\setminus X$ est de codimension 1.
\end{prop}
\begin{proof}
On considère l'application de normalisation $\eta_Y$. Alors l'application induite $\eta_Y^{-1}(X)\rightarrow X$ est l'application de normalisation de $X$. On en déduit que $\eta_Y^{-1}(X)$ est affine. De plus, comme $\eta_Y$ est finie, la dimension du complémentaire de $X$ ne change pas en remplaçant $Y$ par sa normalisation et $X$ par sa préimage. On peut donc supposer $Y$ normal.

Quitte à soustraire de $Y$ les composantes irréductibles de codimension $1$ de $Y\setminus X$, on peut supposer $\codime_Y(Y\setminus X)\geq 2$. Soit $V$ un ouvert affine de $Y$. Par irréductibilité de $Y$, on a $U:=X\cap V$ non-vide et $\codime_V(V\setminus U)=\codime_Y(Y\setminus X)\geq 2$. Ainsi d'après \ref{extregularnormal}, la restriction $\Oo(V)\rightarrow \Oo(U)$ est un isomorphisme. Comme $U$ et $V$ sont affines, cela signifie que l'inclusion $U\subset V$ est un isomorphisme, d'où $U=V$ puis $X=Y$.
\end{proof}


\subsection{Quelques résultats sur les morphismes}
\subsubsection{Généralités}

\begin{defn}[Morphisme affine]
Un morphisme de variétés algébriques $\phi: X\rightarrow Y$ est dit affine si pour tout ouvert affine $V\subset Y$, l'image réciproque $\phi^{-1}(V)$ est affine.
\end{defn}

\begin{ex}\label{exaff}
Un morphisme de variétés affines $\phi: X\rightarrow Y$ est affine. En effet, soit $V$ un ouvert affine de $Y$ et $U=\phi^{-1}(V)$. En considérant le diagramme commutatif ci-dessous on constate que l'on a $U \simeq (\phi\times i_2)^{-1}(\Delta_Y)=\lbrace (x,\phi(x))\mid x\in U \rbrace \subset X\times V$. Comme $X\times V$ est affine, $U$ aussi.
	\begin{center}
	\begin{tikzcd}
  		U \arrow[r, "\phi"] \arrow[d, "i_1"]& V \arrow[d, "i_2"] \\ 
  		X \arrow[r, "\phi"] & Y
	\end{tikzcd}\\
	\end{center}

\end{ex}

\subsubsection{Dimension des fibres}

\subsubsection{Applications rationnelles}

\subsubsection{Morphismes finis, normalité}

\begin{defn}[Morphisme fini, localement fini]
Soit $f:X \rightarrow Y$ un morphisme de variétés affines. On dit que $f$ est fini si la $k[Y]$-algèbre $(k[X], f^*)$ est finie.\\
On dit qu'un morphisme est localement fini en $x\in X$ si ils existe un morphisme fini $\mu:Y' \rightarrow Y$ et un isomorphisme $\nu$ d'un ouvert de $X$ contenant $x$ sur un ouvert de $Y'$, tel que $\mu\nu =f_{|U}$.
\end{defn}

\begin{prop}Soient $X,Y$ deux variétés algébriques affines irréductibles de même dimension et $f:X \mapsto Y$ un morphisme dominant.\\
Alors il existe $g\in k[Y]^*$ tel que le morphisme induit $f:X_g \mapsto Y_g$ soit fini, surjectif avec des fibres de même cardinal.
\end{prop}
\begin{proof}
Par hypothèse, l'extension $k(Y) \xrightarrow{f^*} k(X)$ est algébrique finie, disons de degré n. En caractéristique zéro on peut trouver $u\in k(X)$ tel que $k(X)=k(Y)[u]$. On remarque que l'on peut imposer $u\in k[X]$. On considère $P:=P_{min}(u, k(Y))=T^n+a_1T^{n-1}+...+a_0$. En réduisant au même dénominateur on a $P\in k[Y]_v[T]$ pour un $v\in k[Y]$. De plus, en prenant l'intersection avec d'autres ouverts principaux on peut supposer $k[X]_v$ entier sur $k[Y]_v$, et $k[Y]_v[u]$ intégralement clos, ce qui donne $k[Y]_v[u]=k[X]_v$ et $k[X]_v$ entier sur $k[Y]_v$. Ainsi $f:X_v \rightarrow Y_v$ est fini et donc surjectif car dominant.\\
On a donc une factorisation de $f^*:k[Y]_v\xrightarrow{p_1^*}k[Y]_v[T]\xrightarrow{\pi}k[Y]_v[T]/(P)\xrightarrow{\overline{ev_f}}k[Y]_v[u]$ qui donne $f:X_v \xrightarrow{\simeq} \lbrace (y,t) \in Y_v\times \mathbb{A}^1\mid P(y)(t)=0\rbrace \hookrightarrow Y_v\times \mathbb{A}^1 \xrightarrow{p_1} Y_v$. Ainsi le cardinal de la fibre $f^{-1}(y), y\in Y_v$ est le cardinal de l'ensemble des zéros du polynôme $P(y)(T)$. On peut s'assurer que cet ensemble est de cardinal constant en intersectant à nouveau avec l'ouvert principal du discriminant de $P$ qui est un polynôme en les coefficients de $P$.
\end{proof}

Ce résultat reste vrai en caractéristique positive, voir 
~\cite{LAGSpringer} 5.1.6 pour une preuve légèrement différente dans ce cadre. On y montre que le cardinal de la fibre générale est $[k(X):k(Y)]_s$. En revanche pour le corollaire immédiat suivant, la caractéristique zéro est essentielle (penser par exemple au morphisme de Frobenius $\AAA^1 \xrightarrow{x \mapsto x^p} \AAA^1$ ). 
\begin{cor}
Avec les hypothèses de 5, si de plus $f$ est injectif, alors il existe $g\in k[Y]^*$ tel que le morphisme induit $f:X_g \mapsto Y_g$ soit un isomorphisme.
\end{cor}

\begin{prop}\label{facto}
Soit $f:X \mapsto Y$ un morphisme dominant de variétés irréductibles. Soit $g:X \rightarrow Z$ constant sur les fibres de $f$. Alors il existe $h\in k[Y]^*$ et une factorisation
	\begin{tikzcd}
		X_h \arrow[r,"g"] \arrow[d,"f"] & Z \\
		Y_h \arrow[ru, dashed]
	\end{tikzcd}
\end{prop}
\begin{proof}
	\begin{multicols}{2}
	On considère $\phi=(f,g):X\rightarrow Y\times Z$ et le diagramme commutatif ci-contre. Comme $f$ est dominant, $\pi_1$ l'est aussi. De plus $\overline{\phi(X)}$ est irréductible et $\phi(X)$ contient un ouvert dense de $\overline{\phi(X)}$. Par ailleurs comme $g$ est constante sur les fibres de $f$ on vérifie que $\pi_1$ est injective sur $\phi(X)$. Par le corollaire précédent, $\pi_1$ réalise un isomorphisme $\overline{\phi(X)}_h \xrightarrow{\pi_1} Y_h$ pour un $h\in k[Y]^*$. Finalement, le morphisme recherché est  $Y_h \xrightarrow{\pi_2\pi_1^{-1}} Z$ 
	
	\columnbreak
	\begin{center}
	\begin{tikzcd}
  		& X \arrow[ldd,bend right,swap, "f"] \arrow[d, "\phi"] \arrow[rdd, bend left,"g"]  &\\ 
  		& \overline{\phi(X)} \arrow[ld,swap,"\pi_1"] \arrow[d,"i=\subset"] \arrow[rd,"\pi_2"]  &\\ 
		Y & \arrow[l,"p_1"]  Y\times Z \arrow[r,swap,"p_2"]  & Z
	\end{tikzcd}\\
	\end{center}
	\end{multicols}
\end{proof}

\begin{prop}
Soit $f:X \mapsto Y$ un morphisme de variétés affines et $x\in X$. Si la fibre de $f(x)$ est finie, alors $f$ est localement fini en $x$.
\end{prop}
\begin{proof}
Cf \cite{LAGSpringer} 5.2.6
\end{proof}

\begin{thm}\label{ZMT}
Soit $f:X \mapsto Y$ un morphisme bijectif de variétés irréductibles avec $Y$ normale. Alors $f$ est un isomorphisme.
\end{thm}
\begin{proof}

\end{proof}


\section{Groupes algébriques affines}
\subsection{Généralités}
\subsection{G-variétés, représentations}

\begin{defn}[G-variété]
Soit $G$ un groupe algébrique. Une $G$-variété est une variété algébrique $X$ sur laquelle $G$ agit algébriquement. C'est à dire qu'on a un morphisme de groupes de $G$ dans le groupe d'automorphismes $X$.
\end{defn}

\begin{prop}
Soit $G$ un groupe algébrique, $X$ une $G$-variété et $x\in X$.
\begin{enumerate}
\item $G.x$ est ouvert dans $\overline{G.x}$.
\item Toute composante irréductible de $G.x$ a pour dimension $dim (G)-dim(G.x)$.
\item $\overline{G.x}\setminus G.X$ est une union d'orbites de dimension $<dim(\overline{G.x})$.
\item $G.x$ est ouvert dans $\overline{G.x}$.
\end{enumerate}
\end{prop}
\begin{proof}
On suppose d'abord $G$ connexe.
\begin{enumerate}
\item D'après \ref{}, $G.x$ contient un ouvert dense $U$ de $\overline{G.x}$. Or, $G$ est réunion de translatés de $U$.
\item D'après \ref{}, il existe un ouvert dense de $G.x$ tel que toute les fibres de cet ouvert ont pour dimension dim($G$)$-$dim($G.x$)$=$dim($G_x$).
\item $\overline{G.x}\setminus G.x$ est un fermé propre de $\overline{G.x}$ donc de dimension inférieure d'après \ref{}. Par ailleurs, $\overline{G.x}$ est $G$-stable donc $\overline{G.x}\setminus G.x$ est réunion d'orbites.
\item Enfin si dim($G.x$) est minimal, $\overline{G.x}\setminus G.x$ est vide
\end{enumerate}
Enfin, $G$ n'est pas connexe, on écrit $G=\cup_{i=1}^{n}g_iG^\circ .x$ avec $g_1=e$. D'où $\overline{G.x}=\cup_{i=1}^{n}\overline{g_iG^\circ .x}$. Les $\overline{g_iG^\circ}$ sont égales où disjointes, c'est donc la décomposition en composantes irréductibles. On construit un ouvert de $\overline{G.x}$ inclus dans $G.x$ en posant $U=G^\circ .x\setminus \cup_{i=2}^{n}\overline{g_iG^\circ .x}$. On a dim($G^\circ$)$-$dim($(G^\circ) _x$)$=$dim($G$)$-$dim($G_x$) car $(G_x)^\circ \subset(G^\circ) _x \subset G_x$, d'où dim($G_x$)$=$dim($(G^\circ) _x$). Or chaque composante de $G.x$ est l'adhérence d'un orbite pour $G^\circ$, d'où 2) d'après le cas connexe. On a $\overline{G.x}\setminus G.x=\cup_{i=1}^{n}\overline{g_iG^\circ .x}\setminus g_iG^\circ .x=\cup_{i=1}^{n}g_i(\overline{G^\circ .x}\setminus G^\circ .x)$ qui est une union finie de fermés de dimension inférieure à $\overline{G .x}$ ce qui prouve 3). On utilise le même argument pour prouver 4) dans le cas général.
\end{proof}

\begin{defn}[$G$-module, simple, semi-simple]
Une représentation de $G$, ou $G$-module (rationnel) est un couple $(V, \rho)$ où $V$ est un $k$-espace vectoriel de dimension finie et $\rho$ un morphisme de groupes algébriques de $G$ dans $GL(V)$.\\
On étend cette définition au cas où $V$ est de dimension infinie, on demande alors que $V$ soit réunion de $G$-modules de dimension finie.\\
On dit qu'un $G$-module est simple si il n'admet pas de sous $G$-module non trivial. On dit qu'un $G$-module est semi-simple si tout sous $G$-module admet un $G$-module supplémentaire.
\end{defn}

\begin{prop}
Soit $G$ un groupe algébrique et $X$ une $G$-variété. $k[X]$ est naturellement muni d'une action $(g.f)(x):=f(g^{-1}.x), \forall f\in k[X], g\in G,x\in X$\\
Muni de cette action, $k[X]$ un $G$-module.
\end{prop}
\begin{proof}
On note $a:G\times X\rightarrow X$ le morphisme associé à l'action de $G$. Cela donne $\forall g,x \in G\times X$, $a^*(f)(g,x)=g^{-1}.f(x)=\sum_{i=1}^r\phi_i(g)\psi_i(x)$, d'où $g.f=\sum_{i=1}^r\phi_i(g)\psi_i\in k[X]$. Ainsi les translaté $g.f$ pour $g\in G$ engendrent un $k$-ev $V(f)$ de dimension finie et $G$-stable. \\
De plus l'action est algébrique. En effet $\forall l\in V(f)^*$, qu'on prolonge en $l'\in \textrm{Vect}_k(\psi_1,...,\psi_r)^*$. On a $\forall h\in G, g\mapsto l(g.(h.f))=\sum_{i=1}^r\phi_i((gh)^{-1})l'(\psi_i)\in k[G]$.\\
Finalement $k[X]=\cup_{f\in k[X]}V(f)$ est un G-module.
\end{proof}

\begin{thm}\label{embed}
Soit $G$ un groupe algébrique et $X$ une $G$ variété. $X$ est isomorphe en tant que $G$-variété à une sous $G$-variété fermée d'un $G$-module de dimension finie.
\end{thm}

\begin{cor}
Tout groupe algébrique est linéaire.
\end{cor}

\begin{defn}[Groupe réductif]
Un groupe algébrique $G$ est dit réductif si tout $G$-module est semi-simple.
\end{defn}

\begin{ex}
Les groupes finis et les groupe diagonalisables sont réductifs.
\end{ex}

\subsection{Groupes quotients}
\begin{thm}
Soit $G$ un groupe algébrique et $H\leq G$ fermé. \\Alors il existe un $G$-module $V$ de dimension finie et une ligne $L\subset V$ telle que $H=Stab_G(L):=\lbrace g\in G\mid g.v\in L,\forall v\in L\rbrace$.
\end{thm}

\begin{thm}
Soit $G$ un groupe algébrique et $H\lhd G$ fermé. \\Alors il existe un $G$-module $(V, \rho)$ de dimension finie tel que $H=\Ker{\rho}$.
\end{thm}
Le théorème suivant est le résultat principal de cette section. Il prouve l'existence des groupes quotients dans la catégorie des groupes algébriques. Le groupe quotient est alors unique à isomorphisme près, c'est une conséquence formelle de la propriété universelle du quotient.
\begin{thm}[Car. 0]\label{groupequotient}
Soient $G, H, (V, \rho)$ comme dans le théorème précédent, et $f:G \rightarrow G'$ un morphisme de groupes algébriques tel que $H\subset \Ker f$.\\
Alors il existe une unique factorisation 	
	\begin{tikzcd}
		G \arrow[r,"f"] \arrow[d,"\rho"] & G' \\
		\rho(G) \arrow[ru, "\exists ! \phi", swap, dashed]
	\end{tikzcd}
\end{thm}
\begin{proof}
	Le morphisme $\phi$ recherché existe en tant que morphisme de groupes abstraits, il est G-équivariant pour les actions naturelles de G sur $\rho(G)$ et $G'$ via $\rho$ et $f$. Concrètement cela signifie $\forall g_1, g_2 \in G, \phi(\rho(g_1)\rho(g_2))=f(g_1)\phi(\rho(g_2))$. Si $G$ est connexe, d'après la proposition \ref{facto}, $\phi$ est algébrique sur un ouvert $U$ non-vide de $\rho(G)$. Or on a un recouvrement de $\rho(G)$ par des $g.U$. En écrivant pour $x\in g.U,  \phi(x)=f(g)\phi(g^{-1}.x))$, on constate que $\phi$ est un morphisme de groupes algébriques.
	\\Supposons $G$ quelconque mais $H\leq G^\circ$. Comme $\phi$ est algébrique sur le sous-groupe $G^\circ/H$ d'après ce qui précède, on a $\phi$ algébrique partout à nouveau par G-équivariance.\\
	On peut se ramener au cas précédent en procédant en deux étapes. Dans un premier temps, on quotiente par le sous-groupe normal connexe $H^\circ$ (on a bien $H^\circ\leq G^\circ$), puis on quotiente par le sous-groupe normal fini $H/H^\circ$. Il reste donc à prouver le cas $H$ fini, c'est un corollaire direct du théorème \ref{goodquotientthm}.
	\end{proof}
\subsection{Groupes diagonalisables, actions de groupes diagonalisables}

\subsubsection{Groupes diagonalisables}
Soit $G$ un groupe algébrique. Le groupe $X^*(G)$ des caractères de $G$ est un sous-groupe de $k[G]^\times$. On remarque que $X^*$ est un foncteur contravariant de la catégorie des groupes algébriques dans la catégorie des groupes abéliens de type fini, l'image d'un morphisme $G_1\xrightarrow{\phi}G_2$ étant simplement la (co)-restriction $\phi^* \,_{|X^*(G_2)}^{|X^*(G_1)}$ du comorphisme $\phi^*$ entre les algèbres de coordonnées. On signale qu'en caractéristique $p>0$, les groupes de caractères ont de plus la propriété d'être sans $p$-torsion. Tout ce qui suit reste vrai en caractéristique $p$, avec cette contrainte supplémentaire sur les groupes de caractères.

\begin{ex}
\begin{enumerate}
\item $X^*(GL_n)=\lbrace \textrm{det}^k\mid k\in \ZZ \rbrace\simeq\ZZ$. En effet, $k[GL_n]^\times =\lbrace \lambda\textrm{det}^k\mid k\in \ZZ, \lambda\in k^* \rbrace$, puisque $k[X_{ij}]$ est factoriel et det irréductible. En évaluant en $I_n$, on a nécessairement $\lambda=1$. Comme det$^k$ est un caractère, le résultat suit.
\item $X^*(SL_n)={1}$ car $D(SL_n)=SL_n$.
\item Les unités de $k[\GG_m]$ sont les monômes. On en déduit que les caractères sont exactement les $t\mapsto t^k, k\in \ZZ$. Par ailleurs, on remarque que $X^*(G_1\times G_2)=X^*(G_1)\times X^*(G_2)$, d'où $X^*(D_n)=\lbrace \textrm{monômes à coefficient unitaire} \rbrace \simeq \ZZ^n$.
\end{enumerate}
\end{ex}

On remarque que les caractères de $D_n$ engendrent $k[D_n]$ comme $k$-ev, ils en forment donc une $k$-base par le lemme de Dedekind qui assurent que les caractères sont libres dans Map$(D_n, k)$. On a plus généralement:

\begin{prop}
Soit $G$ un groupe algébrique. Les assertions suivantes sont équivalentes:
\begin{enumerate}
\item $G$ est diagonalisable (i.e. $\simeq$ à un sous-groupe fermé de $D_n$)
\item $k[G]=\textrm{Vect}_k(X^*(G))$.
\item Tout $G$-module est somme directe $G$-modules de dimension 1.
\end{enumerate}
\end{prop}
\begin{proof}
\begin{enumerate}
\item 1)$\implies$2) La restriction $k[D_n]\xrightarrow{res_G} k[G]$ est surjective et la restriction d'un caractère est un caractère.
\item 2)$\implies$3) $G$ est abélien, car $\forall \chi\in X^*(G),\,g,h\in G, \chi(gh)=\chi(hg)$. C'est donc vrai pour toute fonction régulière, on en conclut $gh=hg$.\\
On observe que l'action naturelle de $G$ sur $k[G]$ est semi-simple. En effet, les caractères forment une base de diagonalisation de $k[G]$. Ainsi $G$ est semi-simple par la décomposition de Jordan. Soit $(V,\rho)$ le $G$-module considéré et $W\subset V$ un sous $G$-module de dimension finie, disons $n$. Par la décomposition de Jordan,  $\rho(G)$ est semi-simple. De plus, c'est un sous groupe abélien fermé de $GL_n$. Il est donc conjugué à un sous-groupe fermé de $D_n$. On voit ainsi que $V=\bigoplus_{\chi \in X^*(G)}V_\chi$, où $V_\chi:=\lbrace v\in v\mid g.f=\chi(g)f,\, \forall g\in G \rbrace$. En effet, ils sont en somme directe, et tout élément de $V$ se décompose de cette manière.
\item 3)$\implies$1) On peut supposer $G\subset GL_n$, et considérer l'action naturelle sur $k^n$ après choix d'une base $(e_1,...,e_n)$. Par hypothèse, on peut écrire $k^n=(f_1)\oplus...\oplus (f_n)$, avec les $(f_i)$ sous $G$-module de dimension 1. $G$ est donc conjugué à un sous-groupe de $D_n$.
\end{enumerate}
\end{proof}

Ce constat motive la définition suivante:

\begin{defn}[Tore, Groupe diagonalisable]
Un groupe diagonalisable est un groupe algébrique $G$ tel que $k[G]=\textrm{Vect}_k(X^*(G))$. Un tore est un groupe diagonalisable connexe.
\end{defn}

On travaille désormais dans la catégorie des groupes diagonalisables. On considère un groupe diagonalisable $G$ et le groupe $\chi^{**}(G):=(\chi^*)^2(G)$.

\begin{prop}
$G$ et $\chi^{**}(G)$ sont naturellement isomorphes en tant que groupes abstraits, et aussi en tant que groupes algébriques par transport de structure. L'isomorphisme est $ev_G : G\rightarrow \Hom (X^*(G),\GG _m), g\mapsto (\chi\mapsto \chi(g))$.
\end{prop}
\begin{proof}
ev$_G$ est injective: Soit $g\in G$ tel que $\chi(g)=1=\chi(e_G),\forall \chi\in X^*(G)$. Alors $g=e_G$ car $G$ est un groupe diagonalisable.\\
ev$_G$ est surjective: Soit $\phi\in\chi^{**}(G)$. On a un prolongement unique de $\phi$ en un morphisme de $k$-algèbre $k[X^*(G)]=k[G]\rightarrow k$ qui est donc de la forme $k[G]\rightarrow k, f\mapsto f(g)$ pour un $g\in G$. En restreignant à $X^*(G)$, on trouve que $\phi=\textrm{ev}_G(g)$.
\end{proof}

Autrement dit on a un isomorphisme de foncteurs $(\chi^*)^2\simeq Id$, d'où une équivalence de catégories entre les groupes diagonalisables et les groupes abéliens de type fini. Une autre façon de voir cela est d'introduire l'algèbre de groupe d'un groupe abélien de type fini M, c'est par définition $k[M]=\lbrace \sum_{finie} \lambda_gg,\,\lambda_g\in k,g\in G \rbrace$ avec la multiplication définie par l'opération de groupe de $G$. La propriété suivante montre que l'on construit ainsi un autre inverse de $\chi^*(.)$
\begin{prop}
Soient, $M, M_1, M_2$ des groupes abéliens de type fini, et $G$ un groupe diagonalisable.\\
Alors $k[M]$ est de type fini, réduite et on a $k[M_1\oplus M_2]\simeq k[M_1]\otimes k[M_2]$. De plus, $k[M]$ est naturellement muni d'une structure d'algèbre de Hopf et on a $k[G]=k[X^*(G)]$ et donc $G=\spec\circ k[X^*(G)]$.
\end{prop}
\begin{proof}
On a deux morphismes d'algèbre $k[M_1]\rightarrow k[M_1\oplus M_2], e_{m_1}\mapsto e_{(m_1,0)}$ et $k[M_2]\rightarrow k[M_1\oplus M_2], e_{m_2}\mapsto e_{(0,e_{m_2})}$, d'où l'existence d'un morphisme $k[M_1]\otimes k[M_2]$ dont on vérifie que c'est un isomorphisme.\\
Comme on a $M\simeq \ZZ^r\oplus (\oplus_{i=1}^r \ZZ/d_i\ZZ)$, il suffit de traiter les cas $M=\ZZ$ et $M=\ZZ/d\ZZ$. On a $k[\ZZ]\simeq k[t,t^{-1}]$ qui est intègre, de type fini, et réduite. On a $k[\ZZ/d\ZZ]\simeq k[t]/(t^d-1)$. On voit, par le théorème chinois par exemple, que cette algèbre de type fini non-intègre est réduite si et seulement si les racine de $t^d-1$ sont simples, ce qui est le cas en caractéristique zéro.\\
Enfin la structure d'algèbre de Hopf sur $k[M]$ est donnée par $\Delta(e_m)=e_m\otimes e_m,\, i(e_m)=e_{-m},\, e(e_m)=e_0$.
\end{proof}

\begin{cor}
Soit $G$ un groupe diagonalisable. Alors:
\begin{enumerate}
\item $G$ est isomorphe au produit direct d'un tore et d'un groupe abélien fini.
\item $G$ est un tore $\iff X^*(G)$ est libre de type fini $\iff G$ est connexe.
\end{enumerate}
\end{cor}

\subsubsection{Action d'un groupe diagonalisable sur une variété affine}

On a montré dans la partie précédente que l'algèbre des coordonnées d'un groupe diagonalisable $G$ était naturellement munie d'une $X^*(G)$-graduation. Cela peut être vu comme la traduction algébrique de l'action de $G$ sur lui même par multiplication à gauche. On précise, cela ci-dessous en montrant que les foncteurs $\spec$ et $k[.]$ réalisent une équivalence de catégories entre les variétés affines munies d'une action d'un groupe diagonalisable et les algèbres graduées par un groupe abélien de type fini. On a déjà l'équivalence entre variétés affines et algèbres affines. Il s'agit donc de vérifier que les (co)-restrictions des deux foncteurs $k[.]$ et $\spec$ sont bien définies et que les actions et graduations sont préservées.

\begin{cons}
Soit $H$ un groupe diagonalisable et X une $H$-variété affine. Le $H$-module $k[X]$ est somme directe de $H$-module de dimension 1 d'après la partie précédente et on peut écrire: 
\begin{center}
$k[X]=\bigoplus_{\chi \in X^*(H)}V_\chi$,  où $V_\chi:=\lbrace f\in k[X]\mid h.f=\chi(h)f,\, \forall h\in H \rbrace$ 
\end{center}
On voit immédiatement que cette somme directe est $X^*(H)$-graduée. De plus, cette construction est fonctorielle, en effet en considérant les diagrammes commutatifs ci-dessous. Le premier est la traduction de la donnée d'un morphisme $(\widetilde{\phi},\phi)$ d'une $H$-variété affine $X$ vers une $H'$-variété affine $X'$. Le second est obtenu par passage aux algèbres de coordonnées.
\begin{multicols}{2}
	\begin{center}
		\begin{tikzcd}
  		H\times X \arrow[r, "a_1"] \arrow[d, "\widetilde{\phi} \times \phi"] & X \arrow[d, "\phi"] \\ 
  		H'\times X' \arrow[r, "a_2"] & X'
	\end{tikzcd}
	\end{center}

	\columnbreak
	\begin{center}
		\begin{tikzcd}
  		k[H]\otimes k[X]   & k[X] \arrow[l, "a_1^*"] \\ 
  		k[H']\otimes k[X'] \arrow[u, "\widetilde{\phi}^* \otimes \phi^*"] & k[X'] \arrow[u, "\phi"] \arrow[l, "a_2^*"] 
	\end{tikzcd}
	\end{center}
\end{multicols}
Le petit calcul ci-dessous montre que $\forall f\in V'_{\chi'}\subset k[X'],\,$ on a $\phi^*(f)\in V_{\widetilde{\phi}^*(\chi')}$, ce qui montre que $(\widetilde{\phi}^*,\phi^*)$ est un morphisme d'algèbres graduées. 
$$\forall h\in H,\,x\in X,\, \phi^*(f)(h.x)=a_1^*\phi^*(f)(h, x)=(\widetilde{\phi}\otimes\phi)a_2^*(f)(h,x)=\widetilde{\phi}^*(\chi')(h)\phi^*(f)(x)$$ 

\end{cons}

\begin{cons}\label{consgraduegroupe diagonalisable}
Soit $K$ un groupe abélien de type fini et $A$ une $k$-algèbre affine $K$-graduée. On pose $X:=\spec(A)$ et on choisit des générateurs homogènes $f_{\omega_1},...,f_{\omega_r}$ de $A$, ce qui donne une immersion fermée $i:X\rightarrow k^r, x\mapsto (f_1(x),...,f_r(x))$. On transporte la graduation à $k[k^r]=k[t_1,...,t_r]$ en posant deg$(t_i)=\omega _i$, ce qui fait de $k[i(X)]=k[t_1,...,t_r]/\Ker i^*\xrightarrow{\overline{i^*}}A$ un isomorphisme d'algèbres graduées. Enfin, on munit $k^r$ de l'action diagonale associée aux caractères $\chi^{\omega_1},...,\chi^{\omega_r}$ de $H:=\spec(k[K])$, c'est à dire $\forall h\in H,\, t\in k^r,\, h.t := (\chi^{\omega_1}(h)t_1,...,\chi^{\omega_r}(h)t_r)$. On a donc concrètement pour $f:=\sum\alpha_{i_1,...,i_r}t_1^{i_1}...t_r^{i_r}\in k[k^r]$, $f(h.t)=f(\chi^{\omega_1}(h)t_1,...,\chi^{\omega_r}(h)t_r)=\sum\alpha_{i_1,...,i_r}\chi^{\sum i_k\omega_k}(h)t_1^{i_1}...t_r^{i_r}$. Ainsi on a, $f\in k[k^r]_{\omega}\iff \forall (i_1,...,i_r),\, \sum_k i_k\omega_k=\omega \iff f(h.t)=\chi^{\omega}(h)\sum\alpha_{i_1,...,i_r}t_1^{i_1}...t_r^{i_r}=\chi^{\omega}(h)f(t), \forall h\in H$. Cette condition montre que l'idéal homogène $\Ker i^*$ est $H$-stable et donc que $i(X)$ est une $H$-variété. On transporte en retour cette action sur $X$ et on voit que l'action obtenue est indépendante du choix initiale des $f_i$, en effet la condition d'homogénéité exprimée ci-dessus définit complètement le comorphisme de l'action de $H$: $a^*:A\rightarrow k[K]\otimes A,\, f_{\omega}\mapsto \chi^{\omega}\otimes f_{\omega}$.\\
Pour finir, vérifions la fonctorialité. Soit $(\widetilde{\phi},\phi)$ un morphisme entre la $K$-algèbre graduée $A$ et la $K'$-algèbre graduée $A'$. La construction montre que l'on définit deux morphismes $a_1:A\rightarrow k[K]\otimes A$ et $a_2:A'\rightarrow k[K']\otimes A'$. Et on a, $$\forall h_\omega \in A_\omega,\, (\widetilde{\phi}\otimes \phi)\circ a_1(h_\omega)=(\widetilde{\phi}\otimes \phi)(\chi^{\omega}\otimes h_\omega) = \chi'^{\widetilde{\phi}(\omega)}\otimes \phi(h_\omega)_{\widetilde{\phi}(\omega)}$$ 
$$\forall h_\omega \in A_\omega,\, a_2\circ \phi(h_\omega)=a_2(\phi(h_\omega)_{\widetilde{\phi}(\omega)})=\chi'^{\widetilde{\phi}(\omega)}\otimes \phi(h_\omega)_{\widetilde{\phi}(\omega)}$$
D'où le diagramme commutatif de gauche ci-dessous, qui donne le diagramme de droite par application du foncteur $\spec$. Ce dernier diagramme finit de prouver la fonctorialité.
\begin{multicols}{2}
	\begin{center}
			\begin{tikzcd}
  		K\otimes A  \arrow[d, "\widetilde{\phi}^* \otimes \phi^*"]  & A \arrow[l, "a_1"] \arrow[d, "\phi"]  \\ 
  		K'\otimes A' & A' \arrow[l, "a_2"] 
	\end{tikzcd}
	\end{center}

	\columnbreak
	\begin{center}
		\begin{tikzcd}
  		\spec(k[K])\times \spec(A) \arrow[r, "a_1^{\circ}"]  & \spec(A) \\ 
  		\spec(k[K'])\times \spec(A') \arrow[r, "a_2^{\circ}"] \arrow[u, "\widetilde{\phi} \times \phi"] & \spec(A') \arrow[u, "\phi"] 
	\end{tikzcd}
	\end{center}
\end{multicols}
\end{cons}

On voit que dans les deux constructions on a la même condition sur l'homogénéité qui détermine à la fois l'action et la graduation, c'est ce qu'on voulait vérifier.\\
Ainsi les actions de groupe diagonalisable sur les variétés affines peuvent être caractérisé en terme algébrique  grâce à cette équivalence. C'est l'objet des propositions suivantes.

\begin{prop}\label{stablehomogene}
Soit $A$ une algèbre affine $K$-graduée, $X:=\spec A$ muni de l'action de $H:=\spec k[K]$.  Soit $Y\subset X$ une sous variété fermée, et $I=\mathcal{I}(Y)$. Les assertions suivantes sont équivalentes:
\begin{enumerate}
\item $Y$ est $H$-stable.
\item $I$ est un idéal homogène.
\end{enumerate}
\end{prop}
\begin{proof}
Comme $I$ est radical, $Y$ est $H$-stable si et seulement si $\forall P\in I,\,h\in H,\, y\in Y,\, P(h.y)=0$. Supposons que cette dernière condition soit vérifiée, on écrit $P=\sum_{finie}P_\omega$ sa décomposition en composantes homogènes. On a alors, $P(h.y)=\sum \chi^{\omega}\otimes P_{\omega}(h,y) = \sum \chi^{\omega}(h) P_{\omega}(y)=0$. Les caractères étant libres, on a $P_{\omega}(y)=0, \forall y,\omega$, c'est à dire $P_{\omega}\in I, \forall \omega$. Ainsi $I$ est homogène. La réciproque est immédiate. 
\end{proof}

Dans la construction \ref{consgraduegroupe diagonalisable}, on voit que la situation générale est assez proche de l'exemple le plus simple de l'action diagonale de $(k^*)^n$ sur $k^r$ par un choix de $r$ caractères. On voit bien dans ce cas que la géométrie de l'orbite d'un point $p\in k^r$ va être assez dépendante de la nullité des coordonnées $x_1,..., x_r$ au point $p$. Cela motive la définition suivante.

\begin{defn}[Monoïde d'orbite, Groupe d'orbite]
Soit $A$ une algèbre affine $K$-graduée munie de l'action de $H:=\spec k[K]$ sur $X:=\spec A$.
\begin{enumerate}
\item Le monoïde d'orbite d'un point $x\in X$ est le sous-monoïde $S_x\subset K$ engendré par $\lbrace\omega \in K\mid \exists f\in A_{\omega} \textrm{ telle que } f(x)\neq 0\rbrace$
\item Le groupe d'orbite d'un point $x\in X$ est le sous-groupe $K_x\subset K$ engendré par le monoïde d'orbite.
\end{enumerate}
\end{defn}

\begin{prop}\label{staborbitegroup}
Soit $A$ une algèbre affine $K$-graduée, $X:=\spec A$ muni de l'action de $H:=\spec k[K]$, et $x\in X$. On a le diagramme commutatif suivant, dont les deux lignes sont exactes:
	\begin{center}
		\begin{tikzcd}
  		0  \arrow[r] & K_x\arrow[r] \arrow[d, "\simeq"] & K \arrow[r] \arrow[d, "\omega \xmapsto{\simeq} \chi^{\omega}"] & K/K_x \arrow[d, "\simeq"] \arrow[r] & 0 \\ 
  		0 \arrow[r]& X^*(H/H_x) \arrow[r, "\pi^*"] & X^*(H)  \arrow[r,"i^*"] & X^*(H_x) \arrow[r] & 0
	\end{tikzcd}
	\end{center}
	où $i:H_x\rightarrow H$ est l'inclusion du stabilisateur de $x$, et $\pi:H\rightarrow H/H_x$ la projection canonique. En particulier, on obtient $H_x\simeq \spec(k[K/K_x])$.
\end{prop}
\begin{proof}
La deuxième ligne est obtenue par le théorème \ref{groupequotient}, puis application du foncteur $X^*$. Elle est bien exacte par exactitude du foncteur $X^*$. La flèche verticale centrale est l'isomorphisme canonique déduit de $X^*\circ\spec\circ k[.]\simeq Id$.\\ 
Comme $\overline{H.x}$ est $H$-stable, la graduation est préservée sur $k[\overline{H.x}]$ d'après la proposition \ref{stablehomogene}. De plus si $f\in k[X]_{\omega}$ est telle que $f(x)\neq 0$, cela reste le cas modulo $\mathcal{I}_{X}(\overline{H.x})_\omega$ et réciproquement. Ainsi, $S_x$ et $K_x$ ne sont pas modifiés si on remplace $X$ par $\overline{H.x}$. \\
Considérons le fermé propre $\overline{H.x}\setminus H.x$, éventuellement vide. Il est $H$-stable comme réunion d'orbites. Alors $\mathcal{I}_{\overline{H.x}}(\overline{H.x}\setminus H.x)$ est homogène d'après la proposition \ref{stablehomogene}, et $\neq \lbrace 0\rbrace$. Choisissons $f\neq0$ homogène dans ce $H$-module, c'est donc un vecteur propre. Ainsi, $f$ est non-nulle quelque part sur $H.x$ et donc partout par transitivité et choix de $f$. On en conclut $H.x=(\overline{H.x})_f$. Considérons la graduation naturelle associée à $k[\overline{H.x}]_f$. Comme on a inversé $f$, le monoïde de poids est potentiellement plus gros. En revanche on voit facilement que $K_x$ n'est pas modifié.\\
Supposons donc $X=H.x$. Dans ce cas, on voit qu'une fonction homogène non-nulle est partout non-nulle, donc inversible. On a donc dans ce cas $S_x=K_x=S(k[H.x])=K(k[H.x])$, où $S(k[H.x])$ et $K(k[H.x])$ sont respectivement les monoïdes et groupes de poids de l'algèbre $K$-graduée $k[H.x]$. \\
D'après la proposition \ref{facto} et le théorème \ref{ZMT} on a le diagramme commutatif de gauche ci-dessous. Le diagramme de droite est obtenu par application du foncteur $k[.]$.
\begin{multicols}{2}
	\begin{center}
	\begin{tikzcd}
		H \arrow[r,"h \mapsto h.x", twoheadrightarrow] \arrow[d,"\pi", twoheadrightarrow] & H.x \\
		H/H_x \arrow[ru, "\exists !\simeq \phi", swap, dashed]
	\end{tikzcd}
	\end{center}

	\columnbreak
	\begin{center}
	\begin{tikzcd}
		k[H] & & k[H.x] \arrow[lld, "\exists !\simeq \phi^*", dashed]\arrow[ll, hook, "f_\omega \mapsto f_\omega(x)\chi^{\omega}", swap]  \\
		k[H/H_x]  \arrow[u,"\pi^*", hook]  & &
	\end{tikzcd}
	\end{center}
\end{multicols}
Comme les flèches du diagramme de gauche sont des morphismes de $H$-variétés, les morphismes du diagramme de droite préservent la graduation. Ainsi, $\phi^*$ induit un isomorphisme sur les groupes de poids $K_x \xrightarrow{\omega \mapsto\chi^{\omega}}X^*(H/H_x)$. Enfin, toujours en utilisant le diagramme, cet isomorphisme est l'unique faisant commuter le carré de gauche dans le diagramme de la proposition.
\end{proof}

\begin{prop}
Soit $A$ une algèbre affine $K$-graduée, $X:=\spec A$ muni de l'action de $H:=\spec k[K]$, et $x\in X$. Cette action induit une action de $H/H_x$ sur $\overline{H.x}$. De plus, $\overline{H.x}$ et $\spec(k[S_x])$ sont isomorphes en tant que $H/H_x$-variétés.
\end{prop}
\begin{proof}
On suppose $X=\overline{H.x}$ ce qui ne modifie pas $S_x$ et $K_x$. De plus on a $S(k[\overline{H.x}])=S_x$ et $K(k[\overline{H.x}])=K_x$. D'après la preuve précédente, $k[K_x]$ et $k[H/H_x]$ sont canoniquement isomorphes et on a un isomorphisme de $K_x$-algèbres graduées $k[H.x]\rightarrow k[K_x],\, f_\omega \mapsto f_\omega(x)\chi^{\omega}$. On a également le morphisme de $K_x$-algèbres graduées $k[\overline{H.x}]\rightarrow k[S_x]$ définit par la même formule. On voit facilement qu'il est injectif. Pour la surjectivité, on peut par exemple voir que le premier isomorphisme est en fait le prolongement par localisation en un élément homogène de ce morphisme (voir démonstration précédente), d'où un diagramme commutatif de $K_x$-algèbres graduées. Cela donne la proposition par application de $\spec$.

	\begin{center}
	\begin{tikzcd}
		k[H.x] \arrow[r,"\simeq"] & k[K_x]  \\
		k[\overline{H.x}] \arrow[r, "\simeq"] \arrow[u,"f\mapsto f_{|H.x}"] & k[S_x] \arrow[u,"\subset"] 
	\end{tikzcd}
	\end{center}

\end{proof}

\begin{prop}
Soit $A$ une algèbre affine intègre $K$-graduée, $X:=\spec A$ muni de l'action de $H:=\spec k[K]$. Alors il existe un ouvert affine non-vide $U\subset X$ tel que:
$$S_x=S(A),\,\,\, K_x=K(A),\,\,\, \forall x\in U$$
\end{prop}
\begin{proof}
On choisit des générateurs homogènes $f_1,...,f_r$ de $A$. On pose $U:=X_{f_1...f_r}$ qui est non-vide car $A$ est intègre. $\forall \omega \in S(A),\, \exists g\neq0 \in A_\omega$ car $A$ est intègre. Quitte à décomposer $g$,  on peut le supposer de la forme $f_1^{i_1}...f_r^{i_r}$. On en déduit que $\omega \in S_x$, $\forall x\in U$. Comme l'autre inclusion est immédiate, $U$ satisfait la propriété.
\end{proof}


Pour finir voyons une caractérisation algébrique des actions fidèles de groupes diagonalisables sur des variétés affines.

\begin{prop}
Soit $A$ une algèbre affine intègre $K$-graduée, $X:=\spec A$ muni de l'action de $H:=\spec k[K]$. L'action de $H$ est fidèle si et seulement si $K=K(A)$.
\end{prop}
\begin{proof}
Soit $g\in \cap_{x\in X}H_x$. L'action de $g$ sur les fonctions régulières est triviale, donc en particulier sur les fonctions homogènes. Comme $A$ est intègre, $\forall \omega \in S(A), \exists f_\omega\neq 0\in A_\omega$, on en déduit $g\in \cap_{\chi \in S(A)} \Ker(\chi)$ puis facilement $g\in \cap_{\chi \in K(A)} \Ker(\chi)$ d'où $g=e_H$ si $K=K(A)$ car $H$ est un groupe diagonalisable.\\
Sinon comme $K(A)\subsetneq K$, on peut choisir $g\neq e_H\in H$ tel que $g\in\cap_{\chi \in K(A)} \Ker(\chi)$. En effet $\cap_{\chi \in K(A)} \Ker(\chi)$ est un sous-groupe fermé non-trivial de $H$ car son groupe de caractère est $K/K(A)$. Ainsi $g$ agit trivialement sur les fonctions homogènes et donc sur les fonctions régulières. On en déduit que $g$ agit trivialement sur $X$.
\end{proof}

\section{Théorie des invariants}
\subsection{L'algèbre des invariants}

Soit $G$ un groupe algébrique et $X$ une $G$-variété affine. $k[X]$ est un $G$-module rationnel pour l'action naturelle de $G$ sur les fonctions régulières. on définit la sous-algèbre des invariants $k[X]^G:=\lbrace f\in k[X]\mid g.f=f,\, \forall g\in G\rbrace$. C'est par définition la sous-algèbre des fonctions constantes sur les orbites de l'action de $G$ sur $X$.\\
Une question naturelle est de se demander si cette algèbre est de type fini. Ce n'est pas le cas en général. En effet, dans la perspective de répondre au 14e problème de Hilbert, Nagata exhiba en 1959 une algèbre d'invariants pour l'action d'un groupe algèbrique qui n'est pas de type fini. Avec des hypothèses sur $G$, on peut cependant montrer que c'est le cas, c'est l'objectif de cette partie.\\
Supposons $G$ réductif. Le $G$-module $k[X]$ est alors semi-simple, en particulier, $k[X]^G$ admet un supplémentaire $G$-stable que l'on note $k[X]_G$. On définit l'opérateur de Reynolds $R_{k[X]}$ comme la projection sur $k[X]^G$ associée à cette décomposition. Voici quelques propriétés de $R_{k[X]}$:


\begin{prop}\label{reynolds}
\begin{enumerate}
\item Soit $f:V \rightarrow W$ un morphisme de $G$-module et $f^G:V^G \rightarrow W^G$ le morphisme induit. On a $R_Wf=fR_V$. En particulier, si $f$ est surjective, $f^G$ l'est aussi.
\item $R_{k[X]}$ est $K[X]^G$-linéaire
\end{enumerate}

\end{prop}
\begin{proof}
\begin{enumerate}
\item Ok
\item Soit $a\in k[X]^G$. on considère $m_a$ la multiplication par $a$ dans $k[X]$. C'est un endomorphisme de $G$-module, il commute donc avec $R_{k[X]}$. 
\end{enumerate}
\end{proof}

\begin{thm}[Hilbert]\label{hilbert}
Soit $G$ un groupe réductif et $X$ une $G$-variété affine. Alors l'algèbre des invariant $k[X]^G$ est de type fini.
\end{thm}
\begin{proof}
Supposons que $X$ soit un $G$-module $V$ de dimension finie. L'action de $k^*$ sur $V$ par homothétie donne une $\NN$-graduation $k[V]=\bigoplus_{n=0}^{\infty}k[V]_n$, $k[V]_n$ étant le sous espace des polynômes homogènes de degré n. Cette graduation est $G$-stable et se restreint sur l'algèbre des invariants en une $\NN$-graduation $k[V]^G=\bigoplus_{n=0}^{\infty}k[V]^G_n$. Or on remarque que $k[V]^G$ est noetherien. En effet, soit $I$ un idéal de $k[V]^G$, et $J$ son extension dans $k[V]$. $J$ est un sous $G$-module, donc la contraction de $J$ dans $k[V]^G$ est $R_{k[V]}(J)=IR_{k[V]}(k[V])=I$. On voit donc que la condition de chaîne est satisfaite sur $k[V]^G$ si elle satisfaite sur $k[V]$, ce qui est le cas car ce dernier est noetherien par le théorème de la base de Hilbert. Ainsi, on a le résultat d'après la proposition \ref{noethgrad}.\\
Dans le cas général, on peut d'après le théorème \ref{embed} supposer $X$ inclus dans un $G$-module $V$. On obtient alors un $G$-morphisme surjectif $k[V] \rightarrow k[X]$ qui induit un $G$-morphisme surjectif $k[V]^G \rightarrow k[X]^G$ d'après la proposition \ref{reynolds}. Cela montre que $k[X]^G$ est de type fini.
\end{proof}

On constate que cette preuve n'est pas effective. Il est en général difficile de calculer l'algèbre des invariants. On présente ci-dessous la méthode des sections qui permet le calcul dans certains cas.\\
Soit $S\subset X$ une sous-variété fermée. Définissons $Z(S)=\lbrace g\in G\mid g.s=s,\,\forall s \in S\rbrace$ et $N(S)=\lbrace g\in G\mid g.s\in S,\,\forall s \in S\rbrace$. Clairement, $Z(S)$ est un sous-groupe normal de $N(S)$, et le quotient $W=N(S)/Z(S)$ agit sur $S$. La surjection $k[X]\rightarrow k[S]$, induit un morphisme $k[X]^G\xrightarrow{\phi} k[S]^W$.\\
Supposons que l'on ait un ouvert dense $U\subset X$ tel que $\forall x\in U,\, G.x$ intersecte $S$, alors on voit que $\phi$ est injective. Si de plus, $k[S]^W$ est engendré par des $\phi(f_1),...,\phi(f_r)$, alors $\phi$ est un isomorphisme et $k[X]^G$ est engendré par $f_1,...,f_r$.

\begin{ex}
$G=GL_n$,\, $X=M_n$,\, $g.A=gAg^{-1},\, S=D_n,\, U=X_{disc(\chi )}$. En considérant un élément de $U$, qui a donc ses valeurs propres deux à deux distinctes, on a par un calcul direct $Z(S)=D_n$. Puis on a $N(S)=\lbrace\textrm{matrices monomiales}\rbrace$ car la conjugaison préserve les espaces propres. Ainsi, $W$ est isomorphe au groupe symétrique $\Sigma_n$ et agit sur $S$ en permutant les entrées diagonales. On a ainsi $k[S]^W=k[\sigma_1,...,\sigma _n]$, l'algèbre engendrée par les fonctions symétriques élémentaires. C'est une algèbre de polynômes.\\
Soient $f_1,...,f_n$ les coefficient du polynôme caractéristique générique. Ce sont des éléments de $k[X]^G$, et on a $f_{i| S}=(-1)^i\sigma _i$, d'où $k[X]^G=k[f_1,...,f_n]$.
\end{ex}

\subsection{Quotient d'une variété algébrique sous l'action d'un groupe algébrique}

\subsubsection{Quotient catégorique}

Soit $G$ un groupe algébrique et $X$ une $G$-variété. En tant que groupe abstrait agissant sur un ensemble, le quotient de $X$ par $G$ (noté $X//G$) est par définition l'ensemble des orbites. On note $\pi:X \rightarrow X//G$ l'application qui à un élément de $X$ associe son orbite. $X//G$ satisfait une propriété universelle, il représente le foncteur $\textrm{Ens}\rightarrow \textrm{Ens}, Y\mapsto \lbrace f\in \textrm{Map}(X, Y)\mid f \textrm{ est constante sur les orbites} \rbrace$, il est donc unique à isomorphisme près. Pour cette raison la paire $(X//G, \pi)$ est appelée le quotient catégorique de $X$ par $G$. \\ 
On peut ainsi transporter cette définition dans la catégorie des variétés algébriques. Toutefois, il n'est pas clair que ce quotient existe toujours. L'exemple suivant montre que lorsqu'il existe, le quotient catégorique ne coïncide pas nécessairement avec l'ensemble des orbites.

\begin{ex}
On considère l'action naturelle de $GL_n$ sur $\AAA^n$. Le quotient catégorique existe et est un point. En effet soir $f:\AAA^n\rightarrow Z$ constant sur les orbites, alors $f$ est constante car il existe un orbite dense. En revanche il y a un deuxième orbite, c'est le fermé $\lbrace 0\rbrace$. 
\end{ex}

On suppose $X$ affine et $k[X]^G$ de type fini avec $f_1,...,f_r$ des générateurs, c'est en particulier le cas lorsque $G$ est réductif d'aprés le théorème \ref{hilbert}. Dans ce cadre, l'algèbre des invariants définit une variété algébrique affine, notons la $Y$. On définit le morphisme $\phi :X\rightarrow k^r,\, x\mapsto (f_1(x),...,f_r(x))$. Son comorphisme $\phi^*$ admet une factorisation: $k[t_1,...,t_r]\twoheadrightarrow k[X]^G\xrightarrow{\subset}k[X]$, d'où $Y\simeq \overline{\phi (X)}$. On peut donc voir $X \xrightarrow{\phi} \overline{\phi (X)}$ comme une réalisation du morphisme $\pi: X\rightarrow Y$ associé à $k[X]^G\subset k[X]$. On appelle ce morphisme, le morphisme quotient. De la même manière on voit que tout morphisme $G$-invariant de variétés affines $X\rightarrow Z$ se factorise à travers $Y$. De ce fait, $Y$ semble être un bon candidat pour le quotient catégorique. Toutefois il faut être prudent, dans \cite{LAGFerrer} 6.4.10, on exhibe un exemple de cette situation qui n'admet pas de quotient catégorique. On a toutefois le résultat suivant:

\begin{thm}\label{goodquotientthm}
Soit $G$ un groupe réductif et $X$ une $G$-variété affine.
\begin{enumerate}
\item Le morphisme quotient $\pi:X\rightarrow Y$ est surjectif.
\item $(Y, \pi)$ est un quotient catégorique. On écrit donc $Y=X//G$.
\item Soit $Z\subset X$ une sous $G$-variété fermée. Le morphisme induit $Z//G \rightarrow X//G$ est une immersion fermé. On peut ainsi identifier $\pi_Z$ et $\pi_X$ restreint à $Z$. De plus, soit $Z'$ une autre sous $G$-variété fermée, on a $\pi_X(Z\cap Z')=\pi_X(Z)\cap\pi_X(Z')$.
\item Chaque fibre de $\pi_X$ contient un unique orbite fermé.
\end{enumerate}
\end{thm}
\begin{proof}
\begin{enumerate}
\item Soit $x\in Y$ et $m_x$ l'ideal maximal de $k[X]^G$ correspondant. La fibre $\pi^{-1}(x)$ correspond à l'ensemble des idéaux maximaux contenant l'extension $I$ de $m_x$ dans $k[X]$. Or on a déjà vu que l'extension dans $k[X]$ était injective, $I$ est donc un idéal propre contenu dans au moins un idéal maximal. La fibre étant non-vide, $\pi$ est surjective.
\item L'existence de la factorisation a déjà était vue. Avec 1) on a maintenant l'unicité.
\item On note $i$ l'inclusion $Z\subset X$. $\pi_Xi$ est constant sur les orbites de $Z$ d'où l'existence d'un unique morphisme $\phi:Z//G\rightarrow X//G$ tel que $\phi\pi_Z=\pi_Xi$. La projection $k[X] \rightarrow k[Z]$ est un morphisme de $G$-module surjectif. D'après la proposition \ref{reynolds}, cette projection induit un morphisme de $k$-algèbre surjectif $k[X]^G \xrightarrow{\phi^*} k[Z]^G$, donc $\phi$ est une immersion fermée.
Soit $I$ (resp. $I'$) l'idéal de $Z$ (resp. $Z'$) dans $k[X]$. L'idéal de $Z\cap Z'$ est $I+I'$ et l'idéal de $\pi_X(Z)$ est $\mathcal{I}_{X//G}(\pi_X(Z))=I\cap k[X]^G=R_X(I)$. Ainsi $\mathcal{I}_{X//G}(\pi_X(Z\cap Z'))=R_X(I+I')=R_X(I)+R_X(I')=\mathcal{I}_{X//G}(\pi_X(Z)\cap \pi_X(Z'))$.
\item d'après 3), $\pi_X$ envoie deux orbites fermés distincts sur deux points distincts.
\end{enumerate}
\end{proof}

On remarque que les propriétés du théorème précédent s'étendent automatiquement au cas d'une $G$-variété $X$ si le théorème est vérifié localement sur un recouvrement affine d'un candidat $(Y,\pi)$ pour le quotient $X//G$. De ce constat découle la notion de bon quotient.

\begin{defn}[Bon quotient]
Soit $G$ un groupe réductif et $X$ une $G$-variété. Une paire $(Y, \pi)$ où $Y$ est une variété et $\pi$ un morphisme $X\rightarrow Y$ est un bon quotient si elle vérifie:\\
(i) $\pi$ est affine et $G$-invariant.\\
(ii) $\pi^*:\Oo_Y \rightarrow (\pi_*\Oo_X)^G$ est un isomorphisme.
\end{defn}

\begin{ex}
Soit $G$ un groupe réductif et $X$ une $G$-variété affine. D'après le théorème \ref{goodquotientthm} et l'exemple \ref{exaff}, $X//G$ est un bon quotient.
\end{ex}


\subsubsection{Quotient géométrique}

Parmi les quotients catégoriques $(X//G,\pi)$, on cherche à caractériser ceux ayant les propriétés géométriques intuitivement attendues pour un quotient, c'est à dire que $X//G$ soit l'ensemble des orbites avec une topologie aussi fine que possible. C'est la notion de quotient géométrique:

\begin{defn}[Quotient géométrique]
Soit $G$ un groupe algébrique et $X$ une $G$-variété. Une paire $(Y, \pi)$ où $Y$ est une variété et $\pi$ un morphisme $X\rightarrow Y$ est un quotient géométrique si elle vérifie:\\
(i) $\pi$ est surjective et ses fibres sont exactement les orbites.\\
(ii) La topologie de $Y$ coïncide avec la topologie quotient associée à $\pi$.\\
(iii) $\pi^*:\Oo_Y \rightarrow (\pi_*\Oo_X)^G$ est un isomorphisme.
\end{defn}

On remarque que pour un quotient géométrique $(X//G, \pi)$, tous les orbites sont fermés dans $X$ et l'application quotient est ouverte. En effet, soit $U$ un ouvert de $X$, on a $\pi^{-1}(\pi(U))=\cup_{g\in G}g.U$ qui est ouvert.

\begin{ex}
Soit $G$ un groupe algébrique et $H$ un sous-groupe fermé. Dans la catégorie des ensemble, le quotient $(G/H,\pi)$ est exactement le quotient catégorique $(G//H,\pi)$ pour l'action de $H$ sur $G$ par multiplication à droite. Dans \cite{LAGSpringer} 5.5.5, en caractéristique quelconque, on munit $G/H$ d'une structure d'espace annelé en lui attribuant la topologie quotient puis en définissant le faisceau structural par $\Oo_{G/H}(U):=\lbrace f\in\textrm{Map}(U,k)\mid f\pi \in \Oo_G(\pi^{-1}(U))\rbrace$. Par construction, il vérifie la propriété universelle de factorisation. De plus, on montre ensuite que cet espace annelé est isomorphe à une variété quasi-projective, ce qui montre l'existence du quotient catégorique $(G//H,\pi)$ dans la catégorie des variétés algébriques. Par définition de $\Oo_{G/H}$, on a une flèche $\Oo_Y \xrightarrow{\pi^*} (\pi_*\Oo_X)^G$. Elle est injective par la surjectivité de $\pi$, et elle est surjective, par la propriété universelle du quotient. $(G//H,\pi)$ est donc un quotient géométrique. Cela généralise bien sur le théorème \ref{groupequotient}.
\end{ex}

\begin{ex}
Un bon quotient $(X//G, \pi)$ est un quotient géométrique si les fibres de $\pi$ sont exactement les orbites. En effet, d'après ce qui précède, il reste alors à vérifier que $X//G$ est muni de la topologie quotient. Soit un ouvert de X de la forme $\pi^{-1}(A)$ où $A$ est une partie de $X//G$. En tenant compte de \ref{goodquotientthm} (iii) et de la surjectivité de $\pi$ on a: $\pi(X\setminus \pi^{-1}(A))=Y\setminus A$ qui est fermé, donc $A$ est ouvert.
\end{ex}

\subsubsection{Un exemple: La construction Proj}
Dans cette partie, on va détailler une construction qui à la fois éclaire et généralise la construction de la variété algébrique $\PP^n$($k$). On considère $A$ une algèbre affine $\NN$-graduée et on pose $X:=\spec(A)$. $X$ est donc muni d'une action de $k^*$. Une orbite $k^*.x$ est de dimension 0 ou 1. Si elle est de dimension 0, c'est un point fixe car $k^*$ est connexe. Si elle est de dimension 1, elle est soit fermée, soit son adhérence est constituée de $k^*.x$  et d'une réunion de points fixes, en fait un seul comme on va le voir.

On note $F$ l'ensemble des points fixes et on remarque que $F=\mathcal{V}_X(A_{>0})$, où $A_{>0}:=(f\mid f\in A_d \textrm{ pour un }d>0)$ est l'idéal dit inconvenant. En effet, $F$ est l'ensemble des idéaux maximaux qui sont $k^*$-stables. Or, un idéal maximal et homogène contient nécessairement $A_{>0}$.

On remarque que $A^{k^*}=A_0=A/A_{>0}$, et comme le bon quotient $Y_0:=X//k^*=\spec (A_0)$ paramètre les orbites fermés, on obtient en particulier que si $k^*.x$ n'est pas fermé, son adhérence contient un unique point fixe. En résumé, $W:=X\setminus F$ est la réunion des orbites de dimension 1, et ils sont tous fermés dans $W$. On va montrer que $W$ admet un quotient géométrique, c'est la construction Proj.\\
Pour tout $f\in A_{>0}$ homogène, la localisation $A_f$ est $\ZZ$-graduée de la manière suivante:
\begin{center}
 $A_f=\bigoplus_{d\in \ZZ},\,\,\,\, (A_f)_d:=\lbrace h/f^l\mid \textrm{deg}(h)-l\textrm{deg}(f)=d \rbrace$
\end{center}
On note $A_{(f)}:=(A_f)_0$ en remarquant qu'il s'agit de l'algèbre des invariants de la $k^*$-variété affine $X_f$. On a ainsi trouvé le bon quotient $U_f:=\spec(A_{(f)})=X_f//k^*$. De plus, comme $F\subset\mathcal{V}_X(f)$,  il s'agit d'un quotient géométrique d'après ce qui précède. Or on peut recouvrir $W$ par un nombre fini de $X_{f_i}$ avec les $f_i$ homogènes de degrés $>0$. Considérons les diagrammes commutatifs ci-dessous. Dans le diagramme de gauche, toutes les flèches sont des inclusions. Le diagramme de droite est obtenu par application du foncteur $\spec$:
\begin{multicols}{2}
	\begin{center}
	\begin{tikzcd}
  		A_{f_i} \arrow[r, ""] & A_{f_if_j}  & A_{f_j} \arrow[l, ""]\\ 
  		A_{(f_i)} \arrow[r, ""] \arrow[u, ""] & A_{(f_if_j)}  \arrow[u, ""] & A_{(f_j)} \arrow[l, ""] \arrow[u, ""]
	\end{tikzcd}\\
	\end{center}
	
	\columnbreak
	\begin{center}
	\begin{tikzcd}
  		X_{f_i} \arrow[d, "\pi_i"] & X_{f_i}\cap X_{f_j}=X_{f_jf_j} \arrow[l, ""] \arrow[d, ""] \arrow[r, ""] & X_{f_j}  \arrow[d, "\pi_j"]\\ 
  		U_{f_i}  & U_{f_if_j}  \arrow[l, ""] \arrow[r, ""]   & U_{f_j} 
	\end{tikzcd}\\
	\end{center}
\end{multicols}
Les flèches horizontales du diagramme de droite sont des immersions ouvertes, en effet on a $U_{f_if_j}\simeq (U_{f_i})_{f_j}$. Les conditions de la construction \ref{gluevar} sont satisfaites, on peut former une prevariété Proj($A$) par recollement des $U_{f_i}$ le long de ces immersions. De plus, les $(U_{f_i},\pi_i)$ sont des quotients géométriques et le diagramme exprime que l'on a les conditions de recollement sur les intersections qui font de Proj($A$) le quotient géométrique global $W//k^*$.

Enfin, on remarque que les fermés de Proj($A$) correspondent aux fermés $k^*$-stables de $W$, c'est à dire aux idéaux radicaux homogènes qui ne contiennent pas l'idéal inconvenant. On appelle ces fermés des variétés projectives. La topologie quotient sur Proj($A$) que l'on vient de définir est aussi appelé la topologie de Zariski.

\section{Faisceaux quasi-cohérents}

\subsection{Faisceaux quasi-cohérents sur une variété}

\subsection{Faisceaux quasi-cohérents sur une variété projective}

\subsection{Faisceaux inversibles, Fibrés en droites}
\begin{defn}
Soit $X$ une variété. Un faisceau inversible $\Ll$ sur $X$ est un $\mathcal{O}_X$-module localement libre de rang 1. Autrement dit, tout point $x\in X$ admet un voisinage ouvert $U \subset X$ tel que ${\mathcal L}|_U$ est isomorphe à ${\mathcal O}_U$.
\end{defn}

On peut toujours trivialiser deux faisceaux inversibles sur un même recouvrement ouvert de $X$. On voit ainsi que le produit tensoriel sur $\mathcal{O}_X$ de faisceaux inversibles est inversible. Par ailleurs, le faisceau $\Hh om(\Ll,\Oo_X)$ est clairement inversible car sur les ouverts $U$ où $\Ll$ est trivial, se donner un morphisme $\Ll_{|U}\rightarrow\Oo_{|U}$ revient à se donner une section de $\Oo(U)$. De plus, l'application naturelle $\Ll\otimes_{\Oo_X} \Hh om(\Ll,\Oo_X)\rightarrow \Oo_X, (s,f)\mapsto f(s)$ est un isomorphisme. Ainsi, les classes d'isomorphie de faisceaux inversible sur $X$ munies du produit tensoriel forment un groupe appelé groupe de Picard de $X$, noté $\pic(X)$. 

\begin{ex}\label{expicaff}
Soit $X=\spec (A)$ irréductible. Alors se donner un faisceau inversible sur $X$ revient (à isomorphisme près) à se donner un idéal fractionnaire $I$ de $k[X]$ qui est inversible (cf \ref{ideauxinversibles}), son inverse est alors $I^{-1}:=(A:I)$. Les idéaux fractionnaires inversibles donnant le faisceau inversible trivial sont les idéaux fractionnaires principaux. Le groupe de Picard de $X$ est ainsi isomorphe au groupe des idéaux fractionnaires inversibles modulo les idéaux fractionnaires principaux. Les idéaux de $A$ qui sont inversibles forment une partie génératrice de ce groupe. 

Si de plus $A$ est localement factoriel, par exemple si $X$ est lisse, les idéaux inversibles sont les idéaux de hauteur 1 pure, c'est à dire tels que ses idéaux premiers associés sont tous de hauteur 1. De plus tout idéal inversible s'écrit de manière unique comme produit de puissances d'idéaux premiers de hauteur 1. Pic(X) est donc le quotient du groupe libre sur les idéaux premiers de hauteur 1 par les idéaux fractionnaires principaux.
\end{ex}


On va maintenant voir que les faisceaux inversibles sur $X$ s'incarnent naturellement en des variétés sur $X$, ce sont les fibrés en droites.

\begin{defn}\label{linebundle}
Soit $X$ une variété. Un fibré en droite sur $X$ est une variété $L$ munie d'un morphisme $\pi:L\rightarrow X$ tel que $X$ admet un recouvrement ouvert $(U_i)_{i\in I}$ satisfaisant:

\begin{enumerate}
\item $\forall i\in I$, il existe un isomorphisme $\phi_i :\pi^{-1}(U_i)\rightarrow U_i\times\AAA^1$ de variétés sur $U_i$.
\item $\forall i, j\in I$, l'isomorphisme $\phi_j\circ \phi_i^{-1}: (U_i\cap U_j)\times \AAA^1\rightarrow (U_i\cap U_j)\times \AAA^1$ est de la forme $(x, z)\mapsto (x, a_{ij}(x)z)$.
\end{enumerate}
Un morphisme de fibrés en droites sur $X$ est un morphisme de variétés sur $X$ avec la conditions supplémentaire que les morphismes induits sur les fibres soient linéaires.
Une section d'un fibré en droites $(L, \pi)$ est une section de $\pi$, et on a la version locale de cette notion.
\end{defn}

Considérons un faisceau inversible $\Ll$ sur $X$ trivialisé sur un recouvrement affine $(U_i)_{i\in I}$ avec un générateur $s_i\in\Ll(U_i)$ sur chaque $U_i$. Sur $U_i\cap U_j$, on a $s_j=a_{ij}s_i$ avec $a_{ij} \in \mathcal{O}_X(U_i\cap U_j)^\times$. On considère au dessus de chaque $U_i$ le fibré trivial $(U_i\times\AAA^1, \pi_i)$, et on les recolle avec des isomorphismes définis par $\mathcal{O}_X(U_i\cap U_j)\otimes_k k[t]\rightarrow \mathcal{O}_X(U_i\cap U_j)\otimes k[u]$, $f\otimes_k 1\mapsto f\otimes_k 1$, $1 \otimes_k t \mapsto a_{ij}\otimes_k u$. On a ainsi construit un fibré en droites sur $X$.

Réciproquement, considérons le $\mathcal{O}_X$-module des sections d'un fibré en droite $L$ sur $X$. Sur les ouverts $U_i$ où $L$ est trivialisé on voit que les sections forment un faisceau isomorphe à $\mathcal{O}_{X|U_i}$. En effet se donner une section sur $U_i$ revient à se donner un morphisme $U_i\rightarrow \AAA^1$, c'est à dire un élément de $\mathcal{O}_X(U_i)$. C'est donc un faisceau inversible.

En composant les deux opérations on trouve le faisceau inversible dual du faisceau de départ. Ces opération sont fonctorielles et réalisent une anti-équivalence de catégorie entre faisceaux inversibles sur $X$ et fibrés en droites sur $X$. Cela permet de transporter la structure du groupe de Picard sur les classes d'isomorphie de fibrés en droites.\\

On introduit ci-dessous la version relative du spectre d'un anneau. Ceci va nous permettre de construire les fibrés en droite de manière plus intrinsèque.

\begin{cons}\label{relspec}[Spectre relatif]

\end{cons}

Si on se donne un faisceau inversible $\Ll$ sur une variété $X$, le fibré en droites qu'on lui associe dans la discussion précédente n'est autre que Spec$_X($Sym$(\Ll))$, où Sym$(\Ll))$ est l'algèbre symétrique associée à $\Ll$ sur $\mathcal{O}_X$. En effet, on recolle les Spec$_{U_i}($Sym$(\Ll_{|U_i}))\simeq $ Spec$_{U_i}( \mathcal{O}_{U_i}[t])\simeq U_i \times_k \AAA^1_k$, où $(U_i)_i$ est un recouvrement qui trivialise $\Ll$.


Notons qu'un fibré en droites $(L, \pi)$ est muni d'une action de $\GG_m$ sur ses fibres. L'ensemble $L_0$ des points fixes sous $\GG_m$ est le fermé correspondant à l'image de la section nulle. Son complémentaire $L^\times:=L\setminus L_0$ est une $\GG_m$-variété et $\pi$ se restreint en $\pi^\times: L^\times \rightarrow X$ qui est un quotient géométrique. En effet sur les $U_i$, on a $\pi^{\times-1}(U_i) \simeq U_i\times_k \GG_m$ et l'action de $\GG_m$ se fait par multiplication sur le facteur de droite.\\

\subsection{$G$-linearisation d'un fibré en droite}

Dans cette partie, on se donne un groupe algébrique $G$.

\begin{defn}
Soit $X$ une $G$-variété, et $(L,\pi)$ un fibré en droites. Une $G$-linéarisation de  $(L,\pi)$ est une action de $G$ sur $L$ telle que $\pi$ est $G$-équivariante et pour tout $(g,x)\in G\times X$ l'application $L_x\rightarrow L_{g.x},\, l\mapsto g.l$ est linéaire.
\end{defn}

On note $\alpha: G\times X\rightarrow X$ l'action de $G$ et $p_2:G\times X\rightarrow X$. Pour tout $g\in G$ on a une application:
$$g\times id: X\rightarrow G\times X,\,\, x\mapsto (g,x)$$
qui satisfait $(g\times id)^*\alpha^*(L)=g^*(L)$ et $(g\times id)^*p_2^*(L)=L$. Ainsi, tout morphisme de fibré en droites $\Phi: \alpha^*(L)\rightarrow p_2^*(L)$ induit pour tout $g\in G$ un morphisme $\Phi_g:g^*(L)\rightarrow L$.

\begin{lem}
Avec les notations ci-dessus, on a une correspondance bijective entre les $G$-linéarisations de $L$ et les isomorphismes
$$\Phi: \alpha^*(L)\rightarrow p_2^*(L)$$
de fibrés en droites sur $G\times X$ tels que $\Phi_{gh}=\Phi_h\circ h^*(\Phi_g)$ pour tous $g,h\in G$.
\end{lem}
\begin{proof}
Soit $\beta:G\times L\rightarrow L$ une $G$-linéarisation. Par définition, on a un diagramme commutatif:
	\begin{center}
	\begin{tikzcd}
  		G\times L \arrow[r, "\beta"] \arrow[d, "id\times\pi"]& L\arrow[d, "\pi"] \\ 
  		G\times X \arrow[r, "\alpha"] & X
	\end{tikzcd}\\
	\end{center}
Par la propriété universelle du produit fibré on a donc un morphisme $\gamma:G\times L\rightarrow\alpha^*(L)$ de variété sur $G\tmes X$. On remarque que l'on a un isomorphisme $G\times L\rightarrow p_2^*(L),\, (g,l)\mapsto ((g,\pi (l)),l)$.
\end{proof}

\section{Diviseurs}

\subsection{Diviseurs de Weil}

Partant de l'observation qu'en géométrie classique dans le plan projectif, il existe une dualité entre les droites et les points, il parait intéressant de s'intéresser aux sous-variétés fermées de codimension 1 et de conférer une structure naturelle à cet ensemble. C'est l'idée de diviseur, dont on va voir qu'un cadre privilégié est celui d'une variété normale, que l'on supposera de plus irréductible suivant la remarque \ref{normaluniondisjointe}.

\begin{defn}[Diviseur premier, WDiv($X$), diviseur de Weil, diviseur effectif]
Soit $X$ une variété normale irréductible. Un diviseur premier $D$ est une sous-variété fermée irréductible de codimension 1. On définit WDiv($X$) le groupe libre engendré par les diviseurs premiers. Un élément de WDiv($X$) est appelé un diviseur de Weil. Enfin, un diviseur est dit effectif si il est à coefficients $\geq 0$. 
\end{defn}
On introduit maintenant pour chaque diviseur $D$ une valuation sur $k(X)$ donnant des informations sur le comportement des fonctions rationnelles sur $D$. C'est l'analogue de l'ordre d'un zéro ou d'un pôle d'une fonction rationnelle de la droite affine en un point. Soit $\eta$ le point générique de $D$, et $\mathcal{O}_{\eta,X}$ son anneau local. Par hypothèse et grâce aux propriétés de la localisation, il est noethérien normal et de dimension 1, c'est donc un anneau de valuation discrète. La valuation associée $v_D:k(X)\rightarrow \ZZ$ donne par définition l'ordre d'annulation d'une fonction rationnelle le long de $D$. La propriété ci-dessous montre que les fonctions rationnelles permettent de définir des diviseurs de Weil.

\begin{prop}\label{noetherienPWDivBienDef}
Soit $X$ une variété normale et irréductible et $f\in k(X)^*$. Alors $v_D(f)=0$ sauf pour un nombre fini de diviseurs premiers $D$.
\end{prop}
\begin{proof}
Soit $f=g/h\in k(X)^*$, où l'on peut supposer $X$ affine. Comme $v_D(f)=v_D(g)-v_D(h)$, on peut supposer $f\in k[X]$. Soit $D$ une diviseur premier et $p$ son point générique. Si $f\in k[X]_p^\times$ alors $v_D(f)=0$. Sinon, $f\in p$ et donc $D \subset \mathcal{V}_X(f)$. Or, d'après le théorème \ref{dimsousvariete}, les composantes irréductibles $Z_i$ de $\mathcal{V}_X(f)$ sont des diviseurs premiers. Ainsi $v_D(f)=0$ à moins que $D$ ne soit l'un des $Z_i$.
\end{proof}

Ainsi l'application $k(X)^*\rightarrow $ WDiv$(X),\, f\mapsto $ div$(f):=\sum_D v_D(f)D$ définit un morphisme de groupes. Son image est le \textit{groupe des diviseurs principaux} noté PDiv$(X)$. La relation modulo PDiv$(X)$ s'appelle \textit{l'équivalence linéaire}, et le groupe quotient Cl$(X)$ est le \textit{groupe des classes de diviseurs}.\\
Cl$(X)$ est un invariant en général difficile à calculer. Ci-dessous on liste quelques outils et exemples.


\begin{prop}
Soit $X=\spec(A)$ une variété affine normale et irréductible. Alors $A$ est factoriel si et seulement si Cl$(X)=0$
\end{prop}
\begin{proof}
C'est une conséquence de \ref{factonormal} et \ref{UFDPID}. Voir \cite{Hartshorne} II.6.2.
\end{proof}

\begin{cor}
Cl$(\mathbb{A} ^n_k)=0 pour n\geq 1$
\end{cor}

\begin{thm}\label{divexactseq}
Soit $X$ une variété normale et irréductible et $Z$ une sous-variété fermée propre. On pose $U:=X\setminus Z$. Alors:
\begin{enumerate}
\item Cl$(X)\rightarrow$ Cl$(U)$ défini par $\sum_i n_iD_i\mapsto \sum_i n_i(D_i\cap U)$, avec $D_i\cap U = 0$ si $D_i\cap U=\emptyset $, est un morphisme de groupe surjectif.
\item Si codim$(Z,X)\geq 2$, then Cl$(X)\rightarrow$Cl$(U)$ est un isomorphisme.
\item Soient $D_1,..., D_s$ les composantes irréductibles de $Z$ qui sont des diviseurs. Alors la suite ci-dessous exacte $$\bigoplus_{j=1}^s \ZZ D_j \xrightarrow{\pi} Cl(X) \xrightarrow{.\cap U} Cl(U) \rightarrow 0 $$
\end{enumerate}
\end{thm}
\begin{proof}
\begin{enumerate}
\item Si $D\cap U\neq \emptyset$ alors dim$(X)$=dim$(U)$ et dim$(D)$=dim$(D\cap U)$ car ce sont des ouverts de variétés irréductibles donc la dimension est préservée. Ainsi cela définit une application WDiv$(X)\rightarrow$ WDiv$(U)$ qui est un morphisme par construction. De plus, comme un diviseur principal est envoyé sur un diviseur principal, on a bien le morphisme attendu. Il est surjectif car pour tout $D\in$ WDiv$(U)$ premier, on a $D=\overline{D}\cap U$.
\item Dans ce cas on ne peut avoir $D\subset Z$ cause de la dimension donc $D\cap U\neq \emptyset$. Ainsi, le noyau du morphisme WDiv$(X)\rightarrow$ WDiv$(U)$ est exactement PDiv$(X)$ d'où l'isomorphisme.
\item Le noyau de $.\cap U$ est exactement l'ensemble des $\pi(D)$ où $D$ est un diviseur dont le support est contenu dans $X\setminus U=Z$, d'où le résultat.
\end{enumerate}
\end{proof}

\subsection{Faisceau d'algèbres divisorielles}

La proposition suivante montre que l'on peut caractériser les sections du faisceau structural de $X$ en terme de diviseurs.

\begin{prop}\label{caracfaisceaustructdiv}
Soit $X$ une variété normale et irréductible et $f\in k(X)^*$. Alors 
\begin{enumerate}
\item div$(f)\geq 0\iff f \in\Oo_X(X)$
\item div$(f)= 0\iff f\in\Oo_X(X)^\times$
\end{enumerate}
\end{prop}
\begin{proof}
Il est suffisant de vérifier ces propriétés localement sur les ouverts affines. Or dans ce cas, $f$ est une section globale si et seulement si $f$ appartient à tous les anneaux locaux des diviseurs premier d'après \ref{factonormal}. Cette dernière condition revient à dire que div$(f)$ est effectif, cela prouve 1).  \\
Pour la deuxième assertion, on remarque que div$(f)=0\iff$div$(f)\geq 0$ et div$(f^{-1})\geq 0$.
\end{proof}

Plus généralement, on définit pour chaque diviseur $D$ un $\Oo_X$-module $\Oo_X(D)$ dont les sections sur un ouvert $U\subset X$ sont définies par $$\Gamma(U,\Oo_X(D)):=\lbrace f\in k(X)^* \mid (\textrm{div}(f)+D)_{|U}\geq 0\rbrace\cup \lbrace 0 \rbrace$$
On vérifie, grâce aux propriétés des valuations $v_D$, qu'il s'agit d'un sous $\Oo_X(U)$-module de $k(X)$, autrement dit un idéal fractionnaire de $\Oo_X(U)$, pour tout ouvert affine $U$. C'est donc un sous $\Oo_X$-module de la  $\Oo_X$-algèbre $k(X)$. Dans le cas où $X$ est affine on a une description explicite des sections globales:

\begin{prop}\label{divaff}
Soit $X$ une variété affine normale et irréductible, $A:=k[X]$. 
\begin{enumerate}
\item Soit un diviseur de Weil $D=a_{p_1}Y_{p_1}+...+a_{p_r}Y_{p_r}$, on a $\Gamma(X,\Oo_X(D))=\bigcap_{ht(p)=1}p^{-a_p}A_p$, où $a_p=0$ si $p\notin \lbrace p_1,...,p_r\rbrace$.
\item Soit $(Y_i)_{i\leq r}$ des diviseurs premiers de Cartier et $D=a_{p_1}Y_{p_1}+...+a_{p_r}Y_{p_r}$, on a $\Gamma(X,\Oo_X(D))=\prod_{i=1}^n p^{-a_p}$.
\item Pour un diviseur premier $Y_p$, on a $\Gamma(X,\Oo_X(-Y_p))=p$. Si de plus $p$ est inversible et $a\in \ZZ$, on a $\Gamma(X,\Oo_X(aY_p))=p^{-a}$.
\end{enumerate}
\end{prop}
\begin{proof}
\begin{enumerate}
\item Pour tout idéal premier $p$ de hauteur 1, $A_p$ est un DVR. On a donc $pA_p=(\pi)$ pour un certain $\pi \in A_p\subset k(X)$. Pour tout entier $a\in \ZZ$, on définit un sous $A_p$-module de $k(X)$ isomorphe à $pA_p$ en posant $p^aA_p:= (\pi^a)$. C'est un idéal fractionnaire de $A_p$ et on a pour $f\in k(X)^*$, $v_{Y_p}(f)\geq a \iff f\in p^a A_p$.
Or, $f\in \Gamma(X,\Oo_X(D)) \iff $ div$(f)\geq -D \iff v_{Y_p}(f)\geq -a_p$, pour tout $p$ de hauteur 1, avec $a_p$ le coefficient de $Y_p$ dans $D$.
\item Cela est une conséquence de \ref{isopic} et \ref{expicaff}.

\item Dans ce cas, $Y_p$ est effectif, donc $\Gamma(X,\Oo_X(-Y_p))$ est un sous-module de $A$, donc un idéal. On a donc $\Gamma(X,\Oo_X(-Y_p))=pA_p\cap A=p$. Pour l'autre assertion, c'est immédiat car $\Oo_X(Y_p)$ est alors inversible d'après \ref{expicaff}, c'est à dire que $Y_p$ est de Cartier.
\end{enumerate}
\end{proof}

On forme maintenant la somme directe des $\Oo_X(D)$ et on la munit d'un produit de la façon suivante: pour $f_1\in \Oo_X(D_1),\, f_2\in \Oo_X(D_2)$, on définit le produit de $f_1$ et $f_2$ comme l'élément $f_1f_2$ de $\Oo_X(D_1+D_2)$. On voit que cette algèbre est naturellement WDiv-graduée, avec pour chaque degré $D$ un contrôle prescrit quant au comportement des fonctions sur le support de $D$. Ceci mène à la définition suivante.

\begin{defn}[Faisceau d'algèbres divisorielles]
Soit $X$ une variété normale et irréductible. Le faisceau d'algèbres divisorielles associé à un sous-groupe $K\in \wdiv(X)$ est le faisceau de $\Oo_X$-algèbres $K$-graduées $$\bigoplus_{D\in K}S_D,\,\,\,\,\,\,\,\,\, S_D:=\Oo_X(D)$$ 
\end{defn}


\begin{ex}
On considère la droite projective $\PP^1$, $D=\lbrace\infty\rbrace$ et $K=\ZZ D$. Cherchons la forme d'une section $f\in S_{nD}(\PP^1)$. On se place sur la carte affine $U_0=\PP^1\setminus \lbrace \infty\rbrace$ associée au repère projectif $(\infty, 0, 1)=(e_0,e_1,e_0+e_1)$, on note $z$ la coordonnée associée. Par hypothèse, $f$ est régulière sur $U_0$, c'est donc un polynôme en $z$. On fait agir l'homographie $z\mapsto w=1/z$  pour se placer sur la carte $U_1=\PP^1\setminus \lbrace e_1 \rbrace$ associée au repère $(0, \infty,1)$. Sur cette carte, la fonction qui coïncide avec $f$ sur $U_0\cap U_1$ est $g(w)=f(1/w)$. Or si on écrit $f(z)=z^kh(z)$ avec $z\nmid h(z)$, on obtient $g(w)=w^{-k-\textrm{deg}(h)}h(w)$. Comme on doit avoir $k+\textrm{deg}(h)\leq n$, on obtient que $f$ est un polynôme de degré $\leq n$.\\
Ainsi on voit que l'application $\phi_n:k[t_0,t_1]_n\rightarrow S_{nD}(\PP^1), f\mapsto f(1,z)$ est un isomorphisme de $k$-ev. De plus, on a facilement $\phi_n\phi_m=\phi_{n+m}$. Finalement, $(\phi, \widetilde{\phi})$ avec $\phi: k[t_0,t_n]\rightarrow S(\PP^1), f\mapsto f(1,z)$ et $\widetilde{\phi}:\ZZ\rightarrow K, n\mapsto nD$ est un isomorphisme d'algèbres graduées.
\end{ex}

\begin{prop}
Soit $X$ une variété normale et irréductible et $D\in\wdiv(X)$. Alors $\Oo_X(D)$ est un $\Oo_X$-module cohérent. En particulier, le faisceau d'algèbres divisorielles associé à un sous-groupe $K\in \wdiv(X)$ est une $\Oo_X$-algèbre quasi-cohérente.
\end{prop}
\begin{proof}
On peut supposer $X=\spec A$ affine car le problème est local.  Alors d'après \ref{quasicoaff}, $\Oo_X(D)\simeq \tilde{M}$ où $M=\Gamma(X,\Oo_X(D))$. Il s'agit donc de montrer que $M$ est un $A$-module de type fini. Mais d'après \ref{divaff}, on voit que $\Gamma(X,\Oo_X(D))$ est un idéal fractionnaire de $A$, il est donc isomorphe à un idéal de $A$ en tant $A$-module après multiplication par une certaine fonction rationnelle. Comme $A$ est noetherien cela conclut la preuve. En fait, cela revient à constater que les fonctions $f\in \Gamma(X,\Oo_X(D))$ peuvent admettre des pôles uniquement sur les diviseurs premiers intervenant dans l'écriture de $D$, soit un nombre fini. L'ordre de ces pôle peut  donc être borné par un même $d\in \NN$.
\end{proof}

\begin{prop}\label{somorphismcodim2}
Soit $X$ une variété normale et irréductible et $D\in\wdiv(X)$. Alors pour tout ouvert $U\subset X$ tels que $X\setminus U$ soit de codimension $\geq 2$ dans $X$, on a $\Gamma(U,\Oo_X(D))\simeq \Gamma(X,\Oo_X(D))$.
\end{prop}
\begin{proof}
Encore une fois, on peut traiter le problème localement et supposer $X=\spec A$ affine. La restriction est injective et comme en \ref{extregularnormal}, on remarque que $U$ contient tous les premiers $p$ de hauteur 1. On écrit $D=a_{p_1}Y_{p_1}+...+a_{p_r}Y_{p_r}$ comme en \ref{divaff}, et on considère les injections dans les tiges $\Gamma(X,\Oo_X(D))_p=p^{-a_p}A_p$. On construit ainsi l'inverse de la restriction $\Gamma(U, \Oo_X(D))\xhookrightarrow{}\Gamma(X,\Oo_X(D))=\bigcap_{ht(p)=1}p^{-a_p}A_p$, en procédant comme en \ref{extregularnormal}.
\end{proof}

\subsection{Diviseurs de Cartier et groupe de Picard}

Sur des variétés plus générales, par exemple avec des singularités, les anneaux locaux associés aux diviseurs premiers ne sont plus en général des DVR. On a alors des difficultés pour définir par exemple le diviseur d'une fonction rationnelle. On a néanmoins la notion générale de diviseur de Cartier, qui dans le cadre des variétés normales irréductibles correspondra aux diviseur de Weil "localement principaux".

\begin{defn}[Diviseur de Cartier]
Soit $X$ une variété irréductible. Un diviseur de Cartier sur $X$ est une section globale du faisceau $k(X)^\times/\mathcal{O}_X^\times$. Ainsi un diviseur de Cartier est la donnée d'une famille $(U_i, f_i)_{i\in I}$ telle que pour tout $i$, $U_i$ est un ouvert de $X$, $(U_i)_{i\in I}$ est un recouvrement de $X$, $f_i\in k(X)^\times$, et pour tout $i,j \in I$, $f_if_j^{-1}\in \mathcal{O}_X^\times(U_i\cap U_j)$.\\
Un diviseur de cartier est dit principal si il provient d'une section globale de $k(X)^\times$ c'est à dire d'une fonction rationnelle. Deux diviseurs de Cartier sont dits linéairement équivalents si ils sont égaux modulo le sous-groupe des diviseurs principaux. Le groupe quotient se note CaCl$(X)$.
\end{defn}





Soit $X$ une variété irréductible. On remarque que pour un diviseur de Cartier $D=(U_i, f_i)_{i\in I}$ de $X$, $\mathcal{O}_X(D)_{|U_i}$ est le $(\mathcal{O}_X)_{|U_i}$-module libre de base $(f_i^{-1})$. Il est donc localement libre de rang 1, c'est à dire inversible. On récupère facilement $D$ à partir de $\mathcal{O}_X(D)$ en prenant un recouvrement qui le trivialise. Enfin, pour tout sous-faisceau inversible de $k(X)^\times$ on construit de la même manière un diviseur de Cartier. On a donc une correspondance bijective entre diviseurs de Cartier et sous-faisceau inversible de $k(X)^\times$. Par cette correspondance, deux diviseurs sont linéairement équivalents si et seulement si les faisceaux inversibles sont isomorphes. On a ainsi définit une application injective CaCl$(X)\rightarrow$ Pic$(X)$ dont on voit facilement que c'est un morphisme de groupe. Comme $X$ est supposé irréductible, c'est un isomorphisme car tout faisceau inversible est isomorphe à un sous-faisceau inversible de $k(X)^\times$. En résumé on a le résultat suivant:


\begin{prop}\label{isopic}
Soit $X$ une variété irréductible. L'application $D\mapsto \mathcal{O}_X(D)$ définit un isomorphisme de groupe CaCl$(X)\simeq $ Pic$(X)$.
\end{prop}


On suppose à nouveau $X$ normale et irréductible. Dans ce cadre, tout diviseur de Cartier $(U_i, f_i)_{i\in I}$ définit un unique diviseur de Weil de la façon suivante. Pour tout diviseur premier $Y$, on choisit un indice $i\in I$ tel que $U_i\cap Y\neq \emptyset$ et on prend $v_Y(f_i)$ pour coefficient de $Y$. Cette somme est finie par la même preuve que \ref{noetherienPWDivBienDef}. Par ailleurs elle ne dépend pas du choix des indices car si $j$ est un autre indice possible, $f_if_j^{-1}\in \Oo_X^\times(U_i\cap U_j)$ par définition, donc $v_Y(f_i)=v_Y(f_j)$. On a ainsi un diviseur de Weil tel que sa restriction à tout ouvert du recouvrement $(U_i)_{i\in I}$ est principal. D'où la terminologie "localement principal". Ce constat permet de voir CaCl$(X)$ comme un sous-groupe de Cl$(X)$ (on vérifie que les diviseurs principaux se correspondent).

Ce sous-groupe est propre en général (cf \cite{Hartshorne} 6.11.3). En revanche, si $X$ est lisse, tout diviseur de Weil est localement principal . En effet dans ce cas, les anneaux locaux sont factoriels, on obtient ainsi en tout point une équation locale d'un diviseur premier car un idéal premier de hauteur 1 d'un anneau factoriel est principal, ce qui permet de conclure.\\

\begin{prop}
Soit $X$ une variété normale irréductible, $D,E\in\wdiv X$ avec $D$ de Cartier. Alors le morphisme naturel $\alpha:\Oo_X(D)\otimes_{\Oo_X}\Oo_X(E)\rightarrow \Oo_X(D+E)$ est un isomorphisme.
\end{prop}
\begin{proof}
Écrivons $D=(U_i,f_i)_{i\in I}$. Alors sur chaque $U_i$, $\alpha$ induit un isomorphisme de $\Oo_X$-module. En effet on a un morphisme inverse, il s'agit de la multiplication par $f_i^{-1}$ composée avec l'isomorphisme  $\Oo_{U_i}(E)\simeq   \Oo_{U_i}(D)\otimes_{\Oo_{U_i}}\Oo_{U_i}(E)$.
\end{proof}

Par analogie avec les diviseurs de Weil, un diviseur de Cartier $D=(U_i, f_i)_{i\in I}$ est dit effectif si pour tout $i\in I$, $f_i\in \Oo_X(U_i)$. Dans ce cas $\Oo_X(-D)$ est un sous $\Oo_X$-module de $\Oo_X$, c'est concrètement le faisceau d'idéaux localement généré sur chaque $U_i$ par $f_i$. D'après \ref{dimsousvariete} cela définit un sous-schéma fermé de $X$ de codimension 1. L'inclusion $\Oo_X(-D)\xhookrightarrow{} \Oo_X$ est une section globale de $\Hh om(\Oo_X(-D), \Oo_X)\simeq \Oo_X(D)$ appelée section canonique et notée $1_D$ puisqu'elle correspond à la multiplication par $1$. Réciproquement, la donnée d'un couple $(\Ll, s)$ constitué d'un faisceau inversible sur $X$ et d'une section globale définit un diviseur de Cartier effectif de la manière suivante. Soit $(U_i)_{i\in I}$ un recouvrement qui trivialise $\Ll$. Sur chaque $U_i$ on a un isomorphisme $\phi_i:\Ll_{|U_i}\rightarrow\Oo_{X|U_i}$. On voit que $(U_i,\phi_i(s))_i$ définit un diviseur de Cartier effectif indépendant du choix des $\phi_i$, et donc du couple $(\Ll,s)$ à isomorphisme près. On l'appelle le diviseur des zéros de $s$ et on le note $\divi_D(s)$. Les deux procédés que l'on vient de décrire sont inverses l'un de l'autre, on obtient ainsi une correspondance bijective:

$$
\left\{
\begin{matrix}
\text{Diviseurs de Cartier effectif sur }X
\end{matrix}
\right\}
\leftrightarrow
\left\{
\begin{matrix}
\text{couples }(\mathcal{L}, s)\text{ constitués d'un faisceau}\\
\text{inversible et d'une section globale}
\end{matrix}
\right\}
$$

Considérons un diviseur de Cartier $D$ quelconque et $\Oo_X(D)$ le sous-faisceau inversible de $k(X)$ qui lui correspond. En faisant varier $s$ non-nulle dans les couples $(\Oo_X(D),s)$ tels que ci-dessus, on obtient tous les diviseurs de Cartier effectifs linéairement équivalent à $D$. En effet, $\divi_D(s)$ est par définition un diviseur effectif linéairement équivalent à $D$. Réciproquement un diviseur effectif linéairement équivalent à $D$ s'écrit $D+\divi(s)\geq 0$ où $s$ est donc une section globale $s$ de $\Oo_X(D)$.

\subsection{L'espace projectif $\PP_k^n$}

\subsubsection{Faisceaux inversibles sur $\PP_k^n$}

\subsubsection{Morphismes vers l'espace projectif}
\label{morphismeproj}
Soit $X$ une variété munie d'un morphisme $(f,f^\sharp):X\rightarrow \PP^n_k$, où $\PP^n_k=$Proj$k[x_0,...,x_n]$. On considère le faisceau tordu de Serre $\Oo(1)$ sur $\PP^n_k$. C'est un faisceau inversible engendré par les sections globales $x_0,...,x_n$. $f^*\Oo(1)$ est également inversible et on a un morphisme canonique de $\Oo_{\PP^n_k}$-module $\alpha:\Oo(1)\rightarrow f_*f^*\Oo(1)$ où la structure de $\Oo_{\PP^n_k}$-module sur $f_*f^*\Oo(1)$ est donnée par $\lambda.t=f^\sharp(\lambda)t$ pour tout ouvert $V\subset \PP^n_k$, $\lambda\in \Oo_{\PP^n_k}(V)$, $t\in f_*f^*\Oo(1)(V)$. On définit des sections globales $f^*(x_0):=s_0:=\alpha(\PP^n_k)(x_0),...,f^*(x_n):=s_n:=\alpha(\PP^n_k)(x_n)$ de $f^*\Oo(1)$ dont on voit facilement qu'elles engendrent $f^*\Oo(1)$.\\
Considérons un faisceau inversible $\Ll$ sur une variété $Y$, une section globale $l$, et un morphisme $(g,g^\sharp):X\rightarrow Y$. On définit l'ouvert 
$$Y_l:=\lbrace y\in Y\mid \Oo_{Y,y}l_y=\Ll_y\rbrace=\lbrace y\in Y\mid l_y\notin m_y\Ll_y\rbrace$$
C'est tout simplement le complémentaire du support du diviseur des zéros associé au couple $(\Ll,l)$. Si $\Ll=\Oo_Y$ il s'agit de l'ouvert principal $Y_l$, et on a de plus $g^{-1}(Y_l)=X_{g^\sharp(l)}=X_{g^*(l)}$ car $g^*\Oo_Y=\Oo_X$ et $\alpha=g^\sharp$ dans ce cas. Revenant dans le cas général, on voit facilement que $g^*\Ll$ est inversible et en raisonnant localement sur des ouverts affines qui trivialisent $\Ll$, on voit d'après ce qui précède que $g^{-1}(Y_l)=X_{g^*(l)}$. D'autre part, si $l_1,...,l_n$ sont des sections globales qui engendrent $\Ll$, on voit que $(Y_{l_i})$ est un recouvrement de $Y$ qui trivialise $\Ll$. En effet, sur chaque $Y_{l_i}$, on a un isomorphisme $\Oo_{Y|Y_{l_i}}\rightarrow \Ll_{Y|Y_{l_i}}$, $\lambda \mapsto \lambda l_i$, et par hypothèse, l'intersection des complémentaires des $Y_{l_i}$ est vide. Enfin, on note sans ambiguïté $l_j/l_i$ l'unique élément de $\Oo_{Y|Y_{l_i}}(Y_l)$ tel que $l_j/l_i.l_i=l_j$ par l'isomorphisme précédent.\\
Revenons au cas initial et notons $U_i=D_+(x_i)=\Oo(1)_{x_i}$. La $k$-algèbre $\Oo_{P^n_k}(U_i)$ est engendrée par les éléments $x_j/x_i$ et on a $s_j/s_i.s_i=s_j=\alpha(U_i)(x_j)=\alpha(U_i)(x_j/x_i.x_i)=f^\sharp(x_j/x_i).\alpha(U_i)(x_i)=f^\sharp(x_j/x_i).s_i$. On a donc nécessairement $f^\sharp(x_j/x_i)=s_j/s_i$. Comme les $U_i$ sont affines, cela définit des morphismes $f_i:X_{s_i}\rightarrow P^n_k$ en composant avec l'inclusion. C'est morphismes se recollent en un unique morphisme, car ils coïncident sur les $X_{s_i}\cap X_{s_j}$. Autrement dit, on récupère $f$ par la donnée des $s_i$, et $f$ est ainsi l'unique morphisme tel que $f^*(x_i)=s_i$.\\
Réciproquement montrons que la donnée d'un faisceau inversible $\Ll$ sur $X$ et de sections globales $l_0,...,l_n$ qui l'engendrent définissent un unique morphisme $f:X\rightarrow \PP^n_k$ tel que $\Ll\simeq f^*\Oo(1)$, ce dernier isomorphisme étant celui qui envoie $f^*(x_i)$ sur $l_i$. Si $f$ existe avec ces propriétés, il est unique d'après ce qui précède. Pour l'existence, on construit comme précédemment des morphismes $f_i:X_{l_i}\rightarrow U_i$ qui se recollent en un morphisme $f:X\rightarrow \PP^n_k$. Par construction, $\Ll$ et $f^*\Oo(1)$ se trivialisent sur le même recouvrement $(f^{-1}(U_i)=X_{l_i}=X_{f^*(x_i)})_i$. On a des isomorphismes locaux $\phi_i:\Ll_{|f^{-1}(U_i)}\rightarrow f^*\Oo(1)_{|f^{-1}(U_i)}$, $l_i\mapsto f^*x_i$. Ce sont des sections locales de $\mathscr{H}om_{\Oo_X}(\Ll,\Oo(1))$ sur un recouvrement de $X$. Pour vérifier qu'elles coïncident aux intersections $X_{l_i}\cap X_{l_j}$, on remarque que l'on a $l_i/l_j=f^\sharp(U_i\cap U_j)(x_i/x_j)= f^*(x_i)/f^*(x_j)\in \Oo_X(X_{l_i}\cap X_{l_j})^\times$. Ces sections se recollent donc en un unique isomorphisme $\Ll\simeq f^*\Oo(1)$ qui est bien l'isomorphisme recherché.

Notons que cette propriété caractérise l'espace projectif $\PP^n_k$ à isomorphisme près. En effet il représente le foncteur (voir stacks)...

\subsubsection{Variétés quasi-projectives, faisceaux inversible très amples}

\begin{defn}
Soit $X$ une variété et $\Ll$ un faisceau inversible sur $X$ engendré par une famille finie de sections globales. Si ces sections définissent une immersion $X\xhookrightarrow{}\PP_k^n$, on dit que $\Ll$ est très ample. Cela revient à dire que $\Ll\simeq i^*\Oo(1)$  pour une immersion $i:\xhookrightarrow{}\PP_k^n$.\\
Si $X$ est normale et irréductible, un diviseur de Cartier $D$ est dit très ample si on a $\Oo_X(D)\simeq i^*\Oo(1)$ pour une immersion $i:\xhookrightarrow{}\PP_k^n$.
\end{defn}

Supposons maintenant $X$ lisse (donc normale) et projective, $D\in \wdiv(X)$, $\Oo_X(D)$ le faisceau inversible associé. On a vu que l'ensemble des diviseurs effectifs linéairement équivalents à $D$ est $\lbrace \divi_D(s)\mid s\in\Gamma(X,\Oo_X(D))\setminus\lbrace0\rbrace \rbrace$. D'autre part deux sections globales $s, s'$ non nulles ont même diviseur des zéros si et seulement si elles sont colinéaires dans le $k$-ev $\Gamma(X, \Oo_X(D))$, en effet dans ce cas $s/s'\in \Oo_X^\times(X)=k^*$, car $k$ est algébriquement clos. Notons enfin que $\Gamma(X, \Oo_X(D))$ est de dimension finie (\cite{Hartshorne} II.5.19). Ainsi l'ensemble des diviseurs effectifs linéairement équivalent à $D$ est naturellement muni d'une structure d'espace projectif, cela amène la définition suivante:
\begin{defn}[Système linéaire, point de base]
Soit $X$ une variété lisse et projective, et $D$ un diviseur. Le système linéaire complet définit par $D$ est l'ensemble des diviseurs effectifs linéairement équivalents à $D$, on le note $|D|$. Un système linéaire est une partie de $|D|$ correspondant à un sous espace projectif. On dit que $p\in X$ est un point de base d'un système linéaire $\PP(V)\subset |D|$ si l'intersection des $X\setminus X_s$ pour $s\in V$ est non vide. 
\end{defn}

Autrement dit dans ce langage, se donner un morphisme $X\rightarrow \PP^n_k$ est équivalent à se donner un système linéaire $\PP(V)\subset |D|$ sans point de base et une base de $V$.