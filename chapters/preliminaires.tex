\chapter{Préliminaires}

\section{Algèbres graduées}

\begin{prop}\label{noethgrad}
Soit $A$ un anneau $\ZZ$-gradué. Les assertions suivantes sont équivalentes:\\
(i) $A$ est noetherien\\
(ii) $A_0$ est noetherien et $A$ est de type fini en tant que $A_0$-algèbre
\end{prop}
\begin{proof}
AtiyahMcdo p106
\end{proof}


\section{Variétés algébriques}

\subsection{Généralités}
La référence principale est \cite{LAGSpringer}. On rappelle ci-dessous les définitions et résultats de base. On rappelle que $k$ est un corps algébriquement clos (voir \hyperref[sec:conventions]{conventions}).
\subsubsection{La topologie de Zariski dans $k^n$}
\begin{defn}[Ens. alg. affine]
Un ensemble algébrique affine est une partie de $k^n$ constituée de zéros communs à un ensemble de polynômes de $S:=k[X_1,...,X_n]$.\\
Soient $\Sigma_1,\Sigma_2$ deux ensembles algébriques affine de $k^{n_1}$ (resp. $k^{n_2}$). Un morphisme est une application $\phi:\Sigma_1,\rightarrow \Sigma_2$ telle que les composantes de $\phi$ soient polynomiales.
\end{defn}
On remarque que les ensembles algébriques affines munis de leur morphismes constituent une catégorie. 
Les ensembles algébriques affines sont de la forme $\mathcal{V}(I):=\lbrace x\in k^n\mid \forall P\in I, P(x)=0\rbrace$ où $I$ est un idéal de $S$. Ils sont stables par intersection quelconque, et on a $\mathcal{V}(S)=\emptyset$ et $\mathcal{V}(\lbrace 0\rbrace)=k^n$. Ainsi les ensembles algébriques affines constituent les fermés d'une topologie, dite de Zariski.\\
Par ailleurs, à tout ensemble algébrique affine $\Sigma$ on associe l'idéal $\mathcal{I}(\Sigma):=\lbrace P\in k[X_1,...X_n] \mid P(x)=0, \forall x\in\Sigma \rbrace$. L'ensemble des morphismes $\Sigma\rightarrow k$ est doté d'une structure de k-algèbre évidente que l'on note $k[\Sigma]$. Cette algèbre est naturellement isomorphe à $S/\mathcal{I}(\Sigma)$. Elle est de type fini réduite, on dit que c'est une $k$-algèbre affine. D'après le Nullstellensatz on a $\mathcal{I}(\mathcal{V}(I))= \sqrt{I}$, d'où une bijection entre les sous-ensembles algébriques affines de $\Sigma$ et les idéaux radicaux de $k[\Sigma]$. En particulier les points de $\Sigma$ sont en bijection avec les idéaux maximaux de $k[\Sigma]$. On note $\spm (k[\Sigma])$ cet ensemble.
\begin{prop}\label{eqaffinevaralg}
La construction qui à $\Sigma$ associe $k[\Sigma]$ est fonctorielle. Le foncteur est pleinement fidèle et essentiellement surjectif.
\end{prop}
\begin{proof}

\end{proof}
\subsubsection{Irréductibilité}
\begin{defn}[espace irréductible]

\end{defn}
\subsubsection{La catégorie des $k$-variétés algébriques affines}
\begin{defn}[Espace annelé]\label{espaceannele}
Un espace annelé est un espace topologique $X$ muni d'un faisceau de $k$-algèbres de fonctions sur $X$ à valeurs dans $k$. On dénote $\mathcal{O}_X$ ce faisceau.\\
Soient $(X, \mathcal{O}_X), (Y, \mathcal{O}_Y)$ deux espaces annelés. Un morphisme d'espace annelé est une application continue $\phi: X\rightarrow Y$ telle que que la pré-composition par $\phi$ induit, pour tout ouvert $U$ de $Y$, un morphisme de $k$-algèbres $\phi^*:\mathcal{O}_Y(U) \rightarrow \mathcal{O}_X(\phi^{-1}U)$.
\end{defn}

\begin{defn}[Fonction régulière]
Soit $\Sigma$ un ensemble algébrique affine, $x\in \Sigma$ et $U$ un ouvert contenant $x$. Une application $f:U\rightarrow k\in Map(U,k)$ est dite régulère en $x$ si $\exists\ g,h\in k[\Sigma]$ et un ouvert $V\subset U\cap D(h)$ contenant $x$ tel que $f(y) = g(y)/h(y), \forall y\in V$.\\
$f$ est dite régulière sur $U$ si elle est régulière en tout point de $U$.
\end{defn}

On voit que les fonctions régulières sur les ouverts de X définissent un faisceau. Ainsi un ensemble algébrique affine muni de la topologie de Zariski et de son faisceau de fonctions régulières est un espace annelé. 

\begin{prop}
Soit $\Sigma$ un ensemble algébrique affine et $f\in k[\Sigma]^*$. On a $\mathcal{O}_\Sigma(D(f))\simeq k[\Sigma][1/f]$.
\end{prop}
\begin{proof}

\end{proof}

\begin{defn}[Variété algébrique affine]
Une $k$-variété algébrique affine est un espace annelé isomorphe à un ensemble algébrique affine.
\end{defn}

Soit $\Sigma$ un ensemble algébrique affine. En utilisant la bijection entre $\Sigma$ et $\spm (k[\Sigma])$, on voit comment définir directement la topologie de Zariski sur $\spm (k[\Sigma])$ ainsi que le faisceau structural, faisant de ce dernier une variété algébrique affine. Concrètement, les fermés de $\spm (k[\Sigma])$ sont les $\mathcal{V}(I):=\lbrace m \in \spm (k[\Sigma]) \mid I\subset m \rbrace$. Les éléments de $k[\Sigma]$ définissent des fonctions sur $\spm (k[\Sigma])$ en les considérant modulo $m$, pour $m \in \spm (k[\Sigma])$. On peut définir le faisceau sur la base des ouverts principaux $D(f):=\lbrace m \in \spm (k[\Sigma]) \mid f \notin m\rbrace$ en posant $\mathcal{O}_{\spm (k[\Sigma])}(D(f))\simeq k[\Sigma][1/f]$\\
Par construction $\spm (k[\Sigma])$ est isomorphe à $\Sigma$. Cela donne une manière intrinsèque de définir une variété algébrique affine, indépendamment d'un plongement dans un espace affine quelconque.\\
On remarque que la catégorie des variétés algébriques est équivalente à celle des ensembles algébriques affines.
\begin{ex}
Soit $X$ une variété algébrique affine et $f\in k[X]$. $(D(f), \mathcal{O}_X(D(f)))$ est une variété algébrique affine.
\end{ex}
\begin{proof}

\end{proof}

\begin{prop}
On note $\mathcal{O}_x$ la $k$-algèbre des fonctions régulières en $x\in X$. C'est par définition $ \underset{x\in U}{\varinjlim} \mathcal{O}(U)$. On a $\mathcal{O}_x\simeq k[X]_{m_x}$ (localisé en l'idéal maximal $m_x$).
\end{prop}
\begin{proof}

\end{proof}

\begin{cor}
Soit $X$ une variété algébrique affine. On a $\mathcal{O}_X(X)\simeq k[X]$.
\end{cor}



\begin{prop}Soient $X,Y$ deux variétés algébriques affines.\\
(i) le produit $X\times Y$ existe dans la catégorie des variétés algébriques affines. De plus on a $k[X\times Y]\simeq k[X]\otimes _k k[Y]$.\\
(ii) Si $X$ et $Y$ sont irréductibles, alors $X\times Y$ aussi.
\end{prop}
\begin{proof}

\end{proof}

\subsubsection{La catégorie des $k$-variétés algébriques}

\begin{cons}\label{gluevar}
Soit $(X_i)_{u\in I}$ une famille finie de prevariétés. Supposons $\forall i,j$ avec $i\neq j$ on ait des ouverts $U_{ij}\subset X_i$ et des isomorphismes $f_{ij}:U_{ij}\rightarrow U_{ji}$ tels que $\forall i,j,k\in I$ distincts on ait:
	\begin{enumerate}
	\item $f_{ij}=f_{ji}^{-1}$
	\item $U_{ij}\cup f_{ij}^{-1}(U_{jk})\subset U_{ik}^{-1}$ et $f_{jk}f_{ij}=f_{ik}$ sur $U_{ij}\cup f_{ij}^{-1}(U_{jk})$
	\end{enumerate}
Alors on définit une prevariété X comme la réunion disjointe des $X_i$ modulo la relation d'équivalence $a\sim f_{ji}(a),\forall a\in U_{ij}\subset X_i$, et $a\sim a \forall a$. La topologie est la topologie finale associée aux inclusions $inc_i:X_i\subset X$. Le faisceau structural est définit par $\Oo_X(U):=\lbrace \phi\in \textrm{Map}(U,k)\mid inc_i^*\phi\in\Oo_{X_i}(inc_i^{-1}(U))\rbrace$
\end{cons}


\subsection{Dimension}
\subsection{Quelques résultats sur les morphismes}
\subsubsection{Généralités}

\begin{defn}[Morphisme affine]
Un morphisme de variétés algébriques $\phi X\rightarrow Y$ est dit affine si pour tout ouvert affine $V\subset Y$, l'image réciproque $\phi^{-1}(V)$ est affine.
\end{defn}

\begin{ex}\label{exaff}
Un morphisme de variétés affines $\phi X\rightarrow Y$ est affine. En effet, soit $V$ un ouvert affine de $Y$ et $U=\phi^{-1}(V)$. En considérant le diagramme commutatif ci-dessous on constate que l'on a $U \simeq (\phi\times i_2)^{-1}(\Delta_Y)=\lbrace (x,\phi(x))\mid x\in U \rbrace \subset X\times V$. Comme $X\times V$ est affine, $U$ aussi.
	\begin{center}
	\begin{tikzcd}
  		U \arrow[r, "\phi"] \arrow[d, "i_1"]& V \arrow[d, "i_2"] \\ 
  		X \arrow[r, "\phi"] & Y
	\end{tikzcd}\\
	\end{center}

\end{ex}

\subsubsection{Dimension des fibres}

\subsubsection{Applications rationnelles}

\subsubsection{Morphismes finis, normalité}

\begin{defn}[Morphisme fini, localement fini]
Soit $f:X \rightarrow Y$ un morphisme de variétés affines. On dit que $f$ est fini si la $k[Y]$-algèbre $(k[X], f^*)$ est finie.\\
On dit qu'un morphisme est localement fini en $x\in X$ si ils existe un morphisme fini $\mu:Y' \rightarrow Y$ et un isomorphisme $\nu$ d'un ouvert de $X$ contenant $x$ sur un ouvert de $Y'$, tel que $\mu\nu =f_{|U}$.
\end{defn}

\begin{prop}Soient $X,Y$ deux variétés algébriques affines irréductibles de même dimension et $f:X \mapsto Y$ un morphisme dominant.\\
Alors il existe $g\in k[Y]^*$ tel que le morphisme induit $f:X_g \mapsto Y_g$ soit fini, surjectif avec des fibres de même cardinal.
\end{prop}
\begin{proof}
Par hypothèse, l'extension $k(Y) \xrightarrow{f^*} k(X)$ est algébrique finie, disons de degré n. En \textbf{caractéristique zéro} on peut trouver $u\in k(X)$ tel que $k(X)=k(Y)[u]$. On remarque que l'on peut imposer $u\in k[X]$. On considère $P:=P_{min}(u, k(Y))=T^n+a_1T^{n-1}+...+a_0$. En réduisant au même dénominateur on a $P\in k[Y]_{v_1}[T]$ pour un $v_1\in k[Y]$. \textcolor{red}{Est ce que $k[X]_{v_1}\simeq k[Y]_{v_1}[u]$ ? A priori pas de raison?}
\end{proof}

Ce résultat reste vrai en caractéristique positive, voir 
~\cite{LAGSpringer} 5.1.6 pour une preuve légèrement différente dans ce cadre. On y montre que le cardinal de la fibre générale est $[k(X):k(Y)]_s$. En revanche pour le corollaire immédiat suivant, la caractéristique zéro est essentielle (penser par exemple au morphisme de Frobenius $\AAA^1 \xrightarrow{x \mapsto x^p} \AAA^1$ ). 
\begin{cor}
Avec les hypothèses de 5, si de plus $f$ est injectif, alors il existe $g\in k[Y]^*$ tel que le morphisme induit $f:X_g \mapsto Y_g$ soit un isomorphisme.
\end{cor}

\begin{prop}\label{facto}
Soit $f:X \mapsto Y$ un morphisme dominant de variétés irréductibles. Soit $g:X \rightarrow Z$ constant sur les fibres de $f$. Alors il existe $h\in k[Y]^*$ et une factorisation
	\begin{tikzcd}
		X_h \arrow[r,"g"] \arrow[d,"f"] & Z \\
		Y_h \arrow[ru, dashed]
	\end{tikzcd}
\end{prop}
\begin{proof}
	\begin{multicols}{2}
	On considère $\phi=(f,g):X\rightarrow Y\times Z$ et le diagramme commutatif ci-contre. Comme $f$ est dominant, $\pi_1$ l'est aussi. De plus $\overline{\phi(X)}$ est irréductible et $\phi(X)$ contient un ouvert dense de $\overline{\phi(X)}$. Par ailleurs comme $g$ est constante sur les fibres de $f$ on vérifie que $\pi_1$ est injective sur $\phi(X)$. Par le corollaire précédent, $\pi_1$ réalise un isomorphisme $\overline{\phi(X)}_h \xrightarrow{\pi_1} Y_h$ pour un $h\in k[Y]^*$. Finalement, le morphisme recherché est  $Y_h \xrightarrow{\pi_2\pi_1^{-1}} Z$ 
	
	\columnbreak
	\begin{center}
	\begin{tikzcd}
  		& X \arrow[ldd,bend right,swap, "f"] \arrow[d, "\phi"] \arrow[rdd, bend left,"g"]  &\\ 
  		& \overline{\phi(X)} \arrow[ld,swap,"\pi_1"] \arrow[d,"i=\subset"] \arrow[rd,"\pi_2"]  &\\ 
		Y & \arrow[l,"p_1"]  Y\times Z \arrow[r,swap,"p_2"]  & Z
	\end{tikzcd}\\
	\end{center}
	\end{multicols}
\end{proof}

\begin{prop}
Soit $f:X \mapsto Y$ un morphisme de variétés affines et $x\in X$. Si la fibre de $f(x)$ est finie, alors $f$ est localement fini en $x$.
\end{prop}

\begin{thm}
Soit $f:X \mapsto Y$ un morphisme bijectif de variétés irréductibles avec $Y$ normale. Alors $f$ est un isomorphisme.
\end{thm}



\section{Groupes algébriques affines}
\subsection{Généralités}
\subsection{G-variétés, représentations}

\begin{defn}[G-variété]
Soit $G$ un groupe algébrique. Une $G$-variété est une variété algébrique $X$ sur laquelle $G$ agit algébriquement. C'est à dire qu'on a un morphisme de groupes de $G$ dans le groupe d'automorphismes $X$.
\end{defn}

\begin{prop}
Soit $G$ un groupe algébrique, $X$ une $G$-variété et $x\in X$.\\
(i) $G.x$ est ouvert dans $\overline{G.x}$.\\
(ii) Toute composante irréductible de $G.x$ a pour dimension $dim (G)-dim(G.x)$.\\
(iii) $\overline{G.x}\setminus G.X$ est une union d'orbites de dimension $<dim(\overline{G.x})$.\\
(iv) $G.x$ est ouvert dans $\overline{G.x}$.
\end{prop}
\begin{proof}
On suppose d'abord $G$ connexe. \\
D'après \ref{}, $G.x$ contient un ouvert dense $U$ de $\overline{G.x}$. Comme, $G$ est réunion de translatés de $U$, cela prouve (i).\\ 
D'après \ref{}, il existe un ouvert dense de $G.x$ tel que toute les fibres de cet ouvert ont pour dimension dim($G$)$-$dim($G.x$)$=$dim($G_x$). Cela prouve (ii).\\
$\overline{G.x}\setminus G.x$ est un fermé propre de $\overline{G.x}$ donc de dimension inférieure d'après \ref{}. Par ailleurs, $G$ stable donc est réunion d'orbites. Cela prouve (iii).\\
Enfin si dim($G.x$) est minimal, $\overline{G.x}\setminus G.x$ est vide ce qui prouve (iv).\\
Si $G$ n'est pas connexe, on écrit $G=\cup_{i=1}^{n}g_iG^\circ .x$ avec $g_1=e$. D'où $\overline{G.x}=\cup_{i=1}^{n}\overline{g_iG^\circ .x}$. Les $\overline{g_iG^\circ}$ sont égales où disjointes, c'est donc la décomposition en composantes irréductibles. On construit un ouvert de $\overline{G.x}$ inclus dans $G.x$ en posant $U=G^\circ .x\setminus \cup_{i=2}^{n}\overline{g_iG^\circ .x}$ puis on conclus comme dans le cas connexe pour (i).\\
On a dim($G^\circ$)$-$dim($(G^\circ) _x$)$=$dim($G$)$-$dim($G_x$) car $(G_x)^\circ \subset(G^\circ) _x \subset G_x$, d'où dim($G_x$)$=$dim($(G^\circ) _x$). Or chaque composante de $G.x$ est l'adhérence d'un orbite pour $G^\circ$, d'où (ii) d'après le cas connexe.\\
On a $\overline{G.x}\setminus G.x=\cup_{i=1}^{n}\overline{g_iG^\circ .x}\setminus g_iG^\circ .x=\cup_{i=1}^{n}g_i(\overline{G^\circ .x}\setminus G^\circ .x)$ qui est une union finie de fermés de dimension inférieure à $\overline{G .x}$ ce qui prouve (iii).\\
On utilise le même argument pour prouver (iv) dans le cas général.
\end{proof}

\begin{defn}
Une représentation de $G$, ou $G$-module (rationnel) est un couple $(V, \rho)$ où $V$ est un $k$-espace vectoriel de dimension finie et $\rho$ un morphisme de groupes algébriques de $G$ dans $GL(V)$.\\
On étend cette définition au cas où $V$ est de dimension infinie, on demande alors que $V$ soit réunion de $G$-modules de dimension finie.\\
On dit qu'un $G$-module est simple si il n'admet pas de sous $G$-module non trivial. On dit qu'un $G$-module est semi-simple si tout sous $G$-module admet un $G$-module supplémentaire.
\end{defn}

\begin{prop}
Soit $G$ un groupe algébrique et $X$ une $G$-variété. $k[X]$ est naturellement muni d'une action $(g.f)(x):=f(g^{-1}.x), \forall f\in k[X], g\in G,x\in X$\\
Muni de cette action, $k[X]$ un $G$-module.
\end{prop}
\begin{proof}
On note $a:G\times X\rightarrow X$ le morphisme associé à l'action de $G$. Cela donne $\forall g,x \in G\times X$, $a^*(f)(g,x)=g^{-1}.f(x)=\sum_{i=1}^r\phi_i(g)\psi_i(x)$, d'où $g.f=\sum_{i=1}^r\phi_i(g)\psi_i\in k[X]$. Ainsi les translaté $g.f$ pour $g\in G$ engendrent un $k$-ev $V(f)$ de dimension finie et $G$-stable. \\
De plus l'action est algébrique. En effet $\forall l\in V(f)^*$, qu'on prolonge en $l'\in \textrm{Vect}_k(\psi_1,...,\psi_r)^*$. On a $\forall h\in G, g\mapsto l(g.(h.f))=\sum_{i=1}^r\phi_i((gh)^{-1})l'(\psi_i)\in k[G]$.\\
Finalement $k[X]=\cup_{f\in k[X]}V(f)$ est un G-module.
\end{proof}

\begin{thm}\label{embed}
Soit $G$ un groupe algébrique et $X$ une $G$ variété. $X$ est isomorphe en tant que $G$-variété à une sous $G$-variété fermée d'un $G$-module de dimension finie.
\end{thm}

\begin{cor}
Tout groupe algébrique est linéaire.
\end{cor}

\begin{defn}
Un groupe algébrique $G$ est dit réductif si tout $G$-module est semi-simple.
\end{defn}

\begin{ex}
Les groupes finis et les quasitores sont réductifs.
\end{ex}

\subsection{Groupes quotients}
\begin{thm}
Soit $G$ un groupe algébrique et $H\leq G$ fermé. \\Alors il existe un $G$-module $V$ de dimension finie et une ligne $L\subset V$ telle que $H=Stab_G(L):=\lbrace g\in G\mid g.v\in L,\forall v\in L\rbrace$.
\end{thm}

\begin{thm}
Soit $G$ un groupe algébrique et $H\lhd G$ fermé. \\Alors il existe un $G$-module $(V, \rho)$ de dimension finie tel que $H=\Ker{\rho}$.
\end{thm}
Le théorème suivant est le résultat principal de cette section. Il prouve l'existence des groupes quotients dans la catégorie des groupes algébriques. Le groupe quotient est alors unique à isomorphisme près, c'est une conséquence formelle de la propriété universelle du quotient.
\begin{thm}[Car. 0]\label{groupequotient}
Soient $G, H, (V, \rho)$ comme dans le théorème précédent, et $f:G \rightarrow G'$ un morphisme de groupes algébriques tel que $H\subset \Ker f$.\\
Alors il existe une unique factorisation 	
	\begin{tikzcd}
		G \arrow[r,"f"] \arrow[d,"\rho"] & G' \\
		\rho(G) \arrow[ru, "\exists ! \phi", swap, dashed]
	\end{tikzcd}
\end{thm}
\begin{proof}
	Le morphisme $\phi$ recherché existe en tant que morphisme de groupes abstraits, il est G-équivariant pour les actions naturelles de G sur $\rho(G)$ et $G'$ via $\rho$ et $f$. Concrètement cela signifie $\forall g_1, g_2 \in G, \phi(\rho(g_1)\rho(g_2))=f(g_1)\phi(\rho(g_2))$. Si $G$ est connexe, d'après la proposition \ref{facto}, $\phi$ est algébrique sur un ouvert $U$ non-vide de $\rho(G)$. Or on a un recouvrement de $\rho(G)$ par des $g.U$. En écrivant pour $x\in g.U,  \phi(x)=f(g)\phi(g^{-1}.x))$, on constate que $\phi$ est un morphisme de groupes algébriques.
	\\Supposons $G$ quelconque mais $H\leq G^\circ$. Comme $\phi$ est algébrique sur le sous-groupe $G^\circ/H$ d'après ce qui précède, on a $\phi$ algébrique partout à nouveau par G-équivariance.\\
	On peut se ramener au cas précédent en procédant en deux étapes. Dans un premier temps, on quotiente par le sous-groupe normal connexe $H^\circ$ (on a bien $H^\circ\leq G^\circ$), puis on quotiente par le sous-groupe normal fini $H/H^\circ$. Il reste donc à prouver le cas $H$ fini, c'est un corollaire direct du théorème \ref{goodquotientthm}.
	\end{proof}
\subsection{Quasitores, et actions de quasitores}

\subsubsection{Quasitores}
Soit $G$ un groupe algébrique. Le groupe $X^*(G)$ des caractères de $G$ est un sous-groupe de $k[G]^\times$. On remarque que $X^*(.)$ est un foncteur contravariant de la catégorie des groupes algébriques dans la catégorie des groupes abéliens de type fini, l'image d'un morphisme $G_1\xrightarrow{\phi}G_2$ étant simplement la (co)-restriction $\phi^* \,_{|X^*(G_2)}^{|X^*(G_1)}$ du co-morphisme $\phi^*$ entre les algèbres de coordonnées. On signale qu'en caractéristique $p>0$, les $X^*(G)$ ont de plus la propriété d'être sans $p$-torsion. Tout ce qui suit reste vrai en caractéristique p, avec cette contrainte supplémentaire sur les groupes de caractères.

\begin{ex}
\begin{enumerate}
\item $X^*(GL_n)=\lbrace \textrm{det}^k\mid k\in \ZZ \rbrace\simeq\ZZ$. En effet, $k[GL_n]^\times =\lbrace \lambda\textrm{det}^k\mid k\in \ZZ, \lambda\in k^* \rbrace$, puisque $k[X_{ij}]$ est factoriel et det irréductible.
\item $X^*(SL_n)={1}$ car $D(SL_n)=SL_n$.
\item Les unités de $k[\GG_m]$ sont les monômes. On en déduit que les caractères sont exactement les $t\mapsto t^k, k\in \ZZ$. Par ailleurs, on remarque que $X^*(G_1\times G_2)=X^*(G_1)\times X^*(G_2)$, d'où $X^*(D_n)=\lbrace \textrm{monômes à coefficient unitaire} \rbrace \simeq \ZZ^n$.
\end{enumerate}
\end{ex}

On remarque que les caractères de $D_n$ engendrent $k[D_n]$ comme $k$-ev, ils en forment donc une $k$-base par le lemme de Dedekind qui assurent que les caractères sont libres dans Map$(D_n, k)$. On a plus généralement:

\begin{prop}
Soit $G$ un groupe algébrique. Les assertions suivantes sont équivalentes:
\begin{enumerate}
\item $G$ est diagonalisable (i.e. $\simeq$ à un sous-groupe fermé de $D_n$)
\item $k[G]=\textrm{Vect}_k(X^*(G))$.
\item Tout $G$-module est somme directe $G$-modules de dimension 1.
\end{enumerate}
\end{prop}
\begin{proof}
\begin{enumerate}
\item 1)$\implies$2) La restriction $k[D_n]\xrightarrow{res} G$ est surjective et la restriction d'un caractère est un caractère.
\item 2)$\implies$3) $G$ est abélien, car $\forall \chi\in X^*(G),\,g,h\in G, \chi(gh)=\chi(hg)$. C'est donc vrai pour toute fonction régulière, on en conclut $gh=hg$.\\
On observe que l'action naturelle de $G$ sur $k[G]$ est semi-simple. En effet, les caractères forment une base de diagonalisation, $G$ est donc semi-simple. Soit $(V,\rho)$ le $G$-module considéré, que l'on peut supposer de dimension finie. Par la décomposition de Jordan,  $\rho(G)$ est un sous groupe fermé de $GL_n$, abélien et semi-simple. Il est donc conjugué à un sous-groupe fermé de $D_n$.
\item 3)$\implies$1) On peut supposer $G\subset GL_n$, et considérer l'action naturelle sur $k^n$ après choix d'une base $(e_1,...,e_n)$. Par hypothèse, on peut écrire $k^n=(f_1)\oplus...\oplus (f_n)$, avec les $(f_i)$ sous $G$-module de dimension 1. $G$ est donc conjugué à un sous-groupe de $D_n$.
\end{enumerate}
\end{proof}

Ce constat motive la définition suivante:

\begin{defn}
Un quasitore est un groupe algébrique $G$ tel que $k[G]=\textrm{Vect}_k(X^*(G))$. Un tore est un quasitore connexe.
\end{defn}

On travaille désormais dans la catégorie des quasitores. On considère un quasitore $G$ et le groupe $\chi^{**}(G):=(\chi^*)^2(G)$.

\begin{prop}
$G$ et $\chi^{**}(G)$ sont naturellement isomorphes en tant que groupes abstraits, et aussi en tant que groupes algébriques par transport de structure. L'isomorphisme est $ev_G:G\rightarrow \Hom (X^*(G),\GG _m), g\mapsto (\chi\mapsto \chi(g))$.
\end{prop}
\begin{proof}
$ev_G$ est injective: Soit $g\in G$ tel que $\chi(g)=1=\chi(e_G),\forall \chi\in X^*(G)$. Alors $g=e_G$ car $G$ est un quasitore.\\
$ev_G$ est surjective: Soit $\phi\in\chi^{**}(G)$. On a un prolongement unique de $\phi$ en un morphisme de $k$-algèbre $k[X^*(G]=k[G]\rightarrow k$ qui est donc de la forme $k[G]\rightarrow k, f\mapsto f(g)$ pour un $g\in G$. En restreignant à $X^*(G)$, on trouve que $\phi=ev_G(g)$.
\end{proof}

Autrement dit on a un isomorphisme de foncteurs $(\chi^*)^2(.)\simeq Id$, d'où une équivalence de catégories entre les quasitores et les groupes abéliens de type fini. Une autre façon de voir cela est d'introduire l'algèbre de groupe d'un groupe abélien de type fini M, c'est par définition $k[M]=\lbrace \sum_{finie} \lambda_gg,\,\lambda_g\in k,g\in G \rbrace$ avec la multiplication définie par l'opération de groupe de $G$. La propriété suivante montre que l'on construit ainsi un autre inverse de $\chi^*(.)$
\begin{prop}
Soient, $M, M_1, M_2$ des groupes abéliens de type fini, et $G$ un quasitore.\\
Alors $k[M]$ est de type fini, réduite et on a $k[M_1\oplus M_2]\simeq k[M_1]\otimes k[M_2]$. De plus, $k[M]$ est naturellement muni d'une structure d'algèbre de Hopf et on a $k[G]=k[X^*(G)]$ et donc $G=\spm\circ k[X^*(G)]$.
\end{prop}
\begin{proof}
On a deux morphismes d'algèbre $k[M_1]\rightarrow k[M_1\oplus M_2], e_{m_1}\mapsto e_{(m_1,0)}$ et $k[M_2]\rightarrow k[M_1\oplus M_2], e_{m_2}\mapsto e_{(0,e_{m_2})}$, d'où l'existence d'un morphisme $k[M_1]\otimes k[M_2]$ dont on vérifie que c'est un isomorphisme.\\
Comme on a $M\simeq \ZZ^r\oplus (\oplus_{i=1}^r \ZZ/d_i\ZZ)$, il suffit de traiter les cas $M=\ZZ$ et $M=\ZZ/d\ZZ$. On a $k[\ZZ]\simeq k[t,t^{-1}]$ qui est intègre, de type fini, et réduite. On a $k[\ZZ/d\ZZ]\simeq k[t]/(t^d-1)$. On voit, par le théorème chinois par exemple, que cette algèbre de type fini non-intègre est réduite si et seulement si les racine de $t^d-1$ sont simples, ce qui est le cas en caractéristique zéro.\\
Enfin la structure d'algèbre de Hopf sur $k[M]$ est donnée par $\Delta(e_m)=e_m\otimes e_m,\, i(e_m)=e_{-m},\, e(e_m)=e_0$.
\end{proof}

\begin{cor}
Soit $G$ un quasitore. Alors:
\begin{enumerate}
\item $G$ est isomorphe au produit direct d'un tore et d'un groupe abélien fini.
\item $G$ est un tore $\iff X^*(G)$ est libre de type fini $\iff G$ est connexe.
\end{enumerate}
\end{cor}

\subsubsection{Action d'un quasitore sur une variété affine}

Soient $\chi_1,...,\chi_r$ un choix de caractères de $D_n$. On a une action dite diagonale de $D_n$ sur $k^r$ associée à ce choix par: $g.x = (\chi_1(g)x_1,...,\chi_r(g)x_r)$. On voit, que $k[k^r]=k[t_1,...,t_r]$ est naturellement graduée par le sous-groupe de $X^*(D_n)=\oplus_{i=1}^n\ZZ t_i$ engendré par les $\chi_i$. En fait, on observe ci-dessous que les foncteurs $\spm$ et $k[.]$ réalisent une équivalence de catégorie entre les variétés affines munies d'une action d'un quasitore et les algèbres graduées par un groupe abélien de type fini. La seule chose à vérifier étant que ces restriction sont bien définies, l'équivalence découlant alors de celle entre les variétés affines et les $k$-algèbres affines.

\begin{cons}
Soit $G$ un quasitore et X une $G$-variété affine. Le $G$-module $k[X]$ est somme directe de $G$-module de dimension 1 d'après la partie précédente et on peut écrire: 
\begin{center}
$k[X]=\bigoplus_{\chi \in X^*(G)}V_\chi$,  où $V_\chi:=\lbrace f\in k[X]\mid g.f=\chi(g)f,\, \forall g\in G$
\end{center}
On voit immédiatement que cette somme directe est $X^*(G)$-graduée. De plus, cette construction est fonctorielle, en effet en considérons les diagrammes commutatifs ci-dessous. Le premier est la traduction de la donnée d'un morphisme d'un morphisme $(\widetilde{\phi},\phi)$ d'une $G$-variété affine $X$ vers une $G'$-variété affine $X'$. Le second est obtenu par passage aux algèbres de coordonnées.
\begin{multicols}{2}
	\begin{center}
		\begin{tikzcd}
  		G\times X \arrow[r, "a_1"] \arrow[d, "\widetilde{\phi} \times \phi"] & X \arrow[d, "\phi"] \\ 
  		G'\times X' \arrow[r, "a_2"] & X'
	\end{tikzcd}
	\end{center}

	\columnbreak
	\begin{center}
		\begin{tikzcd}
  		k[G]\otimes k[X]   & k[X] \arrow[l, "a_1^*"] \\ 
  		k[G]'\otimes k[X'] \arrow[u, "\widetilde{\phi}^* \otimes \phi^*"] & X' \arrow[u, "\phi"] \arrow[l, "a_2^*"] 
	\end{tikzcd}
	\end{center}
\end{multicols}
On obtient alors pour $f\in V'_{\chi'}\subset k[X'],g\in G,x\in X,\, \phi^*(f)(g.x)=a_1^*\phi^*(f)(g, x)=(\widetilde{\phi}\otimes\phi)a_2^*(f)(g,x)=\widetilde{\phi}^*(\chi')(g)\phi^*(f)(x)$ donc $\phi^*(f)\in V_{\widetilde{\phi}^*(\chi')}$, cela montre que $(\widetilde{\phi},\phi)$ est un morphisme d'algèbres graduées. Le reste des propriétés découlent du fait que $X^*(.)$ et $k[.]$ sont des foncteurs. 
\end{cons}

\begin{cons}
Soit $K$ un groupe abélien de type fini et $A$ une $k$-algèbre $K$-graduée. On pose $X:=\spm(A)$ et on choisit des générateurs homogènes $f_{\omega_1},...,f_{\omega_r}$ de $A$, ce qui donne une immersion fermée $i:X\rightarrow k^r, x\mapsto (f_1(x),...,f_r(x))$. On transporte la graduation à $k[i(X)]=k[t_1,...,t_r]/\Ker i^*$ en posant $deg(t_i)=\omega _i$. Enfin, on munit $i(X)$ de l'action diagonales associée aux caractères $\chi^{\omega_1},...,\chi^{\omega_r}$ de $G:=\spm(k[K])$. Vu comme cela on a concrètement pour $f\in k[i(X)]$, $f(g.t)=f(\chi^{\omega_1}(g)t_1,...,\chi^{\omega_r}(g)t_r)=\sum\alpha_{i_1,...,i_r}\chi^{\sum i_k\omega_k}t_1^{i_1}...t_r^{i_r}(x)$. D'où $f\in $
\end{cons}





\section{Théorie des invariants}
\subsection{L'algèbre des invariants}

Soit $G$ un groupe algébrique et $X$ une $G$-variété affine. $k[X]$ est un $G$-module rationnel pour l'action naturelle de $G$ sur les fonctions régulières. on définit la sous-algèbre des invariants $k[X]^G:=\lbrace f\in k[X]\mid g.f=f,\, \forall g\in G\rbrace$. C'est par définition la sous-algèbre des fonctions constantes sur les orbites de l'action de $G$ sur $X$.\\
Une question naturelle est de se demander si cette algèbre est de type fini. Ce n'est pas le cas en général. En effet, dans la perspective de répondre au 14e problème de Hilbert, Nagata exhiba en 1959 une algèbre d'invariants pour l'action d'un groupe algèbrique qui n'est pas de type fini. Avec des hypothèses sur $G$, on peut cependant montrer que c'est le cas, c'est l'objectif de cette partie.\\
Supposons $G$ réductif. Le $G$-module $k[X]$ est alors semi-simple, en particulier, $k[X]^G$ admet un supplémentaire $G$-stable que l'on note $k[X]_G$. On définit l'opérateur de Reynolds $R_{k[X]}$ comme la projection sur $k[X]^G$ associée à cette décomposition. Voici quelques propriétés de $R_{k[X]}$:


\begin{prop}\label{reynolds}
\begin{enumerate}
\item Soit $f:V \rightarrow W$ un morphisme de $G$-module et $f^G:V^G \rightarrow W^G$ le morphisme induit. On a $R_Wf=fR_V$. En particulier, si $f$ est surjective, $f^G$ l'est aussi.
\item $R_{k[X]}$ est $K[X]^G$-linéaire
\end{enumerate}

\end{prop}
\begin{proof}
\begin{enumerate}
\item Ok
\item Soit $a\in k[X]^G$. on considère $m_a$ la multiplication par $a$ dans $k[X]$. C'est un endomorphisme de $G$-module, il commute donc avec $R_{k[X]}$. 
\end{enumerate}
\end{proof}

\begin{thm}[Hilbert]\label{hilbert}
Soit $G$ un groupe réductif et $X$ une $G$-variété affine. Alors l'algèbre des invariant $k[X]^G$ est de type fini.
\end{thm}
\begin{proof}
Supposons que $X$ soit un $G$-module $V$ de dimension finie. L'action de $k^*$ sur $V$ par homothétie donne une $\NN$-graduation $k[V]=\bigoplus_{n=0}^{\infty}k[V]_n$, $k[V]_n$ étant le sous espace des polynômes homogènes de degré n. Cette graduation est $G$-stable et se restreint sur l'algèbre des invariants en une $\NN$-graduation $k[V]^G=\bigoplus_{n=0}^{\infty}k[V]^G_n$. Or on remarque que $k[V]^G$ est noetherien. En effet, soit $I$ un idéal de $k[V]^G$, et $J$ son extension dans $k[V]$. $J$ est un sous $G$-module, donc la contraction de $J$ dans $k[V]^G$ est $R_{k[V]}(J)=IR_{k[V]}(k[V])=I$. On voit donc que la condition de chaîne est satisfaite sur $k[V]^G$ si elle satisfaite sur $k[V]$, ce qui est le cas car ce dernier est noetherien par le théorème de la base de Hilbert. Ainsi, on a le résultat d'après la proposition \ref{noethgrad}.\\
Dans le cas général, on peut d'après le théorème \ref{embed} supposer $X$ inclus dans un $G$-module $V$. On obtient alors un $G$-morphisme surjectif $k[V] \rightarrow k[X]$ qui induit un $G$-morphisme surjectif $k[V]^G \rightarrow k[X]^G$ d'après la proposition \ref{reynolds}. Cela montre que $k[X]^G$ est de type fini.
\end{proof}

On constate que cette preuve n'est pas effective. Il est en général difficile de calculer l'algèbre des invariants. On présente ci-dessous la méthode des sections qui permet le calcul dans certains cas.\\
Soit $S\subset X$ une sous-variété fermée. Définissons $Z(S)=\lbrace g\in G\mid g.s=s,\,\forall s \in S\rbrace$ et $N(S)=\lbrace g\in G\mid g.s\in S,\,\forall s \in S\rbrace$. Clairement, $Z(S)$ est un sous-groupe normal de $N(S)$, et le quotient $W=N(S)/Z(S)$ agit sur $S$. La surjection $k[X]\rightarrow k[S]$, induit un morphisme $k[X]^G\xrightarrow{\phi} k[S]^W$.\\
Supposons que l'on ait un ouvert dense $U\subset X$ tel que $\forall x\in U,\, G.x$ intersecte $S$, alors on voit que $\phi$ est injective. Si de plus, $k[S]^W$ est engendré par des $\phi(f_1),...,\phi(f_r)$, alors $\phi$ est un isomorphisme et $k[X]^G$ est engendré par $f_1,...,f_r$.

\begin{ex}
$G=GL_n$,\, $X=M_n$,\, $g.A=gAg^{-1},\, S=D_n,\, U=X_{disc(\chi )}$. En considérant un élément de $U$, qui a donc ses valeurs propres deux à deux distinctes, on a par un calcul direct $Z(S)=D_n$. Puis on a $N(S)=\lbrace\textrm{matrices monomiales}\rbrace$ car la conjugaison préserve les espaces propres. Ainsi, $W$ est isomorphe au groupe symétrique $\Sigma_n$ et agit sur $S$ en permutant les entrées diagonales. On a ainsi $k[S]^W=k[\sigma_1,...,\sigma _n]$, l'algèbre engendrée par les fonctions symétriques élémentaires. C'est une algèbre de polynômes.\\
Soient $f_1,...,f_n$ les coefficient du polynôme caractéristique générique. Ce sont des éléments de $k[X]^G$, et on a $f_{i| S}=(-1)^i\sigma _i$, d'où $k[X]^G=k[f_1,...,f_n]$.
\end{ex}

\subsection{Quotient d'une variété algébrique sous l'action d'un groupe algébrique}

\subsubsection{Quotient catégorique}

Soit $G$ un groupe algébrique et $X$ une $G$-variété. En tant que groupe abstrait agissant sur un ensemble, le quotient de $X$ par $G$ (noté $X//G$) est par définition l'ensemble des orbites. On note $\pi:X \rightarrow X//G$ l'application qui à un élément de $X$ associe son orbite. $X//G$ satisfait une propriété universelle, il représente le foncteur $\textrm{Ens}\rightarrow \textrm{Ens}, Y\mapsto \lbrace f\in \textrm{Map}(X, Y)\mid f \textrm{ est constante sur les orbites} \rbrace$, il est donc unique à isomorphisme près. Pour cette raison la paire $(X//G, \pi)$ est appelée le quotient catégorique de $X$ par $G$. \\ 
On peut ainsi transporter cette définition dans la catégorie des variétés algébriques. Toutefois, il n'est pas clair que ce quotient existe toujours. L'exemple suivant montre que lorsqu'il existe, le quotient catégorique ne coïncide pas nécessairement avec l'ensemble des orbites.

\begin{ex}
On considère l'action naturelle de $GL_n$ sur $\AAA^n$. Le quotient catégorique existe et est un point. En effet soir $f:\AAA^n\rightarrow Z$ constant sur les orbites, alors $f$ est constante car il existe un orbite dense. En revanche il y a un deuxième orbite, c'est le fermé $\lbrace 0\rbrace$. 
\end{ex}

On suppose $X$ affine et $k[X]^G$ de type fini avec $f_1,...,f_r$ des générateurs, c'est en particulier le cas lorsque $G$ est réductif d'aprés le théorème \ref{hilbert}. Dans ce cadre, l'algèbre des invariants définit une variété algébrique affine, notons la $Y$. On définit le morphisme $\phi :X\rightarrow k^r,\, x\mapsto (f_1(x),...,f_r(x))$. Son comorphisme $\phi^*$ admet une factorisation: $k[t_1,...,t_r]\twoheadrightarrow k[X]^G\xrightarrow{\subset}k[X]$, d'où $Y\simeq \overline{\phi (X)}$. On peut donc voir $X \xrightarrow{\phi} \overline{\phi (X)}$ comme une réalisation du morphisme $\pi: X\rightarrow Y$ associé à $k[X]^G\subset k[X]$. On appelle ce morphisme, le morphisme quotient. De la même manière on voit que tout morphisme $G$-invariant de variétés affines $X\rightarrow Z$ se factorise à travers $Y$. De ce fait, $Y$ semble être un bon candidat pour le quotient catégorique. Toutefois il faut être prudent, dans \cite{LAGFerrer} 6.4.10, on exhibe un exemple de cette situation qui n'admet pas de quotient catégorique. On a toutefois le résultat suivant:

\begin{thm}\label{goodquotientthm}
Soit $G$ un groupe réductif et $X$ une $G$-variété affine.
\begin{enumerate}
\item Le morphisme quotient $\pi:X\rightarrow Y$ est surjectif.
\item $(Y, \pi)$ est un quotient catégorique. On écrit donc $Y=X//G$.
\item Soit $Z\subset X$ une sous $G$-variété fermée. Le morphisme induit $Z//G \rightarrow X//G$ est une immersion fermé. On peut ainsi identifier $\pi_Z$ et $\pi_X$ restreint à $Z$. De plus, soit $Z'$ une autre sous $G$-variété fermée, on a $\pi_X(Z\cap Z')=\pi_X(Z)\cap\pi_X(Z')$.
\item Chaque fibre de $\pi_X$ contient un unique orbite fermé.
\end{enumerate}
\end{thm}
\begin{proof}
\begin{enumerate}
\item Soit $x\in Y$ et $m_x$ l'ideal maximal de $k[X]^G$ correspondant. La fibre $\pi^{-1}(x)$ correspond à l'ensemble des idéaux maximaux contenant l'extension $I$ de $m_x$ dans $k[X]$. Or on a déjà vu que l'extension dans $k[X]$ était injective, $I$ est donc un idéal propre contenu dans au moins un idéal maximal. La fibre étant non-vide, $\pi$ est surjective.
\item L'existence de la factorisation a déjà était vue. Avec 1) on a maintenant l'unicité.
\item On note $i$ l'inclusion $Z\subset X$. $\pi_Xi$ est constant sur les orbites de $Z$ d'où l'existence d'un unique morphisme $\phi:Z//G\rightarrow X//G$ tel que $\phi\pi_Z=\pi_Xi$. La projection $k[X] \rightarrow k[Z]$ est un morphisme de $G$-module surjectif. D'après la proposition \ref{reynolds}, cette projection induit un morphisme de $k$-algèbre surjectif $k[X]^G \xrightarrow{\phi^*} k[Z]^G$, donc $\phi$ est une immersion fermée.
Soit $I$ (resp. $I'$) l'idéal de $Z$ (resp. $Z'$) dans $k[X]$. L'idéal de $Z\cap Z'$ est $I+I'$ et l'idéal de $\pi_X(Z)$ est $\mathcal{I}_{X//G}(\pi_X(Z))=I\cap k[X]^G=R_X(I)$. Ainsi $\mathcal{I}_{X//G}(\pi_X(Z\cap Z'))=R_X(I+I')=R_X(I)+R_X(I')=\mathcal{I}_{X//G}(\pi_X(Z)\cap \pi_X(Z'))$.
\item d'après 3), $\pi_X$ envoie deux orbites fermés distincts sur deux points distincts.
\end{enumerate}
\end{proof}

On remarque que les propriétés du théorème précédent s'étendent automatiquement au cas d'une $G$-variété $X$ si le théorème est vérifié localement sur un recouvrement affine d'un candidat $Y$ pour le quotient $X//G$. De ce constat découle la notion de bon quotient.

\begin{defn}
Soit $G$ un groupe réductif et $X$ une $G$-variété. Une paire $(Y, \pi)$ où $Y$ est une variété et $\pi$ un morphisme $X\rightarrow Y$ est un bon quotient si elle vérifie:\\
(i) $\pi$ est affine et $G$-invariant.\\
(ii) $\pi^*:\Oo_Y \rightarrow (\pi_*\Oo_X)^G$ est un isomorphisme.
\end{defn}

\begin{ex}
Soit $G$ un groupe réductif et $X$ une $G$-variété affine. D'après le théorème \ref{goodquotientthm} et l'exemple \ref{exaff}, $X//G$ est un bon quotient.
\end{ex}


\subsubsection{Quotient géométrique}

Parmi les quotients catégoriques $(X//G,\pi)$, on cherche à caractériser ceux ayant les propriétés géométriques intuitivement attendues pour un quotient, c'est à dire que $X//G$ soit l'ensemble des orbites avec une topologie aussi fine que possible. C'est la notion de quotient géométrique:

\begin{defn}
Soit $G$ un groupe algébrique et $X$ une $G$-variété. Une paire $(Y, \pi)$ où $Y$ est une variété et $\pi$ un morphisme $X\rightarrow Y$ est un quotient géométrique si elle vérifie:\\
(i) $\pi$ est surjective et ses fibres sont exactement les orbites.\\
(ii) La topologie de $Y$ coïncide avec la topologie quotient associée à $\pi$.\\
(iii) $\pi^*:\Oo_Y \rightarrow (\pi_*\Oo_X)^G$ est un isomorphisme.
\end{defn}

On remarque que pour un quotient géométrique $(X//G, \pi)$, tous les orbites sont fermés dans $X$ et l'application quotient est ouverte. En effet, soit $U$ un ouvert de $X$, on a $\pi^{-1}(\pi(U))=\cup_{g\in G}g.U$ qui est ouvert.

\begin{ex}
Soit $G$ un groupe algébrique et $H$ un sous-groupe fermé. Dans la catégorie des ensemble, le quotient $(G/H,\pi)$ est exactement le quotient catégorique $(G//H,\pi)$ pour l'action de $H$ sur $G$ par multiplication à droite. Dans \cite{LAGSpringer} 5.5.5, en caractéristique quelconque, on munit $G/H$ d'une structure d'espace annelé en lui attribuant la topologie quotient puis en définissant le faisceau structural par $\Oo_{G/H}(U):=\lbrace f\in\textrm{Map}(U,k)\mid f\pi \in \Oo_G(\pi^{-1}(U))\rbrace$. Par construction, il vérifie la propriété universelle de factorisation. De plus, on montre ensuite que cet espace annelé est isomorphe à une variété quasi-projective, ce qui montre l'existence du quotient catégorique $(G//H,\pi)$ dans la catégorie des variétés algébriques. Par définition de $\Oo_{G/H}$, on a une flèche $\Oo_Y \xrightarrow{\pi^*} (\pi_*\Oo_X)^G$. Elle est injective par la surjectivité de $\pi$, et elle est surjective, par la propriété universelle du quotient. $(G//H,\pi)$ est donc un quotient géométrique. Cela généralise bien sur le théorème \ref{groupequotient}.
\end{ex}

\begin{ex}
Un bon quotient $(X//G, \pi)$ est un quotient géométrique si les fibres de $\pi$ sont exactement les orbites. En effet, d'après ce qui précède, il reste à vérifier que $X//G$ est muni de la topologie quotient. Soit un ouvert de X de la forme $\pi^{-1}(A)$ où $A$ est une partie de $X//G$. En tenant compte de \ref{goodquotientthm} (iii) et de la surjectivité de $\pi$ on a: $\pi(X\setminus \pi^{-1}(A))=Y\setminus A$ qui est fermé, donc $A$ est ouvert.
\end{ex}

\subsubsection{Un exemple: La construction Proj}
Dans cette partie, on va détailler une construction qui à la fois éclaire et généralise la construction de la variété algébrique $\PP^n$($k$). On considère une variété affine $X$ munie d'une action de $k^*$. Une orbite $k^*.x$ est de dimension 0 ou 1. Si elle est de dimension 0, c'est un point fixe car $k^*$ est connexe. Si elle est de dimension 1, elle est soit fermée, soit son adhérence est constituée de $k^*.x$  et d'une réunion de points fixes, en fait un seul comme on va le voir. On note $F$ l'ensemble des points fixes. \\
L'algèbre affine $A:=k[X]$ est $\NN$-graduée par l'action de $k^*$ et on remarque que $F=\mathcal{V}_X(A_{>0})$, où $A_{>0}:=(f\mid f\in A_d \textrm{ pour un }d\geq 0)$ est l'idéal dit inconvenant. En effet, $F$ est l'ensemble des idéaux maximaux qui sont $k^*$-stables. Or on peut remarquer qu'un idéal radical est $k^*$-stable si et seulement si il est homogène. Enfin, un idéal maximal et homogène contient nécessairement $A_{>0}$.\\
On remarque que $A^{k^*}=A_0=A/A_{>0}$, et comme le bon quotient $Y_0:=X//k^*=\spm (A_0)$ paramètre les orbites fermés, on obtient en particulier que si $k^*.x$ n'est pas fermé, son adhérence contient un unique point fixe. En résumé, $W:=X\setminus F$ est la réunion des orbites de dimension 1, et ils sont tous fermés dans $W$. On va montrer que $W$ admet un quotient géométrique, c'est la construction Proj.\\
Pour tout $f\in A_{>0}$ homogène, la localisation $A_f$ est $\ZZ$-graduée de la manière suivante:
\begin{center}
 $A_f=\bigoplus_{d\in \ZZ},\,\,\,\, (A_f)_d:=\lbrace h/f^l\mid \textrm{deg}(h)-l\textrm{deg}(f)=d \rbrace$
\end{center}
On note $A_{(f)}:=(A_f)_0$ en remarquant qu'il s'agit de l'algèbre des invariants de la $k^*$-variété affine $X_f$. On a ainsi trouvé le bon quotient $U_f:=\spm(A_{(f)})=X_f//k^*$. De plus, comme $F\subset\mathcal{V}_X(f)$,  il s'agit d'un quotient géométrique d'après ce qui précède. Or on peut recouvrir $W$ par un nombre fini de $X_{f_i}$ avec les $f_i$ homogènes de degrés $>0$. Considérons les diagrammes commutatifs ci-dessous. Dans le diagramme de gauche, toutes les flèches sont des inclusions. Le diagramme de droite est obtenu par application du foncteur $\spm$:
\begin{multicols}{2}
	\begin{center}
	\begin{tikzcd}
  		A_{f_i} \arrow[r, ""] & A_{f_if_j}  & A_{f_j} \arrow[l, ""]\\ 
  		A_{(f_i)} \arrow[r, ""] \arrow[u, ""] & A_{(f_if_j)}  \arrow[u, ""] & A_{(f_j)} \arrow[l, ""] \arrow[u, ""]
	\end{tikzcd}\\
	\end{center}
	
	\columnbreak
	\begin{center}
	\begin{tikzcd}
  		X_{f_i} \arrow[d, "\pi_i"] & X_{f_i}\cap X_{f_j}=X_{f_jf_j} \arrow[l, ""] \arrow[d, ""] \arrow[r, ""] & X_{f_j}  \arrow[d, "\pi_j"]\\ 
  		U_{f_i}  & U_{f_if_j}  \arrow[l, ""] \arrow[r, ""]   & U_{f_j} 
	\end{tikzcd}\\
	\end{center}
\end{multicols}
Les flèches horizontales du diagramme de droite sont des immersions ouvertes, en effet on a $U_{f_if_j}\simeq (U_{f_i})_{f_j}$. Les conditions de la construction \ref{gluevar} sont satisfaites, on peut former une prevariété Proj($A$) par recollement des $U_{f_i}$ le long de ces immersions. De plus, les $(U_{f_i},\pi_i)$ sont des quotients géométriques et le diagramme exprime que l'on a les conditions de recollement sur les intersections qui font de Proj($A$) le quotient géométrique global $W//k^*$.\\
Enfin, on remarque que les fermés de Proj($A$) correspondent aux fermés $k^*$-stables de $W$, c'est à dire aux idéaux radicaux homogènes qui ne contiennent pas l'idéal inconvenant. On appelle ces fermés des variétés projectives. La topologie quotient sur Proj($A$) que l'on vient de définir est aussi appelé la topologie de Zariski.

\section{Diviseurs}
