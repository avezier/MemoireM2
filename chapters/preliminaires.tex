\chapter{Préliminaires}

\section{Algèbre commutative}

On collecte dans cette partie des définitions et résultats utilisés sans la suite. Soit $A$ un anneau, $\mathfrak{p}$ un idéal premier de $A$, et $f\in A$ un élément. On note $A_\mathfrak{p}$ la localisation de $A$ en la partie multiplicative $A\setminus \mathfrak{p}$, et $A_f$ la localisation de $A$ en la partie multiplicative $\lbrace 1, f, f^2, ....\rbrace$.

\subsection{Extensions entières d'anneaux}

\begin{defn}[Anneau intègralement clos]\label{normalring}
Un anneau $A$ est intégralement clos si il est intègre et égal à sa clôture intégrale. 
\end{defn}

\begin{rem}
Il existe une notion plus générale qu'il est utile de mentionner pour la consistance avec la terminologie de variété normale que nous verrons plus loin. Pour un anneau quelconque $A$, on dit que $A$ est normal si tout localisé de $A$ en idéal premier est un anneau intègre et intégralement clos. On note qu'un anneau intègre est intégralement clos si et seulement si il est normal (\cite{atiyahmacdo} 5.13). Mentionnons que nous ne rencontrerons pas dans ce mémoire d'anneaux normaux non-intègres, donc la définition ci-dessus sera suffisante.
\end{rem}


\begin{thm}\label{factonormal}
Soit $A$ un anneau intègre noetherien intégralement clos. Alors
\begin{enumerate}
\item Tous les premiers associés d'un idéal principal non-nul sont de hauteur 1.
\item $A=\bigcap_{\mathfrak{p}\in \spec A, \haut \mathfrak{p}=1} A_\mathfrak{p}$.
\end{enumerate}
\end{thm}
\begin{proof}
Voir \cite{Matsumura} 11.5
\end{proof}


\begin{thm}\label{UFDPID}
Soit $A$ un anneau noetherien intégralement clos. Alors, \\
$A$ est factoriel $\iff$ Tout idéal premier de hauteur 1 est principal.
\end{thm}
\begin{proof}
Voir \cite{Matsumura} 20.1
\end{proof}

\subsection{Anneaux locaux, Anneaux de valuation discrète}

\begin{defn}
Un anneau local noethérien $A$ de corps résiduel $k=A/\mathfrak{m}$ est régulier si $\dim A =\dim_k\mathfrak{m}/\mathfrak{m}^2$.
\end{defn}

\begin{thm}\label{reglocufd}
Un anneau local régulier est factoriel.
\end{thm}
\begin{proof}
Voir \cite{Matsumura} 20.3
\end{proof}

\begin{thm}\label{ideauxinversibles}
Soit $A$ un anneau intègre et $I$ un idéal fractionnaire de $A$. Les assertions suivantes sont équivalentes:
\begin{enumerate}
\item $I$ est inversible
\item $I$ est un $A$-module projectif
\item $I$ est de type fini, et pour tout idéal maximal $\mathfrak{m}$ de $A$, l'idéal fractionnaire $I_\mathfrak{m}$ de $A_\mathfrak{m}$ est principal.
\end{enumerate}
\end{thm}
\begin{proof}
Voir \cite{Matsumura} 11.3
\end{proof}

\subsection{Algèbres graduées}



\section{Schémas}

Notre principale référence est \cite{Hartshorne}. On résume dans cette partie certains faits et définitions. 

\subsection{Généralités}

Soit $A$ un anneau, on note $X=\spec A$ l'ensemble de ses idéaux premiers, il est naturellement muni d'une topologie en définissant que les fermés sont les parties de la forme $\Vv_X(\mathfrak{a})=\lbrace \mathfrak{p}\in X\mid \mathfrak{a}\subset \mathfrak{p}\rbrace$, où $\mathfrak{a}$ est un idéal de $A$. On munit $X$ d'un faisceau d'anneaux $\Oo_X$, dit faisceau structural, constitué pour tout ouvert $U\subset X$ des sections $s: U\mapsto \bigsqcup_{\mathfrak{p}\in U}A_\mathfrak{p}$ telles que $s(\mathfrak{p})\in A_\mathfrak{p}$, et pour tout $\mathfrak{p}\in U$, il existe un voisinage $V\subset U$ de $\mathfrak{p}$ et des éléments $a,f\in A$ tels que pour tout $\mathfrak{q}\in V$, on ait $f\notin \mathfrak{q}$ et $s(\mathfrak{q})=a/f\in A_\mathfrak{q}$. Les sections $s$ sur un ouvert $U$ sont appelées les fonctions régulières sur $U$. On peut définir le même faisceau structural d'une manière différente en remarquant que $X$ possède une base d'ouverts, dits principaux, de la forme $X_f:=X\setminus \Vv_X((f))$ avec $f\in A$. On pose alors $\Oo(X_f)=A_f$ et on vérifie grâce aux propriétés $(A_f)_g\simeq A_{fg}$ et $X_{fg}=X_f\cap X_g$ que cela définit bien un "faisceau" sur cette base d'ouverts. qui se prolonge donc de manière unique en un faisceau sur $X$. En particulier on obtient immédiatement $A=\Gamma(X, \Oo_X)$, et sans difficulté $\Oo_{X,\mathfrak{p}}\simeq A_\mathfrak{p}$. 

\begin{defn}[Espace annelé]
Un espace annelé est un espace topologique $X$ muni d'un faisceau d'anneau $\Oo_X$ tel que les tiges $\Oo_{X,x}$ soient des anneaux locaux en tout point $x\in X$.

Un morphisme d'espaces annelés $(f, f^\sharp):(X,\Oo_X)\rightarrow (Y,\Oo_Y)$ est une application continue $f:X\rightarrow Y$ et un morphisme $f^\sharp:\Oo_Y\rightarrow f_*\Oo_X$ tel que les morphismes induits entre les tiges soient locaux, c'est à dire $f^{\sharp -1}_x(\mathfrak{m}_x)=\mathfrak{m}_{f(x)}$ pour tout $x\in X$.
\end{defn}

\begin{defn}[Schéma affine, schéma]
Un schéma affine est un espace annelé isomorphe au spectre d'un anneau muni de son faisceau structural. Un schéma est espace annelé $(X,\Oo_X)$ tel que tout point $x\in X$ admet un voisinage ouvert $U$ tel que $(U,\Oo_U)$ soit un schéma affine. Un morphisme de schémas est un morphisme d'espaces annelés.
\end{defn}

Tout morphisme d'anneaux $\phi:A\rightarrow B$ définit un morphisme de schémas affines $(f,f^\sharp):\spec B\rightarrow\spec A$. En effet, on pose $f(\mathfrak{p}):=\phi^{-1}(\mathfrak{p})$ pour tout $\mathfrak{p}\in \spec B$. Puis, pour tout ouvert $V\subset \spec A$, $s\in\Oo_{\spec A}(V)$, et $\mathfrak{p}\in f^{-1}(V)$, on pose $f^\sharp(s)(\mathfrak{p}):=\phi_{\mathfrak{p}}\circ s \circ f(\mathfrak{p})$, où $\phi_{\mathfrak{p}}:A_{f(\mathfrak{p})}\rightarrow B_{\mathfrak{p}}$ est la localisation de $\phi$.
Réciproquement, tout morphisme entre schémas affines $f:\spec B\rightarrow \spec A$ définit un morphisme d'anneaux en prenant les sections globales $\Gamma(\spec A, f^\sharp):A\rightarrow B$. Ces deux procédés sont inverses l'un de l'autre et on obtient:


\begin{prop}\label{EqCat}
Les foncteurs $\spec$ et $\Gamma(.)$ définissent une anti-équivalence de catégories entre la catégorie des schémas affines et la catégorie des anneaux (commutatifs).
\end{prop}

Soit $X$ un schéma, $f\in \Oo(X)$ une fonction régulière sur $X$, et $x$ un point de $X$. On dit que $f$ s'annule en $x$ si son germe n'est pas inversible dans l'anneau local $\Oo_{X,x}$. Soit $U=\spec A$ un voisinage affine de $x$, on a un isomorphisme canonique $\Oo_{X,x}\simeq A_\mathfrak{p}$, où $\mathfrak{p}$ est l'idéal premier correspondant à $x$. Alors $f$ s'annule en $x$ revient à dire que $f_{|U}\in p$. On peut alors par exemple parler de l'idéal de $\Oo(X)$ des fonctions régulières s'annulant sur un fermé $F$ de $X$, on le note $\Ii_X(F)$. Ainsi, $F$ définit un faisceau d'idéaux $\Ii_F:U\mapsto \Ii_U(F\cap U)$ radicaux qui définit une structure de schéma $(F, i^{-1}(\Oo_X/\Ii_F))$ sur $F$, où $i$ est l'inclusion. Si $X=\spec A$ est un schéma affine, on a alors pour un idéal $\mathfrak{a}$ et une partie $Y$ de $X$ les formules
$$\Ii_X(Y)=\bigcap_{\mathfrak{p}\in Y}\mathfrak{p},\,\,\,\,\, \Ii_X(\Vv_X(\mathfrak{a}))=\sqrt{\mathfrak{a}},\,\,\,\,\,\,\Vv_X(\Ii_X(Y))=\overline{Y}$$

\begin{rem}
On remarque que si on considère une algèbre affine, c'est à dire une $k$-algèbre de type fini réduite, et que l'on suppose de plus le corps algébriquement clos. Alors si l'on remplace "premier" par "maximal" dans les formules ci-dessus relatives aux opérations $\Vv$ et $\Ii$, on retombe sur les définitions classiques de ces opérations qui donnent les ensemble algébriques et leurs équations dans un espace affine de dimension finie sur $k$. En effet le Nullstellensatz assure dans ce cas que les points de l'ensemble algébrique correspondent aux idéaux maximaux de l'algèbre et que $\Ii(Y)=\bigcap_{y\in Y}\mathfrak{m}_y$.

Par ailleurs, une fonction régulière $s$ sur un ouvert $U$ d'un schéma $X$ peuvent être vue comme une fonction à valeurs dans les différents corps résiduels $A_\mathfrak{p}/\mathfrak{p}A_\mathfrak{p}$ pour chaque $\mathfrak{p}\in U$ en prenant $s(\mathfrak{p})$ modulo $\mathfrak{p}A_\mathfrak{p}$. Dans le cas d'une variété affine, c'est à dire $X=\spec A$ où $A$ est une $k$-algèbre affine sur un corps algébriquement clos, tous les corps résiduels sont isomorphes à $k$. D'après le Nullstellensatz, l'évaluation de $s\in A$ en les points fermés, c'est à dire les idéaux maximaux, coïncide avec l'évaluation de $s$ en les points correspondants de l'ensemble algébrique associé à $A$. Enfin, considérons deux $k$-algèbres affines $A$ et $B$ sur un corps algébriquement clos, et $X, Y$ les variétés affines correspondantes. Étant donné un morphisme $f:X\rightarrow Y$, le morphisme de $k$-algèbres correspondant s'identifie à $\phi:A\rightarrow B,s\mapsto s\circ f$. En considérant $X$ et $Y$ plongés dans un espace affine et en prenant pour $s$ les différentes coordonnées de $Y$ on retrouve la définition classique que les morphismes entre ensembles algébriques sont les applications polynomiales.
\end{rem}

\begin{cons}[Recollement de schémas]\label{gluevar}
Soit $(X_i)_{u\in I}$ une famille de schémas. Supposons $\forall i,j$ on ait des ouverts $X_{ij}\subset X_i$ et des isomorphismes $f_{ij}:X_{ij}\rightarrow X_{ji}$ tels que $\forall i,j,k\in I$ on ait:
	\begin{enumerate}
	\item $X_{ii}=X_i$ et $f_{ii}=id$
	\item $f_{ij}^{-1}(X_{ji} \cap X_{jk}) =  X_{ij} \cap X_{ik}$
	\item Le diagramme suivant commute:
	$$
		\begin{tikzcd}
		  X_{ij}\cap X_{ik} \arrow[rd, "f_{ij}"] \arrow[rr, "f_{ik}"] & & X_{ki}\cap X_{kj}\\
  		 & X_{ji}\cap X_{jk} \arrow[ru, "f_{jk}"] &
	\end{tikzcd}
	$$
	\end{enumerate}
Alors on peut définir un schéma $X$ comme la réunion disjointe des $X_i$ modulo la relation d'équivalence $x\sim x' \iff (x\in X_{ij}, x'\in X_{ji}$ et $f_{ij}(x)=x')$. On note $f_i:X_i\rightarrow X$ les applications canoniques, et on munit 
$X$ de la topologie finale associée aux $f_i$. On vérifie que chaque $f_i$ est une immersion ouverte dont on note $U_i$ l'image. Le faisceau structural est définit en recollant les $\mathcal{O}_{U_i}:=f_{i\star}\mathcal{O}_{X_i}$, ses sections sur un ouvert $W\subset X$ étant $$
\mathcal{O}_X(W) =
\{
(s_i)_{i \in I} \mid
s_i \in \mathcal{O}_{U_i}(W \cap U_i),
f_{ij}(s_i|_{W \cap U_i \cap U_j}) = s_j|_{W \cap U_i \cap U_j}
\}
$$
\end{cons}


\begin{defn}
Soit $S$ un schéma. Un $S$-schéma, où schéma sur $S$, est un schéma $X$ muni d'un morphisme $X\rightarrow S$ appelé morphisme structural.
\end{defn}

Le produit fibré existe dans la catégorie des schémas. Cette construction se fait par recollement en remarquant que pour des schémas affines $X,Y$ sur un schéma affine $S$, le produit fibré $X\times_S Y$ est naturellement $\spec \Oo_X(X)\otimes_{\Oo_S(S)}\Oo_Y(Y)$. En effet, cela vient de \ref{EqCat} et du fait que le produit tensoriel sur $\Oo_S(S)$ est le co-produit dans la catégorie des $\Oo_S(S)$-algèbres. 
Comme tout anneau est naturellement une $\ZZ$-algèbre, tout schéma est naturellement un schéma sur $\spec \ZZ$. Le produit de deux schéma est par définition le produit sur $\spec \ZZ$.

\subsection{Quelques propriétés des schémas}

\subsubsection{Schémas réduits, irréductibles, intègres, noetheriens}

\begin{defn}[Schéma réduit]
Un schéma $X$ est réduit si pour tout $x\in X$ les anneaux locaux $\Oo_{X,x}$ sont réduits. 
\end{defn}

\begin{prop}
Un schéma $X$ est réduit si et seulement si pour tout ouvert $U\subset X$, l'anneau $\Oo(U)$ est réduit.
\end{prop}
\begin{proof}
Supposons $X$ réduit et $f\in \Oo(U)$ nilpotent. Si $f$ est non-nul, alors il existe un germe $f_x$ non-nul pour un point $x\in U$. Mais alors $f_x^n=0$ pour un $n>0$, et donc $f_x=0$, contradiction.
Réciproquement, soit $f_x\in \Oo_{X,x}$ nilpotent alors $f_x$ est la classe d'une fonction nilpotente $f\in \Oo(U)$ pour un ouvert $U\subset X$. Cela implique $f_x=0$.
\end{proof}

\begin{defn}[Schéma irréductible]
Un schéma irréductible si l'espace topologique sous-jacent est irréductible, c'est à dire qu'il est non vide et n'est pas réunion de deux fermés propres.
\end{defn}

\begin{defn}[Schéma noetherien]
Un schéma $X$ est noetherien si c'est une union finie d'ouverts affines $U_i$ tels que $\Oo(U)$ est un anneau noetherien.
\end{defn}

\begin{prop}
Soit $X$ un schéma noetherien. Alors 
\begin{enumerate}
\item Les tiges de $X$ sont des anneaux noetheriens. Tout sous-schéma ouvert ou fermé de $X$ est noetherien.
\item $X$ est quasi-compact et $\Oo(U)$ est noetherien pour tout ouvert affine $U\subset X$.
\end{enumerate}
\end{prop}

Par le lemme de Zorn, on montre que l'ensemble des fermés irréductibles d'un schéma $X$ admet des éléments maximaux, appelés composantes irréductibles. Si $X$ est noetherien, ce nombre est fini et $X$ s'écrit donc de manière unique comme réunion de ses composantes irréductibles. Pour tout point $x\in X$, les composantes irréductibles de $\spec \Oo_{X,x}$ correspondent bijectivement aux composantes irréductibles de $X$ passant par $x$.

\begin{prop}
Un schéma $X=\spec A$ un schéma affine, et $\mathfrak{a}$ un idéal.
\begin{enumerate}
\item $\Vv_X(\mathfrak{a})$ est irréductible si et seulement si $\sqrt{\mathfrak{a}}$ est premier
\item Soit $(\mathfrak{p}_i)$ les premiers minimaux de $A$. Alors $(\Vv_X(\mathfrak{p}_i))$ sont les composantes irréductibles de $X$.
\item $X$ est irréductible si et seulement si $A$ admet un unique idéal premier minimal. Dans ce cas, cet idéal est nécessairement $(0)$, et $A$ est intègre.
\end{enumerate}
\end{prop}

\begin{defn}[Schéma intègre]
Un schéma $X$ est intègre si il est réduit et irréductible.
\end{defn}

\begin{prop}
Un schéma $X$ est intègre si et seulement si $\Oo(U)$ est intègre pour tout ouvert $U\subset X$ non-vide.
\end{prop}

Un point $x$ d'un espace topologique $X$ est dit générique si $\overline{\lbrace x\rbrace}=X$. Ainsi, si $X$ admet un point générique, il est irréductible. Pour un schéma $X$ on a en fait une correspondance bijective entre fermés irréductibles de $X$ et points génériques donnée par $x\mapsto \overline{\lbrace x\rbrace}$. 

Un schéma affine $X=\spec A$ est intègre si et seulement si $A$ est intègre. Alors le point générique $\eta$ de $X$ correspond à l'idéal $(0)$ est $\Oo_{X,\eta}$ est la localisation $A_{(0)}$, c'est à dire le corps des fractions de $A$. Plus généralement, on définit le corps des fonctions rationnelles d'un schéma intègre $X$ comme les fonctions régulières sur un ouvert non-vide de $X$. Comme tout ouvert non-vide contient le point générique $\eta$ de $X$, il s'agit de $\Oo_{X,\eta}$ qui est bien un corps d'après ce qui précède. On le note $k(X)$.

\begin{prop}
Soit $X$ un schéma intègre et $k(X)$ son corps des fonctions rationnelles. 
\begin{enumerate}
\item Pour tous ouverts $U\subset V\subset X$ non-vides, les applications suivantes sont injectives
$$\Oo(V)\xhookrightarrow{res_{V,U}}\Oo(U)\xhookrightarrow{f\mapsto f_\eta} k(X)$$
\item Pour tout ouvert non-vide $U\subset X$ et tout recouvrement $U=\cup_i U_i$, on a dans $k(X)$:
$$\Oo(U)=\bigcap_i \Oo(U_i)=\bigcap_{x\in U}\Oo_{X,x}$$
\end{enumerate}
\end{prop}

\subsection{Quelques propriétés des morphismes de schémas}



\subsubsection{Immersion ouverte, immersion fermée}

\begin{defn}[Immersion ouverte, immersion fermée]
Un morphisme de schémas $(f,f^\sharp):X\rightarrow Y$ est une immersion ouverte (resp. fermée) si $f$ est une immersion ouverte topologique (resp. immersion fermée topologique), et si $f^\sharp_x:\Oo_{X,f(x)}\rightarrow \Oo_{X,x}$ est un isomorphisme (resp. surjective) pour tout $x\in X$.
\end{defn}

\begin{defn}
Un schéma $X$ est quasi-affine si il est quasi-compact et il existe une immersion ouverte dans un schéma affine.
\end{defn}

\subsubsection{Morphismes de type fini}

\begin{defn}[Morphisme quasi-compact]
Un morphisme de schémas $f:X\rightarrow Y$ est quasi-compact si l'image réciproque tout ouvert affine de $Y$ est quasi-compact. 
\end{defn}

\begin{defn}[Morphisme de type fini]
Un morphisme de schémas $f:X\rightarrow Y$ est de type fini si il est quasi-compact et pour tout ouvert affine $V\subset Y$, et tout ouvert affine $U\subset f^{-1}(V)$, le morphisme canonique $\Oo_Y(V)\rightarrow \Oo_X(U)$ fait de $\Oo_X(U)$ une $\Oo_Y(V)$-algèbre de type fini. Un $S$-schéma est dit de type fini si son morphisme structural est de type fini.
\end{defn}

\begin{rem}
Pour qu'un morphisme soit de type fini il suffit qu'il existe un recouvrement par des ouverts affines qui satisfont la condition de la définition.
\end{rem}

\subsubsection{Morphismes finis, normalité}

\begin{defn}[Morphisme fini]
Soit $f:X \rightarrow Y$ un morphisme de schémas. On dit que $f$ est fini si pour tout ouvert affine $V$ de $Y$, si la $\Oo_Y(V)$-algèbre $(\Oo_X(f^{-1}(V)), f^\sharp(V))$ est finie.
\end{defn}

\begin{rem}
Pour qu'un morphisme soit fini il suffit qu'il existe un recouvrement par des ouverts affines qui satisfont la condition de la définition.
\end{rem}


\begin{defn}[Schéma normal]
Un point d'un schéma est dit normal si l'anneau local en ce point est intégralement clos.
Un schéma est dit normal si il est normal en chacun de ses points.
\end{defn}

\begin{prop}
Un schéma intègre $X$ est normal si et seulement si pour tout ouvert affine $U\subset X$ non-vide, $\Oo(U)$ est intégralement clos.
\end{prop}

Soit $X$ un schéma intègre, et $(U_i)_{i\in I}$ un recouvrement affine de $X$. Pour chaque $U_i=\spec A_i$ on pose $\widetilde{U}_i=\spec \widetilde{A}_i$ où $\widetilde{A}_i$ est la clôture intégrale de $A_i$, c'est naturellement un schéma intègre sur $U_i$. De plus, comme la normalisation commute à la localisation, on a des isomorphismes naturels $\widetilde{U_i\cap U_j}\simeq \widetilde{U_i}\cap \widetilde{U_j}$. On construit alors par recollement un schéma intègre et normal $\widetilde{X}$ sur $X$ qui satisfait de plus la propriété universelle que tout morphisme dominant d'un schéma intègre $Z$ vers $X$ se factorise de manière unique à travers $\widetilde{X}$. On dit que $\widetilde{X}$ est la normalisation de $X$. De plus, si $X$ est de type fini sur un corps $k$, alors d'après \ref{}, le morphisme structural de $\widetilde{X}$ est fini.

\subsubsection{Morphismes affines}

\begin{defn}[Morphisme affine]
Un morphisme de variétés algébriques $\phi: X\rightarrow Y$ est dit affine si pour tout ouvert affine $V\subset Y$, l'image réciproque $\phi^{-1}(V)$ est affine.
\end{defn}

\begin{ex}\label{exaff}
Un morphisme de schémas affines $\phi: X\rightarrow Y$ est affine. En effet, soit $V$ un ouvert affine de $Y$ et $U=\phi^{-1}(V)$. En considérant le diagramme commutatif ci-dessous on constate que l'on a $U \simeq (\phi\times i_2)^{-1}(\Delta_Y)=\lbrace (x,\phi(x))\mid x\in U \rbrace \subset X\times V$. Comme $X\times V$ est affine, $U$ aussi.
	\begin{center}
	\begin{tikzcd}
  		U \arrow[r, "\phi"] \arrow[d, "i_1"]& V \arrow[d, "i_2"] \\ 
  		X \arrow[r, "\phi"] & Y
	\end{tikzcd}\\
	\end{center}

\end{ex}

\subsubsection{Morphismes séparés}

\begin{defn}[Morphisme séparé, schéma séparé]
Soit $f:X\rightarrow Y$ un morphisme de schémas. Le morphisme diagonal $\Delta:X\rightarrow X\times_Y X$ est l'unique morphisme $X\rightarrow X\times_Y X$ tel que la composition avec les projections $p_1,p_2:X\times_Y X\rightarrow X$ est l'identité de $X$. On dit que $f$ est séparé si le morphisme diagonal est une immersion fermée.
Un $S$-schéma est séparé si son morphisme structural est séparé. En particulier un schéma est séparé si il est séparé sur $\spec\ZZ$.
\end{defn}

Les morphismes entre schémas affines sont séparés et en particulier les schémas affines sont séparés. En effet, considérons $X=\spec A$ et $Y=\spec B$, le morphisme diagonal est donné par $\rho: B\otimes_A B\rightarrow B, x\otimes 1\mapsto x, 1\otimes y \mapsto y$ qui est clairement surjectif, c'est donc une immersion fermée. Le critère suivant est utile et nous dit de plus que les intersections d'affines sont affines dans les schémas séparés.

\begin{prop}\label{sepCritere}
Soit $X$ un schéma. Les assertions suivantes sont équivalentes
	\begin{enumerate}
	\item $X$ est séparé
	\item Pour tous ouverts affines $U$ et $V$ de $X$, $U\cap V$ est affine et le morphisme naturel $\Oo(U)\otimes_\ZZ \Oo(V)\rightarrow\Oo(U\cap V), f\otimes g\mapsto f_{U\cap V}g_{U\cap V}$
	\item Il existe un recouvrement affine $(U_i)$ tel que pour tout $i,j$, $U_i\cap U_j$ est affine et le morphisme naturel $\Oo(U_i)\otimes_\ZZ \Oo(U_j)\rightarrow\Oo(U_i\cap U_j), f\otimes g\mapsto f_{U_i\cap U_j}g_{U_i\cap U_j}$
	\end{enumerate}
\end{prop}

\begin{prop}\label{sepCritere2}
Le morphisme composé de deux morphismes séparés est séparé.
\end{prop}


\subsubsection{Morphismes propres}


\section{Variétés algébriques}

Dans ce mémoire, on travaille principalement dans la catégorie des $k$-schémas réduits séparés de type fini sur $k$, où $k$ est un corps algébriquement clos fixé. Ces objets sont appelés des $k$-variétés algèbriques, ou tout simplement variétés.  Cette catégorie est équivalente à la catégorie des $k$-variétés algébriques au sens de \cite{LAGSpringer} en ne considérant que les points fermés. D'ailleurs, par un point d'une variété $X$, on entendra point fermé, sauf mention du contraire. Soit $X_0$ le sous espace des points fermés de $X$ muni de la topologie induite. Alors les treillis des ouverts des topologies de $X$ et $X_0$ sont isomorphes. On bénéficie ainsi des résultats sur les morphismes et la dimension démontrés par exemple dans \cite{LAGSpringer} ou \cite{MumfordRedBook} chap I. On rappel que le corps de base est supposé de caractéristique nulle.

Un schéma de type fini sur un corps est noetherien, on en déduit que toute partie localement fermée d'une variété admet une unique structure de variété, et tout fermé se décompose de manière unique en une union finie de composantes irréductibles. Enfin, le produit sur $k$ préserve l'irréductibilité.

\subsection{Dimension des variétés algébriques}

\subsubsection{Généralités}

\begin{defn}[Dimension d'une variété]
Soit $X$ une variété. La dimension de $X$ est la borne supérieure des longueurs de chaînes de parties fermées irréductibles
$$\emptyset \varsubsetneq Z_0 \varsubsetneq Z_1 \varsubsetneq ... \varsubsetneq Z_r$$
Soit $F\subset X$ un fermé irréductible. La codimension de $F$ dans $X$, notée $\codime_X F$ est la borne supérieure des longueurs de chaînes de parties fermées irréductibles
$$F \varsubsetneq Z_1 \varsubsetneq ... \varsubsetneq Z_r$$
\end{defn}

Soit $X$ une variété, et $X=\cup_{i=1}^rX_i$ sa décomposition en composantes irréductibles. Alors $\dim X=\sup_i \dim X_i$. De plus, si $X$ est irréductible (i.e $r=1$), alors pour tout ouvert $U\subset X$ non-vide, on a $\dim X=\dim U$. On se ramène donc ainsi à l'étude de la dimension des variétés affines. On suppose donc $X=\spec A$ affine. Par la correspondance entre fermés irréductibles de $X$ et idéaux premier de $A$ on obtient $\dim X= \dim A$, où $\dim A$ est la dimension de Krull de $A$. En particulier, comme $A$ est une $k$-algèbre de type fini intègre, on a le résultat fondamental (Voir \cite{Matsumura} 5.6)
$$\dim X=\dim A=\trdeg_k \fract A$$
Cela assure en particulier que la dimension d'une variété est finie. Par ailleurs, la hauteur d'un idéal premier $\mathfrak{p}$ de $A$ correspond à la codimension de la sous-variété $F$ qu'il définit. Pour une $k$-algèbre de type fini intègre, on a $\haut \mathfrak{p}+ \dim A/\mathfrak{p}=\dim A$. On en déduit
$$\dim X=\dim F+\codime_X F,\,\,\,\,\,\,\, \text{ et } \dim X=\dim\Oo_{X,x} \text{, pour tout } x\in X$$

\begin{thm}\label{dimsousvariete}
Soit $X$ une variété irréductible, $U\subset X$ un ouvert non-vide et $f\in \Oo_X(U)$ non nul et non-inversible. Soit $Z$ une composante irréductible de $\Vv_U((f))$. Alors $\dime Z= \dime(X)-1$.
\end{thm}
\begin{proof}
Voir \cite{MumfordRedBook} I.7 Th.2, après réduction au cas $X$ affine, la preuve consiste en une réduction au cas facile où $\Oo(X)$ est factoriel. 
\end{proof}

Ce théorème peut aussi être vu comme une version géométrique du théorème de l'idéal principal de Krull qui assure les idéaux premiers minimaux sur un idéal principal d'un anneau noetherien sont au plus de hauteur $1$.

\subsubsection{Dimension des fibres d'un morphisme}

\begin{thm}\label{dimensionfibres}
Soit $f:X\rightarrow Y$ un morphisme dominant de variétés irréductibles. Alors il existe un ouvert $U\subset Y$ tel que
\begin{enumerate}
\item $U\subset f(X)$
\item Pour tout fermé irréductible $W\subset Y$ tel que $U\cap X\neq \emptyset$, et pour toute composante irréductible $Z$ de $f^{-1}(W)$  telle que $f^{-1}(U)\cap Z\neq \emptyset$, on a $\dim Z=\dim W + r$, où $r:=\dim X -\dim Y$
\end{enumerate}
En particulier, pour tout $Y\in U$, on a $\dim f^{-1}(y)=r.$
\end{thm}

On s'intéresse maintenant aux morphismes dont les fibres sont des ensembles finis. Cette étude s'achève avec un résultat important pour la suite, on obtient un critère pour qu'un morphisme bijectif en caractéristique zéro soit un isomorphisme. 

\begin{defn}[Morphisme quasi-fini]
Soit $f:X \rightarrow Y$ un morphisme de variétés. On dit que $f$ est quasi-fini si les fibres de $f$ sont finies.
\end{defn}

\begin{thm}[Théorème principal de Zariski]\label{ZMT}
Soit $f:X\rightarrow Y$ un morphisme birationnel quasi-fini séparé de variétés irréductibles. On suppose de plus que $Y$ est normale. Alors $f$ est une immersion ouverte.
\end{thm}
\begin{proof}
Voir \cite{QingLiu} 4.4.6
\end{proof}

\begin{prop}\label{fibersCardinal}
Soient $X,Y$ deux variétés algébriques affines irréductibles de même dimension et $f:X \mapsto Y$ un morphisme dominant. Alors il existe $g\in \Oo(Y)$ non nulle tel que le morphisme induit $f:X_g \mapsto Y_g$ soit fini, surjectif avec des fibres de même cardinal.
\end{prop}
\begin{proof}
Par hypothèse, l'extension $k(Y) \xrightarrow{f^*} k(X)$ est algébrique finie, disons de degré n. En caractéristique zéro on peut trouver $u\in k(X)$ tel que $k(X)=k(Y)[u]$. On remarque que l'on peut imposer $u\in \Oo(X)$. On considère $P:=P_{min}(u, k(Y))=T^n+a_1T^{n-1}+...+a_0$. En réduisant au même dénominateur on a $P\in \Oo(Y)_v[T]$ pour un $v\in \Oo(Y)$. De plus, en prenant l'intersection avec d'autres ouverts principaux on peut supposer $\Oo(X)_v$ entier sur $\Oo(Y)_v$, et $\Oo(Y)_v[u]$ intégralement clos, ce qui donne $\Oo(Y)_v[u]=\Oo(X)_v$ et $\Oo(X)_v$ entier sur $\Oo(Y)_v$. Ainsi $f:X_v \rightarrow Y_v$ est fini et donc surjectif car dominant.\\
On a donc une factorisation de $f^*:\Oo(Y)_v\xrightarrow{p_1^*}\Oo(Y)_v[T]\xrightarrow{\pi}\Oo(Y)_v[T]/(P)\xrightarrow{\overline{ev_f}}\Oo(Y)_v[u]$ qui donne $f:X_v \xrightarrow{\simeq} \lbrace (y,t) \in Y_v\times \mathbb{A}^1\mid P(y)(t)=0\rbrace \hookrightarrow Y_v\times \mathbb{A}^1 \xrightarrow{p_1} Y_v$. Ainsi le cardinal de la fibre $f^{-1}(y), y\in Y_v$ est le cardinal de l'ensemble des zéros du polynôme $P(y)(T)$. On peut s'assurer que cet ensemble est de cardinal constant en intersectant à nouveau avec l'ouvert principal du discriminant de $P$ qui est un polynôme en les coefficients de $P$.
\end{proof}

Ce résultat reste vrai en caractéristique positive, voir \cite{LAGSpringer} 5.1.6 pour une preuve légèrement différente dans ce cadre. On y montre que le cardinal de la fibre générale est $[k(X):k(Y)]_s$. En revanche pour le corollaire immédiat suivant, la caractéristique zéro est essentielle (penser par exemple au morphisme de Frobenius $\AAA^1 \xrightarrow{x \mapsto x^p} \AAA^1$ ). 

\begin{cor}\label{injectiveBirationel}
Avec les hypothèses de \ref{fibersCardinal}, si de plus $f$ est injectif, alors il existe $g\in \Oo(Y)$ non-nulle tel que le morphisme induit $f:X_g \mapsto Y_g$ soit un isomorphisme.
\end{cor}

\begin{thm}\label{ZMTCor}
Soit $f:X \mapsto Y$ un morphisme bijectif de variétés irréductibles avec $Y$ normale. Alors $f$ est un isomorphisme.
\end{thm}
\begin{proof}
D'après \ref{dimensionfibres} on peut appliquer \ref{injectiveBirationel} et on obtient ainsi que $f$ est birationnel. Ensuite, \ref{ZMT} nous dit que $f$ est une immersion ouverte. Mais $f$ est surjective par hypothèse, c'est donc un isomorphisme.
\end{proof}

\subsection{Variétés lisses}

\begin{defn}[Variété lisse, point régulier]
Soit $X$ une variété. On dit que $x\in X$ est un point régulier si l'anneau local $\Oo_{X,x}$ est régulier. On dit que la variété est lisse si chacun de ses points est régulier. Un point est singulier si il n'est pas régulier.
\end{defn}

Par définition, un point $x$ d'une variété $X$ irréductible est régulier si l'espace tangent $(\mathfrak{m}_x/\mathfrak{m}_x^2)^*$ en $x$ est de dimension $\dim X$. L'ensemble des point singulier d'une variété est un fermé propre. Par ailleurs, le théorème \ref{reglocufd} entraine qu'un point lisse est un point normal. 



\subsection{Propriétés des variétés normales}

\begin{prop}\label{normaluniondisjointe}
Une variété normale est union disjointe de ses composantes irréductibles.
\end{prop}
\begin{proof}
Si un point $p\in X$ d'une variété se situe à l'intersection de deux composantes irréductibles, l'anneau local en $p$ contient au moins deux premiers minimaux et n'est donc pas intègre.
\end{proof}

\begin{prop}\label{codimesingnormal}
Le lieu singulier d'une variété normale est un fermé de codimension $\geq 2$
\end{prop}
\begin{proof}

\end{proof}


\begin{prop}\label{extregularnormal}
Soit $X$ une variété normale irréductible. Pour toute sous-variété fermée $Y$ de codimension $\geq 2$, la restriction $\Oo(X)\rightarrow \Oo(X\setminus Y)$ est un isomorphisme.
\end{prop}
\begin{proof}
On peut traiter le problème localement et supposer $X=\spec A$ affine. On considère $f$ régulière sur $U:=X\setminus Y$. On remarque que tout $p\in \spec A$ de hauteur $1$ est un point de $U$. En effet, dans le cas contraire il contiendrait les idéaux premiers correspondants aux composantes irréductibles de $Y$, ce qui est impossible car ils sont de hauteur $\geq 2$. On en déduit, en tenant compte de l'irréductibilité de $X$, des injections $\Oo_X(U)\xhookrightarrow{} \Oo_p=A_p$ dans les tiges qui peuvent être vues comme des inclusions dans le corps des fonctions rationnelles de $X$. On a ainsi en tenant compte de \ref{factonormal} un morphisme $\Oo_X(U)\xhookrightarrow{} \cap_p A_p=A=\Oo(X)$ qui est inverse de la restriction, d'où le résultat.
\end{proof}

\begin{prop}\label{codimaffinenormal}
Soit $X$ une variété affine irréductible. Pour toute variété irréductible $Y$ contenant $X$, le complémentaire $Y\setminus X$ est de codimension 1.
\end{prop}
\begin{proof}
On considère l'application de normalisation $\eta_Y$. Alors l'application induite $\eta_Y^{-1}(X)\rightarrow X$ est l'application de normalisation de $X$. On en déduit que $\eta_Y^{-1}(X)$ est affine. De plus, comme $\eta_Y$ est finie, la dimension du complémentaire de $X$ ne change pas en remplaçant $Y$ par sa normalisation et $X$ par sa préimage. On peut donc supposer $Y$ normal.

Quitte à soustraire de $Y$ les composantes irréductibles de codimension $1$ de $Y\setminus X$, on peut supposer $\codime_Y(Y\setminus X)\geq 2$. Soit $V$ un ouvert affine de $Y$. Par irréductibilité de $Y$, on a $U:=X\cap V$ non-vide et $\codime_V(V\setminus U)=\codime_Y(Y\setminus X)\geq 2$. Ainsi d'après \ref{extregularnormal}, la restriction $\Oo(V)\rightarrow \Oo(U)$ est un isomorphisme. Comme $U$ et $V$ sont affines, cela signifie que l'inclusion $U\subset V$ est un isomorphisme, d'où $U=V$ puis $X=Y$.
\end{proof}


