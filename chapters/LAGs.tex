\chapter{Groupes algébriques et théorie des invariants}

Notre principale référence pour les groupes algébriques est \cite{LAGSpringer}. On rappelle que dans ce mémoire, le terme groupe algébrique désigne un groupe algébrique affine.

\section{Groupes algébriques}

\subsection{Généralités}

\begin{defn}[Groupe algébrique (affine)]
Un groupe algébrique est une variété affine $G$ munie d'une structure de groupe telle que l'application produit
$$\mu:G\times G\rightarrow G$$
et l'inverse
$$\iota:G\rightarrow G$$
soient des morphismes de variétés. Les morphismes de groupes algébriques sont les morphisme de groupes  qui sont également des morphismes de variétés.
\end{defn}

\begin{ex}
\begin{enumerate}
\item Tout groupe fini est algébrique
\item Le groupe linéaire $\gln_n(k)$ est l'ouvert principal de $\mn_n(k)$ définit par le déterminant. En effet, le produit et l'inverse sont des applications polynomiales en les coordonnées. Par exemple le groupe $\GG_m:=\gln_1(k)$.
\item Tout sous groupe fermé de $\gln_n(k)$ est un groupe algébrique. Par exemple $\sln_n(k)$ le sous-groupe des matrices de déterminant $1$. Mentionnons aussi D$_n$, le sous groupe des matrices diagonales est défini par l'annulation des coefficients hors-diagonale.
\end{enumerate}
\end{ex}

Pour un groupe algébrique $G$, les composantes irréductibles et les composantes connexes coïncident. On note $G^o$ la composante connexe contenant le neutre $e_G$, on l'appelle la composante neutre. C'est un sous-groupe normal et fermé de $G$, et les autres composantes connexes de $G$ sont les classes à gauche de $G^o$.

\subsection{G-variétés, représentations}

\begin{defn}[$G$-variété]
Soit $G$ un groupe algébrique. Une $G$-variété est une variété algébrique $X$ sur laquelle $G$ agit algébriquement. C'est à dire que l'action de $G$ est définie par un morphisme de variétés $a:G\times X\rightarrow X$. Un morphisme de $G$-variétés est un morphisme qui commute à l'action de $G$. 
\end{defn}

\begin{defn}[Orbite, stabilisateur]
Soit $G$ un groupe algébrique, $X$ une $G$-variété, et $x\in X$. L'orbite de $x$ est l'image de l'application d'orbite $a_x=(.,x):G\rightarrow X$, on le note $G.x$. Le stabilisateur est l'ensemble des $g\in G$ tels que $g.x=x$, on le note $Gx$. 
\end{defn}

Le stabilisateur d'un point $x$ de $G$ est un sous-groupe fermé de $G$. En effet, c'est la fibre de $x$ pour l'application d'orbite $a_x$. De plus pour tout $g.x$ dans l'image de $a_x$, la fibre de $g.x$ est $g.G_x$, donc toutes les fibres de $a_x$ sont isomorphes.

\begin{prop}
Soit $G$ un groupe algébrique, $X$ une $G$-variété et $x\in X$.
\begin{enumerate}
\item $G.x$ est ouvert dans $\overline{G.x}$.
\item Toute composante irréductible de $G.x$ a pour dimension $\dim G-\dim G.x$.
\item $\overline{G.x}\setminus G.X$ est une union d'orbites de dimension strictement inférieure à $\dim \overline{G.x}$.
\item $G.x$ est ouvert dans $\overline{G.x}$.
\end{enumerate}
\end{prop}
\begin{proof}
On suppose d'abord $G$ connexe.
\begin{enumerate}
\item D'après \ref{dimensionfibres}, $G.x$ contient un ouvert dense $U$ de $\overline{G.x}$. Or, $G$ est réunion de translatés de $U$.
\item D'après \ref{dimensionfibres}, il existe un ouvert dense de $G.x$ tel que toute les fibres de cet ouvert ont pour dimension $\dim G-\dim G.x=\dim G_x$.
\item $\overline{G.x}\setminus G.x$ est un fermé propre de $\overline{G.x}$ donc de dimension inférieure. Par ailleurs, $\overline{G.x}$ est $G$-stable donc $\overline{G.x}\setminus G.x$ est réunion d'orbites.
\item Enfin si dim($G.x$) est minimal, $\overline{G.x}\setminus G.x$ est vide
\end{enumerate}
Enfin, $G$ n'est pas connexe, on écrit $G=\cup_{i=1}^{n}g_iG^\circ .x=\cup_{i=1}^{n}G^\circ g_i.x$ car $G^o$ est normal. D'où $\overline{G.x}=\cup_{i=1}^{n}\overline{g_iG^\circ .x}$. Les $\overline{g_iG^\circ}$ sont égales où disjointes, c'est donc la décomposition en composantes irréductibles. On construit un ouvert de $\overline{G.x}$ inclus dans $G.x$ en posant $U=G^\circ .x\setminus \cup_{i=2}^{n}\overline{g_iG^\circ .x}$. On a $\dim G^o-\dim (G^o) _x=\dim G -\dim G_x$ car $(G_x)^o \subset(G^o)_x \subset G_x$, d'où $\dim G_x=\dim (G^o) _x$. Or chaque composante de $G.x$ est l'adhérence d'un orbite pour $G^o$, d'où 2) d'après le cas connexe. On a $\overline{G.x}\setminus G.x=\cup_{i=1}^{n}\overline{g_iG^o .x}\setminus g_iG^o .x=\cup_{i=1}^{n}g_i(\overline{G^\circ .x}\setminus G^o .x)$ qui est une union finie de fermés de dimension inférieure à $\overline{G .x}$ ce qui prouve 3). On utilise le même argument pour prouver 4) dans le cas général.
\end{proof}

\begin{prop}
Soit $f:G\rightarrow H$ un morphisme de groupes algébriques. Alors $\Im f$ est fermé, et $\dim G=\dim\Ker f+\dim\Im f$.
\end{prop}
\begin{proof}
$G$ agit sur $H$ par $g.h=f(g).h$. Les orbites sont les classes à droite de $f(G)$, et elles sont toutes isomorphes. Comme il existe un orbite fermé, on obtient que $\Im f$ est fermé. Pour $h\neq e_H$, le stabilisateur $G_h$ est $\Ker f$, d'où $\dim f(G)=\dim G-\ker f$.
\end{proof}


\begin{defn}[$G$-module, simple, semi-simple]
Une représentation de $G$, ou $G$-module (rationnel) est un couple $(V, \rho)$ où $V$ est un $k$-espace vectoriel de dimension finie et $\rho$ un morphisme de groupes algébriques de $G$ dans $GL(V)$. On étend cette définition au cas où $V$ est de dimension infinie en demandant que $V$ soit réunion de $G$-modules de dimension finie.

On dit qu'un $G$-module est simple si il n'admet pas de sous $G$-module non trivial. On dit qu'un $G$-module est semi-simple si tout sous $G$-module admet un $G$-module supplémentaire.
\end{defn}

\begin{prop}\label{GmoduleSectionsGlobales}
Soit $G$ un groupe algébrique et $X$ une $G$-variété. $\Oo(X)$ est naturellement muni d'une action $(g.f)(x):=f(g^{-1}.x), \forall f\in \Oo(X)$, pour tout $g\in G$, $x\in X$. Muni de cette action, $\Oo(X)$ un $G$-module.
\end{prop}
\begin{proof}
On a pour tout $(g,x) \in G\times X$ et $f\in \Oo(X)$, $a^\sharp(X)(f)(g,x)=f(g.x)=\sum_{i=1}^r\phi_i(g)\psi_i(x)$, d'où $g^{-1}.f=\sum_{i=1}^r\phi_i(g)\psi_i\in \Oo(X)$. Ainsi les translaté $g.f$ pour $g\in G$ engendrent un $k$-espace vectoriel $V(f)$ de dimension finie et $G$-stable.

Montrons que l'action est algébrique. Comme les translatés de $f$ engendrent $V(f)$, il suffit de montrer que pour tout $l\in V(f)^*$, $h,g\in G$, on a $g\mapsto l(g.(h.f))\in \Oo(G)$. On prolonge $l$ en $l'\in \vect_k(\psi_1,...,\psi_r)^*$ et on a $ g\mapsto l(g.(h.f))=\sum_{i=1}^r\phi_i((gh)^{-1})l'(\psi_i)\in \Oo(G)$. Finalement $\Oo(X)=\cup_{f\in \Oo(X)}V(f)$ est un G-module.
\end{proof}

\begin{thm}\label{embed}
Soit $G$ un groupe algébrique et $X$ une $G$ variété affine. $X$ est isomorphe en tant que $G$-variété à une sous $G$-variété fermée d'un $G$-module de dimension finie.
\end{thm}
\begin{proof}
D'après la proposition précédente on voit que $\Oo(X)=k[f_1,...,f_n]$ est engendré comme $k$-algèbre par un sous $G$-module $V$ de $\Oo(X)$ de dimension finie, en prenant par exemple la somme des $V(f_i)$ pour $1\leq i\leq n$. On introduit le morphisme $\phi:X\rightarrow V^*, x\mapsto ev_x$, c'est une immersion fermée. En effet, il suffit de montrer que $V$ est dans l'image de $\phi^\sharp$, or $V^{**}\subset \Oo(V^*)$, et pour tout $ev_f\in V^{**}$, on a $\phi^\sharp(ev_f)=ev_f\circ\phi=f$. Enfin, en munissant $V^*$ de la représentation duale, on a pour tout $g\in G, x\in X,f\in V$, $\phi(g.x)(f)=ev_{g.x}(f)=ev_x( g^{-1}.f)=(g^{-1})^t(ev_x)(f)=(g.\phi(x))(f)$, ainsi $\phi$ est un morphisme de $G$-variétés.
\end{proof}

\begin{cor}
Tout groupe algébrique est linéaire.
\end{cor}
\begin{proof}
On munit $G$ de son action sur lui même par multiplication à gauche. D'après le théorème précédent on peut supposer $G\subset V$, où $(V,\rho)$ est un $G$-module avec $\rho(g).h=g.h$ pour tout $g,h\in G$. Montrons que $\rho$ est une immersion fermée. On a $\rho^\sharp(k[\gln(V)])=k[\lbrace g\in G\mapsto l(\rho(g).v)\mid l\in V^*,v\in V\rbrace]$, donc en prenant $v=e_G$, on voit que cette algèbre contient la restriction des fonctions de $\Oo(V)$ à $G$. Cela montre que $\rho^\sharp$ est surjectif car $G$ est fermé dans $V$.
\end{proof}

\begin{defn}[Groupe linéairement réductif]
Un groupe algébrique $G$ est linéairement réductif si tout $G$-module est semi-simple.
\end{defn}

\begin{ex}
Les groupes finis et les groupe diagonalisables sont linéairement réductifs.
\end{ex}

\subsection{Groupes quotients}
\begin{thm}[Chevalley]
Soit $G$ un groupe algébrique et $H\subset G$ un sous-groupe fermé. Alors il existe un $G$-module $V$ de dimension finie et une droite $L\subset V$ telle que $H=Stab_G(L):=\lbrace g\in G\mid g.v\in L,\forall v\in L\rbrace$.
\end{thm}
\begin{proof}
Soit $\mathfrak{a}:=\Ii_G(H)$, on a  $\mathfrak{a}=\lbrace g\in G\mid g.\mathfrak{a}=\mathfrak{a}\rbrace$. En effet, on a les équivalences
$$g.\mathfrak{a}=\mathfrak{a}\iff Hg=H\iff g\in H$$
L'idéal $\mathfrak{a}$ est de type fini, donc engendré par un sous $k$-espace vectoriel $E$ de dimension finie $H$-stable. Alors $H=\lbrace g\in G\mid g.E=E\rbrace$ car si $g$ stabilise $E$, il stabilise aussi l'idéal engendré par $E$, c'est à dire $\mathfrak{a}$. On choisit maintenant un $G$-module $F\subset \Oo(G)$ tel que $E\subset F$. Alors $G$ agit sur les sous-espaces vectoriels de $F$, et c'est en particulier le stabilisateur de $E$. Soit $m=\dim_k E$, alors $\wedge^mE$ est une droite de $\wedge^mF$, et ce dernier est naturellement un $G$-module. Or on a $\stab_G(E)=\stab_G(\wedge^mE)$. Une inclusion est facile, pour l'autre on choisit une base $(v_1,...,v_m)$ de $E$ que l'on complète en une base $(v_1,...,v_n)$ de $F$. Alors $(v_1\wedge...\wedge v_m)$ est une base de $\wedge^mE$, et on remarque que $E=\lbrace v\in F\mid v\wedge v_1\wedge...\wedge v_m=0\rbrace$.
\end{proof}


\begin{thm}[Chevalley]
Soit $G$ un groupe algébrique et $H\lhd G$ n sous-groupe normal fermé. Alors il existe un $G$-module $(V, \rho)$ de dimension finie tel que $H=\Ker{\rho}$.
\end{thm}
\begin{proof}
Soit $V,L$ comme dans le théorème précédent, alors $H$ agit sur $L$ via un caractère $\chi_1$, et on note $V_{\chi_1}:=\lbrace v\in V\mid h.v=\chi_1(h)v,\, \forall h\in H\rbrace$ le sous-espace propre associé à $\chi_1$. On note $V_{\chi_1},...,V_{\chi_n}$ tous les sous-espaces propres de $V$, ils sont en somme directe et cette somme est $G$-stable car $H$ est normal. On peut donc supposer $V=\bigoplus_{i=1}^nV_{\chi_i}$, et comme $L\subset V_{\chi_1}$, on a $H=\lbrace g\in G\mid g.V_{\chi_i}=\chi_i(g)V_{\chi_i}\rbrace$. Maintenant, $G$ agit linéairement sur $\End(V)$ par conjugaison, et en particulier $G$ agit sur $\prod_{i=1}^n\End V_{\chi_i}$. On peut voir $H$ est le noyau de cette action.
\end{proof}

Le théorème suivant est le résultat principal de cette section. Il prouve l'existence des groupes quotients dans la catégorie des groupes algébriques. Le groupe quotient est alors unique à isomorphisme près, c'est une conséquence formelle de la propriété universelle du quotient.

\begin{thm}[Car. 0]\label{groupequotient}
Soient $G, H, (V, \rho)$ comme dans le théorème précédent, et $f:G \rightarrow G'$ un morphisme de groupes algébriques tel que $H\subset \Ker f$.\\
Alors il existe une unique factorisation 	
	\begin{tikzcd}
		G \arrow[r,"f"] \arrow[d,"\rho"] & G' \\
		\rho(G) \arrow[ru, "\exists ! \phi", swap, dashed]
	\end{tikzcd}
\end{thm}
\begin{proof}
	Le morphisme $\phi$ recherché existe en tant que morphisme de groupes abstraits, il est G-équivariant pour les actions naturelles de G sur $\rho(G)$ et $G'$ via $\rho$ et $f$. Concrètement cela signifie $\forall g_1, g_2 \in G, \phi(\rho(g_1)\rho(g_2))=f(g_1)\phi(\rho(g_2))$. Si $G$ est connexe, d'après le lemme suivant, $\phi$ est un morphisme sur un ouvert $U$ non-vide de $\rho(G)$. Or on a un recouvrement de $\rho(G)$ par les ouverts $g.U$ pour $g\in G$. En écrivant pour $x\in g.U,  \phi(x)=f(g)\phi(g^{-1}.x))$, on constate que $\phi$ est un morphisme de groupes algébriques.
	\\Supposons $G$ quelconque mais $H\leq G^\circ$. Comme $\phi$ est algébrique sur le sous-groupe $G^\circ/H$ d'après ce qui précède, on a $\phi$ algébrique partout à nouveau par G-équivariance.\\
	On peut se ramener au cas précédent en procédant en deux étapes. Dans un premier temps, on quotiente par le sous-groupe normal connexe $H^\circ$ (on a bien $H^\circ\leq G^\circ$), puis on quotiente par le sous-groupe normal fini $H/H^\circ$. Il reste donc à prouver le cas $H$ fini, qui sera un corollaire direct du théorème \ref{goodquotientthm}.
\end{proof}

\begin{lem}\label{factoInvariantFibers}
Soit $f:X \mapsto Y$ un morphisme dominant de variétés irréductibles. Soit $g:X \rightarrow Z$ constant sur les fibres de $f$. Alors il existe $h\in \Oo(Y)^*$ et une factorisation
	\begin{tikzcd}
		X_h \arrow[r,"g"] \arrow[d,"f"] & Z \\
		Y_h \arrow[ru, dashed]
	\end{tikzcd}
\end{lem}
\begin{proof}
	\begin{multicols}{2}
	On considère $\phi=(f,g):X\rightarrow Y\times Z$ et le diagramme commutatif ci-contre. Comme $f$ est dominant, $\pi_1$ l'est aussi. De plus $\overline{\phi(X)}$ est irréductible et $\phi(X)$ contient un ouvert dense de $\overline{\phi(X)}$. Par ailleurs comme $g$ est constante sur les fibres de $f$ on vérifie que $\pi_1$ est injective sur $\phi(X)$. Par le corollaire \ref{injectiveBirationel}, $\pi_1$ réalise un isomorphisme $\overline{\phi(X)}_h \xrightarrow{\pi_1} Y_h$ pour un $h\in \Oo(Y)$ non-nul. Finalement, le morphisme recherché est  $Y_h \xrightarrow{\pi_2\pi_1^{-1}} Z$.
	
	\columnbreak
	\begin{center}
	\begin{tikzcd}
  		& X \arrow[ldd,bend right,swap, "f"] \arrow[d, "\phi"] \arrow[rdd, bend left,"g"]  &\\ 
  		& \overline{\phi(X)} \arrow[ld,swap,"\pi_1"] \arrow[d,"i=\subset"] \arrow[rd,"\pi_2"]  &\\ 
		Y & \arrow[l,"p_1"]  Y\times Z \arrow[r,swap,"p_2"]  & Z
	\end{tikzcd}\\
	\end{center}
	\end{multicols}
\end{proof}


\subsection{Groupes diagonalisables, actions de groupes diagonalisables}

\subsubsection{Groupes diagonalisables}
Soit $G$ un groupe algébrique. Le groupe $X^*(G)$ des caractères de $G$ est un sous-groupe de $\Oo(G)^*$ qui est de type fini comme on le verra dans la partie \ref{unitGroupFiniteType}. On remarque que $X^*$ est un foncteur contravariant de la catégorie des groupes algébriques dans la catégorie des groupes abéliens de type fini, l'image d'un morphisme $G_1\xrightarrow{\phi}G_2$ étant simplement la (co)-restriction $\phi^* \,_{|X^*(G_2)}^{|X^*(G_1)}$ du comorphisme $\phi^*$ entre les algèbres de coordonnées. On signale qu'en caractéristique $p>0$, les groupes de caractères ont de plus la propriété d'être sans $p$-torsion. Tout ce qui suit reste vrai en caractéristique $p$, avec cette contrainte supplémentaire sur les groupes de caractères.

\begin{ex}
\begin{enumerate}
\item $X^*(\gln_n)=\lbrace \det^k\mid k\in \ZZ \rbrace\simeq\ZZ$. En effet, $\Oo(\gln_n)^* =\lbrace \lambda\textrm{det}^k\mid k\in \ZZ, \lambda\in k^* \rbrace$, puisque $k[X_{ij}]$ est factoriel et det irréductible. En évaluant en $I_n$, on a nécessairement $\lambda=1$. Comme det$^k$ est un caractère, le résultat suit.
\item $X^*(\sln_n)={1}$ car $D(\sln_n)=\sln_n$.
\item Les unités de $k[\GG_m]$ sont les monômes. On en déduit que les caractères sont exactement les $t\mapsto t^k, k\in \ZZ$. Par ailleurs, on montrera en \ref{unitGroupFiniteType} que $X^*(G_1\times G_2)\simeq X^*(G_1)\times X^*(G_2)$ pour tous groupes algébriques connexes $G_1$ et $G_2$, d'où $X^*(\dn_n)=\lbrace \textrm{monômes à coefficient unitaire} \rbrace \simeq \ZZ^n$.
\end{enumerate}
\end{ex}

\begin{defn}[Groupe diagonalisable, tore]
Un groupe diagonalisable est un groupe algébrique isomorphe à un sous groupe fermé de $\dn_n$. Un tore est un groupe diagonalisable connexe.
\end{defn}

On remarque que les caractères de $\dn_n$ engendrent $\Oo(D_n)$ comme $k$-espace vectoriel, ils en forment donc une $k$-base par le lemme de Dedekind qui assurent que les caractères sont libres dans Map$(\dn_n, k)$. On a plus généralement:

\begin{prop}\label{GroupeDiagCarac}
Soit $G$ un groupe algébrique. Les assertions suivantes sont équivalentes:
\begin{enumerate}
\item $G$ est diagonalisable.
\item $\Oo(G)=\vect_k(X^*(G))$.
\item Tout $G$-module est somme directe $G$-modules de dimension $1$.
\end{enumerate}
Si l'une de ces assertions est vérifiée, tout $G$-module $V$ est naturellement un $k$-espace vectoriel $X^*(G)$ gradué
$$V=\bigoplus_{\chi \in X^*(G)}V_{\chi}$$
où les $V_{\chi}$ sont les sous-espaces propres pour l'action $G$. Réciproquement, tout $k$-espace vectoriel $X^*(G)$ gradué est naturellement un $G$-module.
\end{prop}
\begin{proof}
\begin{enumerate}
\item 1)$\implies$2) La restriction $\Oo(D_n)\xrightarrow{res_G} \Oo(G)$ est surjective et la restriction d'un caractère est un caractère.
\item 2)$\implies$3) $G$ est abélien, car $\forall \chi\in X^*(G),\,\,g,h\in G$, on a $\chi(gh)=\chi(hg)$. Cette relation est donc vérifiée pour toute fonction régulière sur $G$, on en conclut $gh=hg$. On observe que l'action naturelle de $G$ sur $\Oo(G)$ est semi-simple. En effet, les caractères forment une base de diagonalisation de $\Oo(G)$. Ainsi $G$ est semi-simple par la décomposition de Jordan. Soit $(V,\rho)$ le $G$-module considéré et $W\subset V$ un sous $G$-module de dimension finie, disons $n$. Par la décomposition de Jordan,  $\rho_{|W}(G)$ est semi-simple. De plus, c'est un sous-groupe abélien fermé de $\gln_n$. Il est donc conjugué à un sous-groupe fermé de $\dn_n$. On voit ainsi que $V=\bigoplus_{\chi \in X^*(G)}V_\chi$, où $V_\chi:=\lbrace v\in v\mid g.f=\chi(g)f,\, \forall g\in G \rbrace$. En effet, ils sont en somme directe, et tout élément de $V$ se décompose de cette manière.
\item 3)$\implies$1) On peut supposer $G\subset \gln_n$, et considérer l'action naturelle sur $k^n$ après choix d'une base $(e_1,...,e_n)$. Par hypothèse, on peut écrire $k^n=(f_1)\oplus...\oplus (f_n)$, avec les $(f_i)$ sous $G$-module de dimension 1. On obtient donc que $G$ est conjugué à un sous-groupe de $\dn_n$.
\end{enumerate}
\end{proof}


On travaille désormais dans la catégorie des groupes diagonalisables. On considère un groupe diagonalisable $G$ et le groupe $X^{**}(G):=(X^*)^2(G)$.

\begin{prop}
$G$ et $X^{**}(G)$ sont naturellement isomorphes en tant que groupes abstraits, et aussi en tant que groupes algébriques par transport de structure. L'isomorphisme est $ev_G : G\rightarrow \Hom (X^*(G),\GG _m), g\mapsto (\chi\mapsto \chi(g))$.
\end{prop}
\begin{proof}
$ev_G$ est injective, en effet considérons $g\in G$ tel que $\chi(g)=1=\chi(e_G)$, pour tout $\chi\in X^*(G)$. Alors $g=e_G$ car $G$ est un groupe diagonalisable. $ev_G$ est surjective, en effet considérons $\phi\in X^{**}(G)$. On a un prolongement unique de $\phi$ en un morphisme de $k$-algèbre $k[X^*(G)]=\Oo(G)\rightarrow k$ qui est donc de la forme $\Oo(G)\rightarrow k, f\mapsto f(g)$ pour un $g\in G$. En restreignant à $X^*(G)$, on trouve que $\phi=\textrm{ev}_G(g)$.
\end{proof}

\begin{cor}\label{EqCatGpDiagGpAb}
Le foncteur $X^*$ réalise une anti-équivalence de catégories entre la catégorie des groupes diagonalisables et la catégorie des groupes abéliens de type fini. 
\end{cor}
\begin{proof}
En effet on a un isomorphisme de foncteurs $(X^*)^2\simeq Id$ d'après la proposition précédente.
\end{proof}

Une autre façon de voir cela est d'introduire l'algèbre de groupe d'un groupe abélien de type fini $M$, c'est par définition $k[M]=\lbrace \sum_{\text{finie}} \lambda_gg,\,\lambda_g\in k,g\in G \rbrace$ avec la multiplication définie par l'opération de groupe de $G$. La propriété suivante montre que l'on construit ainsi un autre inverse de $X^*(.)$.

\begin{prop}
Soient, $M, M_1, M_2$ des groupes abéliens de type fini, et $G$ un groupe diagonalisable.\\
Alors $k[M]$ est de type fini, réduite et on a $k[M_1\oplus M_2]\simeq k[M_1]\otimes k[M_2]$. De plus, $k[M]$ est naturellement muni d'une structure d'algèbre de Hopf et on a $G=\spec k[X^*(G)]$.
\end{prop}
\begin{proof}
On a deux morphismes d'algèbre $k[M_1]\rightarrow k[M_1\oplus M_2], e_{m_1}\mapsto e_{(m_1,0)}$ et $k[M_2]\rightarrow k[M_1\oplus M_2], e_{m_2}\mapsto e_{(0,e_{m_2})}$, d'où l'existence d'un morphisme $k[M_1]\otimes_k k[M_2]\rightarrow k[M_1\oplus M_2], $ dont on vérifie que c'est un isomorphisme.
Pour la deuxième assertion, comme on a $M\simeq \ZZ^r\oplus (\oplus_{i=1}^r \ZZ/d_i\ZZ)$, il suffit de traiter les cas $M=\ZZ$ et $M=\ZZ/d\ZZ$. On a $k[\ZZ]\simeq k[t,t^{-1}]$ qui est intègre, de type fini, et réduite. On a $k[\ZZ/d\ZZ]\simeq k[t]/(t^d-1)$. On voit, par le théorème chinois par exemple, que cette algèbre de type fini non-intègre est réduite si et seulement si les racine de $t^d-1$ sont simples, ce qui est le cas en caractéristique zéro. La structure d'algèbre de Hopf sur $k[M]$ est donnée par $\Delta(e_m)=e_m\otimes e_m,\, i(e_m)=e_{-m},\, e(e_m)=e_0$. Pour la dernière assertion, il suffit de voir que $\Oo(G)=k[X^*(G)]$.
\end{proof}

\begin{cor}
Soit $G$ un groupe diagonalisable. Alors:
\begin{enumerate}
\item $G$ est isomorphe au produit direct d'un tore et d'un groupe abélien fini.
\item $G$ est un tore $\iff X^*(G)$ est libre de type fini $\iff G$ est connexe.
\end{enumerate}
\end{cor}


\subsubsection{Action d'un groupe diagonalisable sur une variété affine}

Les variétés affines munies d'une action d'un groupe diagonalisable forment une catégorie dans laquelle les morphismes sont les paires $(\phi,\widetilde{\phi})$, où $\phi:X\rightarrow X'$ est un morphisme de variétés affines, $\widetilde{\phi}:H\rightarrow H'$ est un morphisme de groupes diagonalisables, et pour tout $x\in X, h\in H$, on a $\phi(h.x)=\widetilde{\phi}(h)\phi(x)$.

D'après \ref{GmoduleSectionsGlobales}, $\Oo(X)$ est un $H$-module, et aussi un $k$-espace vectoriel $K$-gradué avec $K:=X^*(H)$, d'après \ref{GroupeDiagCarac}. C'est en fait une $k$-algèbre $K$-graduée, en effet considérons $(f_1, f_2)\in \Oo(X)_{\chi_1}\times \Oo(X)_{\chi_2}$, alors pour tout $g\in G$, $g.f_1f_2=(g.f_1)(g.f_2)=(\chi_1(g)f_1)(\chi_2(g)f_2)=(\chi_1\chi_2)(g)f_1f_2$, ce qui prouve que $f_1f_2\in \Oo(X)_{\chi_1\chi_2}$. De plus, cette construction est fonctorielle, en effet considérons une $H'$-variété $X'$, ainsi qu'un morphisme $(\phi,\widetilde{\phi}):X\rightarrow X'$. On a donc par définition le diagramme de gauche ci-dessous, le second est obtenu par application de $\Oo(.)$.

\begin{multicols}{2}
	\begin{center}
		\begin{tikzcd}
  		H\times X \arrow[r, "a_1"] \arrow[d, "\widetilde{\phi} \times \phi"] & X \arrow[d, "\phi"] \\ 
  		H'\times X' \arrow[r, "a_2"] & X'
	\end{tikzcd}
	\end{center}

	\columnbreak
	\begin{center}
		\begin{tikzcd}
  		\Oo(H)\otimes \Oo(X)   & \Oo(X) \arrow[l, "a_1^\sharp"] \\ 
  		\Oo(H')\otimes \Oo(X') \arrow[u, "\widetilde{\phi}^\sharp \otimes \phi^\sharp"] & \Oo(X') \arrow[u, "\phi"] \arrow[l, "a_2^\sharp"] 
	\end{tikzcd}
	\end{center}
\end{multicols}

Le petit calcul ci-dessous montre que $\forall f\in \Oo(X')_{\chi'},\,$ on a $\phi^\sharp(f)\in \Oo(X)_{\widetilde{\phi}^\sharp(\chi')}$. cela montre que $(\widetilde{\phi}^\sharp,\phi^\sharp)$ est un morphisme d'algèbres graduées car $\widetilde{\phi}^\sharp$ envoie des caractères sur des caractères en tant que comorphisme d'un morphisme de groupes diagonalisables. 
$$\forall h\in H,\,x\in X,\, \phi^\sharp(f)(h.x)=a_1^\sharp\phi^\sharp(f)(h, x)=(\widetilde{\phi}\otimes\phi)a_2^\sharp(f)(h,x)=\widetilde{\phi}^\sharp(\chi')(h)\phi^\sharp(f)(x)$$ 

Réciproquement, considérons une $k$-algèbre affine $A$ graduée par un groupe abélien $K$ de type fini. On a un morphisme de algèbres affine naturel que l'on peut définir sur les éléments homogènes par $a:A\rightarrow k[K]\otimes A,\, f_{\omega}\mapsto \chi^{\omega}\otimes f_{\omega}$. On voit facilement que ce morphisme définit une action de $H:=\spec k[K]$ sur $X:=\spec A$. De plus, cette association est fonctorielle, en effet considérons $A'$ une algèbre affine $K'$-graduée , ainsi qu'un morphisme $(\widetilde{\phi},\phi):A\rightarrow A'$ d'algèbres graduées. On a pour tout $f_w\in A_w$
$$(\widetilde{\phi}\otimes \phi)\circ a(f_\omega)=(\widetilde{\phi}\otimes \phi)(\chi^{\omega}\otimes f_\omega) = \chi'^{\widetilde{\phi}(\omega)}\otimes \phi(f_\omega)_{\widetilde{\phi}(\omega)}$$ 
$$a'\circ \phi(f_\omega)=a'(\phi(f_\omega)_{\widetilde{\phi}(\omega)})=\chi'^{\widetilde{\phi}(\omega)}\otimes \phi(f_\omega)_{\widetilde{\phi}(\omega)}$$
D'où le diagramme commutatif de gauche ci-dessous, qui donne le diagramme de droite par application du foncteur $\spec$. Ce dernier diagramme finit de prouver la fonctorialité.
\begin{multicols}{2}
	\begin{center}
			\begin{tikzcd}
  		K\otimes A  \arrow[d, "\widetilde{\phi}^* \otimes \phi^*"]  & A \arrow[l, "a"] \arrow[d, "\phi"]  \\ 
  		K'\otimes A' & A' \arrow[l, "a'"] 
	\end{tikzcd}
	\end{center}

	\columnbreak
	\begin{center}
		\begin{tikzcd}
  		\spec(k[K])\times \spec(A) \arrow[r, "a^o"]  & \spec(A) \\ 
  		\spec(k[K'])\times \spec(A') \arrow[r, "a'^o"] \arrow[u, "\widetilde{\phi} \times \phi"] & \spec(A') \arrow[u, "\phi"] 
	\end{tikzcd}
	\end{center}
\end{multicols}

Finalement on a obtenu le résultat suivant qui va nous permettre dans la suite caractériser l'action d'un groupe diagonalisable sur une variété affines en termes algébriques.

\begin{prop}
Les foncteurs $\spec$ et $\Oo(.)$ réalisent une équivalence de catégories entre les variétés affines munies d'une action d'un groupe diagonalisable et les algèbres affines graduées par un groupe abélien de type fini. 
\end{prop}

\begin{prop}\label{stablehomogene}
Soit $A$ une algèbre affine $K$-graduée, $X:=\spec A$ muni de l'action de $H:=\spec k[K]$.  Soit $Y\subset X$ une sous variété fermée, et $\mathfrak{a}=\mathcal{I}_X(Y)$. Les assertions suivantes sont équivalentes:
\begin{enumerate}
\item $Y$ est $H$-stable.
\item $\mathfrak{a}$ est un idéal homogène.
\end{enumerate}
\end{prop}
\begin{proof}
$Y$ est $H$-stable si et seulement si pour tout $P\in \mathfrak{a},\,h\in H,\, y\in Y,\, P(h.y)=0$. Supposons que cette dernière condition soit vérifiée, on écrit $P=\sum P_w$ sa décomposition en composantes homogènes. On a alors, $P(h.y)=\sum \chi^w\otimes P_w(h,y) = \sum \chi^w(h) P_w(y)=0$. Les caractères étant libres, on a $P_{\omega}(y)=0$, pour tout $w\in K$ intervenant dans la décomposition, et tout $y\in Y$, c'est à dire $P_w\in \sqrt{\mathfrak{a}}=\mathfrak{a}$. Ainsi $\mathfrak{a}$ est homogène. La réciproque est immédiate. 
\end{proof}

Soit $X$ une $H$-variété affine où $H$ est un groupe diagonalisable. On souhaite étudier l'orbite d'un point $x\in X$. En choisissant des générateurs homogènes $f_{w_1},...,f_{w_r}$ de $\Oo(X)$, on peut supposer $X$ plongée dans un espace affine $\AAA^n$ avec l'action de $H$ donnée par $h.t=(\chi^{w_1}(h)t_1,...,\chi^{w_r}(h)t_r)$, pour tout $t=(t_1,...,t_r)\in X\subset \AAA^n$ et $h\in H$. On voit alors que la géométrie de l'orbite d'un point $p\in X$ va être assez dépendante de la nullité des coordonnées $t_1,..., t_r$ en $p$. Cela motive la définition suivante.

\begin{defn}[Monoïde d'orbite, Groupe d'orbite]
Soit $A$ une algèbre affine $K$-graduée munie de l'action de $H:=\spec k[K]$ sur $X:=\spec A$.
\begin{enumerate}
\item Le monoïde d'orbite d'un point $x\in X$ est le sous-monoïde $S_x\subset K$ engendré par $\lbrace\omega \in K\mid \exists f\in A_{\omega} \textrm{ telle que } f(x)\neq 0\rbrace$
\item Le groupe d'orbite d'un point $x\in X$ est le sous-groupe $K_x\subset K$ engendré par le monoïde d'orbite.
\end{enumerate}
\end{defn}

Pour l'étude de la géométrie de l'orbite d'un point, les diagrammes ci-dessous seront d'une grande aide. Le diagramme de gauche est obtenu avec \ref{factoInvariantFibers} (ou \ref{groupQuotientGeometrique}) et \ref{ZMTCor}. Le diagramme de droite est obtenu par application du foncteur $\Oo(.)$.

	\begin{equation}\label{eq:DiagOrbite1}
	\begin{tikzcd}
		H \arrow[r,"h \mapsto h.x", twoheadrightarrow] \arrow[d,"\pi", twoheadrightarrow] & H.x \\
		H/H_x \arrow[ru, "\exists !\simeq \phi", swap, dashed]
	\end{tikzcd}
	\,\,\,\,\,\,\,\,\,\,
	\begin{tikzcd}
		\Oo(H) & & \Oo(H.x) \arrow[lld, "\exists !\simeq \phi^\sharp", dashed]\arrow[ll, hook, "f_\omega \mapsto f_\omega(x)\chi^{\omega}", swap]  \\
		\Oo(H/H_x)  \arrow[u,"\pi^*", hook]  & &
	\end{tikzcd}
	\end{equation}



\begin{prop}\label{staborbitegroup}
Soit $A$ une algèbre affine $K$-graduée, $X:=\spec A$ muni de l'action de $H:=\spec k[K]$, et $x\in X$. On a le diagramme commutatif suivant, dont les deux lignes sont exactes:
	\begin{center}
		\begin{tikzcd}
  		0  \arrow[r] & K_x\arrow[r] \arrow[d, "\simeq"] & K \arrow[r] \arrow[d, "\omega \xmapsto{\simeq} \chi^{\omega}"] & K/K_x \arrow[d, "\simeq"] \arrow[r] & 0 \\ 
  		0 \arrow[r]& X^*(H/H_x) \arrow[r, "\pi^*"] & X^*(H)  \arrow[r,"i^*"] & X^*(H_x) \arrow[r] & 0
	\end{tikzcd}
	\end{center}
	où $i:H_x\rightarrow H$ est l'inclusion du stabilisateur de $x$, et $\pi:H\rightarrow H/H_x$ la projection canonique. En particulier, on obtient $H_x\simeq \spec(k[K/K_x])$.
\end{prop}
\begin{proof}
La deuxième ligne est obtenue par le théorème \ref{groupequotient}, puis application du foncteur $X^*$, elle est bien exacte par exactitude du foncteur $X^*$. La flèche verticale centrale est l'isomorphisme canonique $X^*(\spec k[K])\simeq K$. Comme $\overline{H.x}$ est $H$-stable, la graduation est préservée sur $\Oo(\overline{H.x})$ d'après la proposition \ref{stablehomogene}. De plus si $f\in \Oo(X)_{\omega}$ est telle que $f(x)\neq 0$, cela reste le cas modulo $\mathcal{I}_{X}(\overline{H.x})_\omega$ et réciproquement. Ainsi, $S_x$ et $K_x$ ne sont pas modifiés si on remplace $X$ par $\overline{H.x}$. 
Considérons le fermé propre $\overline{H.x}\setminus H.x$, éventuellement vide. Il est $H$-stable comme réunion d'orbites. Alors $\mathcal{I}_{\overline{H.x}}(\overline{H.x}\setminus H.x)$ est homogène et $\neq \lbrace 0\rbrace$. Choisissons $f\neq0$ homogène dans ce $H$-module, c'est donc un vecteur propre. Ainsi, $f$ est non-nulle quelque part sur $H.x$ et donc partout par transitivité et choix de $f$. On en conclut $H.x=(\overline{H.x})_f$. Considérons la graduation naturelle associée à $\Oo(\overline{H.x})_f$. Comme on a inversé $f$, le monoïde de poids est potentiellement plus gros. En revanche on voit facilement que $K_x$ n'est pas modifié. Finalement, on peut donc supposer $X=H.x$, qui est affine comme on va le voir. 

Dans ce cas, on voit qu'une fonction homogène non-nulle est partout non-nulle, donc inversible. On a donc dans ce cas $S_x=K_x=S(\Oo(H.x))=K(\Oo(H.x))$, où $S(\Oo(H.x))$ et $K(\Oo(H.x))$ sont respectivement les monoïdes et groupes de poids de l'algèbre $K$-graduée $\Oo(H.x)$. Dans les deux diagrammes de \ref{eq:DiagOrbite1}, les flèches sont respectivement des morphismes de $H$-variétés, et des morphismes d'algèbres graduées. Ainsi, $\phi^\sharp$ induit un isomorphisme sur les groupes de poids $K(\Oo(H.x))=K_x \xrightarrow{w \mapsto\chi^w}K(\Oo(H/H_x))=X^*(H/H_x)$. Enfin, toujours en utilisant le diagramme, cet isomorphisme est l'unique faisant commuter le carré de gauche dans le diagramme de la proposition.
\end{proof}

Dans la preuve précédente, on en particulier le résultat intéressant suivant

\begin{lem}
Soit $A$ une algèbre affine $K$-graduée, $X:=\spec A$ muni de l'action de $H:=\spec k[K]$, et $x\in X$. Alors il existe $f\in\Oo(\overline{H.x})$ homogène non-nul tel que $H.x=(\overline{H.x})_f$.
\end{lem}
\begin{proof}
\end{proof}

\begin{prop}
Soit $A$ une algèbre affine $K$-graduée, $X:=\spec A$ muni de l'action de $H:=\spec k[K]$, et $x\in X$. Cette action induit une action de $H/H_x$ sur $\overline{H.x}$. De plus, $\overline{H.x}$ et $\spec(k[S_x])$ sont isomorphes en tant que $H/H_x$-variétés.
\end{prop}
\begin{proof}
On suppose $X=\overline{H.x}$ ce qui ne modifie pas $S_x$ et $K_x$. De plus on a $S(k[\overline{H.x}])=S_x$ et $K(k[\overline{H.x}])=K_x$. D'après la proposition précédente et \ref{EqCatGpDiagGpAb}, on a des isomorphismes canoniques $k[K_x]\simeq k[X^*(H/H_x)]\simeq \Oo(H/H_x)$ et le diagramme \ref{eq:DiagOrbite1} induit un isomorphisme d'algèbres $K_x$-graduées $\Oo(H.x)\rightarrow k[K_x],\, f_w \mapsto f_\omega(x)\chi^w$. On a également le morphisme de $K_x$-algèbres graduées $\Oo(\overline{H.x})\rightarrow k[S_x]$ définit par la même formule. On voit facilement qu'il est injectif. Pour la surjectivité, on peut par exemple voir en utilisant le lemme précédent que le premier isomorphisme est en fait le prolongement par localisation en un élément homogène $g\in\Oo(\overline{H.x})$ de ce morphisme. On obtient donc le diagramme commutatif d'algèbres affines $K_x$-graduées ci-dessous. Cela donne la proposition par application de $\spec$.

	\begin{center}
	\begin{tikzcd}
		\Oo(H.x)=\Oo(\overline{H.x})_g \arrow[r,"\simeq"] & k[K_x]  \\
		\Oo(\overline{H.x}) \arrow[r, "\simeq"] \arrow[u,"f\mapsto f_{|H.x}"] & k[S_x] \arrow[u,"\subset"] 
	\end{tikzcd}
	\end{center}

\end{proof}


\begin{prop}
Soit $A$ une algèbre affine intègre $K$-graduée, $X:=\spec A$ muni de l'action de $H:=\spec k[K]$. Alors il existe un ouvert affine non-vide $U\subset X$ tel que:
$$S_x=S(A),\,\,\, K_x=K(A),\,\,\, \forall x\in U$$
\end{prop}
\begin{proof}
On choisit des générateurs homogènes $f_1,...,f_r$ de $A$. On pose $U:=X_{f_1...f_r}$ qui est non-vide car $A$ est intègre. Pour tout $w \in S(A)$, il existe $g\neq0 \in A_w$ car $A$ est intègre. Quitte à décomposer $g$,  on peut le supposer de la forme $f_1^{i_1}...f_r^{i_r}$. On en déduit que $w \in S_x$, pour tout $x\in U$. Comme l'autre inclusion est immédiate, $U$ satisfait la propriété.
\end{proof}


Pour finir voyons une caractérisation algébrique des actions fidèles de groupes diagonalisables sur des variétés affines.

\begin{prop}
Soit $A$ une algèbre affine intègre $K$-graduée, $X:=\spec A$ muni de l'action de $H:=\spec k[K]$. L'action de $H$ est fidèle si et seulement si $K=K(A)$.
\end{prop}
\begin{proof}
Soit $g\in \cap_{x\in X}H_x$. L'action de $g$ sur les fonctions régulières est triviale, donc en particulier sur les fonctions homogènes. Comme $A$ est intègre, $\forall \omega \in S(A), \exists f_\omega\neq 0\in A_\omega$, on en déduit $g\in \cap_{\chi \in S(A)} \Ker(\chi)$ puis facilement $g\in \cap_{\chi \in K(A)} \Ker(\chi)$ d'où $g=e_H$ si $K=K(A)$ car $H$ est un groupe diagonalisable.

Sinon comme $K(A)\subsetneq K$, on peut choisir $g\neq e_H\in H$ tel que $g\in\cap_{\chi \in K(A)} \Ker(\chi)$. En effet $\cap_{\chi \in K(A)} \Ker(\chi)$ est un sous-groupe fermé non-trivial de $H$ car son groupe de caractère est $K/K(A)$. Ainsi $g$ agit trivialement sur les fonctions homogènes et donc sur les fonctions régulières. On en déduit que $g$ agit trivialement sur $X$.
\end{proof}

\section{Théorie des invariants}
\subsection{L'algèbre des invariants}

Soit $G$ un groupe algébrique et $X$ une $G$-variété affine. $\Oo(X)$ est un $G$-module rationnel pour l'action naturelle de $G$ sur les fonctions régulières. On définit la sous-algèbre des invariants $\Oo(X)^G:=\lbrace f\in \Oo(X)\mid g.f=f,\, \forall g\in G\rbrace$. C'est par définition la sous-algèbre des fonctions constantes sur les orbites de l'action de $G$ sur $X$.

Une question naturelle est de se demander si cette algèbre est de type fini sur $k$, c'est le $14$e des $23$ problèmes que Hilbert posa en $1900$ à la communauté mathématique. En $1959$, Nagata exhiba une algèbre d'invariants pour l'action d'un groupe algébrique qui n'est pas de type fini sur $k$, répondant ainsi au problème. Avec des hypothèses sur $G$, on peut cependant s'assurer de la finitude de l'algèbre des invariants, c'est l'objectif de cette partie.

Soit $G$ linéairement réductif, et $V$ un $G$-module. Le sous-espace $V^G$ des éléments $G$-invariants admet un supplémentaire $G$-stable que l'on note $V_G$. On définit l'opérateur de Reynolds $R_V$ comme la projection sur $V^G$ associée à cette décomposition. Voici quelques propriétés de $R_V$:


\begin{prop}\label{reynolds}
\begin{enumerate}
\item Soit $f:V \rightarrow W$ un morphisme de $G$-module et $f^G:V^G \rightarrow W^G$ le morphisme induit. On a $R_Wf=f^GR_V$. En particulier, si $f$ est surjective, $f^G$ l'est aussi.
\item $R_{\Oo(X)}$ est $\Oo(X)^G$-linéaire
\end{enumerate}
\end{prop}
\begin{proof}
Seule la deuxième assertion n'est pas évidente. Considérons $a\in \Oo(X)^G$ et $m_a$ la multiplication par $a$ dans $\Oo(X)$. C'est un endomorphisme de $G$-module, il commute donc avec $R_{\Oo(X)}$. 
\end{proof}

\begin{thm}[Hilbert]\label{hilbert}
Soit $G$ un groupe linéairement réductif et $X$ une $G$-variété affine. Alors l'algèbre des invariant $\Oo(X)^G$ est de type fini.
\end{thm}
\begin{proof}
Supposons que $X$ soit un $G$-module $V$ de dimension finie. L'action de $\GG_m$ sur $V$ par homothétie donne une $\NN$-graduation $\Oo(V)=\bigoplus_{n=0}^{\infty}\Oo(V)_n$, $\Oo(V)_n$ étant le sous espace des polynômes homogènes de degré $n$. Cette graduation est $G$-stable et se restreint sur l'algèbre des invariants en une $\NN$-graduation $\Oo(V)^G=\bigoplus_{n=0}^{\infty}\Oo(V)^G_n$. Or on remarque que $\Oo(V)^G$ est noetherien. En effet, soit $I$ un idéal de $\Oo(V)^G$, et $J$ son extension dans $\Oo(V)$. L'idéal $J$ est un sous $G$-module, donc la contraction de $J$ dans $\Oo(V)^G$ est $R_{\Oo(V)}(J)=IR_{\Oo(V)}(\Oo(V))=I$. On voit donc que la condition de chaîne est satisfaite sur $\Oo(V)^G$ si elle satisfaite sur $\Oo(V)$, ce qui est le cas car ce dernier est noetherien par le théorème de la base de Hilbert. 

Dans le cas général, on peut d'après le théorème \ref{embed} supposer $X$ inclus dans un $G$-module $V$. On obtient alors un $G$-morphisme surjectif $\Oo(V) \rightarrow \Oo(X)$ qui induit un $G$-morphisme surjectif $\Oo(V)^G \rightarrow \Oo(X)^G$ d'après la proposition \ref{reynolds}. Cela montre que $\Oo(X)^G$ est de type fini.
\end{proof}

On constate que cette preuve n'est pas effective. Il est en général difficile de calculer l'algèbre des invariants. On présente maintenant la méthode des sections qui permet le calcul dans certains cas. Soit $S\subset X$ une sous-variété fermée. Définissons 
$$Z(S):=\lbrace g\in G\mid g.s=s,\,\forall s \in S\rbrace\text{ et }N(S):=\lbrace g\in G\mid g.s\in S,\,\forall s \in S\rbrace$$
Clairement, $Z(S)$ est un sous-groupe normal de $N(S)$, et le quotient $W=N(S)/Z(S)$ agit sur $S$. La surjection $\Oo(X)\rightarrow k[S]$, induit un morphisme $\Oo(X)^G\xrightarrow{\phi} k[S]^W$. Supposons que l'on ait un ouvert dense $U\subset X$ tel que pour tout $x\in U,\, G.x$ intersecte $S$, alors on voit que $\phi$ est injective. Si de plus, $k[S]^W$ est engendré par des éléments $\phi(f_1),...,\phi(f_r)$, alors $\phi$ est un isomorphisme et $\Oo(X)^G$ est engendré par $f_1,...,f_r$.

\begin{ex}
$G=\gln_n$,\, $X=\mn_n$,\, $g.A=gAg^{-1},\, S=\dn_n,\, U=X_{\disc(\chi )}$, où $\chi$ est le polynôme caractéristique générique sur $X$ (c'est bien une fonction régulière sur $X$). En considérant un élément de $U$, qui a donc ses valeurs propres deux à deux distinctes, on a par un calcul direct $Z(S)=\dn_n$. Puis on a $N(S)=\lbrace\textrm{matrices monomiales}\rbrace$ car la conjugaison préserve les espaces propres. Ainsi, $W$ est isomorphe au groupe symétrique $\Sigma_n$ et agit sur $S$ en permutant les entrées diagonales. On a ainsi $k[S]^W=k[\sigma_1,...,\sigma _n]$, l'algèbre engendrée par les fonctions symétriques élémentaires. C'est une algèbre de polynômes car les $\sigma_i$ sont algébriquement libres. Soient $f_1,...,f_n$ les coefficient du polynôme caractéristique générique. Ce sont des éléments de $\Oo(X)^G$, et on a $f_{i| S}=(-1)^i\sigma _i$, d'où $\Oo(X)^G=k[f_1,...,f_n]$.
\end{ex}

\subsection{Quotient d'une variété algébrique sous l'action d'un groupe algébrique}

\subsubsection{Quotient catégorique}

Soit $G$ un groupe algébrique et $X$ une $G$-variété. En tant que groupe abstrait agissant sur un ensemble, le quotient de $X$ par $G$ (noté $X/G$) est par définition l'ensemble des orbites. On note $\pi:X \rightarrow X/G$ l'application qui à un élément de $X$ associe son orbite. $X/G$ satisfait une propriété universelle, il représente le foncteur $\textrm{Ens}\rightarrow \textrm{Ens}, Y\mapsto \lbrace f\in \textrm{Map}(X, Y)\mid f \textrm{ est constante sur les orbites} \rbrace$, il est donc unique à isomorphisme près. Pour cette raison la paire $(X
/G, \pi)$ est appelée le quotient catégorique de $X$ par $G$.

On peut ainsi transporter cette définition dans la catégorie des variétés algébriques. Toutefois, il n'est pas clair que ce quotient existe toujours. L'exemple suivant montre que lorsqu'il existe, le quotient catégorique ne coïncide pas nécessairement avec l'ensemble des orbites.

\begin{ex}
On considère l'action naturelle de $\gln_n$ sur $\AAA^n$. Le quotient catégorique existe et est un point. En effet soir $f:\AAA^n\rightarrow Z$ constant sur les orbites, alors $f$ est constante car il existe un orbite dense. En revanche il y a un deuxième orbite, c'est le fermé $\lbrace 0\rbrace$. 
\end{ex}

On suppose $X$ affine et $\Oo(X)^G$ de type fini, c'est en particulier le cas lorsque $G$ est linéairement réductif d'aprés le théorème \ref{hilbert}. Dans ce cadre, l'algèbre des invariants définit une variété algébrique affine $Y=\spec \Oo(X)^G$ muni d'un morphisme $\pi$ défini par l'inclusion $\Oo(X)^G\subset \Oo(X)$. On constate que tout morphisme $G$-invariant de variétés affines $X\rightarrow Z$ se factorise à travers $Y$, car le comorphisme est alors à valeurs dans $\Oo(X)^G$. De ce fait, $Y$ semble être un bon candidat pour le quotient catégorique. Toutefois il faut être prudent, dans \cite{LAGFerrer} 6.4.10, on exhibe un exemple de cette situation qui n'admet pas de quotient catégorique. En effet, le morphisme d'un quotient catégorique est toujours surjectif, c'est un conséquence de la propriété universelle de factorisation (Voir  \cite{LAGFerrer} 6.4.5). Or $\pi$ n'est pas nécessairement surjectif. On a toutefois le résultat suivant:

\begin{thm}\label{goodquotientthm}
Soit $G$ un groupe linéairement réductif et $X$ une $G$-variété affine.
\begin{enumerate}
\item Le morphisme quotient $\pi:X\rightarrow Y$ est surjectif.
\item $(Y, \pi)$ est un quotient catégorique.
\item Soit $Z\subset X$ une sous $G$-variété fermée. Le morphisme induit $Z/G \rightarrow X/G$ est une immersion fermé. On peut ainsi identifier $\pi_Z$ et $\pi_X$ restreint à $Z$. De plus, soit $Z'$ une autre sous $G$-variété fermée, on a $\pi_X(Z\cap Z')=\pi_X(Z)\cap\pi_X(Z')$.
\item Chaque fibre de $\pi_X$ contient un unique orbite fermé.
\end{enumerate}
\end{thm}
\begin{proof}
\begin{enumerate}
\item Soit $x\in Y$ et $\mathfrak{m}_x$ l'idéal maximal de $\Oo(X)^G$ correspondant. La fibre $\pi^{-1}(x)$ correspond à l'ensemble des idéaux maximaux contenant l'extension $I$ de $\mathfrak{m}_x$ dans $\Oo(X)$. Or on a déjà vu que l'extension des idéaux de $\Oo(X)^G$ dans $\Oo(X)$ était injective, $I$ est donc un idéal propre contenu dans au moins un idéal maximal. La fibre étant non-vide, $\pi$ est surjective.
\item L'existence de la factorisation a déjà était vue. Avec 1) on a maintenant l'unicité.
\item On note $i$ l'inclusion $Z\subset X$. $\pi_Xi$ est constant sur les orbites de $Z$ d'où l'existence d'un unique morphisme $\phi:Z/G\rightarrow X/G$ tel que $\phi\pi_Z=\pi_Xi$. La projection $\Oo(X) \rightarrow \Oo(Z)$ est un morphisme de $G$-module surjectif. D'après la proposition \ref{reynolds}, cette projection induit un morphisme de $k$-algèbre surjectif $\Oo(X)^G \xrightarrow{\phi^*} k[Z]^G$, donc $\phi$ est une immersion fermée.
Soit $I$ (resp. $I'$) l'idéal de $Z$ (resp. $Z'$) dans $\Oo(X)$. L'idéal de $Z\cap Z'$ est $I+I'$ et l'idéal de $\pi_X(Z)$ est $\mathcal{I}_{X/G}(\pi_X(Z))=I\cap \Oo(X)^G=R_X(I)$. Ainsi $\mathcal{I}_{X/G}(\pi_X(Z\cap Z'))=R_X(I+I')=R_X(I)+R_X(I')=\mathcal{I}_{X/G}(\pi_X(Z)\cap \pi_X(Z'))$.
\item D'après 3), $\pi_X$ envoie deux orbites fermés distincts sur deux points distincts.
\end{enumerate}
\end{proof}

On remarque que les propriétés du théorème précédent s'étendent automatiquement au cas d'une $G$-variété $X$ si le théorème est vérifié localement sur un recouvrement affine d'un candidat $(Y,\pi)$ pour le quotient $X/G$. De ce constat découle la notion de bon quotient.

\begin{defn}[Bon quotient]
Soit $G$ un groupe linéairement réductif et $X$ une $G$-variété. Une paire $(Y, \pi)$ où $Y$ est une variété et $\pi$ un morphisme $X\rightarrow Y$ est un bon quotient si elle vérifie:
\begin{enumerate}
\item $\pi$ est affine et $G$-invariant.
\item $\pi^\sharp:\Oo_Y \rightarrow (\pi_*\Oo_X)^G$ est un isomorphisme.
\end{enumerate}
Un bon quotient est noté $X//G$.
\end{defn}

\begin{ex}
Soit $G$ un groupe linéairement réductif et $X$ une $G$-variété affine. D'après le théorème \ref{goodquotientthm} et l'exemple \ref{exaff}, $X//G$ est un bon quotient.
\end{ex}


\subsubsection{Quotient géométrique}

Parmi les quotients catégoriques $(X/G,\pi)$, on cherche à caractériser ceux ayant les propriétés géométriques intuitivement attendues pour un quotient, c'est à dire que $X/G$ soit l'ensemble des orbites avec une topologie aussi fine que possible. C'est la notion de quotient géométrique:

\begin{defn}[Quotient géométrique]
Soit $G$ un groupe algébrique et $X$ une $G$-variété. Une paire $(Y, \pi)$ où $Y$ est une variété et $\pi$ un morphisme $X\rightarrow Y$ est un quotient géométrique si elle vérifie:\\
(i) $\pi$ est surjective et ses fibres sont exactement les orbites.\\
(ii) La topologie de $Y$ coïncide avec la topologie quotient associée à $\pi$.\\
(iii) $\pi^*:\Oo_Y \rightarrow (\pi_*\Oo_X)^G$ est un isomorphisme.
\end{defn}

On remarque que pour un quotient géométrique $(X/G, \pi)$, tous les orbites sont fermés dans $X$ et l'application quotient est ouverte. En effet, soit $U$ un ouvert de $X$, on a $\pi^{-1}(\pi(U))=\cup_{g\in G}g.U$ qui est ouvert.

\begin{ex}\label{groupQuotientGeometrique}
Soit $G$ un groupe algébrique et $H$ un sous-groupe fermé. Dans la catégorie des ensemble, le quotient $(G/H,\pi)$ est exactement le quotient catégorique pour l'action de $H$ sur $G$ par multiplication à droite. Dans \cite{LAGSpringer} 5.5.5, en caractéristique quelconque, on munit $G/H$ d'une structure d'espace annelé en lui attribuant la topologie quotient puis en définissant le faisceau structural par $\Oo_{G/H}(U):=\lbrace f\in\textrm{Map}(U,k)\mid f\pi \in \Oo_G(\pi^{-1}(U))\rbrace$. Par construction, cet espace annelé vérifie la propriété universelle de factorisation. De plus, on montre ensuite qu'il est isomorphe à une variété quasi-projective, ce qui montre l'existence du quotient catégorique $(G/H,\pi)$ dans la catégorie des variétés algébriques. Par définition de $\Oo_{G/H}$, on a une flèche $\Oo_{G/H} \xrightarrow{\pi^\sharp} (\pi_*\Oo_G)^H$. Elle est injective par la surjectivité de $\pi$, et elle est surjective construction du faisceau sur $G/H$. Ainsi, $(G/H,\pi)$ est un quotient géométrique. Cela généralise bien sur le théorème \ref{groupequotient}.
\end{ex}

\begin{ex}
Un bon quotient $(X//G, \pi)$ est un quotient géométrique si les fibres de $\pi$ sont exactement les orbites. En effet, d'après ce qui précède, il reste alors à vérifier que $X//G$ est muni de la topologie quotient. Soit un ouvert de X de la forme $\pi^{-1}(A)$ où $A$ est une partie de $X//G$. En tenant compte de \ref{goodquotientthm} (iii) et de la surjectivité de $\pi$ on a: $\pi(X\setminus \pi^{-1}(A))=Y\setminus A$ qui est fermé, donc $A$ est ouvert.
\end{ex}

\subsubsection{Un exemple: La construction Proj}
Dans cette partie, on va détailler une construction qui à la fois éclaire et généralise la construction de la variété algébrique $\PP^n$($k$). On considère $A$ une algèbre affine $\NN$-graduée et on pose $X:=\spec(A)$, ainsi muni d'une action de $\GG_m$. Une orbite $\GG_m.x$ est de dimension 0 ou 1. Si elle est de dimension 0, c'est un point fixe car $k^*$ est connexe. Si elle est de dimension 1, elle est soit fermée, soit son adhérence est constituée de $\GG_m.x$  et d'une réunion de points fixes, en fait un seul comme on va le voir.

On note $F$ l'ensemble des points fixes et on remarque que $F=\mathcal{V}_X(A_{>0})$, où $A_{>0}:=(f\mid f\in A_d \textrm{ pour un }d>0)$ est l'idéal dit inconvenant. En effet, $F$ est l'ensemble des idéaux maximaux qui sont $\GG_m$-stables. Or, un idéal maximal et homogène contient nécessairement $A_{>0}$.

On remarque que $A^{\GG_m}=A_0=A/A_{>0}$, et comme le bon quotient $Y_0:=X//\GG_m=\spec A_0$ paramètre les orbites fermés, on obtient en particulier que si $\GG_m.x$ n'est pas fermé, son adhérence contient un unique point fixe. En résumé, $W:=X\setminus F$ est la réunion des orbites de dimension 1, et ils sont tous fermés dans $W$. On va maintenant montrer que $W$ admet un quotient géométrique, c'est la construction Proj.

Pour tout $f\in A_{>0}$ homogène, la localisation $A_f$ est $\ZZ$-graduée de la manière suivante:
\begin{center}
$A_f=\bigoplus_{d\in \ZZ}(A_f)_d,\,\,\,\, $ avec $(A_f)_d:=\lbrace h/f^l\mid \textrm{deg}(h)-l\textrm{deg}(f)=d \rbrace$
\end{center}
On note $A_{(f)}:=(A_f)_0$ en remarquant qu'il s'agit de l'algèbre des invariants de la $\GG_m$-variété affine $X_f$. On a ainsi trouvé le bon quotient $U_f:=\spec(A_{(f)})=X_f//\GG_m$. De plus, comme $F\subset\mathcal{V}_X(f)$,  il s'agit d'un quotient géométrique d'après ce qui précède. Or on peut recouvrir $W$ par un nombre fini de $X_{f_i}$ avec les $f_i$ homogènes de degrés $>0$. Considérons les diagrammes commutatifs ci-dessous. Dans le diagramme de gauche, les flèches sont soit des inclusions, soit des localisations. Le diagramme de droite est obtenu par application du foncteur $\spec$:
\begin{multicols}{2}
	\begin{center}
	\begin{tikzcd}
  		A_{f_i} \arrow[r, ""] & A_{f_if_j}  & A_{f_j} \arrow[l, ""]\\ 
  		A_{(f_i)} \arrow[r, ""] \arrow[u, ""] & A_{(f_if_j)}  \arrow[u, ""] & A_{(f_j)} \arrow[l, ""] \arrow[u, ""]
	\end{tikzcd}\\
	\end{center}
	
	\columnbreak
	\begin{center}
	\begin{tikzcd}
  		X_{f_i} \arrow[d, "\pi_i"] & X_{f_i}\cap X_{f_j}=X_{f_jf_j} \arrow[l, ""] \arrow[d, ""] \arrow[r, ""] & X_{f_j}  \arrow[d, "\pi_j"]\\ 
  		U_{f_i}  & U_{f_if_j}  \arrow[l, ""] \arrow[r, ""]   & U_{f_j} 
	\end{tikzcd}\\
	\end{center}
\end{multicols}
Les flèches horizontales du diagramme de droite sont des immersions ouvertes, en effet on a $U_{f_if_j}\simeq (U_{f_i})_{f_j}$. Les conditions de la construction \ref{gluevar} sont satisfaites, on peut former un schéma $\proj(A)$ par recollement des $U_{f_i}$ le long de ces immersions. Il est facile de montrer que $\proj k[t_1,...,t_n]$, où les $t_i$ sont des indéterminées, est séparé en utilisant \ref{sepCritere}. On en déduit que $\proj(A)$ est séparé en utilisant \ref{sepCritere2}, et donc que c'est une variété. Enfin, les $(U_{f_i},\pi_i)$ sont des quotients géométriques et le diagramme exprime que l'on a les conditions de recollement sur les intersections qui font de $\proj(A)$ le quotient géométrique global $W/\GG_m$. 

Enfin, on remarque que les fermés de $\proj(A)$ correspondent aux fermés $k^*$-stables de $W$, c'est à dire aux idéaux radicaux homogènes qui ne contiennent pas l'idéal inconvenant. On appelle ces fermés des variétés projectives. La topologie quotient sur $\proj(A)$ que l'on vient de définir est aussi appelé la topologie de Zariski.
